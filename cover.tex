 
% ====================================================================
% <<Titelei>>
% ====================================================================

% Kopf- und Fußzeile für die Titelseite
\newpagestyle{first}{
    {}
    {\hspace{10mm} \vspace{-3mm} \parbox{70mm}{
            \Large\upshape
            \hfill \\[2mm]
            Universität Heidelberg\\
            Fachschaft MathPhysInfo\\
            Luka Thomé \vspace{4mm}
        } \hfill \parbox{80mm}{
            \centering\includegraphics[height=20mm]{./_img/mathphysinfo-logo_4c.pdf}
        } \\[0.5em]
    }
    {}
}{
    {}
    {\begin{center}
        \itshape Habt ihr Fragen, Wünsche, Anregungen? Schreibt uns einfach eine\\
        E-Mail: \upshape\href{mailto:fachschaft@mathphys.stura.uni-heidelberg.de}{fachschaft@mathphys.stura.uni-heidelberg.de}
    \end{center}}
    {}
}

\newcommand{\firstpage}{
    % Satzspiegel für den Rest des Skripts
    \KOMAoptions{
        headinclude=true, % Kopfzeilen in den Textkörper mit einbeziehen
        DIV=14, % Teilungsfaktor zur Berechnung des Satzspiegels
        twoside=semi % Doppelseitige Kopfzeilen bei gleichen Rändern
    }
    \begin{titlepage}
    \thispagestyle{first}
    \newgeometry{centering, text={1.3\textwidth, \paperheight}} % Seitenlayout für die Titelseite
    \setlength{\headheight}{50mm} % Kopfzeile vergrößern
    \setlength{\footheight}{90mm} % Fußzeile vergrößern
    \vspace*{30mm}
    \begin{center}
        {\Large\itshape Skript zum}\\[3ex]
        {\Huge\ifluatex\setmainfont{EB Garamond}\fi\scshape Mathematischen Vorkurs}\\[5ex]
        {\Large\itshape Wintersemester \oldstylenums{2024/25}}
    \end{center}
    \end{titlepage}
    \begin{small}
        \noindent Fassung Stand \today \\[1.5em]
        3. Auflage 2024 \\[0.5em]
        Autor der dritten Auflage: Luka Thomé \\[1.5em]
        2. Auflage 2022 \\[0.5em]
        Autor der zweiten Auflage: Luka Thomé \\[0.5em]
        Besonderen Dank an Florian Frauen \\[1.5em]
        1. Auflage 2021 \\[0.5em]
        Autoren der ersten Auflage: \\[0.25em]
        Matthis Scholz \\
        Nikolaus Betker \\
        Maximilian Bur \\
        Luna Cielibak \\
        Luka Thomé (Leitung) \\[1.5em]
        {\textcopyright} Fachschaft MathPhysInfo \\
        Im Neuenheimer Feld 205 (Mathematikon), Raum 01.301\\
        69120 Heidelberg \\
        Telefon: +49 6221 54 14 999 \\
        Fax: +49 6221 54 161 14 999 \\
        E-Mail: \href{mailto:fachschaft@mathphys.stura.uni-heidelberg.de}{fachschaft@mathphys.stura.uni-heidelberg.de}\\
        Webseite: \href{https://mathphys.stura.uni-heidelberg.de}{https://mathphys.stura.uni-heidelberg.de}
    \end{small}
}


% ====================================================================
% <<Rückseite>>
% ====================================================================

\newcommand{\lastpage}{
    \cleardoubleevenpage
    % Letzte Seite zentrieren und ohne Seitenzahl
    \thispagestyle{scrplain}
    \newgeometry{centering}
    
    \pdfbookmark[-1]{Griechisches Alphabet}{alphabet}
    \centering{\color{gray}\huge\sffamily\textbf{Griechisches Alphabet}}
    \vspace*{15mm}
    \Large
    \begin{longtable}{lcccc}
        & Majuskel & Minuskel & Deutsches Pendant \\
        \midrule
        Alpha & $\mathrm{A}$ & $\alpha$ & A & \\
        Beta & $\mathrm{B}$ & $\beta$ & B & \\
        Gamma & $\Gamma$ & $\gamma$ & G & \\
        Delta & $\Delta$ & $\delta$ & D & \\
        Epsilon & $\mathrm{E}$ & $ \varepsilon$, $\epsilon$ & E \\
        Zeta & $\mathrm{Z}$ & $\zeta$ & Z & \\
        Eta & $\mathrm{H}$ & $\eta$ & E \\
        Theta & $\Theta$ & $\vartheta$, $\theta$ & Th \\
        Iota & $\mathrm{I}$ & $\iota$ & I & \\
        Kappa & $\mathrm{K}$ & $\kappa$ & K & \\
        Lambda & $\Lambda$ & $\lambda$ & L & \\
        My & $\mathrm{M}$ & $\mu$ & M & \\
        Ny & $\mathrm{N}$ & $\nu$ & N & \\
        Xi & $\Xi$ &$\xi$ & X & \\
        Omikron & $\mathrm{O}$ & $o$ & O \\
        Pi & $\Pi$ & $\pi$, $\varpi$ & P & \\
        Rho & $\mathrm{P}$ & $\varrho$, $\rho$ & R & \\
        Sigma & $\Sigma$ & $\sigma$, $\varsigma$  & S \\
        Tau & $\mathrm{T}$ & $\tau$ & T & \\
        Ypsilon & $\Upsilon$ & $\upsilon$ & Y & \\
        Phi & $\Phi$ & $\varphi$, $\phi$ & Ph & \\
        Chi & $\mathrm{X}$ & $\chi$ & Ch \\
        Psi & $\Psi$ & $\psi$ & Ps \\
        Omega & $\Omega$ & $\omega$ & O
    \end{longtable}
}
