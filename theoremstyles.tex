
% ====================================================================
% <<Makros>>
% ====================================================================

\newcommand{\thmheadstyle}{\llap{\color{gray}\footnotesize\mdseries\NUMBER\ }\bfseries\NAME\small\mdseries\itshape\NOTE}

% Gemeinsamen Counter für alle theorems einstellen
\newcounter{thmcounter}
\counterwithin{thmcounter}{section}

% Umstellungen für den Anhang
\newcommand{\switchtoappendixthms}{
    \counterwithout{thmcounter}{section}
    \counterwithin{thmcounter}{chapter}
    \renewcommand{\thmheadstyle}{\llap{\color{gray}\footnotesize\mdseries\S\Alph{chapter}.\arabic{thmcounter} }\bfseries\NAME\small\mdseries\itshape\NOTE} % Pragraphenzeichen am linken Rand
}


% ====================================================================
% <<Definitionen der theoremstyles>>
% ====================================================================

% Für Absätze vom Typ „Resultat“
\declaretheoremstyle[
    spaceabove=\topsep,
    spacebelow=\topsep,
    headfont=\colorres\bfseries,
    notefont=\itshape,
    bodyfont=\itshape,
    postheadspace=0.5em,
    headformat=\thmheadstyle
]{styres}

% Für Absätze vom Typ „Beweis“
\declaretheoremstyle[
    qed=\qedsymbol,
    spaceabove=0.5\topsep,
    spacebelow=\topsep,
    headfont=\colorbew\bfseries,
    notefont=\itshape,
    postheadspace=0.5em,
    headformat=\thmheadstyle,
    preheadhook=\KOMAoptions{parskip=half-},
    postheadhook=\vspace{-\parskip}
]{stybew}

% Für Absätze vom Typ „Definition“
\declaretheoremstyle[
    spaceabove=\topsep,
    spacebelow=\topsep,
    headfont=\colordefin\bfseries,
    notefont=\itshape,
    postheadspace=0.5em,
    headformat=\thmheadstyle
]{stydefin}

% Für Absätze vom Typ „Schreib- und Sprechweise“
\declaretheoremstyle[
    spaceabove=\topsep,
    spacebelow=\topsep,
    headfont=\colorsch\bfseries,
    notefont=\itshape,
    postheadspace=0.5em,
    headformat=\thmheadstyle
]{stysch}

% Für Absätze vom Typ „Axiom“
\declaretheoremstyle[
    spaceabove=\topsep,
    spacebelow=\topsep,
    headfont=\coloraxiom\bfseries,
    notefont=\itshape,
    postheadspace=0.5em,
    headformat=\thmheadstyle
]{styaxiom}

% Für Absätze vom Typ „Bemerkung“
\declaretheoremstyle[
    spaceabove=\topsep,
    spacebelow=\topsep,
    headfont=\colorbem\bfseries,
    notefont=\itshape,
    postheadspace=0.5em,
    headformat=\thmheadstyle
]{stybem}

% Für Absätze vom Typ „Beispiel“
\declaretheoremstyle[
    spaceabove=\topsep,
    spacebelow=\topsep,
    headfont=\colorbsp\bfseries,
    notefont=\itshape,
    postheadspace=0.5em,
    headformat=\thmheadstyle
]{stybsp}

% Für Absätze vom Typ „Aufgabe“
\declaretheoremstyle[
    spaceabove=\topsep,
    spacebelow=2.0\topsep,
    headfont=\coloraufg\bfseries,
    notefont=\itshape,
    postheadspace=0.5em,
    headformat=\thmheadstyle
]{styaufg}


% ====================================================================
% <<Einrichtung der einzelnen amsthm-Umgebungen>>
% ====================================================================

% Satz
\declaretheorem[
    style=styres, % verwendeter theoremstyle
    name=Satz,
    refname={\colorres Satz\color{gray}}, % Anzeige in \cref-Referenzen.
    sharenumber=thmcounter % counter festlegen
]{satz}

% Lemma
\declaretheorem[
    style=styres,
    name=Lemma,
    refname={\colorres Lemma\color{gray}},
    sharenumber=thmcounter
]{lem}

% Korollar
\declaretheorem[
    style=styres,
    name=Korollar,
    refname={\colorres Korollar\color{gray}},
    sharenumber=thmcounter
]{kor}

% Beweis
\declaretheorem[
    style=stybew,
    name=Beweis,
    refname={\colorbew Beweis\color{gray}},
    sharenumber=thmcounter
]{bew}

% Axiom
\declaretheorem[
    style=styaxiom,
    name=Axiom,
    refname={\coloraxiom Axiom\color{gray}},
    sharenumber=thmcounter
]{axiom}

% Definition
\declaretheorem[
    style=stydefin,
    name=Definition,
    refname={\colordefin Definition\color{gray}},
    sharenumber=thmcounter
]{defin}

% Notation
\declaretheorem[
    style=stysch,
    name=Notation,
    refname={\colorsch Notation\color{gray}},
    sharenumber=thmcounter
]{nota}

% Bemerkung
\declaretheorem[
    style=stybem,
    name=Bemerkung,
    refname={\colorbem Bemerkung\color{gray}},
    sharenumber=thmcounter
]{bem}

% Anmerkung
\declaretheorem[
    style=stybem,
    name=Anmerkung,
    refname={\colorbem Anmerkung\color{gray}},
    sharenumber=thmcounter
]{anm}

% Vorschau
\declaretheorem[
    style=stybem,
    name=Vorschau,
    refname={\colorbem Vorschau\color{gray}},
    sharenumber=thmcounter
]{vorschau}

% Beispiel
\declaretheorem[
    style=stybsp,
    name=Beispiel,
    refname={\colorbsp Beispiel\color{gray}},
    sharenumber=thmcounter
]{bsp}

% Aufgabe
\declaretheorem[
    style=styaufg,
    name=Aufgabe,
    refname={\coloraufg Aufgabe\color{gray}},
    sharenumber=thmcounter
]{aufg}

% Lösung
\declaretheorem[
    style=styaufg,
    name=Lösung,
    refname={\coloraufg Lösung\color{gray}},
    sharenumber=thmcounter
]{loes}


% Einmalige theorems für den Abschnitt „Entstehungsprozess eines Beweises“
\declaretheorem[style=stysch,name={Phase 1},sharenumber=thmcounter]{phaseone}
\declaretheorem[style=stysch,name={Phase 2},sharenumber=thmcounter]{phasetwo}
\declaretheorem[style=stysch,name={Phase 3},sharenumber=thmcounter]{phasethree}
