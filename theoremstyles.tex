

% Makros für die theoremstyles
\newcommand{\thmheadstyle}{\llap{\color{gray}\footnotesize\mdseries\S\NUMBER\ }\bfseries\NAME\small\mdseries\itshape\NOTE}
\newlength{\thmrefnumbertweakwidth} % Ad-hoc-Lösung, um die Platzierung des Paragraphenzeichens in \cref-Theoremnummern zu fixen
\newcommand{\thmrefstyle}{\color{gray}\S\settowidth{\thmrefnumbertweakwidth}{\space}\kern-\thmrefnumbertweakwidth} % Anzeige der Theoremnummern in cref-Links


% ====================================================================
% <<Definitionen der theoremstyles>>
% ====================================================================

% Für Absätze vom Typ „Resultat“
\declaretheoremstyle[
    spaceabove=\topsep,
    spacebelow=\topsep,
    headfont=\colorres\bfseries,
    notefont=\itshape,
    bodyfont=\itshape,
    postheadspace=0.5em,
    headformat=\thmheadstyle % Gestalt der theorem-Titel (wird im classfile definiert)
]{styres}

% Für Absätze vom Typ „Beweis“
\declaretheoremstyle[
    qed=\qedsymbol,
    spaceabove=0.5\topsep,
    spacebelow=\topsep,
    headfont=\colorbew\bfseries,
    notefont=\itshape,
    postheadspace=0.5em,
    headformat=\thmheadstyle
]{stybew}

% Für Absätze vom Typ „Definition“
\declaretheoremstyle[
    spaceabove=\topsep,
    spacebelow=\topsep,
    headfont=\colordefin\bfseries,
    notefont=\itshape,
    postheadspace=0.5em,
    headformat=\thmheadstyle
]{stydefin}

% Für Absätze vom Typ „Schreib- und Sprechweise“
\declaretheoremstyle[
    spaceabove=\topsep,
    spacebelow=\topsep,
    headfont=\colorsch\bfseries,
    notefont=\itshape,
    postheadspace=0.5em,
    headformat=\thmheadstyle
]{stysch}

% Für Absätze vom Typ „Axiom“
\declaretheoremstyle[
    spaceabove=\topsep,
    spacebelow=\topsep,
    headfont=\coloraxiom\bfseries,
    notefont=\itshape,
    postheadspace=0.5em,
    headformat=\thmheadstyle
]{styaxiom}

% Für Absätze vom Typ „Bemerkung“
\declaretheoremstyle[
    spaceabove=\topsep,
    spacebelow=\topsep,
    headfont=\colorbem\bfseries,
    notefont=\itshape,
    postheadspace=0.5em,
    headformat=\thmheadstyle
]{stybem}

% Für Absätze vom Typ „Beispiel“
\declaretheoremstyle[
    spaceabove=\topsep,
    spacebelow=\topsep,
    headfont=\colorbsp\bfseries,
    notefont=\itshape,
    postheadspace=0.5em,
    headformat=\thmheadstyle
]{stybsp}

% Für Absätze vom Typ „Aufgabe“
\declaretheoremstyle[
    spaceabove=\topsep,
    spacebelow=2.0\topsep,
    headfont=\coloraufg\bfseries,
    notefont=\itshape,
    postheadspace=0.5em,
    headformat=\thmheadstyle
]{styaufg}


% ====================================================================
% <<Einrichtung der einzelnen amsthm-Umgebungen>>
% ====================================================================

% Proposition (Dummy-Umgebung, die den counter für alle anderen Umgebungen bereitstellt)
\declaretheorem[
    style=styres, % verwendeter theoremstyle
    name=Proposition,
    refname={\colorres Proposition \thmrefstyle}, % Anzeige in \cref-Referenzen. Das Makro \thmrefstyle wird im classfile definiert
    parent=section % Nummerierung jede section erneut beginnen
]{prop}

% Satz
\declaretheorem[
    style=styres,
    name=Satz,
    refname={\colorres Satz \thmrefstyle},
    sibling=prop % Nummerierung vom "prop"-Counter übernehmen
]{satz}

% Lemma
\declaretheorem[
    style=styres,
    name=Lemma,
    refname={\colorres Lemma \thmrefstyle},
    sibling=prop
]{lem}

% Korollar
\declaretheorem[
    style=styres,
    name=Korollar,
    refname={\colorres Korollar \thmrefstyle},
    sibling=prop
]{kor}

% Beweis
\declaretheorem[
    style=stybew,
    name=Beweis,
    refname={\colorbew Beweis \thmrefstyle},
    sibling=prop
]{bew}

% Axiom
\declaretheorem[
    style=styaxiom,
    name=Axiom,
    refname={\coloraxiom Axiom \thmrefstyle},
    sibling=prop
]{axiom}

% Definition
\declaretheorem[
    style=stydefin,
    name=Definition,
    refname={\colordefin Definition \thmrefstyle},
    sibling=prop
]{defin}

% Notation
\declaretheorem[
    style=stysch,
    name=Notation,
    refname={\colorsch Notation \thmrefstyle},
    sibling=prop
]{nota}

% Bemerkung
\declaretheorem[
    style=stybem,
    name=Bemerkung,
    refname={\colorbem Bemerkung \thmrefstyle},
    sibling=prop
]{bem}

% Anmerkung
\declaretheorem[
    style=stybem,
    name=Anmerkung,
    refname={\colorbem Anmerkung \thmrefstyle},
    sibling=prop
]{anm}

% Vorschau
\declaretheorem[
    style=stybem,
    name=Vorschau,
    refname={\colorbem Vorschau \thmrefstyle},
    sibling=prop
]{vorschau}

% Beispiel
\declaretheorem[
    style=stybsp,
    name=Beispiel,
    refname={\colorbsp Beispiel \thmrefstyle},
    sibling=prop
]{bsp}

% Aufgabe
\declaretheorem[
    style=styaufg,
    name=Aufgabe,
    refname={\coloraufg Aufgabe \thmrefstyle},
    sibling=prop
]{aufg}


% ====================================================================
% <<Umgebungen, die nur im Anhang verwendet werden>>
% ====================================================================

\declaretheorem[
    style=styres, % verwendeter theoremstyle
    name=Proposition,
    refname={\colorres Proposition \thmrefstyle},
    parent=chapter
]{propchap} % Genau wie prop, nur mit kapitelweiser Nummerierung


% Beweis
\declaretheorem[
    style=stybew,
    name=Beweis,
    refname={\colorbew Beweis \thmrefstyle},
    sibling=propchap
]{appendixbew}

% Aufgabe
\declaretheorem[
    style=styaufg,
    name=Aufgabe,
    refname={\coloraufg Aufgabe \thmrefstyle},
    sibling=propchap
]{appendixaufg}

% Bemerkung
\declaretheorem[
    style=stybem,
    name=Bemerkung,
    refname={\colorbem Bemerkung \thmrefstyle},
    sibling=propchap
]{appendixbem}


% Einmalige theorems für den Abschnitt „Entstehungsprozess eines Beweises“
\declaretheorem[style=stysch,sibling=propchap,name={\colorsch\bfseries Phase 1}]{phaseone}
\declaretheorem[style=stysch,sibling=propchap,name={\colorsch\bfseries Phase 2}]{phasetwo}
\declaretheorem[style=stysch,sibling=propchap,name={\colorsch\bfseries Phase 3}]{phasethree}
