% ====================================================================
%                                                                 
%
%       §01
%
%
% %%ts latex start%%[2019-03-07 Thu 14:45]%%ts latex end%%
% ====================================================================
%
% --------------------------------------------------------------------
% §1 Section <<Klassische Aussagenlogik>>
% --------------------------------------------------------------------





\section{Buchstaben in mathematischen Formeln}


\begin{de}[Zeichen mit einer konkreten Bedeutung versehen]
 Gelegentlich ist es bequem ein konkretes, kompliziert definiertes Objekt mit einem Zeichen zu bezeichnen. Beispielsweise lässt sich mit Methoden der Analysis zeigen, dass die Gleichung $x^5=x+1$ genau eine Lösung in den reellen Zahlen besitzt\footnote{Die Werkzeuge dafür werden in der Analysis 1-Vorlesung (oder vielleicht auch schon in der Schule) vermittelt.} wohingegen sich mit Methoden der Algebra zeigen lässt, dass sich diese Lösung nicht mit den Operationen $+$, $-$, $\cdot$, $:$, $\sqrt{}$ konstruieren lässt\footnote{Die Werkzeuge dafür werden in der Drittsemestervorlesung „Algebra 1“ vermittelt.}. Möchte man nun mit dieser Lösung arbeiten, so ist es umständlich, immer wieder „die eindeutige reelle Lösung der Gleichung $x^5=x+1$“ zu schreiben und es ist bequemer, einen Buchstaben zu verwenden. Dafür benutzen Mathematiker einen Imperativ:
 \begin{quote}
„Sei $\xi$ die eindeutige reelle Lösung der Gleichung $x^5=x+1$.“  
 \end{quote}
 Nun lassen sich komfortabel Sätze wie etwa „Es gilt $\xi^2\cdot (\xi^{8}-1)=2\xi +1$“ formulieren. \\
Um formal ein Zeichen mit einer konkreten Bedeutung zu versehen, kann man das Symbol
 \begin{align*}
  := &&& (\text{lies: „ist definiert als“ oder „ist per Definition gleich“})
 \end{align*}
verwenden. Beispielsweise wird im Ausdruck
 \[ d:= \text{ggT}(108,2048) \]
 festgelegt, dass der Buchstabe $d$ den größten gemeinsamen Teiler von $108$ und $2048$ bezeichnet.  \\
Manchmal ist es deutlich lesbarer, Umgangssprache zu verwenden, als auf Teufel komm raus eine Formel mit dem Zeichen „$:=$“ hinzuschreiben. Z.B. ist
\begin{quote}
 Sei $x$ die kleinste positive Nullstelle der Kosinusfunktion.
\end{quote}
sicherlich für viele Leute eine schneller verständlichere Definition als
\[ x:= \min \{y\in \Rz_{> 0} \mid \cos(y)=0 \}\]
Dies ist auch eine Geschmacksfrage.
\end{de}



\begin{de}[Variable] \label{variable}
 Eine \textbf{Variable} ist ein Zeichen, das als Platzhalter dient, an dessen Stelle Objekte von einer gewissen Sorte eingesetzt werden können. Die Sorte von Objekten, die für eine Variable eingesetzt werden kann, heißt der \textbf{Typ} dieser Variable. In modernen Mathebüchern werden Variablen meistens als kursive Buchstaben gedruckt. Ansonsten ist aber alles erlaubt: Großbuchstaben, Kleinbuchstaben, griechische Buchstaben, Hiraganazeichen usw. Beispielsweise können die folgenden Zeichen alle als Variablen verwendet werden:
 \[ A,B,m,n,\delta,\mu, x,y,\dots \]
 Prinzipiell darfst du jedes Zeichen als Variable verwenden, solange du festlegst, für welche Sorte von Objekten es als Platzhalter dient (z.B. „Sei $n$ eine ungerade natürliche Zahl“). Allerdings haben sich in der Mathematik diverse Variablen-Konventionen eingebürgert, die du befolgen solltest, um deinen Text für andere Mathematiker leichter lesbar zu machen. Zum Beispiel:
 \begin{itemize}
  \item Natürliche Zahlen werden meist mit den Buchstaben $n,m,k,\dots$ bezeichnet.
  \item Reelle Zahlen mit den Buchstaben $x,y,\dots$.
    \item Komplexe Zahlen mit den Buchstaben $z,w,\dots$.
  \item Funktionen werden meist mit den Buchstaben $f,g,\dots$ bezeichnet.
  \item Vektoren mit den Buchstaben $v,w,\dots$.
  \item In der mathematischen Logik kommt es sogar vor, dass „Metavariablen“ vom Typ „Variable“ auftauchen („Seien $x,y$ zwei Variablen“).
 \end{itemize}
 Das alles sind aber keine strikten Regeln und du solltest, wann immer du Variablen verwendest, klarstellen, von welchem Typ sie sind. Um die Festlegung eines Variablentyps sprachlich hervorzuheben, verwenden Mathematiker dafür den Imperativ. Wie beispielsweise in „Seien $x,y$ zwei reelle Zahlen“ oder „Es bezeichne $n$ eine beliebige natürliche Zahl“.
\end{de}



\begin{bem}[Variablen nie vom Himmel fallen lassen!]
Erstsemester vergessen nicht selten, ihre Variablen sachgemäß einzuführen. Wann immer du eine Variable wie „$x$“ oder „$A$“ verwendest, solltest du, z.B. mit einem „Sei\dots“-Imperativ, klarstellen, auf welche Sorte von Objekten sie sich bezieht. Es ist \emph{sehr} nervig und verwirrend, wenn in Texten plötzlich Buchstaben auftreten, für die nie klargestellt wurde, was sie zu bedeuten haben. 
\end{bem}






\section{Ein bisschen Aussagenlogik}

\begin{de}[Aussage]
Eine \textbf{Aussage} ist ein feststellender Satz, dem ein Wahrheitswert wie „wahr“ oder „falsch“ zugeordnet werden kann und
der vermittels der in diesem Paragraphen besprochenen \emph{Junktoren} mit anderen Aussagen verknüpft werden kann. %Als Aussagenvariablen werden in der deutschen Literatur oft die Buchstaben $A,B,\dots$ verwendet, in der englischen Literatur dagegen $P,Q,\dots$ ($P$ wie ``Proposition").
	\end{de}

	


	\begin{bsp}
Parallel zur abstrakten Theorie werden uns in diesem Paragraphen die folgenden Beispielaussagen begleiten:
		\begin{itemize}
			\item $B_1 :=$ „Der Döner wurde in Deutschland erfunden.“
			\item $B_2 :=$ „Heute ist Mittwoch.“
			\item $B_3 :=$ „Es gibt außerirdisches Leben.“
			\item $B_4 :=$ „Der FC Bayern spielte eine schlechte Hinrunde.“
			\item $B_5 :=$ „Die Relativitätstheorie ist fehlerhaft.“
		\end{itemize}
Nicht jeder deutsche Satz ist eine Aussage. Sätze, die eher nicht als Aussagen durchgehen würden, sind zum Beispiel: \quad
 \begin{itemize}
\item „Hoffentlich kommt der Kellner bald.“
\item „Was möchten Sie trinken?“
\item „Ein großes Bier, bitte!“
\end{itemize}
	\end{bsp}
	


Eine systematische Operation, die aus einer Handvoll Aussagen eine neue Aussage hervorbringt (ähnlich wie etwa die Operation „$+$“ aus zwei Zahlen ihre Summe macht) heißt \textbf{Junktor} oder auch \textbf{logischer Operator}. In diesem Abschnitt werden die wichtigsten Junktoren vorgestellt.
	\begin{de}[Negation]
		Für eine Aussage $A$ wird mit
    \begin{align*}
   \neg A   && (\text{lies: „nicht $A$“})
    \end{align*}
die \emph{Negation} von $A$ notiert. $\neg A$ ist die Verneinung von $A$, d.h. $\neg A$ besagt, dass $A$ nicht zutrifft. Beispielsweise sind die Negationen der Beispielaussagen von vorhin gegeben durch:
 \begin{itemize}
			\item $\neg B_1 =$ „Der Döner wurde nicht in Deutschland erfunden.“
			\item $\neg B_2 =$ „Heute ist nicht Mittwoch.“
			\item $\neg B_3 =$ „Es gibt kein außerirdisches Leben.“
			\item $\neg B_4 =$ „Der FC Bayern spielte keine schlechte Hinrunde.“
			\item $\neg B_5 =$ „Die Relativitätstheorie ist fehlerfrei.“
    \end{itemize}
    An den Beispielen wird deutlich, dass die Negation einer Aussage nicht immer durch das Signalwort „nicht“ erfolgen muss. Manchmal wird die Negation einer Aussage $A$ auch mit einem Oberstrich notiert: $\overline{A}$.
	\end{de}
	

	
	\begin{comment}
	\begin{bem}[Zeichen mit mehr als einer Bedeutung „überladen“]
In dieser Definition haben wir das Zeichen „$A$“ überladen. Denn einerseits bezeichnet $A$ ja schon die Aussage „Der Döner wurde in Deutschland erfunden“, andererseits beginnt die Definition mit „Für eine Aussage $A$ wird\dots“. Wir haben das Zeichen also zugleich als Konstantensymbol und als Variable verwendet. Solche Überladungen von Zeichen sind häufige Fehlerquellen, sollten soweit es geht vermieden und wirklich nur dann verwendet werden, wenn keine Verwirrung auftreten kann und es unmittelbar klar ist, dass eine Überladung vorliegt.
	\end{bem}
\end{comment}


	
	\begin{de}[Und-Verknüpfung]
            Zwei Aussagen $A,B$ können zu ihrer \textbf{Konjunktion}
\begin{align*}
  A\land B && (\text{lies: „$A$ und $B$“})
\end{align*}
 verknüpft werden, deren Bedeutung ist, dass sowohl $A$ als auch $B$ zutreffen. Beispiele für Konjunktionen sind etwa:
\begin{itemize}
    \item $B_2\land B_4 =$ „Heute ist Mittwoch und der FC Bayern spielte eine schlechte Hinrunde.“
\item $B_3\land B_5 =$ „Es gibt außerirdisches Leben, aber die Relativitätstheorie ist fehlerhaft.“
    \item $B_5\land B_1 =$ „Nicht nur ist die Relativitätstheorie fehlerhaft -- auch der Döner wurde in Deutschland erfunden.“
\end{itemize}
	\end{de}
	
	
	
\begin{bem}[Fachbegriffe]
 Du brauchst dir im Vorkurs nicht gleich alle Fachbegriffe zu merken. Sofern du weißt, dass es eine Und- und eine Oder-Verknüpfung gibt, brauchst du dir nicht merken, dass sie auch „Konjunktion“ und „Disjunktion“ genannt werden. In diesem und den folgenden Vorträgen werden wir dennoch oft mehrere Wörter für dasselbe Konzept nennen, um dir zu erleichtern, die Begriffe im Internet nachzuschlagen.
\end{bem}



	\begin{de}[Oder-Verknüpfung]
Zwei Aussagen $A,B$ können zu ihrer \textbf{Disjunktion}
\begin{align*}
  A\lor B && (\text{lies: „$A$ oder $B$“})
\end{align*}
verknüpft werden, deren Bedeutung ist, dass von $A, B$ mindestens eine Aussage zutrifft. Beispiele für Disjunktionen sind:
\begin{itemize}
    \item $B_1 \lor B_3 =$ „Der Döner wurde in Deutschland erfunden oder es gibt außerirdisches Leben.“
\item $B_2\lor B_5 =$ „Wahlweise ist heute Mittwoch oder die Relativitätstheorie fehlerhaft.“
\item $B_4\lor B_4=$ „Der FC Bayern spielte eine schlechte Hinrunde oder der FC Bayern spielte eine schlechte Hinrunde.“
%\item Achtung: $B_2\lor\phantom{C} $ = „Heute ist Mittwoch, oder?“ ist \textbf{keine} erlaubte Verwendung des $\lor$-Symbols in der Aussagenlogik! Sowohl „$\land$“ als auch „$\lor$“ verknüpfen stets \emph{zwei} Aussagen (die nicht unbedingt voneinander verschieden sein müssen) miteinander.
\end{itemize}
	\end{de}

	
\begin{bem}[Ausschließendes Oder]
 Die Disjunktion bezeichnet ein \emph{einschließendes Oder} (lateinisch: „vel“), d.h. $A\lor B$ schließt auch den Fall ein, dass $A$ und $B$ beide gelten. In einer Mathematiker-Beziehung würde das Ultimatum „Unsere Beziehung -- oder deine dummen Fernsehserien!“ keine Besorgnis erregen. Das „oder“ lässt ja auch zu, dass beides vorliegen kann. Möchte man ein ausschließendes Oder (Informatiker sprechen vom XOR, im Lateinischen wird das Wort „aut“ benutzt) verwenden, kann man dies durch
 \begin{align*}
  A\ \dot\lor\ B && (\text{lies: „Entweder $A$ oder $B$“})
\end{align*}
notieren. Das „Entweder $A$ oder $B$“ soll soviel wie „$A$ oder $B$ aber nicht beides“ bedeuten. Das Ultimatum „\emph{Entweder} unsere Beziehung oder deine dummen Fernsehserien“ könnte selbst bei einem Mathematiker-Pärchen eine handfeste Beziehungskrise auslösen. \\
Beispielsweise ist
\begin{itemize}
 \item „Eine natürliche Zahl ist entweder eine gerade oder eine ungerade Zahl” eine korrekte Aussage, während
 \item „Jeder Vorkursteilnehmer ist entweder MathestudentIn oder InformatikstudentIn“ falsch ist, da es ja VorkursteilnehmerInnen gibt, die beides studieren.
\end{itemize}

\end{bem}



	\begin{de}[Implikationspfeil]
Zwei Aussagen $A,B$ können zur („materiellen“) \textbf{Implikation}
\[ A\to B \]
verknüpft werden: Deren Bedeutung ist, dass $B$ von $A$ impliziert wird. Dafür gibt es verschiedene Lesarten:
\begin{itemize}
\item „$A$ impliziert $B$“.
 \item „Wenn $A$ so auch $B$“.
 \item „Falls $A$, dann $B$“.
 \item „$B$ folgt aus $A$“.
 \item „$A$ ist eine hinreichende Bedingung für $B$“.
 \item „$B$ ist eine Konsequenz von $A$“.
 \item usw.
\end{itemize}
Man nennt den Pfeil „$\to$“ auch den \textbf{Implikationspfeil} und die Aussage $A\to B$ auch ein \emph{Konditional}. Beispiele für $\to$-Aussagen sind:
\begin{itemize}
 \item $B_1\to B_5$: „Wenn der Döner in Deutschland erfunden wurde, ist die Relativitätstheorie fehlerhaft“.
 \item $B_2\to B_4$: „Sofern der FC Bayern eine schlechte Hinrunde gespielt hat, ist heute Mittwoch.
 \item $B_3\to B_5$: „Unter der Annahme, dass es außerirdisches Leben gibt, ist die Relativitätstheorie fehlerhaft.“
\end{itemize}
Beachte, dass es beim Implikationspfeil „$\to$“ wesentlich auf die Reihenfolge ankommt. Während sich etwa die Aussagen $A\land B$ und $B\land A$ nicht in ihrer Bedeutung unterscheiden, sind $A\to B$ und $B\to A$ zwei grundlegend verschiedene Aussagen. Beispielsweise sind
\begin{itemize}
 \item[(1)] „Wenn heute Freitag ist, ist morgen Wochenende“.
 \item[(2)] „Falls morgen Wochenende ist, ist heute Freitag“.
\end{itemize}
zwei erheblich verschiedene Aussagen. Aussage (1) ist korrekt aber Aussage (2) ist falsch, da ja auch Samstag sein könnte.
\end{de}
	\begin{de}[Äquivalenz]
		Zwei Aussagen $A$ und $B$ lassen sich zur \textbf{Äquivalenz}
    \[ A\leftrightarrow B \]
    verknüpfen, dessen Bedeutung ist, dass sowohl $B$ von $A$ impliziert wird als auch $A$ von $B$ impliziert wird. Lesarten dafür sind:
    \begin{itemize}
    \item „$A$ ist äquivalent zu $B$“.
     \item „$A$ genau dann wenn $B$“. Ist wenig Platz vorhanden, schreibt man abkürzend „$A$ gdw. $B$“. In der englischen Literatur schreibt man ``$A$ iff $B$''.
    \item „$A$ gilt dann und nur dann, wenn $B$“.
    \end{itemize}
Man nennt den Doppelpfeil „$\leftrightarrow$“ \textbf{Äquivalenzzeichen} und die Aussage $A\leftrightarrow B$ auch ein \emph{Bikonditional}. Beispiele für Äquivalenzaussagen sind:
\begin{itemize}
     \item „Genau dann ist heute Mittwoch, wenn morgen Donnerstag ist“.
     \item $B_1\leftrightarrow B_3$: „Dass der Döner in Deutschland erfunden wurde, ist äquivalent dazu, dass es außerirdisches Leben gibt“.
     \item Eine reelle Zahl $x$ ist dann und nur dann eine negative reelle Zahl, wenn $-x$ eine positive reelle Zahl ist.
    \end{itemize}
	\end{de}


	
\begin{bem}[* Die Pfeile $\to$ und $\Rightarrow$]
 In der Analysis formuliert man Aussagen, die von Funktionen und Folgen handeln; in der Algebra geht es um Aussagen, die von Vektoren, Polynomen, etc. handeln. -- In der mathematischen Logik formuliert man Aussagen, die von Aussagen handeln. Daher liegen dort oft zwei Sprachebenen vor, die \emph{Objektsprache}, in der diejenigen mathematischen Aussagen formuliert sind, die vom Logiker untersucht werden, sowie die \emph{Metasprache}, der sich der Logiker bedient, um die objektsprachlichen Aussagen zu untersuchen. Ein Beispiel:
 \[ \underbrace{\text{Die Aussage „}}_{\text{Metasprache}}\underbrace{\text{$5$ ist eine gerade Zahl}}_{\text{Objektsprache}} \underbrace{\text{“ ist falsch.}}_{\text{Metasprache}} \]
 Die objektsprachliche Implikation bzw. Äquivalenz wird mit einem einfachen Pfeil „$\to$“ bzw. „$\leftrightarrow$“ notiert, während die metasprachliche Implikation bzw. Äquivalenz mit einem doppelten Pfeil „$\Rightarrow$“ bzw. „$\Leftrightarrow$“ notiert wird. \\[0.5em]
 Abseits der mathematischen Logik ist diese Unterscheidung aber überflüssig und die meisten Mathematiker verwenden die Implikationspfeile „$\to$“ und „$\Rightarrow$“ synonym. Benutze einfach den, der dir besser gefällt.
\end{bem}

	
	
	\begin{bem}[Klammern setzen]
Mithilfe der Junktoren lassen sich bereits beliebig kompliziert verschachtelte Aussagen bilden wie z.B. $(B_1\lor \neg B_2) \to (B_3\land \neg B_5)$:
\begin{quote}
 „Sofern der Döner in Deutschland erfunden wurde oder heute nicht Mittwoch ist, gibt es außerirdisches Leben und die Relativitätstheorie ist fehlerfrei.“
\end{quote}
oder
$B_1\lor (\neg B_2 \to (B_3\land \neg B_5))$:
\begin{quote}
„Der Döner wurde in Deutschland erfunden oder aber es gilt: wenn heute nicht Mittwoch ist, gibt es außerirdisches Leben und die Relativitätstheorie ist fehlerfrei.“
\end{quote}
Bei verschachtelten Aussagen sollte man Klammern verwenden, um deutlich zu machen, welche Junktoren „weiter innen liegen“ und welche „als letztes angewendet“ werden. Möchte man Klammern vermeiden, kann man dies alternativ auch durch verschieden große Leerstellen zwischen den Zeichen deutlich machen oder ein Hybrid aus beidem verwenden:
\begin{align*}
 B_1\lor \neg B_2\quad &\to \quad B_3\land \neg B_5 \\[0.5em]
 B_1\quad  &\lor \quad \neg B_2 \to (B_3\land \neg B_5)
\end{align*}
Es gibt auch Konventionen, die die „Erstausführung“ gewisser Junktoren vor anderen Junktoren regeln, ähnlich der Regel „Punkt- vor Strichrechnung“. Die solltest du aber nur dann stillschweigend verwenden, wenn du dir sicher bist, dass dein Leser dieselbe Konvention auch kennt und benutzt.
\end{bem}


	
\begin{bem}[* weitere Junktoren]
Es gibt noch weitere Junktoren wie etwa:
 \begin{itemize}
  \item Der Sheffer-Strich „$A\mid B$“ (lies: „Nicht sowohl $A$ als auch $B$“). In der Informatik spricht man auch von der NAND-Verknüpfung.
  \item Die Peirce-Funktion „$A\downarrow B$“ (lies: „Weder $A$ noch $B$“). In der Informatik spricht man auch von der NOR-Verknüpfung.
 \end{itemize}
% Eine Liste weiterer Junktoren und Schreibvarianten findet sich etwa \href{https://de.wikipedia.org/wiki/Junktor#M%C3%B6gliche_Junktoren}{auf der deutschen Wikipedia}. %Beachte allerdings, dass die Junktoren-Tabelle der Wikipedia eher die Gesamtheit der Verknüpfungen auf der Menge $\{\text{wahr},\ \text{falsch}\}$ der gewöhnlichen Wahrheitswerte bezeichnet. Junktoren im aussagenlogischen Sinn müssen nicht an Wahrheitswerten ausgerichtet sein, sondern vielmehr müssen die Wahrheitswerte so beschaffen sein, dass sie die Junktoren interpretieren können.
\end{bem}

	
\section{Ein bisschen Prädikatenlogik}
%Während in der Aussagenlogik lediglich Aussagen mithilfe von Junktoren in einfachere Aussagen zerlegt werden, beschäftigt sich die Prädikatenlogik mit einer weiteren Zerlegung von Aussagen in Objekt und Prädikat.









\subsection{Prädikate}
\begin{de}[Prädikat]
 Es sei $n$ eine natürliche Zahl. Ein \textbf{$n$-stelliges Prädikat}\footnote{Beachte, dass das Wort „Prädikat“ in der Logik eine andere Bedeutung als in der Grammatik, wo es das Verb in einem Satz bezeichnet, trägt. Es handelt sich also um ein Homonym, d.h. ein Wort, das mehrere Bedeutungen, die nichts miteinander zu tun haben, hat.} ist ein sprachliches Gebilde, in dem $n$-viele Variablen vorkommen und das zu einer Aussage wird, wenn für jede dieser Variablen ein konkretes Objekt eingesetzt wird. \\
 $1$-stellige Prädikate nennt man auch \textbf{Eigenschaften}. Sprechen Mathematiker nur von „Prädikaten“, so meinen sie damit in der Regel einstellige Prädikate. \\
 Ist $n\geq 2$, so spricht man auch von \textbf{$n$-stelligen Relationen}. Sprechen Mathematiker einfach nur von „Relationen“, so meinen sie damit in der Regel zweistellige Relationen. Drei- oder höherstellige Relationen tauchen in der Mathematik selten auf.
\end{de}


\begin{bem}
In der Notation werden die Variablen eines Prädikats manchmal mitgeschrieben: ist $E$ ein $n$-stelliges Prädikat, das die Variablen $x_1,\dots , x_n$ enthält, so schreibt man auch „$E(x_1,\dots , x_n)$“ um zu betonen, dass die Variablen von $E$ genau $x_1,\dots , x_n$ sind. Dies ist ähnlich zur Funktionen-Notation in der Schule: ist $f$ eine Funktion in der Variablen $x$, so notiert man diese Funktion in der Schule auch als „$f(x)$“.% (In der Uni-Mathematik ist diese Schreibweise eher rar, da dort „$f(x)$“ meistens nicht die Funktion $f$, sondern den Funktionswert von $f$ an der Stelle $x$ bezeichnet)
\end{bem}


\begin{bsp}
Beispiele für einstellige Prädikate, also für Eigenschaften, sind etwa:
 \begin{enumerate}[a)]
  \item $E(m):\Leftrightarrow$ „$m$ ist eine gerade Zahl“, wobei $m$ eine Variable vom Typ „natürliche Zahl“ sei. Setzt man hier für die Variable $m$ beispielsweise die konkreten Zahlen $4$ und $5$ ein, erhält man die Aussagen „$4$ ist eine gerade Zahl“ bzw. „$5$ ist eine gerade Zahl“.
  \item $D(X):\Leftrightarrow$ „$X$ wurde in Deutschland erfunden“, wobei sich die Variable $X$ auf kulinarische Errungenschaften beziehen soll. Setzt man hier für die Variable $X$ das Objekt „Der Döner“ ein, erhält man gerade die Aussage „Der Döner wurde in Deutschland erfunden“. Setzt man dagegen das Objekt „Die Pizza“ ein, erhielte man die Aussage „Die Pizza wurde in Deutschland erfunden“.
  \item $M(x):\Leftrightarrow $ „$x$ ist der größte Mathematiker“, wobei die Variable $x$ vom Typ „MathematikerIn“ sei. Setzt man hier für die Variable $x$ z.B. das Objekt „Alexander Grothendieck“ ein, erhält man die Aussage „Alexander Grothendieck ist der größte Mathematiker“. Dagegen ergäbe es keinen Sinn, für $x$ das Objekt „Der Döner“ einzusetzen.
 \end{enumerate}
 Wir haben hier, um eine Eigenschaft mit einem Buchstaben zu bezeichnen, nicht das Symbol „$:=$“ sondern das Symbol „$:\Leftrightarrow$“ (lies: „ist per Definition äquivalent zu“ oder „soll per Definition bedeuten, dass“) verwendet. Bei der Definition von Aussagen und Prädikaten kommt das schonmal vor, man könnte aber genausogut auch immer „$:=$“ verwenden. Ist Geschmackssache.
\end{bsp}


%\begin{de}[Diskursuniversum]
%Oftmals arbeitet man mit mehreren Variablen, die sich allesamt auf dieselbe Sorte von Objekten beziehen. In diesem Fall kann man die Gesamtheit all dieser Objekte als \textbf{Diskursuniversum} festlegen.
%\end{de}



\begin{bsp}
 Zweistellige Prädikate sind zum Beispiel:
 \begin{enumerate}[a)]
  \item „$x$ ist kleiner als $y$“, wobei $x,y$ zwei Variablen vom Typ „reelle Zahl“ seien. Diese Relation lässt sich auch kompakt als Formel „$x<y$“ notieren.
  \item $A(X,Y)=$ „$X$ ist älter als $Y$“, wobei für die Variablen konkrete Menschen eingesetzt werden sollen.
  \item $L(X,Y)=$ „$X$ liebt $Y$“, wobei die Variablen vom Typ „Figur aus Jane Austens `Stolz und Vorurteil'“ seien.
  \item „$n$ lässt bei der Division durch $3$ denselben Rest wie $m$ übrig“, wobei für $n$ und $m$ jeweils ganze Zahlen eingesetzt werden sollen. Mathematiker schreiben auch kurz
  \[ n\equiv m\mod 3 \]
  und sagen „$n$ ist kongruent zu $m$ modulo $3$“.
 \end{enumerate}
\end{bsp}





\subsection{Quantoren}
Mengen werden das zentrale Thema im dritten Vortrag sein. Daher wird an dieser Stelle nur das Allernötigste eingeführt, um den Umgang mit Quantoren bequem zu machen.%\footnote{Ein ausführliches Lehrbuch der Mengenlehre mit vielen historischen Anmerkungen ist die „\href{https://www.aleph1.info/?call=Puc&permalink=mengenlehre1}{Einführung in die Mengenlehre}“ von Oliver Deiser.}


\begin{de} \label{mengenimlogikkapitel}
 Eine \textbf{Menge}\footnote{Was in diesem Vortrag „Menge“ genannt wird, würde in der formalen Mengelehre eher als „Klasse“ bezeichnet werden.} ist eine Gesamtheit von Dingen und als solche selbst wiederum ein Gegenstand des Denkens. Sie ist allein dadurch bestimmt, welche Dinge ihr angehören und welche nicht. Diejenigen Objekte, die einer Menge angehören, werden ihre \textbf{Elemente} genannt. Für ein Objekt $a$ und eine Menge $M$ ergibt es nur Sinn, zu fragen ob $a$ ein Element von $M$ ist oder nicht -- dagegen ergäbe es keinen Sinn, danach zu fragen, „auf welche Weise“ oder „wie oft“ $a$ ein Element von $M$ wäre.
\end{de}


\begin{de}[Extension einer Eigenschaft] \label{extension}
 Sei $E$ ein einstelliges Prädikat. Dann wird mit
 \[ \{ x\mid E(x) \} \qquad (\text{lies: „Menge aller $x$, für die gilt: $E(x)$“})\]
 die Menge all derjenigen Objekte, die die Eigenschaft $E$ besitzen, bezeichnet. Sie heißt die \textbf{Extension} (oder auch „Umfang“ oder „Ausdehnung“) des Prädikats $E$. Manche Autoren schreiben anstelle des Querstrichs $\vert$ auch einen Doppelpunkt:
 \[ \{x:\ E(x) \}\]
\end{de}



\begin{bsp}
 Beispielsweise sind
 \begin{itemize}
  \item $\{M\mid M\ \text{ist ein Mensch}\}$ die Menge aller Menschen.
  \item $\Zz:= \{n\mid n\ \text{ist eine ganze Zahl}\}$ die Menge der ganzen Zahlen.
 \end{itemize}
\end{bsp}


\begin{de}[Elementzeichen]
 Sind $M$ eine Menge und $a$ ein Objekt, so schreibt man
 \begin{align*}
  a\in M\qquad&:\Leftrightarrow\qquad a\ \text{ist ein Element von}\ M \\
    a\notin M\qquad&:\Leftrightarrow\qquad a\ \text{ist kein Element von}\ M
 \end{align*}
Insbesondere gilt für jedes Objekt $a$ und jedes Prädikat $E$:
\[ E(a) \qquad\leftrightarrow\qquad a\in \{x\mid E(x)\} \]
\end{de}

\begin{bem}[Eigenschaften vs. Mengen]
Vermöge ihrer Extension bestimmt jede Eigenschaft eine Menge. Umgekehrt bestimmt auch jede Menge eine Eigenschaft, nämlich die Eigenschaft, ein Element von ihr zu sein. Auf diese Weise hat man eine wechselseitige Beziehung zwischen Eigenschaften und Mengen.
\end{bem}




\begin{de}[Diskursuniversum]
 Um Quantoren sinnvoll verwenden zu können, muss vorher ein \textbf{Diskursuniversum} festgelegt werden. Dies ist eine Menge, auf deren Elemente sich die Quantoren beziehen sollen. Jede beliebige Menge kann als Diskursuniversum gewählt werden. Auch die Gesamtheit \emph{aller} Objekte kann als Diskursuniversum gewählt werden.
\end{de}



\begin{de}[Allaussage]
 Es sei $E(x)$ eine Eigenschaft. Dann lässt sich die \textbf{Allaussage}
 \begin{align*}
    \forall x&:\ E(x) && (\text{lies: „Für jedes $x$ gilt $E(x)$“})
 \end{align*}
bilden, deren Bedeutung ist, dass \emph{jedes} Objekt aus dem Diskursuniversum die Eigenschaft $E$ besitzt. Beispiele:
\begin{itemize}
 \item Sind das Diskursuniversum die Menge der Bewohner meiner WG und $A(m):\Leftrightarrow$ „$m$ ist heute früh aufgestanden“, so besagt $\forall m:\ A(m)$, dass jeder in meiner WG heute früh aufgestanden ist.
 \item Sind das Diskursuniversum die Menge der Primzahlen und $U(p):\Leftrightarrow$ „$p$ ist eine ungerade Zahl“, so bezeichnet $\forall p:\ U(p)$ die (falsche) Aussage, dass jede Primzahl eine ungerade Zahl ist.
\end{itemize}
Das Zeichen $\forall$ heißt \textbf{Allquantor}.
\end{de}


\begin{de}[Existenzaussage]
  Es sei $E(x)$ eine Eigenschaft. Dann lässt sich die \textbf{Existenzaussage}
 \begin{align*}
    \exists x&:\ E(x) && (\text{lies: „Es gibt ein $x$, für das $E(x)$ gilt“})
 \end{align*}
bilden, deren Bedeutung ist, dass \emph{mindestens ein} Objekt aus dem Diskursuniversum die Eigenschaft $E$ besitzt. Beispiele:
\begin{itemize}
 \item Sind das Diskursuniversum die Menge der Bewohner meiner WG und $A(m):\Leftrightarrow$ „$m$ ist heute früh aufgestanden“, so besagt $\exists m:\ A(m)$, dass mindestens einer in meiner WG heute früh aufgestanden ist.
 \item Sind das Diskursuniversum die Menge der Primzahlen und $U(p):\Leftrightarrow$ „$p$ ist eine ungerade Zahl“, so bezeichnet $\exists p:\ U(p)$ die (wahre) Aussage, dass es mindestens eine Primzahl gibt, die ungerade ist.
\end{itemize}
Das Zeichen $\forall$ heißt \textbf{Existenzquantor}. \\[0.5em]
Die Negation einer Existenzaussage notiert man mit dem Zeichen $\nexists$. D.h. anstelle von „$\neg (\exists x:\ E(x))$“ schreibt man
\begin{align*}
  \nexists x&:\ E(x) && (\text{lies: „Es existiert kein $x$, für das $E(x)$ gilt“})
 \end{align*}
 Im Beispiel von gerade eben hieße „$\nexists m:\ A(m)$“, dass in meiner WG heute niemand früh aufgestanden ist. \\[0.5em]
 Für den Allquantor gibt es keine dazu analoge Notation. So etwas wie „$\not\kern-0.15em \forall$“ ist meiner Erfahrung nach nicht gebräuchlich.
\end{de}





\begin{bem}[Diskursuniversen sind prinzipiell obsolet] \label{boundedquant}
Sofern nicht anders angegeben, wird als Diskursuniversum zumeist die Gesamtheit \emph{aller} mathematischen Objekte gewählt. Der Bereich von Dingen, der in der momentanen Situation gerade relevant ist, wird dann direkt in die Aussagen integriert. Beispielsweise kann die Aussage „Es gibt eine natürliche Zahl, die kleiner als $3$ ist“ auch durch
\[ \exists x:\quad (x\ \text{ist eine natürliche Zahl}) \land (x<3) \]
formalisiert werden und die Aussage „Jede natürliche Zahl ist kleiner als $3$“ kann durch
\[ \forall x:\quad (x\ \text{ist eine natürliche Zahl}) \to (x<3) \]
formalisiert werden. Hierbei wird dann implizit vorausgesetzt, dass sich die Quantoren auf die Gesamtheit \emph{aller} Objekte beziehen. \\
Sind $M$ irgendeine Menge und $E$ eine Eigenschaft, so schreibt man abkürzend
\begin{align*}
\forall x\in M:\ E(x)\quad&:\Leftrightarrow\quad \forall x:\ (x\in M)\to E(x) && (\text{lies: „Für jedes $x$ aus $M$ gilt $E(x)$“})\\
\exists x\in M:\ E(x)\quad&:\Leftrightarrow\quad \exists x:\ (x\in M)\land E(x) &&  (\text{lies: „Es gibt ein $x$ aus $M$, für das $E(x)$ gilt“})
\end{align*}
Auf diese Weise lässt sich kompakt notieren, dass alle Elemente von $M$ die Eigenschaft $E$ besitzen bzw. dass es mindestens ein Element von $M$ gibt, das die Eigenschaft $E$ besitzt.
\end{bem}


\begin{bsp}[Syllogistik]
Die Prädikatenlogik ist in der Lage, die Satzformen der mittelalterlichen Syllogistik zu formalisieren. Für ein Beispiel seien
 \begin{align*}
  M & := \{x\mid x\ \text{ist ein Mensch} \} \\
  G & := \{g\mid g\ \text{ist ein Gott} \} \\
  H(x) & :\Leftrightarrow\ \text{„$x$ ist ein Grieche“} \\
  S(x) & :\Leftrightarrow\ \text{„$x$ ist sterblich“}
 \end{align*}
Dann ist
\begin{align*}
 \forall x\in M& :\ S(x) && \text{„Alle Menschen sind sterblich“} \\
 \exists x\in M& :\ H(x)&& \text{„Einige Menschen sind Griechen“} \\
 \exists x\in M& :\ \neg H(x) && \text{„Einige Menschen sind keine Griechen“} \\
 \nexists x\in G& :\ S(x)&& \text{„Keine Götter sind sterblich“}
\end{align*}
\end{bsp}



\begin{bem}[* freie Variablen vs. gebundene Variablen]
 Im Ausdruck „$x$ ist eine negative Zahl“ kann für die Variable $x$ eine beliebige reelle Zahl eingesetzt werden. Ebenso z.B. in der Gleichung „$x(x+1)=2$“. Um die Freiheit in der Belegung einer Variable zu betonen, spricht man auch von einer \textbf{freien Variable}. Dagegen ist der Buchstabe „$n$“ im Ausdruck „Für jede gerade natürliche Zahl $n$ ist auch $n^2$ eine gerade Zahl“ oder der Buchstabe „$x$“ in „$\forall x:\ x(x+1)=2$“ keine Variable mehr, da es keinen Sinn ergäbe, für sie ein konkretes Objekt einzusetzen, wie etwa
 \begin{align*}
  \forall 5:\ 5(5+1) = 2 && (\text{dieser Ausdruck ergibt keinen Sinn})
 \end{align*}
 Man sagt, „die Variable wird durch den Quantor gebunden“ und nennt das Zeichen „$x$“ in „$\forall x:\ x(x+1)=2$“ eine \textbf{gebundene Variable}. Es handelt sich nicht mehr um eine Variable im Sinn von \cref{variable}, sondern nur noch um ein „Dummy-Zeichen“, das in der Umgangssprache sogar meistens vermieden werden kann. So würde man den Ausdruck
  \[ \forall x:\ (x\ \text{ist ein Mensch})\to (x\ \text{ist sterblich}) \]
  umgangssprachlich als „Alle Menschen sind sterblich“ lesen und nicht etwa als „Für jedes $x$ gilt: sofern $x$ ein Mensch ist, ist $x$ sterblich“. \\
  Gebundene Variablen kennst du auch schon aus der Schule: Beispielsweise sind die Variablen „$a,b,c,x$“ im Ausdruck „$ax^2 +bx+c$“ jeweils freie Variablen etwa vom Typ „reelle Zahl“, für die jede beliebige reelle Zahl eingesetzt werden kann. Im Ausdruck
  \[ \int_0^1 (ax^2+bx+c)\ dx \]
  ist das Zeichen „$x$“ dagegen eine gebundene Variable, ein „Dummy-Zeichen“ das nur noch deutlich machen soll, über welche Variable integriert wird (man spricht von der „Integrationsvariable“). Es ergäbe keinen Sinn, eine konkrete Zahl einzusetzen wie etwa
  \begin{align*}
  \int_0^1 (a4^2+b4+c)\ d4 && (\text{dieser Ausdruck ergibt keinen Sinn})
   \end{align*}
Dagegen sind hier die Zeichen $a,b,c$ nachwievor freie Variablen, in die Zahlen eingesetzt werden können wie zum Beispiel
\[ \int_0^1 (2x^2+3x+1)\ dx\]
\end{bem}


\begin{bem}
 Bisher wurden die beiden Quantoren „$\forall$“ und „$\exists$“ verwendet, um aus Eigenschaften Aussagen zu erhalten. Allgemein können sie $n$-stellige Prädikate zu $(n-1)$-stelligen Prädikaten machen. Beispielsweise wird das zweistellige Prädikat
 \[ x < y \]
 durch Binden der Variable $x$ zum einstelligen Prädikat
 \[ \forall x:\ x<y \]
Hierbei ist nun $x$ eine gebundene Variable und $y$ eine freie Variable, es liegt also ein einstelliges Prädikat vor. Das kann nun zu einer Aussage gemacht werden, indem man wahlweise für $y$ ein konkretes Objekt einsetzt wie z.B. „$\forall x:\ x < 3$“ oder aber auch $y$ mit einem Quantor bindet wie z.B.
 \[ \exists y:\ \forall x:\ x < y \]
 Ist $n$ eine natürliche Zahl, so kann jedes $n$-stellige Prädikat zu einer Aussage gemacht werden, indem jede der Variablen wahlweise durch ein konkretes Objekt ersetzt oder aber durch einen Quantor gebunden wird.
 \end{bem}
 
 
 \begin{bem}[Schreibkonventionen bei mehreren Quantoren]
 Verwendet man mehrere Quantoren unmittelbar hintereinander, schreibt man den Doppelpunkt oft nur hinter den letzten Quantor:
 \begin{align*}
  \exists y\ \forall x& :\ x < y && (\text{lies: „Es gibt ein $y$, sodass für alle $x$ gilt, dass $x<y$“}) \\
  \forall x\ \exists y& :\ x < y && (\text{lies: „Für jedes $x$ gibt es ein $y$, für das $x<y$ gilt“})  
 \end{align*}
 Manche Autoren lassen die Doppelpunkte hinter Quantoren auch ganz weg.% Kommen mehrere Quantoren derselben Art hintereinander vor, schreibt man oft nur ein Quantorzeichen auf und trennt die gebundenen Variablen durch ein Komma:
%\begin{align*}
% \forall x,y&:\ x<y && (\text{lies: „Für alle $x,y$ gilt $x<y$“}) \\
% \exists x,y&:\ x<y && (\text{lies: „Es gibt $x,y$, für die $x<y$ gilt“}) 
%\end{align*}
 \end{bem}

 
 \begin{bem} \label{quantortausch}
 Beachte, dass es bei Quantoren verschiedener Art auf die Reihenfolge ankommt. Seien etwa das Diskursuniversum die Menge aller Menschen und $M(x,y):\Leftrightarrow$ „$y$ ist Mutter von $x$“. Dann sind
 \begin{itemize}
  \item $\forall x\ \exists y:\ M(x,y)$: „Für jeden Menschen $x$ gilt: es gibt einen Menschen $y$, der Mutter von $x$ ist“.
  \item $\exists y\ \forall y:\ M(x,y)$: „Es gibt einen Menschen $y$, sodass für jeden Menschen $x$ gilt: $y$ ist Mutter von $x$“.
 \end{itemize}
zwei grundlegend verschiedene Aussagen. Die erste Aussage ist wahr, da jeder Mensch eine Mutter hat. Die zweite Aussage aber ist falsch, weil ja nicht alle Menschen Geschwister sind. %Die komplexeste Verschachtelung von Quantoren, die dir im ersten Semester (und möglicherweise sogar im gesamten Studium) begegnen wird, ist das \emph{Stetigkeitskriterium} aus der reellen Analysis: sind das Diskursuniversum der Bereich der reellen Zahlen und $f$ eine Funktion, so heißt $f$ \emph{stetig}, falls gilt:
%\[ \forall p\ \forall \varepsilon:\ ( (\epsilon >0) \to \exists \delta:\ ((\delta >0) \land \forall x:\ (\vert x-p\vert < \delta)\to(\vert f(x)-f(p)\vert < \epsilon)))\]
%Darüber brauchst du dir aber noch nicht den Kopf zerbrechen. In der Mengenlehre und der Analysis wird genug Sprache entwickelt werden, um diese Formel bequemer lesbar zu machen. Außerhalb der mathematischen Logik würde kein Mathematiker das Stetigkeitskriterium so kompliziert aufschreiben, wie es hier steht. Ein erfahrener Mathematiker würde so etwas sagen wie z.B.
%\begin{quote}
% An jedem Punkt $p$ gilt: Für jede Umgebung $V$ von $f(p)$ gibt es eine Umgebung $U$ von $p$, sodass $f(U)\subseteq V$.
%\end{quote}
%Ein nicht unerheblicher Teil der Mathematik besteht ausschließlich darin, mittels Definitionen einen sprachlichen Werkzeugkasten (in diesem Fall die Begriffe „Punkt“, „Umgebung“, „$f(U)$“ und „$\subseteq$“) zu entwickeln, der es erlaubt, komplizierte Aussagen vergleichsweise kompakt und leicht verständlich aufzuschreiben, damit wir leichter darüber nachdenken können.
\end{bem}



\begin{de}[Eindeutigkeitsquantor]
 Das Zeichen $\exists !$ heißt \textbf{Eindeutigkeitsquantor}. Ist $E(x)$ eine Eigenschaft, so bezeichnet
 \begin{align*}
   \exists ! x& :\ E(x) && (\text{lies: „Es gibt genau ein $x$, für das $E(x)$ gilt“})
 \end{align*}
 die Aussage, dass es \emph{genau ein} Objekt im Diskursuniversum gibt, dass die Eigenschaft $E$ besitzt. Ist $M$ eine Menge, so besagt die Formel
 \begin{align*}
  \exists ! x\in M :\ E(x)  &&(\text{lies: „Es gibt genau ein $x$ aus $M$, für das $E(x)$ gilt“})
 \end{align*}
dass es genau ein Element von $M$ gibt, das die Eigenschaft $E$ besitzt (außerhalb von $M$ darf es aber auch andere Elemente geben, die die Eigenschaft $E$ besitzen). Beispiele:
 \begin{itemize}
  \item Ist $B$ die Menge der Bewohner meiner WG und $A(m):\Leftrightarrow$ „$m$ ist heute früh aufgestanden“, so besagt $\exists ! m\in B:\ A(m)$, dass genau ein Bewohner meiner WG heute früh aufgestanden ist.
  \item Die Formel „$\exists ! n\in \Nz :\ 32+n = 101$“ bezeichnet die Aussage: „Es gibt genau eine natürliche Zahl $n$, für die $32+n=101$ ist.“
  \item Die Formel „$\exists ! x\in \Rz:\ x^2=3$“ bezeichnet die (falsche) Aussage „Es gibt genau eine reelle Zahl $x$, für die $x^2=3$ gilt“.
 \end{itemize}
\end{de}




\begin{bem}[Zurückführung von $\exists !$ auf die anderen beiden Quantoren] \label{zerlegung des eindeutigkeitsquantors}
 Mithilfe der Gleichheitrelation „$=$“ kann der Eindeutigkeitsquantor aus den anderen beiden Quantoren zusammengesetzt werden:
 \[ \exists ! x:\ E(x)\qquad :\Leftrightarrow\qquad \exists x:\ E(x) \quad \land\quad \forall x\ \forall y:\ (E(x)\land E(y)) \to x=y \]
Dies besagt, dass es mindestens ein Objekt mit der Eigenschaft $E$ gibt und dass je zwei Objekte, die beide die Eigenschaft $E$ besitzen, bereits übereinstimmen müssen. \\
Für eine Menge $M$ kann man definieren:
 \[ \exists ! x\in M:\ E(x)\quad :\Leftrightarrow\quad (\exists x\in M:\ E(x))\ \land\ (\forall x,y\in M:\ (E(x)\land E(y)) \to x=y) \]
\end{bem}


\begin{bem}[Mäßigung in der Verwendung von Formelsprache!]
Nach den ganzen Formeln aus diesem Abschnitt eine \textbf{Warnung}: Einige Mathe-Anfis gelangen zu der Meinung, in der Mathematik käme es darauf an, Aussagen möglichst formelhaft zu notieren und Quantoren und Junktoren möglichst nie in Umgangssprache, sondern so oft wie möglich als Formelzeichen aufzuschreiben. Manche schreiben auch furchtbare Hybride wie „Daher $\exists$ eine Zahl $n$, die ein Teiler von $a$ $\land$ ein Teiler von $b$ ist“ auf. \\
 Widerstehe dieser Idee! Mathematische Texte und Beweise sind zuallererst mal ein Akt der Kommunikation, in dem der Autor / die Autorin dem Leser eine Information übermitteln möchte. Die Effizienz dieser Informationsübermittlung muss für dich immer an erster Stelle stehen. Lass dich nicht von (unter Mathematikern recht verbreiteten) Formel-Neurosen unterwerfen! Die Einführung der Symbole $\neg,\land,\to,\forall,\exists$ usw. geschieht \textbf{nicht}, damit wir ab sofort alles in diesen Zeichen aufschreiben. Sondern sie dient uns dazu, die Strukturen mathematischer Aussagen und Argumente analysieren und in aller Allgemeinheit besprechen und reflektieren zu können.
\end{bem}



\begin{comment}
\subsection{Gleichheit}






\begin{de}[Gleichheit]
 Sofern zwei Objekte $x,y$ identisch sind, schreibt man $x=y$ (lies: „$x$ ist gleich $y$“). Sind $x$ und $y$ nicht identisch, so schreibt man $x\neq y$ (lies: „$x$ ist ungleich $y$“). \\
 %Gleichheit ist ein zweistelliges Prädikat, das eine Sonderrolle in der Prädikatenlogik spielt, da es keine Freiheit in der Interpretation einer Aussage der Form „$a=b$“ gibt. Die Formel $a=b$ ist genau dann als wahr zu interpretieren, wenn die beiden Objekte $a$ und $b$ übereinstimmen.
\end{de}





\begin{bem}[Anzahlen]
Es seien $n$ eine natürliche Zahl und $E(x)$ eine Eigenschaft. Dann besagt die Formel
\begin{align*}
\exists x_1,\dots, x_n :\ & ((E(x_1)\land\dots \land E(x_n)) \\
& \land (x_1\neq x_2) \land \ldots \land (x_1\neq x_n)\\
& \land (x_2\neq x_3)\land\ldots\land (x_2\neq x_n) \\
&\ldots \\
& \land (x_{n-1}\neq x_n)) 
\end{align*}
dass es mindestens $n$-viele paarweise verschiedene Objekte gibt, die die Eigenschaft $E$ erfüllen. Ebenso besagt die Formel
\begin{align*}
\forall x_1,\dots, x_n,x_{n+1} :\   ((E(x_1)\land\dots \land E(x_{n+1})) \to &( (x_1= x_2) \lor \ldots \lor (x_1= x_{n+1})\\
& \lor (x_2= x_3)\lor\ldots\lor (x_2= x_{n+1}) \\
&  \ldots \\
& \lor (x_{n}= x_{n+1})) 
\end{align*}
dass es höchstens $n$-viele Objekte gibt, die die Eigenschaft $E$ besitzen (um genau zu sein besagt sie, dass von je $n+1$-vielen Objekten, die die Eigenschaft $E$ besitzen, mindestens zwei identisch sein müssen). Diese furchtbar langen Formeln werden in Logik-Lehrbüchern manchmal abgekürzt durch
\begin{align*}
 \exists_{\geq n} x:\ E(x) && (\text{Es gibt mindestens $n$-viele})\\
 \exists_{\leq n} x:\ E(x) && (\text{Es gibt höchstens $n$-viele})
\end{align*}
Schließlich besagt die Formel
\[ \exists_n x:\ E(x) \qquad\:\Leftrightarrow\qquad  (\exists_{\geq n} x:\ E(x))\land ( \exists_{\leq n} x:\ E(x))  \]
dass es \emph{genau $n$-viele} Objekte gibt, die die Eigenschaft $E$ besitzen.
\end{bem}
Der Sonderfall $n=1$ ist so wichtig, dass er ein eigenes Symbol bekommt:
\begin{de}[Eindeutigkeitsquantor]
Das Symbol $\exists !$ heißt \textbf{Eindeutigkeitsquantor}. Ist $E(x)$ irgendeine Eigenschaft, so wird mit
 \[ \exists ! x:\ E(x) \]
 die Aussage „Es gibt \emph{genau ein} $x$, für das $E(x)$ gilt“ notiert. Mittels Allquantor, Existenzquantor und Gleichheit kann diese Aussage folgendermaßen zusammengesetzt werden:
 \[ (\exists x:\ E(x) )\land (\forall x,y:\ (E(x)\land E(y))\to x=y)\]
\end{de}




\end{comment}


\section{Zweiwertige Interpretationen}



\subsection{Wahrheitswerte}


\begin{bem}[Bivalenzprinzip] \label{bivalenz}
In der klassischen Aussagenlogik ist eine „natürliche“ Struktur, die die Menge der Wahrheitswerte trägt, die Struktur einer sogenannten „boolschen Algebra“. Im Vorkurs beschränken wir uns auf diejenige boolsche Algebra, die ausschließlich aus den beiden Wahrheitswerten „wahr“ und „falsch“ besteht. Diese Einschränkung nennt man auch das „Prinzip der Zweiwertigkeit“ oder „Bivalenzprinzip“\footnote{Philosophen sprechen auch vom „Satz vom ausgeschlossenen Dritten“. Dieser besitzt in der mathematischen Logik aber eine andere Bedeutung als das Bivalenzprinzip, siehe \cref{tnd}}.
\end{bem}


\begin{comment}
\begin{bem}[* „Konstante“ Aussagen]
In der Aussagenlogik kann es bequem sein, Aussagezeichen einzuführen, die für eine Aussage stehen, die stets wahr oder stets falsch sein sollen:
\begin{itemize}
 \item Mit „$\top$“ (wie englisch ``true'') ist eine Aussage gemeint, die in einem absoluten Sinn immer wahr sein soll.
 \item Mit „$\bot$“ ist eine Aussage gemeint, die in einem absoluten Sinn falsch sein soll, unabhängig davon, wie sie interpretiert wird.
 \end{itemize}
 \end{bem}
 \end{comment}
 



\begin{de}[Interpretation] \label{interpretation}
 Eine (zweiwertige) \textbf{Interpretation} einer Aussage $X$ ist die Zuweisung eines der beiden Wahrheitswerte „wahr“ oder „falsch“ zu $X$. Diese Zuweisung darf allerdings nicht vollkommen frei erfolgen, sondern muss den folgenden Regeln gehorchen:
 \begin{itemize}
 % \item Die Aussage „$\top$“ muss stets als wahr und die Aussage „$\bot$“ muss stets als falsch interpretiert werden.
  \item  Ist $X$ eine Aussage, die sich mittels der Junktoren $\neg,\land,\lor,\to,\leftrightarrow$ aus anderen Aussagen $A,B$, denen ebenfalls ein Wahrheitswert zugewiesen wurde, zusammensetzt, so muss sich der Wahrheitswert von $X$ nach den folgenden Regeln aus den Wahrheitswerten von $A$ und $B$ ergeben:
		\[\begin{tabular}{c|c||c|c|c|c|c}
			$A$ & $B$ & $\neg A$ & $A\land B$ & $A\lor B$ & $A\to B$ & $A\leftrightarrow B$ \\
			\hline
			w&w&f & w & w & w & w\\
			w&f&f & f & w & f & f\\
			f&w&w & f & w & w & f\\
			f&f&w & f & f & w & w\\
		\end{tabular}\]
Diese sogenannte \textbf{Wahrheitstafel} ist folgendermaßen zu lesen: In den beiden linken Spalten sind alle möglichen Kombinationen aufgelistet, wie $A$ und $B$ mit Wahrheitswerten belegt sein können. Für jede solche Kombination muss dann der Wahrheitswert von $\neg A$, $A\land B$, $A\lor B$ etc. aus der jeweiligen Zeile übernommen werden. Beispielsweise darf die Aussage „$A\lor B$“ nur dann als falsch interpretiert werden, wenn sowohl $A$ als auch $B$ als falsch interpretiert wurden; in den anderen drei Fällen, also falls mindestens eine der beiden Aussagen $A,B$ als wahr verstanden wird, muss auch $A\lor B$ als wahr interpretiert werden.
\item Ist $E$ eine Eigenschaft, so ist die Allaussage $\forall x:\ E(x)$ als wahr zu interpretieren, falls für jedes Objekt $a$ aus dem Diskursuniversum die Aussage $E(a)$ als wahr interpretiert ist. Wurde dagegen für ein Objekt $a$ aus dem Diskursuniversum die Aussage $E(a)$ als falsch interpretiert, so ist auch „$\forall x:\ E(x)$“ als falsch zu interpretieren.
\item Ist $E$ eine Eigenschaft, so ist die Existenzaussage $\exists x:\ E(x)$ als wahr zu interpretieren, falls es mindestens ein Objekt $a$ aus dem Diskursuniversum gibt, bei dem die Aussage $E(a)$ als wahr interpretiert ist. Wurde dagegen für jedes Objekt $a$ aus dem Diskursuniversum die Aussage $E(a)$ als falsch interpretiert, so ist auch „$\exists x:\ E(x)$“ als falsch zu interpretieren.
 \end{itemize}
\end{de}



\begin{bsp}
 Seien $A,B,C$ drei Aussagen. Um den Wahrheitswert von
 \[ D:= \quad (A\lor \neg B) \to C\quad \land\quad \neg C\]
 für alle möglichen Interpretationen von $A,B,C$ zu ermitteln, stellt man eine Wahrheitstafel auf, die auf der linken Seite mit allen möglichen Wahrheitswerte-Kombinationen für $A,B,C$ startet und in den rechten Spalten in wachsender Komplexität mit den relevanten Verschachtelungen von $A$, $B$ und $C$ fortfährt, bis in der Spalte ganz rechts die gesuchten Wahrheitswerte stehen:
 \[ \begin{tabular}{c|c|c||c|c|c|c|c}
    $A$ & $B$ & $C$ & $\neg B$ & $A\lor \neg B$ & $(A\lor \neg B)\to C$ & $\neg C$ & $((A\lor \neg B) \to C)\land \neg C$ \\
    \hline
    w & w & w & f & w & w & f & f\\
    w & w & f & f & w & f & w & f\\
    w & f & w & w & w & w & f & f\\
    w & f & f & w & w & f & w & f\\
    f & w & w & f & f & w & f & f\\
    f & w & f & f & f & w & w & w\\
    f & f & w & w & w & w & f & f\\
    f & f & f & w & w & f & w & f
    \end{tabular} \]
Also gibt es nur einen Fall, in dem $D$ eine wahre Aussage ist; nämlich wenn $B$ wahr ist und $A,C$ falsch sind. \\
Schon an diesem Beispiel wird vielleicht deutlich, dass das Aufstellen von Wahrheitstafeln eine einfache, aber auch mechanische Tätigkeit ist, die anfällig für Flüchtigkeitsfehler ist und sehr gut einem Computer überlassen werden kann.
\end{bsp}



\begin{bem}
 Die Wahrheitstafel des Implikationspfeils
 		\[\begin{tabular}{c|c||c}
			$A$ & $B$ &  $A\to B$  \\
			\hline
			w&w&w\\
			w&f&f\\
			f&w&w\\
			f&f&w\\
		\end{tabular}\]
 verwirrt manche Anfänger, da sie nicht das umgangssprachliche Verständnis von „$B$ folgt aus $A$“ wiedergibt. Setzt man beispielsweise
 \begin{itemize}
   \item $A:=$ „Der Döner wurde in Deutschland erfunden“
  \item $B:=$ „$529$ ist eine Quadratzahl.“
 \end{itemize}
so ist $B$ eine wahre Aussage. Egal, ob $A$ nun wahr oder falsch ist, ergibt sich aus der Wahrheitstafel des Implikationspfeils, dass „Sofern der Döner in Deutschland erfunden wurde, ist $529$ eine Quadratzahl“ eine wahre Aussage ist, obwohl $B$ mit $A$ ja gar nichts zu tun hat. \\
Der Implikationspfeil braucht in der Mathematik \textbf{nichts} mit einem kausalen Zusammenhang zu tun zu haben. „$A\to B$“ besagt eher soviel wie „Wenn ich $A$ annehme, kann ich $B$ beweisen“. Im nächsten Vortrag wird ausführlich darauf eingegangen, siehe \cref{direkterbeweis} und \cref{modusponens}. %Und insofern ist „Sofern der Döner in Deutschland erfunden wurde, ist $529$ eine Quadratzahl“ auch korrekt, da ich mit Annahme von $A$ beweisen kann, dass $529$ eine Quadratzahl ist. Beweis: $529=23\cdot 23$. In diesem Beweis musste ich sogar gar nicht auf die Annahme $A$ zurückgreifen.
\end{bem}






\subsection{Tautologien}


\begin{de}
 Eine Aussage heißt
 \begin{itemize}
  \item \textbf{Tautologie} oder auch \textbf{allgemeingültig}, falls sie unter jeder möglichen Interpretation eine wahre Aussage ist.
  \item \textbf{erfüllbar}, falls es mindestens eine Interpretation gibt, unter der sie eine wahre Aussage ist.
  \item \textbf{unerfüllbar}, falls sie unter keiner möglichen Interpretation eine wahre Aussage ist.
 \end{itemize}
 \end{de}


\begin{bsp}
 Es gilt:
 \begin{enumerate}[a)]
  \item Die Aussage „Heute ist Mittwoch“ ist erfüllbar, aber keine Tautologie.
  \item Die Aussage „Genau dann ist heute Mittwoch, wenn heute Mittwoch ist“ ist eine Tautologie.
\item Für beliebige Aussagen $A,B$ sind die Aussagen
\[ A\to (A\lor B) \qquad A\to \neg\neg A \qquad \neg(A\land \neg A) \]
jeweils Tautologien, was mithilfe von Wahrheitstafeln überprüft werden kann. Hier ist eine Wahrheitstafel für die Formel $\neg(A\land\neg A)$:
 		\[\begin{tabular}{c||c|c|c}
			$A$ &  $\neg A$ & $A\land \neg A$ & $\neg(A\land \neg A)$ \\
			\hline
			w&f&f&w\\
			f&w&f&w
		\end{tabular}\]
\item Für beliebige Aussagen $A,B$ sind die Aussagen
\[  \neg(A\to (A\lor B)) \qquad A \leftrightarrow \neg A \qquad A\land \neg A\]
unerfüllbar. Hier ist eine Wahrheitstafel für $A\leftrightarrow \neg A$:
 		\[\begin{tabular}{c||c|c}
			$A$ &  $\neg A$ & $A \leftrightarrow \neg A$ \\
			\hline
			w&f&f\\
			f&w&f
		\end{tabular}\]
 \end{enumerate}
 \end{bsp}
 
 
 
 
 \begin{sat}[Nützliche Tatsachen für Tautologien] \label{tauto}
Seien $A,B$ zwei beliebige Aussagen. Dann gilt:
  \begin{enumerate}[a)]
   \item Genau dann ist $A$ unerfüllbar, wenn $\neg A$ eine Tautologie ist.
   \item Genau dann ist $A\leftrightarrow B$ eine Tautologie, wenn die Interpretationen, unter denen $A$ wahr ist, genau dieselben sind, unter denen $B$ wahr ist.
   \item Genau dann ist $A\to B$ eine Tautologie, wenn unter jeder Interpretation, unter der $A$ eine wahre Aussage ist, auch $B$ eine wahre Aussage ist. Diejenigen Interpretationen, unter denen $A$ falsch ist, spielen hierbei keine Rolle.
  \end{enumerate}
 \end{sat}
\begin{bew}
a) Betrachte die Wahrheitstafel der Negation:
\[\begin{tabular}{c||c}
    $A$ &  $\neg A$ \\
			\hline
			w&f\\
			f&w
\end{tabular}\]
Unter einer festen Interpretation ist $\neg A$ genau dann wahr, wenn $A$ falsch ist. Dass $\neg A$ unter allen Interpretationen wahr ist, heißt dann genau, dass $A$ unter allen Interpretationen falsch ist. \\[0.5em]
b) Aus der Wahrheitstafel der Äquivalenz
 		\[\begin{tabular}{c|c||c}
			$A$ &  $B$ & $A \leftrightarrow A$ \\
			\hline
			w&w&w\\
			w&f&f \\
			f & w & f\\
			f & f & w
		\end{tabular}\]
liest man ab, dass $A\leftrightarrow B$ genau dann wahr ist, wenn $A$ und $B$ denselben Wahrheitswert haben. Also ist $A\leftrightarrow B$ genau dann eine Tautologie, wenn $A$ und $B$ unter jeder Interpretation denselben Wahrheitswert haben, was, da wir gemäß dem Bivalenzprinzip nur mit zwei verschiedenen Wahrheitswerten „w“ und „f“ arbeiten, gleichwertig dazu ist, dass $A$ und $B$ unter genau denselben Interpretationen wahr sind. \\[0.5em]
c) Betrachte die Wahrheitstafel der Implikation:
 		\[\begin{tabular}{c|c||c}
			$A$ &  $B$ & $A \to A$ \\
			\hline
			w&w&w\\
			w&f&f \\
			f & w & w\\
			f & f & w
		\end{tabular}\]
Dass $A\to B$ eine Tautologie ist, heißt, dass $A\to B$ unter keiner möglichen Interpretation falsch sein kann. Dies ist äquivalent dazu, dass der Fall, dass $A$ wahr und $B$ falsch ist, niemals auftreten kann. Und das heißt gerade, dass, wann immer $A$ wahr ist, auch $B$ wahr sein muss. \qed
\end{bew}

 
 

\begin{bem}[*]
 Eine große Liste aussagenlogischer Tautologien findest du im Anhang dieses Skripts. Wenn du Lust hast, versuche mal, dir intuitiv für ein paar der Formeln klarzumachen, dass es sich um Tautologien handeln muss. So kannst du ein besseres Verständnis für die Junktoren erwerben.
\end{bem}

 
 

\begin{comment}
\begin{bem}
Besonders erwähnenswerte Tautologien aus dieser Liste sind die folgenden:
\begin{align*}
     A & \leftrightarrow \neg\neg A && (\text{Regel der doppelten Verneinung}) \\[1em]
            \begin{split} ( \neg A \land \neg B) & \leftrightarrow \neg (A \lor B) \\
  (\neg A \lor \neg B) & \leftrightarrow \neg(A \land B) 
  \end{split} && (\text{Regeln von De Morgan}) \\[1em]
    \neg (A\to B) & \leftrightarrow (A \land \neg B) \\
     (A\to B) & \leftrightarrow (\neg A \lor B)
\end{align*}
Die ersten vier Formeln erlauben es, Negationen „in Formeln hineinzuziehen“, ohne ihren Wahrheitswert zu verändern. Au diese Weise kann jede mit Junktoren verschachtelte Aussage bei gleichbleibendem Wahrheitswert so „umgeformt“ werden. Die letzte Formel ermöglicht es Implikationspfeile in Oder-Aussagen „umzuformen“. Hier ist ein Beispiel für eine Umformung der Kontrapositionsformel in eine Formel, die nur noch die Junktoren $\neg$, $\land$ und $\lor$ enthält:
\begin{align*}
 (A\to B)\to (\neg B\to \neg A) & \Leftrightarrow  (\neg A\lor B) \to (\neg\neg B\lor\neg A) \\
 & \Leftrightarrow \neg(\neg A\lor B) \lor (B\lor \neg A) \\
 & \Leftrightarrow (\neg \neg A\land \neg B) \lor (B\lor \neg A) \\
 & \Leftrightarrow (A\land \neg V) \lor B\lor \neg A
\end{align*}
\end{bem}




\begin{bem}[Einige Tautologien der Gleichheit]
Es gilt:
 \begin{itemize}
  \item Für jedes beliebige Objekt $x$ gilt $x=x$. Mit anderen Worten: die Aussage $\forall x:\ x=x$ ist eine Tautologie.
  \item Sei $E$ irgendeine Eigenschaft. Sind $a,b$ zwei Objekte und gilt $a=b$, so gilt genau dann $E(a)$, wenn $E(b)$ gilt. Mit anderen Worten: gleiche Objekte besitzen dieselben Eigenschaften bzw. für jede beliebige Eigenschaft $E$ ist die Formel
  \[ \forall a,b:\ (a=b) \to (E(a)\leftrightarrow E(b)) \]
  eine Tautologie.
  \item Die Idee, diese zweite Tatsache zu einer \emph{Definition} der Gleichheit zu erheben; also zu sagen: Zwei Objekte $a,b$ sind \emph{per Definition} gleich, wenn sie genau dieselben Eigenschaften erfüllen, heißt auf latein „Principium identitatis indiscernibilium“, also soviel wie „Identität des Ununterscheidbaren“.
 \end{itemize}
\end{bem}
\end{comment}

 \begin{bem}[* Entscheidbarkeit der Aussagenlogik] \label{entscheidbar}
  Ist $A$ eine noch so kompliziert verschachtelte Aussage, die keine Prädikate und Quantoren enthält, sondern sich ausschließlich mittels der Junktoren aus anderen Aussagen zusammensetzt, so lässt sich mithilfe von Wahrheitstafeln stets überprüfen ob $A$ eine Tautologie ist. Mit genügend Rechenkapazität kann mir mein Computer also einfach ausrechnen, ob eine Tautologie vorliegt oder nicht. Man sagt, die Aussagenlogik sei (algorithmisch) \textbf{entscheidbar}. Sobald Quantoren ins Spiel kommen, reichen Wahrheitstafeln aber nicht mehr aus: in der mathematischen Logik wird bewiesen, dass es keinen Algorithmus gibt, der für eine beliebige, sich mittels Junktoren und Quantoren aus Prädikaten und Aussagen zusammensetzende Aussage entscheiden kann, ob es sich um eine Tautologie handelt oder nicht. Man spricht von der \textbf{Unentscheidbarkeit der Prädikatenlogik}. Das wäre auch zu schön, denn ein solcher Algorithmus wäre ein „mathematisches Orakel“, das für jede mathematische Aussage ausrechnen könnte, ob sie stets wahr ist oder nicht. Gäbe es so etwas, wären die meisten Mathematiker auf einen Schlag arbeitslos. \\
  Für ein Beispiel seien das Diskursuniversum die natürlichen Zahlen und $E(n)$ das Prädikat „Wenn $n$ eine gerade Zahl und größer als drei ist, dann lässt sich $n$ als Summe zweier Primzahlen schreiben“. Die Aussage $\forall n:\ E(n)$ heißt \href{https://de.wikipedia.org/wiki/Goldbachsche_Vermutung}{\emph{Goldbachsche Vermutung}} und konnte bislang weder bewiesen noch widerlegt werden. Sie lässt sich als eine Art „unendliche Konjunktion“
  \[E(1)\land  E(2)\land E(3)\land E(4)\land E(5) \land \dots \]
  auffassen. Solche „unendlichen Konjunktionen“ entziehen sich aber aussagenlogischen Methoden und sind nicht mehr mit Wahrheitstafeln beherrschbar. Würde man die Konjunktionenkette ab einer bestimmten Zahl abbrechen
  \[ E(1)\land E(2)\land\dots \land E(10^{30}) \]
  so befände sich alles im Rahmen der Aussagenlogik und man müsste nur nacheinander prüfen, ob jede der Zahlen $1,\dots , 10^{30}$ die Eigenschaft $E$ besitzt (was mithilfe von Hochleistungsrechnern tatsächlich verifiziert werden konnte). Aber das reicht ja nicht aus, um die Aussage für \emph{alle} natürlichen Zahlen zu verifizieren. Dafür bräuchte der Rechner unendlich viel Zeit. \\
  Da es kein Patenzrezept gibt, für eine allgemeine prädikatenlogisch aufgebaute Aussage zu entscheiden, ob sie eine Tautologie ist oder nicht, müssen andere Techniken verwendet werden, um zumindest für einige Aussagen beurteilen zu können, ob sie als „wahre“ mathematische Aussagen gehandelt werden können. Dafür sind die Beweismethoden, die im nächsten Vortrag behandelt werden, da.
 \end{bem}










% --------------------------------------------------------------------
% §5 Section <<Übungsaufgaben>>
% --------------------------------------------------------------------
\newpage
\section{Aufgabenvorschläge}

% --------------------------------------------------------------------
% §5.1 Subsection <<Aufgabe 1>>
% --------------------------------------------------------------------



\begin{aufg}[Umgangssprache in Formeln übersetzen]
Zerlegt die folgenden Aussagen mithilfe der im Vortrag behandelten Junktoren und Quantoren in möglichst einfache Grundbausteine.
\begin{enumerate}[a)]
\item Wenn ich entweder alle Prüfungen im ersten Versuch bestehe oder aber durch alle Prüfungen im ersten Versuch durchfalle, werde ich die ganze Nacht hindurch feiern.
 \item Sofern er morgen Abend weder arbeiten muss noch Besuch von seiner Familie kriegt, würde er sich mit mir treffen.
  \item Alle reellen Lösungen der Ungleichung $x^3-3x<3$ sind kleiner als $10$.
 \item Nobody’s perfect.
 \item Wenn es irgendjemand schafft, dann Henrik.
 \item Wenn Henrik es schafft, dann schafft es jeder.
 \item Eine natürliche Zahl $p\neq 1$ ist genau dann eine Primzahl, falls gilt: für alle natürlichen Zahlen $a,b$ mit $p=ab$ ist $a=1$ oder $b=1$.
\end{enumerate}
\end{aufg}





\begin{aufg}[Formeln in Umgangssprache übersetzen]
Übersetzt die folgenden Aussagenformeln in Umgangssprache und beurteilt, ob es sich um wahre oder falsche Aussagen handelt:
	\begin{enumerate}[a)]
	\item $\exists x\in \Rz\ \exists y\in \Zz:\ x<y$
	\item $\forall x\in \Rz\ \forall y\in \Zz:\ x<y$
		\item $\forall x\in \Rz\ \exists y\in \Zz:\ x<y$
		\item $\exists y\in \Zz\ \forall x\in \Rz:\ x<y$
		\item $\forall x\in \Rz:\ (x^2 -x=0\leftrightarrow(x=1\vee x=0))$
		\item $\exists! x\in \Rz\ \exists y\in \Rz:\ (y\neq0\ \land\ x\cdot y=0)$
	\end{enumerate}
\end{aufg}


	
	
\begin{aufg}[Wahrheitstafeln]
Seien $A,B,C$ drei beliebige Aussagen. Entscheidet mithilfe von Wahrheitstafeln, unter welchen Umständen die folgenden Aussagen wahr sind:
\begin{align*}
%\text{b)} && \neg A & \to A\\
\text{a)} && A\to B\quad & \leftrightarrow \quad A\land \neg B \\
\text{b)} && A\to (B\to C) \quad&\leftrightarrow\quad (A\land B)\to C \\
\text{c)} && A\to B \quad&\leftrightarrow\quad \neg A \to \neg B
%\text{d)} && (A\to B)&\lor (B\to C)\\
\end{align*}
\end{aufg}





	
\begin{comment}
\subsection{Aufgabe}
Seien $A,B,C$ drei beliebige Aussagen. Zeigt mithilfe einer Wahrheitstafel, dass die Formel
\[ (A\to B)\lor (B\to C) \]
eine Tautologie ist. Seien nun konkret
\begin{itemize}
 \item $A:=$ „Der Döner wurde in Deutschland erfunden“.
 \item $B:=$ „Heute ist Mittwoch“.
\end{itemize}
Wie muss man „$\to$“ und „$\lor$“ auffassen, damit die Tatsache, dass $(A\to B)\lor (B\to A)$ eine Tautologie ist, kein Nonsens ist? \\%Ergibt so auch die Tatsache, dass $(B\to \neg B)\lor (\neg B\to B)$ eine Tautologie ist, Sinn? \\
Eine Liste weiterer Tautologien des Implikationspfeils, die kontraintuitiv wirken können, findet ihr \href{https://de.wikipedia.org/wiki/Paradoxien_der_materialen_Implikation}{auf der deutschen Wikipedia}. Der nächste Vortrag wird euch dabei helfen, sie besser zu verstehen.
\end{comment}


\begin{aufg}[Seltsame Formeln]
An der Tafel von Captain Chaos stehen die folgenden Ausdrücke:
\begin{align*}
(i)\quad& A\ {\neg}{\to}\ \neg A&(ii)\quad& \exists x\in x:\ x(x)  \\
 (iii)\quad&   \forall n\in \Nz:\ 1<3 & (iv)\quad& \forall E :\ E(x) \\
  (v)\quad& \forall x\in \Rz :\ x &(vi)\quad& \forall x_1\ \exists x_2\ \forall x_3\ \exists x_4\ \forall x_5\ \ldots:\ E(x_1,x_2,x_3,\dots)
\end{align*}
Was haltet ihr davon?
\end{aufg}
	
	
	

\begin{comment}
\subsection{Aufgabe}

Seine A,B,C Aussagen. Beweise mittels Wahrheitstafeln und bekannter Sätze:
\begin{enumerate}
	\item DeMorgan'sche Regeln:
	\begin{enumerate}
		\item $\neg(A\wedge B)\Leftrightarrow(\neg A\vee \neg B)$
		\item $\neg(A\vee B)\Leftrightarrow(\neg A\wedge \neg B)$
	\end{enumerate}
	\item Distributivgesetze:
	\begin{enumerate}
		\item $(A\wedge(B\vee C))\Leftrightarrow((A\wedge B)\vee(A\wedge C))$
	\end{enumerate}
\end{enumerate}

% --------------------------------------------------------------------
% §5.3 Subsection <<Aufgabe 3>>
% --------------------------------------------------------------------
\subsection{Aufgabe}

\begin{enumerate}
	\item Formalisiere die folgende Aussage: (In einzelne Aussagen unterteilen und daraus ein Aussagengebilde mit Operatoren erstellen) \\
	"Wenn ein hartgekochtes Ei nicht mit kaltem Wasser abgeschreckt wird, dann klebt die Schale am Eiweiß und das Ei lasst sich nicht gut schälen."
	\item Andreas, Benedikt, Carolin und Dora sind auf eine Party eingeladen. Folgendes ist bekannt:
	\begin{enumerate}
		\item Wenn Andreas geht, dann geht auch Benedikt
		\item Carolin und Dora gehen nicht beide
		\item Von Andreas und Dora geht mindestens einer
		\item Wenn Benedikt oder Dora geht, dann geht auch Carolin
	\end{enumerate}
	Wer geht auf die Party?
\end{enumerate}

% --------------------------------------------------------------------
% §5.4 Subsection <<Aufgabe 4>>
% --------------------------------------------------------------------
\subsection{Aufgabe}

\begin{enumerate}
	\item Negiere die folgenden Aussagen A:
	\begin{enumerate}
		\item A = $\forall x\in\mathbb{R}:\exists y\in\mathbb{R}:\: x < y^2 $
		\item A = $\forall x,y\in\mathbb{R}:\forall\epsilon>0: \exists\delta>0: (|x-y|<\delta \Rightarrow |g(x)-g(y)|<\epsilon)$
		\item Es gibt eine Universität, an der es keinen Studenten gibt, der Spaß am Negieren von Aussagen hat.
	\end{enumerate}
	\item Entscheide, ob die folgenden Aussagen wahr oder falsch sind oder ob weitere Informationen benötigt werden:
	\begin{enumerate}
		\item $\forall x\in\mathbb{R}:\exists y\in\mathbb{R}: x<y$
		\item $\exists y\in\mathbb{R}: \forall x\in\mathbb{R}: x<y$
		\item $\forall x \in\mathbb{R}:(x^2 -x=0\Rightarrow(x=1\vee x=0))$
		\item $\exists x\in\mathbb{N}: (x\neq0\wedge(\forall y\in \mathbb{N}: (x\cdot y < x+y)))$
	\end{enumerate}
\end{enumerate}
\end{comment}

% --------------------------------------------------------------------
% §6 Section <<Lösungen>>
% --------------------------------------------------------------------
\begin{comment}
\section{Lösungen}

% --------------------------------------------------------------------
% §6.1 Subsection <<Aufgabe 1>>
% --------------------------------------------------------------------
\subsection{Aufgabe}

\begin{enumerate}
	\item \begin{tabular}{c|c||c}
		A & $\neg(\neg$ A) &$ \neg(\neg$ A) $\Leftrightarrow$ A \\
		\hline
		w & w & w\\
		f & f & w \\
	\end{tabular}
	\item \begin{tabular}{c||c|c||c|c}
		A & A$\wedge$ w & (A$\wedge$w)$\Leftrightarrow$ A & A$\wedge$f & (A$\wedge$ f) $\Leftrightarrow$ f \\
		\hline
		w & w & w & f & w \\
		f & f & w & w & w \\
	\end{tabular}
	\item \begin{tabular}{c||c|c||c|c}
		A & A$\vee$ w & (A$\vee$w)$\Leftrightarrow$ w & A$\vee$f & (A$\vee$ f) $\Leftrightarrow$ A \\
		\hline
		w & w & w & w & w \\
		f & w & w & f & w \\
	\end{tabular}
	\item \begin{tabular}{c|c||c|c|c}
		A & B & A$\wedge$B & B$\wedge$A & (A$\wedge$B)$\Leftrightarrow$(B$\wedge$A) \\
		\hline
		w & w & w & w & w \\
		w & f & f & f & w \\
		f & w & f & f & w \\
		f & f & f & f & w \\
	\end{tabular}
	\item \begin{tabular}{c|c|c||c|c|c|c|c}
		A & B & C & B$\wedge$C & A$\wedge$(B$\wedge$C) & A$\wedge$B & (A$\wedge$B)$\wedge$C & A$\wedge$(B$\wedge$C) $\Leftrightarrow$ (A$\wedge$B)$\wedge$C \\
		\hline
		w & w & w & w & w & w & w & w \\
		w & w & f & f & f & w & f & w \\
		w & f & w & f & f & f & f & w \\
		w & f & f & f & f & f & f & w \\
		f & w & w & w & f & f & f & w \\
		f & w & f & f & f & f & f & w \\
		f & f & w & f & f & f & f & w \\
		f & f & f & f & f & f & f & w \\
	\end{tabular}
\end{enumerate}

% --------------------------------------------------------------------
% §6.2 Subsection <<Aufgabe 2>>
% --------------------------------------------------------------------
\subsection{Aufgabe}

\begin{enumerate}
	\item \begin{enumerate}
		\item \begin{tabular}{c|c||c|c|c}
			A & B & A$\wedge$B & $\neg$A$\vee\neg$B & $\neg$(A$\wedge$B) $\Leftrightarrow$ ($\neg$A$\vee\neg$B) \\
			\hline
			w & w & w & f & w \\
			w & f & f & w & w \\
			f & w & f & w & w \\
			f & f & f & w & w \\
		\end{tabular}
		\item \begin{tabular}{c|c||c|c|c}
			A & B & A$\vee$B & $\neg$A$\wedge\neg$B & $\neg$(A$\vee$B) $\Leftrightarrow$ ($\neg$A$\wedge\neg$B) \\
			\hline
			w & w & w & f & w \\
			w & f & w & f & w \\
			f & w & w & f & w \\
			f & f & f & w & w \\
		\end{tabular}
	\end{enumerate}
	\item \begin{tabular}{c|c|c||c|c|c|c|c|c}
		A & B & C & B$\vee$C & A$\wedge$(B$\vee$C) & A$\wedge$B & A$\wedge$C & (A$\wedge$B)$\vee$(A$\wedge$C) & (A$\wedge$(B$\vee$C))$\Leftrightarrow$((A$\wedge$B)$\vee$(A$\wedge$C)) \\
		\hline
		w & w & w & w & w & w & w & w & w \\
		w & w & f & w & w & w & f & w & w \\
		w & f & w & w & w & f & w & w & w \\
		w & f & f & f & f & f & f & f & w \\
		f & w & w & w & f & f & f & f & w \\
		f & w & f & w & f & f & f & f & w \\
		f & f & w & w & f & f & f & f & w\\
		f & f & f & f & f & f & f & f & w \\
	\end{tabular}
\end{enumerate}

% --------------------------------------------------------------------
% §6.3 Subsection <<Aufgabe 3>>
% --------------------------------------------------------------------
\subsection{Aufgabe}

\begin{enumerate}
	\item Definiere die Aussagen:
	\begin{itemize}
		\item A = “Ein hartgekochtes Ei wird mit kaltem Wasser abgeschreckt.”,
		\item B = “Die Schale klebt am Eiweiß.”,
		\item C = “Das Ei laßt sich gut schälen.”
	\end{itemize}
	Erhalte die Formalisierung: $(\neg A)\Rightarrow (B\wedge (\neg C))$
	\item In Grundaussagen einteilen: A: Andreas geht; B: Benedikt geht; C: Carolin geht; D: Dora geht. \\
	Laut Aufgabe muss gelten:
	\begin{enumerate}
		\item $A\Rightarrow B$ muss wahr sein.
		\item $C\wedge D$ muss falsch sein.
		\item $A\vee D$ muss wahr sein.
		\item $B\vee D\Rightarrow C$ muss wahr sein.
	\end{enumerate}
	\textbf{Fall 1:} Andreas geht, d.h. A wahr $\stackrel{a)}{\Rightarrow}$ B wahr $\stackrel{d)}{\Rightarrow}$ C wahr $\stackrel{b)}{\Rightarrow}$ D falsch \\
	D.h. wenn Andreas geht, gehen insgesamt Andreas, Benedikt und Carolin, nur Dora geht dann nicht. \\
	\textbf{Fall 2:} Andreas geht nicht, führt zu Widerspruch: A falsch $\stackrel{c)}{\Rightarrow}$ D wahr $\stackrel{b)}{\Rightarrow}$ C falsch, aus D wahr ergibt sich aber mit (d) auch C wahr, d.h. Widerspruch, d.h. erste Annahme ergibt einzige Lösung. \\
	Insgesamt also: Andreas, Benedikt und Carolin gehen zur Party. Dora geht nicht.
\end{enumerate}

% --------------------------------------------------------------------
% §6.4 Subsection <<Aufgabe 4>>
% --------------------------------------------------------------------
\subsection{Aufgabe}

\begin{enumerate}
	\item \begin{enumerate}
		\item $\neg$A = $\exists x\in\mathbb{R}:\forall y\in\mathbb{R}: x \geq y^2$
		\item Sei B(x,y) = $|x-y|<\delta$, C(x,y) = $|g(x)-g(y)|<\epsilon$. Verneine nun Schritt für Schritt: 
		\begin{align*}
			\neg(\forall x,y\in\mathbb{R}:\forall\epsilon>0: \exists\delta>0: (A(x,y) &\Rightarrow B(x,y)) \\
			\Leftrightarrow \exists x,y\in\mathbb{R}:\neg(\forall\epsilon>0: \exists\delta>0: (A(x,y) &\Rightarrow B(x,y)) \\
			\Leftrightarrow \exists x,y\in\mathbb{R}:\exists\epsilon>0: \neg(\exists\delta>0: (A(x,y) &\Rightarrow B(x,y)) \\
			\Leftrightarrow \exists x,y\in\mathbb{R}: \exists\epsilon>0: \forall\delta>0: \neg((A(x,y) &\Rightarrow B(x,y)) \\
		\end{align*}
		Nun müssen wir noch $\neg( A(x,y)\Rightarrow B(x,y))$ umformulieren: \\
		$\neg( A(x,y)\Rightarrow B(x,y) \Leftrightarrow A(x,y)\wedge\neg B(x,y)$ (Mit Wahrheitstafel beweisen). \\
		$\neg B(x,y) \Leftrightarrow |g(x)-g(y)|\geq\epsilon$ \\
		Also: \[\neg A= \exists x,y\in\mathbb{R}: \exists\epsilon>0: \forall\delta>0: (|x-y|<\delta \wedge |g(x)-g(y)|\geq \epsilon\]
		\item An allen Universitäten gibt es einen Studenten, der keinen Spaß am Negieren von Aussagen hat.
	\end{enumerate}
	\item \begin{enumerate}
		\item wahr.
		\item falsch, denn a) gilt.
		\item wahr, denn: $x^2 - x = x\cdot(x-1) = 0 \Leftrightarrow x=0 \vee x=1$
		\item wahr, $x=1$ erfüllt die Aussage.
	\end{enumerate}
\end{enumerate}


%%% Local Variables:
%%% mode: latex
%%% TeX-master: "Skript"
%%% End:




\begin{comment}
\section{Einige Aussagenlogische Beweistechniken}






\begin{de}[Satz/Beweis]
		Ein \swdde{Satz} ist eine Aussage, die überprüfbar immer wahr ist. Ein \swdde{Beweis} eines Satzes ist eine logische Herleitung dieser Wahrheits-Aussage aus Axiomen und Sätzen.
	\deEnd
	\end{de}
	
	\noindent Andere Bezeichnungen für einen Satz sind auch (je nach Kontext): Lemma, Korollar, Proposition, etc. \\
	Anstatt \anf{Die Aussage ist immer wahr} verwenden wir auch \anf{Die Aussage gilt.} Als Beispiel wollen wir nun den \anf{Satz vom Widerspruch} beweisen. Dies machen wir mit Hilfe von Wahrheitstafeln, d.h. wir gehen alle logischen Möglichkeiten durch und zeigen in jedem Fall, dass die Aussage des Satzes wahr ist.

%------------------
%
	\begin{sat}[Satz vom Widerspruch]
		Für eine beliebige Aussage $A$ gilt:
		\[
			\Leftrightarrow(\wedge(A,\neg(A)),f)=(A\wedge\neg A)\Leftrightarrow f
		\]
		Alternative Formulierung: Es gilt $\neg(A\wedge\neg A)$.
	\end{sat}

%------------------
%
	\begin{bew}
		Wir leiten spaltenweise die Wahrheitswerte der in der Kopfzeile der Tafel stehenden Aussagen her:
		\[\begin{tabular}{c||c|c|c}
			$A$ & $\neg A$ & $A\wedge\neg A$ & $(A\wedge\neg A)\Leftrightarrow f$ \\
			\hline
			w&f&f&w\\
			f&w&f&w\\
		\end{tabular}\] \hfill $\Box$
	\bewEnd
	\end{bew}
	
	\noindent Aussgen mit Äquivalenzzeichen (\anf{$\Leftrightarrow$}), die immer wahr sind, können dazu genutzt werden, andere Aussagen umzuformen, was einer vereinfachten Kommunikation dient. Kommt beispielsweise in einer anderen Aussage $A\wedge\neg A$ vor, so können wir stattdessen einfach $f$ schreiben. \\
	Zum Beispiel gilt für Aussagen $A,B$:
	\[
		B\vee(A\wedge\neg A)\Leftrightarrow B\vee f
	\]
	
	\noindent Nun können wir auch den \anf{modus ponens} beweisen:

%------------------
%
	\begin{bew}[zu \anf{modus ponens}]
		\[\begin{tabular}{c|c||c|c|c}
			A & B & $A\Rightarrow B$ & $A\wedge(A\Rightarrow B)$ & $(A\wedge(A\Rightarrow B))\Rightarrow B$\\
			\hline
			w&w&w&w&w\\
			w&f&f&f&w\\
			f&w&w&f&w\\
			f&f&w&f&w\\
		\end{tabular}\] \hfill $\Box$
	\bewEnd
	\end{bew}
	
	\noindent Auf ähnliche Weise kann der \anf{Satz vom ausgeschlossenen Dritten} bewiesen werden, dies bleibt den Studierenden allerdings als Übungsaufgabe überlassen.

%------------------
%
	\begin{sat}[Satz vom ausgeschlossenen Dritten]
		Für eine beliebige Aussage $A$ gilt:
		\[
			A\vee\neg A
		\]
		Alternative Formulierung: Es gilt $(A\vee\neg A)\Leftrightarrow w$.
	\end{sat}

\end{comment}

% --------------------------------------------------------------------
% §3 Section <<Prädikatenlogik erster Stufe>>
% --------------------------------------------------------------------
\begin{comment}
\section{Prädikatenlogik erster Stufe}

% --------------------------------------------------------------------
% §3.1 Subsection <<Quantoren>>
% --------------------------------------------------------------------
\subsection{Quantoren}

	Im Gegensatz zur Aussagenlogik, welche die Zerlegung von Aussagen in nicht weiter teilbare
	Aussagen (sog. Elementaraussagen) untersucht, beschäftigt sich die Prädikatenlogik mit der
	Struktur dieser Elementaraussagen. \\
	Das Problem in der oben formulierten Aussagenlogik liegt darin, dass wir nur Sätze über Aussagen formulieren können, die uns vollständig bekannt sind.

%------------------
%
	\begin{bsp}
		Wir betrachten die Aussagen $A$ = \anf{Steffen darf Alkohol kaufen} und $B$ = \anf{Steffen ist älter als 10 Jahre}. \\
		\\
		Nehmen wir an, dass wir zeigen konnten, dass wenn $A$ wahr ist, auch $B$ wahr ist. Dann haben wir aber nur eine Aussage über die spezifische Person Steffen. Vielleicht können wir zeigen, dass die Implikation für die Person Claudia auch gilt. Aber das müssten wir dann für jede einzelne Person erneut überprüfen. Unser Ziel ist also, diese Implikation möglichst abstrakt zu formulieren. Vielleicht könnten wir die Implikation für alle Personen in Deutschland zeigen und sie auf diese Menschenmenge verallgemeinern. Also stellt sich die Frage, was die größte Menge an Menschen ist, auf die sich diese Aussage erweitern ließe. Wir wollen so etwas beweisen können wie
		\begin{center}
			Alle Leute, die Alkohol kaufen dürfen, sind älter als 10 Jahre.
		\end{center}
	\end{bsp}
	
	\noindent Die Situation von Beispiel 3.1 tritt auch direkt in der Mathematik auf: Oft kommt es vor, dass mathematische Sätze mit Unbekannten formuliert werden, deren Wert erst später (zum Beispiel bei Anwendung des Satzes) bekannt ist. Wir werden nun Aussagen einführen, die auch Variablen erlauben, so genannte Prädikate.

%------------------
%
	\begin{de}[Prädikat]
		Ein \swdde{Prädikat} $A(X_1,...,X_n)$ ist ein sprachliches Gebilde mit Variablen $X_1,...,X_n$, das zu einer Aussage wird, wenn für jede Variable $X_1,...,X_n$ ein konkreter Wert eingesetzt wird.
	\deEnd
	\end{de}

%------------------
%
	\begin{bem}
		Auf Prädikaten können dieselben Operatoren angewandt werden wie auf Aussagen. Beachte aber, dass das Ergebnis dann immernoch ein Prädikat ist und keine Aussage!
	\end{bem}

%------------------
%
	\begin{bsp}
		Definiere
		\begin{center}
			$A(X)$ := \anf{$X$ darf Alkohol kaufen},
		\end{center}
		und
		\begin{center}
			$B(X)$ := \anf{$X$ ist älter als 10 Jahre}.
		\end{center}
		Die Interpretation des Prädikats $A(\text{Steffen})\Rightarrow B(\text{Steffen})$ lautet: \anf{Wenn Steffen Alkohol kaufen darf, dann ist Steffen älter als 10 Jahre.}
	\end{bsp}

%------------------
%
	\begin{de}[Prädikat Elementrelation]
		Für $X$ ein Objekt und $M$ eine beliebige Menge können wir das Prädikat
		\[
			E(X,M) := X\in M
		\]
		definieren. \\
		Interpretation: \anf{$X$ ist Element von $M$}.
	\deEnd
	\end{de}

%------------------
%
	\begin{bsp}
		Die Aussage $E(2,\{1,2,3\}) =$\anf{2 ist Element von \{1,2,3\}} ist wahr. Also gilt $2\in \{1,2,3\}$ (Infix-Notation). \\
		Die Aussage $E(4,\{1,2,3\})$ ist falsch, also gilt $\neg(4\in\{1,2,3\})$.
	\end{bsp}
	
	\noindent Wie können wir aus Prädikaten wieder Aussagen gewinnen? Eine Möglichkeit ist, konkrete Werte für die Variablen einzusetzen. Die resultierenden Aussagen für sich genommen sind aber sehr schwach, damit erreichen wir keine Aussage wie in Bsp.3.1. In der Prädikatenlogik wurden daher \anf{Quantoren} eingeführt. Diese geben formal richtig an, für wie viele Objekte $C$ ein bestimmtes Prädikat gilt und erleichtern somit auch die Formalisierung von Gedanken.

%------------------
%
	\begin{de}[Quantoren: All-Quantor und Existenz-Quantor]
		Die Aussage
		\[
			\forall X:A(X)
		\]		
		ist wahr, wenn für alle Objekte $X$ die Aussage $A(X)$ wahr ist. $\forall$ heißt \swdde{All-Quantor}. \\
		Die Aussage
		\[
			\exists X:A(X)
		\]
		ist wahr, wenn es (mindestens) ein Objekt $X$ gibt, sodass die Aussage $A(X)$ wahr ist. $\exists$ heißt \swdde{Existenz-Quantor}.
	\deEnd
	\end{de}
	
	\noindent Zum Erreichen einer Aussage wie in Bsp.1.3 können wir also auch Quantoren als Werkzeug verwenden. \\
	Die Quantoren alleine reden von allen möglichen Objekten/Variablen $X$, d.h. mit $X$ können alle möglichen Zahlen, Mengen, etc. gemeint sein. Wollen wir unsere Aussagen auf bestimmte Teilmengen einschränken, brauchen wir dazu zusätzliche Prädikate.

%------------------
%
	\begin{bsp}
		Für zwei Prädikate $A(X),B(X)$ ist
		\[
			\forall X:(A(X)\Rightarrow B(X))
		\]
		eine Aussage. \\
		Interpretation: \anf{Für alle $X$ gilt: Wenn $A(X)$ wahr ist, dann ist auch $B(X)$ wahr}.\\
		Für eine feste Menge $M$ und ein Prädikat $A(X)$ sind
		\begin{align}
			\forall X&: (E(X,M)\Rightarrow A(X)) \\
			\exists X&: (E(X,M)\wedge A(X))
		\end{align}
		Aussagen. \\
		Interpretation: (1): \anf{Für alle $X$ aus der Menge $M$ ist $A(X)$ wahr}, bzw. (2): \anf{Es gibt ein $X$ aus der Menge $M$, für welches $A(X)$ wahr ist}. \\
		Für Aussagen der Form (1) und (2) können wir auch die Infix-Notation als Abkürzungen, bzw. einfachere und übersichtlichere Schreibweise, verwenden, die in der Mathematik allgemein gebräuchlich ist:
		\begin{align*}
			\forall X\in M &: A(X) \\
			\exists X\in M &: A(X)
		\end{align*}
	\end{bsp}

% --------------------------------------------------------------------
% §3.2 Subsection <<Negation von Quantoren>>
% --------------------------------------------------------------------
\subsection{Negation von Quantoren}

%------------------
%
	\begin{bem}[Negation von Quantoren]
		Für ein Prädikat $A(X)$ gilt:
		\[
			\neg(\exists X:A(X)) \Leftrightarrow \forall X:\neg A(X),
		\]
		die Negation von \anf{Es existiert ein $ X $, sodass $A(X)$ wahr} ist also \anf{Für alle $X$, ist $A(X)$ falsch}. \\
		Es gilt weiter:
		\[
			\neg(\forall X:A(X)) \Leftrightarrow \exists X:\neg A(X),
		\]
		die Negation von \anf{Für alle $X$ ist $A(X)$ wahr} ist also \anf{Es existiert ein $X$, für das $A(X)$ falsch ist}.
	\end{bem}
	
	\noindent (Achtung: Die Negation der Aussage $(\forall X:A(X))$ lautet also nicht etwa \anf{Für alle $X$ ist $A(X)$ falsch}! Denn offensichtlich würde dann die Aussage und ihre Negation nicht alle Möglichkeiten abdecken: Der Fall, dass $A(X)$ für genau zwei $X$ falsch ist, wäre weder in der Aussage noch ihrer Negation enthalten.)  

%------------------
%
	\begin{lem}
		Für zwei Aussagen $A,B$ gilt:
		\[
			\neg(A\Rightarrow B) \Leftrightarrow A\wedge \neg B
		\]
	\end{lem}

%------------------
%
	\begin{bew}
		\[\begin{tabular}{c|c||c|c|c|c|c}
			A & B & $A\Rightarrow B$ & $\neg(A\Rightarrow B)$ & $ \neg B $ & $(A\wedge\neg B$ & $ \neg(A\Rightarrow B) \Leftrightarrow A\wedge \neg B $\\
			\hline
			w&w&w&f&f&f&w\\
			w&f&f&w&w&w&w\\
			f&w&w&f&f&f&w\\
			f&f&w&f&w&f&w\\
		\end{tabular}\] \hfill $\Box$
	\bewEnd
	\end{bew}

%------------------
%
	\begin{kor}
		Für eine Menge $M$ und ein Prädikat $A(X)$ gilt:
		\begin{align*}
			\neg(\forall X\in M:A(X)) & \Leftrightarrow \exists X\in M: \neg A(X) \\
			\neg(\exists X\in M:A(X)) & \Leftrightarrow \forall X\in M: \neg A(X)
		\end{align*}
	\end{kor}

%------------------
%
	\begin{bew}
		Es gilt
		\begin{align*}
			\neg(\forall X\in M:A(X)) & \Leftrightarrow \neg(\forall X:(E(X,M)\Rightarrow A(X))) \\
			&\Leftrightarrow \exists X: \neg(E(X,M)\Rightarrow A(X)) \\
			& \stackrel{3.10}{\Leftrightarrow} \exists X: E(X,M) \wedge \neg A(X) \\
			& \Leftrightarrow \exists X\in M: \neg A(X)
		\end{align*}
		Der zweite Teil kann mit einer ähnlichen Rechnung bewiesen werden. \hfill $\Box$
	\bewEnd
	\end{bew}


% --------------------------------------------------------------------
% §4 Section <<Rückblick>>
% --------------------------------------------------------------------
\end{comment}
