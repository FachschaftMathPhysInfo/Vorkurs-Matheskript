% ====================================================================
%                                                                 
%
%	Skript Statistik 1
%
% %%ts latex start%%[2019-02-14 Thu 15:13]%%ts latex end%%
% ====================================================================

\documentclass[12pt,BCOR1cm,ngerman,DIV15,fleqn,chapterprefix,headings=small]{ST1-book}
\synctex=1

% ====================================================================
% <<Seitenlayout>>
% ====================================================================

% Titel und aktuelles Datum in den Seitenkopf setzen
\ohead[]{\dgrau\small\normalfont\bfseries \leftmark}
\ihead[]{\dgrau\small\normalfont  \rightmark}
\ofoot[]{\small\dgrau \thepage}
\ifoot[]{\dgrau\small  Mathematischer Vorkurs WS 2021/22}
\setheadsepline{.5pt}[\dgrau]
\setfootsepline{.4pt}[\dgrau]
\renewcommand*{\partformat}{\dgrau\partname~\thepart}
\renewcommand\sectionmark[1]{\markright{\MakeMarkcase {\S \thesection\hskip .5em\relax#1}}}
\addtokomafont{part}{\dgrau}
\addtokomafont{chapter}{\dgrau}
\addtokomafont{section}{\dgrau} %\slshape
\addtokomafont{subsection}{\dgrau}
\addtokomafont{subsubsection}{\dgrau}
\addtokomafont{paragraph}{\dgrau}

\defpagestyle{first}{{}{\parbox{160mm}{%
\parbox{120mm}{\hfill\\\mbox{\Large\normalfont\rmfamily {Ruprecht-Karls-Universität  Heidelberg}}\\ {\large\normalfont\rmfamily Fachschaft MathPhysInfo}\\[1ex]
  { }}
\parbox{35mm}{\center\includegraphics[scale=.75]{./_src/UniHei_Logo_4C_small.eps}
  \\[.5ex]}\\[4pt]
 \hspace*{-.1\textwidth}\rule{1.2\textwidth}{.5pt}}}{}}{(\textwidth,0pt){}{\parbox{160mm}{%
\center\cmssfamily\small M$\Lambda$THEM$\Lambda$TIKON, Im Neuenheimer Feld 205, 69120 Heidelberg\\[.25em]
 \cmssfamily Telefon: +49 6221 54.14.999\ --  Fax: +49 6221 54.161.14.999\\[.25em]
   \cmssfamily  eMail: 
   \href{mailto:fachschaft@mathphys.stura.uni-heidelberg.de}{fachschaft@mathphys.stura.uni-heidelberg.de}\\[.25em]
   \cmssfamily  Webseite: \href{https://mathphys.stura.uni-heidelberg.de/w/}{https://mathphys.stura.uni-heidelberg.de}}}{}}

\defpagestyle{chapter}{{}{}{}(\textwidth,0pt)}{{\small\dgrau\thepage\hfill Mathematischer Vorkurs}{\small\dgrau Mathematischer Vorkurs\hfill\thepage}{\hfill\small\dgrau\thepage\hfill}}
\renewcommand*{\chapterpagestyle}{chapter}

\defpagestyle{tablematiere}{{\dgrau\small\normalfont\bfseries
    Inhaltsverzeichnis\hfill}{\hfill\dgrau\small\normalfont\bfseries
    Inhaltsverzeichnis}{}}{{\small\dgrau\thepage\hfill Mathematischer Vorkurs}{\small\dgrau Mathematischer Vorkurs\hfill\thepage}{\hfill\small\dgrau\thepage\hfill}}

\defpagestyle{intro}{{\dgrau\small\normalfont\bfseries
    Einleitung\hfill}{\hfill\dgrau\small\normalfont\bfseries Einleitung}{}}{{\small\dgrau\thepage\hfill  Mathematischer Vorkurs}{\small\dgrau Mathematischer Vorkurs\hfill\thepage}{\hfill\small\dgrau\thepage\hfill}}

%%%% Index
\usepackage{makeidx} % Indexpaket laden
\makeindex
  


\usepackage{ST1-abrev-package}
\usepackage{float}
\usepackage{tikz-cd}
\usetikzlibrary{babel}
\DeclareMathOperator{\iden}{id}
\DeclareMathOperator{\im}{im}



% ====================================================================
% <<MetaPost-Spiele>>
% ====================================================================
\usepackage{ifpdf}
\usepackage{array}
\newcolumntype{L}{>{$}l<{$}}
\usepackage{graphicx}
\ifpdf
\DeclareGraphicsRule{*}{mps}{*}{}
\else
\fi


% ====================================================================
% <<Schalter-Skizze>>
% ====================================================================
% 
\usepackage{environ}
\newtoggle{SK}
%\togglefalse{SK} % ohne Skizzen
\toggletrue{SK} % mit Skizzen
\NewEnviron{envSK}{%
  \iftoggle{SK}%
    {\BODY}%
    {\kwdsch{Skizzen werden hier eingfügt}}%
}
\NewEnviron{testSK}{%
  \iftoggle{SK}
    {\kwdsch{Skizzen werden hier eingfügt}}%
    {\BODY}%
}
\newtoggle{BiVL}
\togglefalse{BiVL} % ohne Beweis
%\toggletrue{BiVL} % mit Beweis
\NewEnviron{BiVL}[3][In der Vorlesung.]{%
\begin{bew}[von \cref{#2}]\label{#3}
\iftoggle{BiVL}%
{\BODY}%
{#1}%
\bewEnd
\end{bew}%
}

% ====================================================================
% Abstände im Inhaltsverzeichnis manipulieren
\setcounter{tocdepth}{1}
\setcounter{secnumdepth}{3}
\makeatletter
\def\tableofcontents{
\vspace{-2em}
\chapter*{\dgrau Inhaltsverzeichnis}
\vspace*{-2em}
\@starttoc{toc}
}
\makeatother

\usepackage[titles]{tocloft}
\newcommand*\toccolor{\dgrau}
\renewcommand{\cftsecleader}{\toccolor\cftdotfill{\cftsecdotsep}} %Sectionpunkte  einfaerben
\renewcommand{\cftsubsecleader}{\toccolor\cftdotfill{\cftsubsecdotsep}} % Subsectionpunkte  einfaerben
\renewcommand{\cftsubsubsecleader}{\toccolor\cftdotfill{\cftsubsubsecdotsep}} %Subsubsectionpunkte  einfaerben
\renewcommand*\cftpartpagefont{\toccolor\bfseries} %Partie einfaerben
\renewcommand*\cftchappagefont{\toccolor\bfseries} %Kapitel einfaerben
\renewcommand*\cftsecpagefont{\toccolor} % Section einfaerben
\renewcommand*\cftsubsecpagefont{\toccolor} % Subsection einfaerben
\renewcommand*\cftsubsubsecpagefont{\toccolor} % Subsection einfaerben
\renewcommand{\cftsecpresnum}{\S }%Vor der Nummer das Paragraph Zeichen

\usepackage{natbib} 
% BIBLIOGRAPHY 
\renewcommand{\cite}{\citet} 
\bibliographystyle{abbrvnat} 

% Change manually label in list and allows reference 
\makeatletter
\newcommand{\mylabel}[2]{#2\def\@currentlabel{#2}\label{#1}}
\makeatother


% change equation label
\makeatletter
\def\tagform@#1{\maketag@@@{\dgrau(\ignorespaces{#1}\unskip\@@italiccorr)}}
\makeatother


\makeatletter
%\renewcommand{\thesection}{\two@digits{\value{section}}}
% Section mit führender Null
\renewcommand{\thesubsection}{\arabic{subsection}.}
% Section mit führendem Paragraphzeichen
\renewcommand\sectionformat{\S \thesection\xspace}%

\renewcommand{\thechapter}{\arabic{chapter}} 
\renewcommand{\thesat}{\thesection.\two@digits{\value{sat}}}   
\renewcommand{\thekor}{\thesection.\two@digits{\value{kor}}}   
\renewcommand{\thepro}{\thesection.\two@digits{\value{pro}}}
\renewcommand{\thelem}{\thesection.\two@digits{\value{lem}}} 
\renewcommand{\theaxi}{\thesection.\two@digits{\value{axi}}}% von Janne  
\renewcommand{\thede}{\thesection.\two@digits{\value{de}}}   
\renewcommand{\thesch}{\thesection.\two@digits{\value{sch}}} 
\renewcommand{\thenotion}{\thesection.\two@digits{\value{notion}}} % von Janne
\renewcommand{\theschn}{\thesection.\two@digits{\value{schn}}}   
\renewcommand{\thespr}{\thesection.\two@digits{\value{spr}}}   
\renewcommand{\thesprn}{\thesection.\two@digits{\value{sprn}}}   
\renewcommand{\thebem}{\thesection.\two@digits{\value{bem}}}
\renewcommand{\theobem}{\thesection.\two@digits{\value{obem}}}   
\renewcommand{\thebsp}{\thesection.\two@digits{\value{bsp}}}   
\renewcommand{\thebspe}{\thesection.\two@digits{\value{bspe}}}   
\renewcommand{\thebew}{\thesection.\two@digits{\value{bew}}}
\renewcommand{\theaufg}{\thesection.\two@digits{\value{aufg}}}  
\renewcommand{\theski}{\thesection.\two@digits{\value{ski}}}   
\makeatother
    

\usepackage{chngcntr}
\counterwithin{equation}{section}

%%%%%%%%%%%%%%%%%%
%%%%%% hyperref ganz zum Schluss laden mit options: hypertexnames=TRUE ,pdfpagelabels=TRUE scheint nicht noetig zu sein
%%%%%%%%%%%%%%%%%%
\usepackage[pdfpagemode=UseOutlines ,plainpages=false,
hypertexnames=false,pdfpagelabels=TRUE,
hyperindex=true,colorlinks=true]{hyperref}
	\makeatletter
	\Hy@breaklinkstrue
	\makeatother
\hypersetup{colorlinks ,linkcolor=gray ,filecolor=darkgreen
,urlcolor=darkblue ,citecolor=black}

%%%%%%%%%%%%%%%%%%
%%%% cleverref muss ans ende
\usepackage{cleveref}

\crefname{axi}{Axiom}{Axiome}% von Janne
\crefformat{axi}{#2{\rmfamily{\color{\colorres} Axiom} {\dgrau\S}{#1}}#3}
\crefname{pro}{Proposition}{Propositionen}
\crefformat{pro}
{#2{\rmfamily{\color{\colorres} Proposition} {\dgrau\S}{#1}}#3}

\crefname{sprache}{Sprache}{Propositionen} % von Luka
\crefformat{sprache}{#2{\rmfamily{\color{\colorres} Sprache}{\dgrau\S}{#1}}#3}

\crefname{kor}{Korollar}{Korollare}
\crefformat{kor}{#2{\rmfamily{\color{\colorres} Korollar} {\dgrau\S}{#1}}#3}
\crefname{sat}{Satz}{Sätze}
\crefformat{sat}{#2{\rmfamily{\color{\colorres} Satz} {\dgrau\S}{#1}}#3}
\crefname{lem}{Lemma}{Lemmata}
\crefformat{lem}{#2{\rmfamily{\color{\colorres} Lemma} {\dgrau\S}{#1}}#3}
\crefname{eig}{Eigenschaft}{Eigenschaften}
\crefformat{eig}{#2{\rmfamily{\color{\colorres} Eigenschaft} {\dgrau\S}{#1}}#3}
\crefname{eign}{Eigenschaft}{Eigenschaften}
\crefformat{eign}{#2{\rmfamily{\color{\colorres} Eigenschaften} {\dgrau\S}{#1}}#3}

\crefname{de}{Definition}{Definitionen}
\crefformat{de}{#2{\rmfamily{\color{\colorde} Definition} {\dgrau\S}{#1}}#3}
\crefname{notion}{Notation}{Notation}
\crefformat{notion}{#2{\rmfamily{\color{\colorsch} Notation} {\dgrau\S}{#1}}#3} % von Janne
\crefname{sch}{Schreibweise}{Schreibweisen}
\crefformat{sch}{#2{\rmfamily{\color{\colorsch} Schreibweise} {\dgrau\S}{#1}}#3}
\crefname{spr}{Sprechweise}{Sprechweisen}
\crefformat{spr}{#2{\rmfamily{\color{\colorsch} Sprechweise} {\dgrau\S}{#1}}#3}
\crefname{sprn}{Sprechweisen}{Sprechweisen}
\crefformat{sprn}{#2{\rmfamily{\color{\colorsch} Sprechweisen} {\dgrau\S}{#1}}#3}
\crefname{sch-spr}{Schreib- und Sprechweise}{Schreib- und Sprechweisen}
\crefformat{sprn}{#2{\rmfamily{\color{\colorsch} Sprechweisen} {\dgrau\S}{#1}}#3}
\crefname{schn-sprn}{Schreib- und Sprechweisen}{Schreib- und Sprechweisen}
\crefformat{sprn}{#2{\rmfamily{\color{\colorsch} Sprechweisen} {\dgrau\S}{#1}}#3}

\crefname{ann}{Annahme}{Annahmen}
\crefformat{ann}{#2{\rmfamily{\color{\colorde} Annahme} {\dgrau\S}{#1}}#3}

\crefname{bem}{Bemerkung}{Bemerkungen}
\crefformat{bem}{#2{\rmfamily{\color{\colorbem} Bemerkung} {\dgrau\S}{#1}}#3} 

\crefname{obem}{Bemerkung}{Bemerkungen}
\crefformat{obem}{#2{\rmfamily{\color{\colorbem} Bemerkung} {\dgrau\S}{#1}}#3} 

\crefname{anm}{Anmerkung}{Anmerkung}
\crefformat{anm}{#2{\rmfamily{\color{\colorbem} Anmerkung} {\dgrau\S}{#1}}#3}

\crefname{bsp}{Beispiel}{Beispiele}
\crefformat{bsp}{#2{\rmfamily{\color{\colorbsp} Beispiel} {\dgrau\S}{#1}}#3}
\crefname{bspe}{Beispiele}{Beispiele}
\crefformat{bspe}{#2{\rmfamily{\color{\colorbsp} Beispiele} {\dgrau\S}{#1}}#3}
\crefname{obsp}{Beispiel}{Beispiele}
\crefformat{obsp}{#2{\rmfamily{\color{\colorbsp} Beispiel} {\dgrau\S}{#1}}#3}
\crefname{ill}{Illustration}{Illustrationen}
\crefformat{ill}{#2{\rmfamily{\color{\colorbsp} Illustration} {\dgrau\S}{#1}}#3}
\crefname{illn}{Illustrationen}{Illustrationen}
\crefformat{illn}{#2{\rmfamily{\color{\colorbsp} Illustrationen} {\dgrau\S}{#1}}#3}


\crefname{bew}{Beweis}{Beweise}
\crefformat{bew}{#2{\rmfamily{\color{\colorres} Beweis} {\dgrau\S}{#1}}#3}


\crefname{aufg}{Aufgabe}{Aufgaben}
\crefformat{aufg}{#2{\rmfamily{\color{\coloraufg} Aufgabe} {\dgrau\S}{#1}}#3}

\crefname{section}{Abschnitt}{Abschnitte}
\crefformat{section}{#2{\rmfamily{\dgrau Abschnitt} {\dgrau\S}{#1}}#3}

\crefname{chapter}{Kapitel}{Kapitel}
\crefformat{chapter}{#2{\rmfamily{\dgrau Kapitel} {#1}}#3}

\newcommand{\myref}[1]{{\normalfont\rmfamily{\dgrau\S}\ref{#1}}}
\newcommand{\mycref}[1]{\cref{#1}}
\newcommand{\myrefi}[2]{{\dgrau\S}\ref{#1}\,{\dgrau#2}}




\renewcommand{\qedsymbol}{\scriptsize{\ensuremath{\blacksquare}}}
%%%%%%%%%%%%%%%%%%%%%%%%%%%%%%%%%%%%%%%%%%%%%%%%
% \usepackage[right]{showlabels}
% \renewcommand{\showlabelfont}{\dr\normalfont\tiny\ttfamily}
\begin{document}
\frontmatter
\renewcommand{\thepage}{\roman{page}}
\thispagestyle{first}
\vspace*{5cm}
\begin{center}
{\large\normalfont\itshape  Skript zum}\\[3ex]
{\huge\normalfont\scshape  Mathematischen Vorkurs}\\[5ex]
{\large\normalfont\itshape  Wintersemester 2021/22}\end{center}
 \vfill
\centerline{{\normalfont\itshape Fassung Stand \today}}\hfill\\[1ex]
\centerline{{\db\small Habt ihr Fragen, Wünsche, Anregungen? Schreibt uns einfach eine}}  
\centerline{{\db\small  Mail:
\href{fachschaft@mathphys.stura.uni-heidelberg.de}{fachschaft@mathphys.stura.uni-heidelberg.de}
mit.}}
\cleardoublepage
\setcounter{page}{1}
\pagestyle{tablematiere}
\tableofcontents
\mainmatter

% ====================================================================
% <<Inhalt>>
% ====================================================================
\pagestyle{scrheadings}

% ====================================================================
% <<Einleitung>>
% ====================================================================
\pagestyle{intro}
\noindent Das Skript zum mathematischen Vorkurs wurde zum Wintersemester 2021/2022 von Luka Thomé, Matthis Scholz, Nikolaus Betker, Maximilian Bur und Luna Cielibak grundlegend überarbeitet. Autoren der einzelnen Kapitel sind:
\begin{itemize}
 \item Kapitel 1 Logik -- Luka Thomé
 \item Kapitel 2 Beweise -- Luka Thomé
 \item Kapitel 3 Mengen und Familien -- Matthis Scholz
 \item Kapitel 4 Abbildungen -- Nikolaus Betker, Luka Thomé
  \item Kapitel 5 Relationen -- Matthis Scholz
 \item Kapitel 6 Folgen, Abstand und Grenzwerte -- Maximilian Bur, Luka Thomé
 \item Kapitel 7 Verknüpfungen -- Luka Thomé
\end{itemize}

\noindent Wir wünschen euch zwei schöne Vorkurswochen, einen guten Start ins Studentenleben und hoffen, euch mit diesem Skript einen ersten Einblick in die Uni-Mathematik bieten zu können, der neugierig auf mehr macht. \\

Nikolaus, Max, Luna, Matthis, Luka 

% ====================================================================
% 1. Vortrag <<Logik>>
% ====================================================================
\setchapterpreamble[c][.7\textwidth]{%
  \itshape\dgrau\small  In diesem Vortrag werden diejenigen sprachlichen Strukturen vorgestellt, die in der Mathematik verwendet werden, um Definitionen, Sätze und Beweise zu formulieren. 
  \vspace{24pt}%
}
\chapter{Logik}
% ====================================================================
%                                                                 
%
%       §01
%
%
% %%ts latex start%%[2019-03-07 Thu 14:45]%%ts latex end%%
% ====================================================================
%
% --------------------------------------------------------------------
% §1 Section <<Klassische Aussagenlogik>>
% --------------------------------------------------------------------





\section{Buchstaben in mathematischen Formeln}


\begin{de}[Zeichen mit einer konkreten Bedeutung versehen]
 Gelegentlich ist es bequem ein konkretes, kompliziert definiertes Objekt mit einem Zeichen zu bezeichnen. Beispielsweise lässt sich mit Methoden der Analysis zeigen, dass die Gleichung $x^5=x+1$ genau eine Lösung in den reellen Zahlen besitzt\footnote{Die Werkzeuge dafür werden in der Analysis 1-Vorlesung (oder vielleicht auch schon in der Schule) vermittelt.} wohingegen sich mit Methoden der Algebra zeigen lässt, dass sich diese Lösung nicht mit den Operationen $+$, $-$, $\cdot$, $:$, $\sqrt{}$ konstruieren lässt\footnote{Die Werkzeuge dafür werden in der Drittsemestervorlesung „Algebra 1“ vermittelt.}. Möchte man nun mit dieser Lösung arbeiten, so ist es umständlich, immer wieder „die eindeutige reelle Lösung der Gleichung $x^5=x+1$“ zu schreiben und es ist bequemer, einen Buchstaben zu verwenden. Dafür benutzen Mathematiker einen Imperativ:
 \begin{quote}
„Sei $\xi$ die eindeutige reelle Lösung der Gleichung $x^5=x+1$.“  
 \end{quote}
 Nun lassen sich komfortabel Sätze wie etwa „Es gilt $\xi^2\cdot (\xi^{8}-1)=2\xi +1$“ formulieren. \\
Um formal ein Zeichen mit einer konkreten Bedeutung zu versehen, kann man das Symbol
 \begin{align*}
  := &&& (\text{lies: „ist definiert als“ oder „ist per Definition gleich“})
 \end{align*}
verwenden. Beispielsweise wird im Ausdruck
 \[ d:= \text{ggT}(108,2048) \]
 festgelegt, dass der Buchstabe $d$ den größten gemeinsamen Teiler von $108$ und $2048$ bezeichnet.  \\
Manchmal ist es deutlich lesbarer, Umgangssprache zu verwenden, als auf Teufel komm raus eine Formel mit dem Zeichen „$:=$“ hinzuschreiben. Z.B. ist
\begin{quote}
 Sei $x$ die kleinste positive Nullstelle der Kosinusfunktion.
\end{quote}
sicherlich für viele Leute eine schneller verständlichere Definition als
\[ x:= \min \{y\in \Rz_{> 0} \mid \cos(y)=0 \}\]
Dies ist auch eine Geschmacksfrage.
\end{de}



\begin{de}[Variable] \label{variable}
 Eine \textbf{Variable} ist ein Zeichen, das als Platzhalter dient, an dessen Stelle Objekte von einer gewissen Sorte eingesetzt werden können. Die Sorte von Objekten, die für eine Variable eingesetzt werden kann, heißt der \textbf{Typ} dieser Variable. In modernen Mathebüchern werden Variablen meistens als kursive Buchstaben gedruckt. Ansonsten ist aber alles erlaubt: Großbuchstaben, Kleinbuchstaben, griechische Buchstaben, Hiraganazeichen usw. Beispielsweise können die folgenden Zeichen alle als Variablen verwendet werden:
 \[ A,B,m,n,\delta,\mu, x,y,\dots \]
 Prinzipiell darfst du jedes Zeichen als Variable verwenden, solange du festlegst, für welche Sorte von Objekten es als Platzhalter dient (z.B. „Sei $n$ eine ungerade natürliche Zahl“). Allerdings haben sich in der Mathematik diverse Variablen-Konventionen eingebürgert, die du befolgen solltest, um deinen Text für andere Mathematiker leichter lesbar zu machen. Zum Beispiel:
 \begin{itemize}
  \item Natürliche Zahlen werden meist mit den Buchstaben $n,m,k,\dots$ bezeichnet.
  \item Reelle Zahlen mit den Buchstaben $x,y,\dots$.
    \item Komplexe Zahlen mit den Buchstaben $z,w,\dots$.
  \item Funktionen werden meist mit den Buchstaben $f,g,\dots$ bezeichnet.
  \item Vektoren mit den Buchstaben $v,w,\dots$.
  \item In der mathematischen Logik kommt es sogar vor, dass „Metavariablen“ vom Typ „Variable“ auftauchen („Seien $x,y$ zwei Variablen“).
 \end{itemize}
 Das alles sind aber keine strikten Regeln und du solltest, wann immer du Variablen verwendest, klarstellen, von welchem Typ sie sind. Um die Festlegung eines Variablentyps sprachlich hervorzuheben, verwenden Mathematiker dafür den Imperativ. Wie beispielsweise in „Seien $x,y$ zwei reelle Zahlen“ oder „Es bezeichne $n$ eine beliebige natürliche Zahl“.
\end{de}



\begin{bem}[Variablen nie vom Himmel fallen lassen!]
Erstsemester vergessen nicht selten, ihre Variablen sachgemäß einzuführen. Wann immer du eine Variable wie „$x$“ oder „$A$“ verwendest, solltest du, z.B. mit einem „Sei\dots“-Imperativ, klarstellen, auf welche Sorte von Objekten sie sich bezieht. Es ist \emph{sehr} nervig und verwirrend, wenn in Texten plötzlich Buchstaben auftreten, für die nie klargestellt wurde, was sie zu bedeuten haben. 
\end{bem}






\section{Ein bisschen Aussagenlogik}

\begin{de}[Aussage]
Eine \textbf{Aussage} ist ein feststellender Satz, dem ein Wahrheitswert wie „wahr“ oder „falsch“ zugeordnet werden kann und
der vermittels der in diesem Paragraphen besprochenen \emph{Junktoren} mit anderen Aussagen verknüpft werden kann. %Als Aussagenvariablen werden in der deutschen Literatur oft die Buchstaben $A,B,\dots$ verwendet, in der englischen Literatur dagegen $P,Q,\dots$ ($P$ wie ``Proposition").
	\end{de}

	


	\begin{bsp}
Parallel zur abstrakten Theorie werden uns in diesem Paragraphen die folgenden Beispielaussagen begleiten:
		\begin{itemize}
			\item $B_1 :=$ „Der Döner wurde in Deutschland erfunden.“
			\item $B_2 :=$ „Heute ist Mittwoch.“
			\item $B_3 :=$ „Es gibt außerirdisches Leben.“
			\item $B_4 :=$ „Der FC Bayern spielte eine schlechte Hinrunde.“
			\item $B_5 :=$ „Die Relativitätstheorie ist fehlerhaft.“
		\end{itemize}
Nicht jeder deutsche Satz ist eine Aussage. Sätze, die eher nicht als Aussagen durchgehen würden, sind zum Beispiel: \quad
 \begin{itemize}
\item „Hoffentlich kommt der Kellner bald.“
\item „Was möchten Sie trinken?“
\item „Ein großes Bier, bitte!“
\end{itemize}
	\end{bsp}
	


Eine systematische Operation, die aus einer Handvoll Aussagen eine neue Aussage hervorbringt (ähnlich wie etwa die Operation „$+$“ aus zwei Zahlen ihre Summe macht) heißt \textbf{Junktor} oder auch \textbf{logischer Operator}. In diesem Abschnitt werden die wichtigsten Junktoren vorgestellt.
	\begin{de}[Negation]
		Für eine Aussage $A$ wird mit
    \begin{align*}
   \neg A   && (\text{lies: „nicht $A$“})
    \end{align*}
die \emph{Negation} von $A$ notiert. $\neg A$ ist die Verneinung von $A$, d.h. $\neg A$ besagt, dass $A$ nicht zutrifft. Beispielsweise sind die Negationen der Beispielaussagen von vorhin gegeben durch:
 \begin{itemize}
			\item $\neg B_1 =$ „Der Döner wurde nicht in Deutschland erfunden.“
			\item $\neg B_2 =$ „Heute ist nicht Mittwoch.“
			\item $\neg B_3 =$ „Es gibt kein außerirdisches Leben.“
			\item $\neg B_4 =$ „Der FC Bayern spielte keine schlechte Hinrunde.“
			\item $\neg B_5 =$ „Die Relativitätstheorie ist fehlerfrei.“
    \end{itemize}
    An den Beispielen wird deutlich, dass die Negation einer Aussage nicht immer durch das Signalwort „nicht“ erfolgen muss. Manchmal wird die Negation einer Aussage $A$ auch mit einem Oberstrich notiert: $\overline{A}$.
	\end{de}
	

	
	\begin{comment}
	\begin{bem}[Zeichen mit mehr als einer Bedeutung „überladen“]
In dieser Definition haben wir das Zeichen „$A$“ überladen. Denn einerseits bezeichnet $A$ ja schon die Aussage „Der Döner wurde in Deutschland erfunden“, andererseits beginnt die Definition mit „Für eine Aussage $A$ wird\dots“. Wir haben das Zeichen also zugleich als Konstantensymbol und als Variable verwendet. Solche Überladungen von Zeichen sind häufige Fehlerquellen, sollten soweit es geht vermieden und wirklich nur dann verwendet werden, wenn keine Verwirrung auftreten kann und es unmittelbar klar ist, dass eine Überladung vorliegt.
	\end{bem}
\end{comment}


	
	\begin{de}[Und-Verknüpfung]
            Zwei Aussagen $A,B$ können zu ihrer \textbf{Konjunktion}
\begin{align*}
  A\land B && (\text{lies: „$A$ und $B$“})
\end{align*}
 verknüpft werden, deren Bedeutung ist, dass sowohl $A$ als auch $B$ zutreffen. Beispiele für Konjunktionen sind etwa:
\begin{itemize}
    \item $B_2\land B_4 =$ „Heute ist Mittwoch und der FC Bayern spielte eine schlechte Hinrunde.“
\item $B_3\land B_5 =$ „Es gibt außerirdisches Leben, aber die Relativitätstheorie ist fehlerhaft.“
    \item $B_5\land B_1 =$ „Nicht nur ist die Relativitätstheorie fehlerhaft -- auch der Döner wurde in Deutschland erfunden.“
\end{itemize}
	\end{de}
	
	
	
\begin{bem}[Fachbegriffe]
 Du brauchst dir im Vorkurs nicht gleich alle Fachbegriffe zu merken. Sofern du weißt, dass es eine Und- und eine Oder-Verknüpfung gibt, brauchst du dir nicht merken, dass sie auch „Konjunktion“ und „Disjunktion“ genannt werden. In diesem und den folgenden Vorträgen werden wir dennoch oft mehrere Wörter für dasselbe Konzept nennen, um dir zu erleichtern, die Begriffe im Internet nachzuschlagen.
\end{bem}



	\begin{de}[Oder-Verknüpfung]
Zwei Aussagen $A,B$ können zu ihrer \textbf{Disjunktion}
\begin{align*}
  A\lor B && (\text{lies: „$A$ oder $B$“})
\end{align*}
verknüpft werden, deren Bedeutung ist, dass von $A, B$ mindestens eine Aussage zutrifft. Beispiele für Disjunktionen sind:
\begin{itemize}
    \item $B_1 \lor B_3 =$ „Der Döner wurde in Deutschland erfunden oder es gibt außerirdisches Leben.“
\item $B_2\lor B_5 =$ „Wahlweise ist heute Mittwoch oder die Relativitätstheorie fehlerhaft.“
\item $B_4\lor B_4=$ „Der FC Bayern spielte eine schlechte Hinrunde oder der FC Bayern spielte eine schlechte Hinrunde.“
%\item Achtung: $B_2\lor\phantom{C} $ = „Heute ist Mittwoch, oder?“ ist \textbf{keine} erlaubte Verwendung des $\lor$-Symbols in der Aussagenlogik! Sowohl „$\land$“ als auch „$\lor$“ verknüpfen stets \emph{zwei} Aussagen (die nicht unbedingt voneinander verschieden sein müssen) miteinander.
\end{itemize}
	\end{de}

	
\begin{bem}[Ausschließendes Oder]
 Die Disjunktion bezeichnet ein \emph{einschließendes Oder} (lateinisch: „vel“), d.h. $A\lor B$ schließt auch den Fall ein, dass $A$ und $B$ beide gelten. In einer Mathematiker-Beziehung würde das Ultimatum „Unsere Beziehung -- oder deine dummen Fernsehserien!“ keine Besorgnis erregen. Das „oder“ lässt ja auch zu, dass beides vorliegen kann. Möchte man ein ausschließendes Oder (Informatiker sprechen vom XOR, im Lateinischen wird das Wort „aut“ benutzt) verwenden, kann man dies durch
 \begin{align*}
  A\ \dot\lor\ B && (\text{lies: „Entweder $A$ oder $B$“})
\end{align*}
notieren. Das „Entweder $A$ oder $B$“ soll soviel wie „$A$ oder $B$ aber nicht beides“ bedeuten. Das Ultimatum „\emph{Entweder} unsere Beziehung oder deine dummen Fernsehserien“ könnte selbst bei einem Mathematiker-Pärchen eine handfeste Beziehungskrise auslösen. \\
Beispielsweise ist
\begin{itemize}
 \item „Eine natürliche Zahl ist entweder eine gerade oder eine ungerade Zahl” eine korrekte Aussage, während
 \item „Jeder Vorkursteilnehmer ist entweder MathestudentIn oder InformatikstudentIn“ falsch ist, da es ja VorkursteilnehmerInnen gibt, die beides studieren.
\end{itemize}

\end{bem}



	\begin{de}[Implikationspfeil]
Zwei Aussagen $A,B$ können zur („materiellen“) \textbf{Implikation}
\[ A\to B \]
verknüpft werden: Deren Bedeutung ist, dass $B$ von $A$ impliziert wird. Dafür gibt es verschiedene Lesarten:
\begin{itemize}
\item „$A$ impliziert $B$“.
 \item „Wenn $A$ so auch $B$“.
 \item „Falls $A$, dann $B$“.
 \item „$B$ folgt aus $A$“.
 \item „$A$ ist eine hinreichende Bedingung für $B$“.
 \item „$B$ ist eine Konsequenz von $A$“.
 \item usw.
\end{itemize}
Man nennt den Pfeil „$\to$“ auch den \textbf{Implikationspfeil} und die Aussage $A\to B$ auch ein \emph{Konditional}. Beispiele für $\to$-Aussagen sind:
\begin{itemize}
 \item $B_1\to B_5$: „Wenn der Döner in Deutschland erfunden wurde, ist die Relativitätstheorie fehlerhaft“.
 \item $B_2\to B_4$: „Sofern der FC Bayern eine schlechte Hinrunde gespielt hat, ist heute Mittwoch.
 \item $B_3\to B_5$: „Unter der Annahme, dass es außerirdisches Leben gibt, ist die Relativitätstheorie fehlerhaft.“
\end{itemize}
Beachte, dass es beim Implikationspfeil „$\to$“ wesentlich auf die Reihenfolge ankommt. Während sich etwa die Aussagen $A\land B$ und $B\land A$ nicht in ihrer Bedeutung unterscheiden, sind $A\to B$ und $B\to A$ zwei grundlegend verschiedene Aussagen. Beispielsweise sind
\begin{itemize}
 \item[(1)] „Wenn heute Freitag ist, ist morgen Wochenende“.
 \item[(2)] „Falls morgen Wochenende ist, ist heute Freitag“.
\end{itemize}
zwei erheblich verschiedene Aussagen. Aussage (1) ist korrekt aber Aussage (2) ist falsch, da ja auch Samstag sein könnte.
\end{de}
	\begin{de}[Äquivalenz]
		Zwei Aussagen $A$ und $B$ lassen sich zur \textbf{Äquivalenz}
    \[ A\leftrightarrow B \]
    verknüpfen, dessen Bedeutung ist, dass sowohl $B$ von $A$ impliziert wird als auch $A$ von $B$ impliziert wird. Lesarten dafür sind:
    \begin{itemize}
    \item „$A$ ist äquivalent zu $B$“.
     \item „$A$ genau dann wenn $B$“. Ist wenig Platz vorhanden, schreibt man abkürzend „$A$ gdw. $B$“. In der englischen Literatur schreibt man ``$A$ iff $B$''.
    \item „$A$ gilt dann und nur dann, wenn $B$“.
    \end{itemize}
Man nennt den Doppelpfeil „$\leftrightarrow$“ \textbf{Äquivalenzzeichen} und die Aussage $A\leftrightarrow B$ auch ein \emph{Bikonditional}. Beispiele für Äquivalenzaussagen sind:
\begin{itemize}
     \item „Genau dann ist heute Mittwoch, wenn morgen Donnerstag ist“.
     \item $B_1\leftrightarrow B_3$: „Dass der Döner in Deutschland erfunden wurde, ist äquivalent dazu, dass es außerirdisches Leben gibt“.
     \item Eine reelle Zahl $x$ ist dann und nur dann eine negative reelle Zahl, wenn $-x$ eine positive reelle Zahl ist.
    \end{itemize}
	\end{de}


	
\begin{bem}[* Die Pfeile $\to$ und $\Rightarrow$]
 In der Analysis formuliert man Aussagen, die von Funktionen und Folgen handeln; in der Algebra geht es um Aussagen, die von Vektoren, Polynomen, etc. handeln. -- In der mathematischen Logik formuliert man Aussagen, die von Aussagen handeln. Daher liegen dort oft zwei Sprachebenen vor, die \emph{Objektsprache}, in der diejenigen mathematischen Aussagen formuliert sind, die vom Logiker untersucht werden, sowie die \emph{Metasprache}, der sich der Logiker bedient, um die objektsprachlichen Aussagen zu untersuchen. Ein Beispiel:
 \[ \underbrace{\text{Die Aussage „}}_{\text{Metasprache}}\underbrace{\text{$5$ ist eine gerade Zahl}}_{\text{Objektsprache}} \underbrace{\text{“ ist falsch.}}_{\text{Metasprache}} \]
 Die objektsprachliche Implikation bzw. Äquivalenz wird mit einem einfachen Pfeil „$\to$“ bzw. „$\leftrightarrow$“ notiert, während die metasprachliche Implikation bzw. Äquivalenz mit einem doppelten Pfeil „$\Rightarrow$“ bzw. „$\Leftrightarrow$“ notiert wird. \\[0.5em]
 Abseits der mathematischen Logik ist diese Unterscheidung aber überflüssig und die meisten Mathematiker verwenden die Implikationspfeile „$\to$“ und „$\Rightarrow$“ synonym. Benutze einfach den, der dir besser gefällt.
\end{bem}

	
	
	\begin{bem}[Klammern setzen]
Mithilfe der Junktoren lassen sich bereits beliebig kompliziert verschachtelte Aussagen bilden wie z.B. $(B_1\lor \neg B_2) \to (B_3\land \neg B_5)$:
\begin{quote}
 „Sofern der Döner in Deutschland erfunden wurde oder heute nicht Mittwoch ist, gibt es außerirdisches Leben und die Relativitätstheorie ist fehlerfrei.“
\end{quote}
oder
$B_1\lor (\neg B_2 \to (B_3\land \neg B_5))$:
\begin{quote}
„Der Döner wurde in Deutschland erfunden oder aber es gilt: wenn heute nicht Mittwoch ist, gibt es außerirdisches Leben und die Relativitätstheorie ist fehlerfrei.“
\end{quote}
Bei verschachtelten Aussagen sollte man Klammern verwenden, um deutlich zu machen, welche Junktoren „weiter innen liegen“ und welche „als letztes angewendet“ werden. Möchte man Klammern vermeiden, kann man dies alternativ auch durch verschieden große Leerstellen zwischen den Zeichen deutlich machen oder ein Hybrid aus beidem verwenden:
\begin{align*}
 B_1\lor \neg B_2\quad &\to \quad B_3\land \neg B_5 \\[0.5em]
 B_1\quad  &\lor \quad \neg B_2 \to (B_3\land \neg B_5)
\end{align*}
Es gibt auch Konventionen, die die „Erstausführung“ gewisser Junktoren vor anderen Junktoren regeln, ähnlich der Regel „Punkt- vor Strichrechnung“. Die solltest du aber nur dann stillschweigend verwenden, wenn du dir sicher bist, dass dein Leser dieselbe Konvention auch kennt und benutzt.
\end{bem}


	
\begin{bem}[* weitere Junktoren]
Es gibt noch weitere Junktoren wie etwa:
 \begin{itemize}
  \item Der Sheffer-Strich „$A\mid B$“ (lies: „Nicht sowohl $A$ als auch $B$“). In der Informatik spricht man auch von der NAND-Verknüpfung.
  \item Die Peirce-Funktion „$A\downarrow B$“ (lies: „Weder $A$ noch $B$“). In der Informatik spricht man auch von der NOR-Verknüpfung.
 \end{itemize}
% Eine Liste weiterer Junktoren und Schreibvarianten findet sich etwa \href{https://de.wikipedia.org/wiki/Junktor#M%C3%B6gliche_Junktoren}{auf der deutschen Wikipedia}. %Beachte allerdings, dass die Junktoren-Tabelle der Wikipedia eher die Gesamtheit der Verknüpfungen auf der Menge $\{\text{wahr},\ \text{falsch}\}$ der gewöhnlichen Wahrheitswerte bezeichnet. Junktoren im aussagenlogischen Sinn müssen nicht an Wahrheitswerten ausgerichtet sein, sondern vielmehr müssen die Wahrheitswerte so beschaffen sein, dass sie die Junktoren interpretieren können.
\end{bem}

	
\section{Ein bisschen Prädikatenlogik}
%Während in der Aussagenlogik lediglich Aussagen mithilfe von Junktoren in einfachere Aussagen zerlegt werden, beschäftigt sich die Prädikatenlogik mit einer weiteren Zerlegung von Aussagen in Objekt und Prädikat.









\subsection{Prädikate}
\begin{de}[Prädikat]
 Es sei $n$ eine natürliche Zahl. Ein \textbf{$n$-stelliges Prädikat}\footnote{Beachte, dass das Wort „Prädikat“ in der Logik eine andere Bedeutung als in der Grammatik, wo es das Verb in einem Satz bezeichnet, trägt. Es handelt sich also um ein Homonym, d.h. ein Wort, das mehrere Bedeutungen, die nichts miteinander zu tun haben, hat.} ist ein sprachliches Gebilde, in dem $n$-viele Variablen vorkommen und das zu einer Aussage wird, wenn für jede dieser Variablen ein konkretes Objekt eingesetzt wird. \\
 $1$-stellige Prädikate nennt man auch \textbf{Eigenschaften}. Sprechen Mathematiker nur von „Prädikaten“, so meinen sie damit in der Regel einstellige Prädikate. \\
 Ist $n\geq 2$, so spricht man auch von \textbf{$n$-stelligen Relationen}. Sprechen Mathematiker einfach nur von „Relationen“, so meinen sie damit in der Regel zweistellige Relationen. Drei- oder höherstellige Relationen tauchen in der Mathematik selten auf.
\end{de}


\begin{bem}
In der Notation werden die Variablen eines Prädikats manchmal mitgeschrieben: ist $E$ ein $n$-stelliges Prädikat, das die Variablen $x_1,\dots , x_n$ enthält, so schreibt man auch „$E(x_1,\dots , x_n)$“ um zu betonen, dass die Variablen von $E$ genau $x_1,\dots , x_n$ sind. Dies ist ähnlich zur Funktionen-Notation in der Schule: ist $f$ eine Funktion in der Variablen $x$, so notiert man diese Funktion in der Schule auch als „$f(x)$“.% (In der Uni-Mathematik ist diese Schreibweise eher rar, da dort „$f(x)$“ meistens nicht die Funktion $f$, sondern den Funktionswert von $f$ an der Stelle $x$ bezeichnet)
\end{bem}


\begin{bsp}
Beispiele für einstellige Prädikate, also für Eigenschaften, sind etwa:
 \begin{enumerate}[a)]
  \item $E(m):\Leftrightarrow$ „$m$ ist eine gerade Zahl“, wobei $m$ eine Variable vom Typ „natürliche Zahl“ sei. Setzt man hier für die Variable $m$ beispielsweise die konkreten Zahlen $4$ und $5$ ein, erhält man die Aussagen „$4$ ist eine gerade Zahl“ bzw. „$5$ ist eine gerade Zahl“.
  \item $D(X):\Leftrightarrow$ „$X$ wurde in Deutschland erfunden“, wobei sich die Variable $X$ auf kulinarische Errungenschaften beziehen soll. Setzt man hier für die Variable $X$ das Objekt „Der Döner“ ein, erhält man gerade die Aussage „Der Döner wurde in Deutschland erfunden“. Setzt man dagegen das Objekt „Die Pizza“ ein, erhielte man die Aussage „Die Pizza wurde in Deutschland erfunden“.
  \item $M(x):\Leftrightarrow $ „$x$ ist der größte Mathematiker“, wobei die Variable $x$ vom Typ „MathematikerIn“ sei. Setzt man hier für die Variable $x$ z.B. das Objekt „Alexander Grothendieck“ ein, erhält man die Aussage „Alexander Grothendieck ist der größte Mathematiker“. Dagegen ergäbe es keinen Sinn, für $x$ das Objekt „Der Döner“ einzusetzen.
 \end{enumerate}
 Wir haben hier, um eine Eigenschaft mit einem Buchstaben zu bezeichnen, nicht das Symbol „$:=$“ sondern das Symbol „$:\Leftrightarrow$“ (lies: „ist per Definition äquivalent zu“ oder „soll per Definition bedeuten, dass“) verwendet. Bei der Definition von Aussagen und Prädikaten kommt das schonmal vor, man könnte aber genausogut auch immer „$:=$“ verwenden. Ist Geschmackssache.
\end{bsp}


%\begin{de}[Diskursuniversum]
%Oftmals arbeitet man mit mehreren Variablen, die sich allesamt auf dieselbe Sorte von Objekten beziehen. In diesem Fall kann man die Gesamtheit all dieser Objekte als \textbf{Diskursuniversum} festlegen.
%\end{de}



\begin{bsp}
 Zweistellige Prädikate sind zum Beispiel:
 \begin{enumerate}[a)]
  \item „$x$ ist kleiner als $y$“, wobei $x,y$ zwei Variablen vom Typ „reelle Zahl“ seien. Diese Relation lässt sich auch kompakt als Formel „$x<y$“ notieren.
  \item $A(X,Y)=$ „$X$ ist älter als $Y$“, wobei für die Variablen konkrete Menschen eingesetzt werden sollen.
  \item $L(X,Y)=$ „$X$ liebt $Y$“, wobei die Variablen vom Typ „Figur aus Jane Austens `Stolz und Vorurteil'“ seien.
  \item „$n$ lässt bei der Division durch $3$ denselben Rest wie $m$ übrig“, wobei für $n$ und $m$ jeweils ganze Zahlen eingesetzt werden sollen. Mathematiker schreiben auch kurz
  \[ n\equiv m\mod 3 \]
  und sagen „$n$ ist kongruent zu $m$ modulo $3$“.
 \end{enumerate}
\end{bsp}





\subsection{Quantoren}
Mengen werden das zentrale Thema im dritten Vortrag sein. Daher wird an dieser Stelle nur das Allernötigste eingeführt, um den Umgang mit Quantoren bequem zu machen.%\footnote{Ein ausführliches Lehrbuch der Mengenlehre mit vielen historischen Anmerkungen ist die „\href{https://www.aleph1.info/?call=Puc&permalink=mengenlehre1}{Einführung in die Mengenlehre}“ von Oliver Deiser.}


\begin{de} \label{mengenimlogikkapitel}
 Eine \textbf{Menge}\footnote{Was in diesem Vortrag „Menge“ genannt wird, würde in der formalen Mengelehre eher als „Klasse“ bezeichnet werden.} ist eine Gesamtheit von Dingen und als solche selbst wiederum ein Gegenstand des Denkens. Sie ist allein dadurch bestimmt, welche Dinge ihr angehören und welche nicht. Diejenigen Objekte, die einer Menge angehören, werden ihre \textbf{Elemente} genannt. Für ein Objekt $a$ und eine Menge $M$ ergibt es nur Sinn, zu fragen ob $a$ ein Element von $M$ ist oder nicht -- dagegen ergäbe es keinen Sinn, danach zu fragen, „auf welche Weise“ oder „wie oft“ $a$ ein Element von $M$ wäre.
\end{de}


\begin{de}[Extension einer Eigenschaft] \label{extension}
 Sei $E$ ein einstelliges Prädikat. Dann wird mit
 \[ \{ x\mid E(x) \} \qquad (\text{lies: „Menge aller $x$, für die gilt: $E(x)$“})\]
 die Menge all derjenigen Objekte, die die Eigenschaft $E$ besitzen, bezeichnet. Sie heißt die \textbf{Extension} (oder auch „Umfang“ oder „Ausdehnung“) des Prädikats $E$. Manche Autoren schreiben anstelle des Querstrichs $\vert$ auch einen Doppelpunkt:
 \[ \{x:\ E(x) \}\]
\end{de}



\begin{bsp}
 Beispielsweise sind
 \begin{itemize}
  \item $\{M\mid M\ \text{ist ein Mensch}\}$ die Menge aller Menschen.
  \item $\Zz:= \{n\mid n\ \text{ist eine ganze Zahl}\}$ die Menge der ganzen Zahlen.
 \end{itemize}
\end{bsp}


\begin{de}[Elementzeichen]
 Sind $M$ eine Menge und $a$ ein Objekt, so schreibt man
 \begin{align*}
  a\in M\qquad&:\Leftrightarrow\qquad a\ \text{ist ein Element von}\ M \\
    a\notin M\qquad&:\Leftrightarrow\qquad a\ \text{ist kein Element von}\ M
 \end{align*}
Insbesondere gilt für jedes Objekt $a$ und jedes Prädikat $E$:
\[ E(a) \qquad\leftrightarrow\qquad a\in \{x\mid E(x)\} \]
\end{de}

\begin{bem}[Eigenschaften vs. Mengen]
Vermöge ihrer Extension bestimmt jede Eigenschaft eine Menge. Umgekehrt bestimmt auch jede Menge eine Eigenschaft, nämlich die Eigenschaft, ein Element von ihr zu sein. Auf diese Weise hat man eine wechselseitige Beziehung zwischen Eigenschaften und Mengen.
\end{bem}




\begin{de}[Diskursuniversum]
 Um Quantoren sinnvoll verwenden zu können, muss vorher ein \textbf{Diskursuniversum} festgelegt werden. Dies ist eine Menge, auf deren Elemente sich die Quantoren beziehen sollen. Jede beliebige Menge kann als Diskursuniversum gewählt werden. Auch die Gesamtheit \emph{aller} Objekte kann als Diskursuniversum gewählt werden.
\end{de}



\begin{de}[Allaussage]
 Es sei $E(x)$ eine Eigenschaft. Dann lässt sich die \textbf{Allaussage}
 \begin{align*}
    \forall x&:\ E(x) && (\text{lies: „Für jedes $x$ gilt $E(x)$“})
 \end{align*}
bilden, deren Bedeutung ist, dass \emph{jedes} Objekt aus dem Diskursuniversum die Eigenschaft $E$ besitzt. Beispiele:
\begin{itemize}
 \item Sind das Diskursuniversum die Menge der Bewohner meiner WG und $A(m):\Leftrightarrow$ „$m$ ist heute früh aufgestanden“, so besagt $\forall m:\ A(m)$, dass jeder in meiner WG heute früh aufgestanden ist.
 \item Sind das Diskursuniversum die Menge der Primzahlen und $U(p):\Leftrightarrow$ „$p$ ist eine ungerade Zahl“, so bezeichnet $\forall p:\ U(p)$ die (falsche) Aussage, dass jede Primzahl eine ungerade Zahl ist.
\end{itemize}
Das Zeichen $\forall$ heißt \textbf{Allquantor}.
\end{de}


\begin{de}[Existenzaussage]
  Es sei $E(x)$ eine Eigenschaft. Dann lässt sich die \textbf{Existenzaussage}
 \begin{align*}
    \exists x&:\ E(x) && (\text{lies: „Es gibt ein $x$, für das $E(x)$ gilt“})
 \end{align*}
bilden, deren Bedeutung ist, dass \emph{mindestens ein} Objekt aus dem Diskursuniversum die Eigenschaft $E$ besitzt. Beispiele:
\begin{itemize}
 \item Sind das Diskursuniversum die Menge der Bewohner meiner WG und $A(m):\Leftrightarrow$ „$m$ ist heute früh aufgestanden“, so besagt $\exists m:\ A(m)$, dass mindestens einer in meiner WG heute früh aufgestanden ist.
 \item Sind das Diskursuniversum die Menge der Primzahlen und $U(p):\Leftrightarrow$ „$p$ ist eine ungerade Zahl“, so bezeichnet $\exists p:\ U(p)$ die (wahre) Aussage, dass es mindestens eine Primzahl gibt, die ungerade ist.
\end{itemize}
Das Zeichen $\forall$ heißt \textbf{Existenzquantor}. \\[0.5em]
Die Negation einer Existenzaussage notiert man mit dem Zeichen $\nexists$. D.h. anstelle von „$\neg (\exists x:\ E(x))$“ schreibt man
\begin{align*}
  \nexists x&:\ E(x) && (\text{lies: „Es existiert kein $x$, für das $E(x)$ gilt“})
 \end{align*}
 Im Beispiel von gerade eben hieße „$\nexists m:\ A(m)$“, dass in meiner WG heute niemand früh aufgestanden ist. \\[0.5em]
 Für den Allquantor gibt es keine dazu analoge Notation. So etwas wie „$\not\kern-0.15em \forall$“ ist meiner Erfahrung nach nicht gebräuchlich.
\end{de}





\begin{bem}[Diskursuniversen sind prinzipiell obsolet] \label{boundedquant}
Sofern nicht anders angegeben, wird als Diskursuniversum zumeist die Gesamtheit \emph{aller} mathematischen Objekte gewählt. Der Bereich von Dingen, der in der momentanen Situation gerade relevant ist, wird dann direkt in die Aussagen integriert. Beispielsweise kann die Aussage „Es gibt eine natürliche Zahl, die kleiner als $3$ ist“ auch durch
\[ \exists x:\quad (x\ \text{ist eine natürliche Zahl}) \land (x<3) \]
formalisiert werden und die Aussage „Jede natürliche Zahl ist kleiner als $3$“ kann durch
\[ \forall x:\quad (x\ \text{ist eine natürliche Zahl}) \to (x<3) \]
formalisiert werden. Hierbei wird dann implizit vorausgesetzt, dass sich die Quantoren auf die Gesamtheit \emph{aller} Objekte beziehen. \\
Sind $M$ irgendeine Menge und $E$ eine Eigenschaft, so schreibt man abkürzend
\begin{align*}
\forall x\in M:\ E(x)\quad&:\Leftrightarrow\quad \forall x:\ (x\in M)\to E(x) && (\text{lies: „Für jedes $x$ aus $M$ gilt $E(x)$“})\\
\exists x\in M:\ E(x)\quad&:\Leftrightarrow\quad \exists x:\ (x\in M)\land E(x) &&  (\text{lies: „Es gibt ein $x$ aus $M$, für das $E(x)$ gilt“})
\end{align*}
Auf diese Weise lässt sich kompakt notieren, dass alle Elemente von $M$ die Eigenschaft $E$ besitzen bzw. dass es mindestens ein Element von $M$ gibt, das die Eigenschaft $E$ besitzt.
\end{bem}


\begin{bsp}[Syllogistik]
Die Prädikatenlogik ist in der Lage, die Satzformen der mittelalterlichen Syllogistik zu formalisieren. Für ein Beispiel seien
 \begin{align*}
  M & := \{x\mid x\ \text{ist ein Mensch} \} \\
  G & := \{g\mid g\ \text{ist ein Gott} \} \\
  H(x) & :\Leftrightarrow\ \text{„$x$ ist ein Grieche“} \\
  S(x) & :\Leftrightarrow\ \text{„$x$ ist sterblich“}
 \end{align*}
Dann ist
\begin{align*}
 \forall x\in M& :\ S(x) && \text{„Alle Menschen sind sterblich“} \\
 \exists x\in M& :\ H(x)&& \text{„Einige Menschen sind Griechen“} \\
 \exists x\in M& :\ \neg H(x) && \text{„Einige Menschen sind keine Griechen“} \\
 \nexists x\in G& :\ S(x)&& \text{„Keine Götter sind sterblich“}
\end{align*}
\end{bsp}



\begin{bem}[* freie Variablen vs. gebundene Variablen]
 Im Ausdruck „$x$ ist eine negative Zahl“ kann für die Variable $x$ eine beliebige reelle Zahl eingesetzt werden. Ebenso z.B. in der Gleichung „$x(x+1)=2$“. Um die Freiheit in der Belegung einer Variable zu betonen, spricht man auch von einer \textbf{freien Variable}. Dagegen ist der Buchstabe „$n$“ im Ausdruck „Für jede gerade natürliche Zahl $n$ ist auch $n^2$ eine gerade Zahl“ oder der Buchstabe „$x$“ in „$\forall x:\ x(x+1)=2$“ keine Variable mehr, da es keinen Sinn ergäbe, für sie ein konkretes Objekt einzusetzen, wie etwa
 \begin{align*}
  \forall 5:\ 5(5+1) = 2 && (\text{dieser Ausdruck ergibt keinen Sinn})
 \end{align*}
 Man sagt, „die Variable wird durch den Quantor gebunden“ und nennt das Zeichen „$x$“ in „$\forall x:\ x(x+1)=2$“ eine \textbf{gebundene Variable}. Es handelt sich nicht mehr um eine Variable im Sinn von \cref{variable}, sondern nur noch um ein „Dummy-Zeichen“, das in der Umgangssprache sogar meistens vermieden werden kann. So würde man den Ausdruck
  \[ \forall x:\ (x\ \text{ist ein Mensch})\to (x\ \text{ist sterblich}) \]
  umgangssprachlich als „Alle Menschen sind sterblich“ lesen und nicht etwa als „Für jedes $x$ gilt: sofern $x$ ein Mensch ist, ist $x$ sterblich“. \\
  Gebundene Variablen kennst du auch schon aus der Schule: Beispielsweise sind die Variablen „$a,b,c,x$“ im Ausdruck „$ax^2 +bx+c$“ jeweils freie Variablen etwa vom Typ „reelle Zahl“, für die jede beliebige reelle Zahl eingesetzt werden kann. Im Ausdruck
  \[ \int_0^1 (ax^2+bx+c)\ dx \]
  ist das Zeichen „$x$“ dagegen eine gebundene Variable, ein „Dummy-Zeichen“ das nur noch deutlich machen soll, über welche Variable integriert wird (man spricht von der „Integrationsvariable“). Es ergäbe keinen Sinn, eine konkrete Zahl einzusetzen wie etwa
  \begin{align*}
  \int_0^1 (a4^2+b4+c)\ d4 && (\text{dieser Ausdruck ergibt keinen Sinn})
   \end{align*}
Dagegen sind hier die Zeichen $a,b,c$ nachwievor freie Variablen, in die Zahlen eingesetzt werden können wie zum Beispiel
\[ \int_0^1 (2x^2+3x+1)\ dx\]
\end{bem}


\begin{bem}
 Bisher wurden die beiden Quantoren „$\forall$“ und „$\exists$“ verwendet, um aus Eigenschaften Aussagen zu erhalten. Allgemein können sie $n$-stellige Prädikate zu $(n-1)$-stelligen Prädikaten machen. Beispielsweise wird das zweistellige Prädikat
 \[ x < y \]
 durch Binden der Variable $x$ zum einstelligen Prädikat
 \[ \forall x:\ x<y \]
Hierbei ist nun $x$ eine gebundene Variable und $y$ eine freie Variable, es liegt also ein einstelliges Prädikat vor. Das kann nun zu einer Aussage gemacht werden, indem man wahlweise für $y$ ein konkretes Objekt einsetzt wie z.B. „$\forall x:\ x < 3$“ oder aber auch $y$ mit einem Quantor bindet wie z.B.
 \[ \exists y:\ \forall x:\ x < y \]
 Ist $n$ eine natürliche Zahl, so kann jedes $n$-stellige Prädikat zu einer Aussage gemacht werden, indem jede der Variablen wahlweise durch ein konkretes Objekt ersetzt oder aber durch einen Quantor gebunden wird.
 \end{bem}
 
 
 \begin{bem}[Schreibkonventionen bei mehreren Quantoren]
 Verwendet man mehrere Quantoren unmittelbar hintereinander, schreibt man den Doppelpunkt oft nur hinter den letzten Quantor:
 \begin{align*}
  \exists y\ \forall x& :\ x < y && (\text{lies: „Es gibt ein $y$, sodass für alle $x$ gilt, dass $x<y$“}) \\
  \forall x\ \exists y& :\ x < y && (\text{lies: „Für jedes $x$ gibt es ein $y$, für das $x<y$ gilt“})  
 \end{align*}
 Manche Autoren lassen die Doppelpunkte hinter Quantoren auch ganz weg.% Kommen mehrere Quantoren derselben Art hintereinander vor, schreibt man oft nur ein Quantorzeichen auf und trennt die gebundenen Variablen durch ein Komma:
%\begin{align*}
% \forall x,y&:\ x<y && (\text{lies: „Für alle $x,y$ gilt $x<y$“}) \\
% \exists x,y&:\ x<y && (\text{lies: „Es gibt $x,y$, für die $x<y$ gilt“}) 
%\end{align*}
 \end{bem}

 
 \begin{bem} \label{quantortausch}
 Beachte, dass es bei Quantoren verschiedener Art auf die Reihenfolge ankommt. Seien etwa das Diskursuniversum die Menge aller Menschen und $M(x,y):\Leftrightarrow$ „$y$ ist Mutter von $x$“. Dann sind
 \begin{itemize}
  \item $\forall x\ \exists y:\ M(x,y)$: „Für jeden Menschen $x$ gilt: es gibt einen Menschen $y$, der Mutter von $x$ ist“.
  \item $\exists y\ \forall y:\ M(x,y)$: „Es gibt einen Menschen $y$, sodass für jeden Menschen $x$ gilt: $y$ ist Mutter von $x$“.
 \end{itemize}
zwei grundlegend verschiedene Aussagen. Die erste Aussage ist wahr, da jeder Mensch eine Mutter hat. Die zweite Aussage aber ist falsch, weil ja nicht alle Menschen Geschwister sind. %Die komplexeste Verschachtelung von Quantoren, die dir im ersten Semester (und möglicherweise sogar im gesamten Studium) begegnen wird, ist das \emph{Stetigkeitskriterium} aus der reellen Analysis: sind das Diskursuniversum der Bereich der reellen Zahlen und $f$ eine Funktion, so heißt $f$ \emph{stetig}, falls gilt:
%\[ \forall p\ \forall \varepsilon:\ ( (\epsilon >0) \to \exists \delta:\ ((\delta >0) \land \forall x:\ (\vert x-p\vert < \delta)\to(\vert f(x)-f(p)\vert < \epsilon)))\]
%Darüber brauchst du dir aber noch nicht den Kopf zerbrechen. In der Mengenlehre und der Analysis wird genug Sprache entwickelt werden, um diese Formel bequemer lesbar zu machen. Außerhalb der mathematischen Logik würde kein Mathematiker das Stetigkeitskriterium so kompliziert aufschreiben, wie es hier steht. Ein erfahrener Mathematiker würde so etwas sagen wie z.B.
%\begin{quote}
% An jedem Punkt $p$ gilt: Für jede Umgebung $V$ von $f(p)$ gibt es eine Umgebung $U$ von $p$, sodass $f(U)\subseteq V$.
%\end{quote}
%Ein nicht unerheblicher Teil der Mathematik besteht ausschließlich darin, mittels Definitionen einen sprachlichen Werkzeugkasten (in diesem Fall die Begriffe „Punkt“, „Umgebung“, „$f(U)$“ und „$\subseteq$“) zu entwickeln, der es erlaubt, komplizierte Aussagen vergleichsweise kompakt und leicht verständlich aufzuschreiben, damit wir leichter darüber nachdenken können.
\end{bem}



\begin{de}[Eindeutigkeitsquantor]
 Das Zeichen $\exists !$ heißt \textbf{Eindeutigkeitsquantor}. Ist $E(x)$ eine Eigenschaft, so bezeichnet
 \begin{align*}
   \exists ! x& :\ E(x) && (\text{lies: „Es gibt genau ein $x$, für das $E(x)$ gilt“})
 \end{align*}
 die Aussage, dass es \emph{genau ein} Objekt im Diskursuniversum gibt, dass die Eigenschaft $E$ besitzt. Ist $M$ eine Menge, so besagt die Formel
 \begin{align*}
  \exists ! x\in M :\ E(x)  &&(\text{lies: „Es gibt genau ein $x$ aus $M$, für das $E(x)$ gilt“})
 \end{align*}
dass es genau ein Element von $M$ gibt, das die Eigenschaft $E$ besitzt (außerhalb von $M$ darf es aber auch andere Elemente geben, die die Eigenschaft $E$ besitzen). Beispiele:
 \begin{itemize}
  \item Ist $B$ die Menge der Bewohner meiner WG und $A(m):\Leftrightarrow$ „$m$ ist heute früh aufgestanden“, so besagt $\exists ! m\in B:\ A(m)$, dass genau ein Bewohner meiner WG heute früh aufgestanden ist.
  \item Die Formel „$\exists ! n\in \Nz :\ 32+n = 101$“ bezeichnet die Aussage: „Es gibt genau eine natürliche Zahl $n$, für die $32+n=101$ ist.“
  \item Die Formel „$\exists ! x\in \Rz:\ x^2=3$“ bezeichnet die (falsche) Aussage „Es gibt genau eine reelle Zahl $x$, für die $x^2=3$ gilt“.
 \end{itemize}
\end{de}




\begin{bem}[Zurückführung von $\exists !$ auf die anderen beiden Quantoren] \label{zerlegung des eindeutigkeitsquantors}
 Mithilfe der Gleichheitrelation „$=$“ kann der Eindeutigkeitsquantor aus den anderen beiden Quantoren zusammengesetzt werden:
 \[ \exists ! x:\ E(x)\qquad :\Leftrightarrow\qquad \exists x:\ E(x) \quad \land\quad \forall x\ \forall y:\ (E(x)\land E(y)) \to x=y \]
Dies besagt, dass es mindestens ein Objekt mit der Eigenschaft $E$ gibt und dass je zwei Objekte, die beide die Eigenschaft $E$ besitzen, bereits übereinstimmen müssen. \\
Für eine Menge $M$ kann man definieren:
 \[ \exists ! x\in M:\ E(x)\quad :\Leftrightarrow\quad (\exists x\in M:\ E(x))\ \land\ (\forall x,y\in M:\ (E(x)\land E(y)) \to x=y) \]
\end{bem}


\begin{bem}[Mäßigung in der Verwendung von Formelsprache!]
Nach den ganzen Formeln aus diesem Abschnitt eine \textbf{Warnung}: Einige Mathe-Anfis gelangen zu der Meinung, in der Mathematik käme es darauf an, Aussagen möglichst formelhaft zu notieren und Quantoren und Junktoren möglichst nie in Umgangssprache, sondern so oft wie möglich als Formelzeichen aufzuschreiben. Manche schreiben auch furchtbare Hybride wie „Daher $\exists$ eine Zahl $n$, die ein Teiler von $a$ $\land$ ein Teiler von $b$ ist“ auf. \\
 Widerstehe dieser Idee! Mathematische Texte und Beweise sind zuallererst mal ein Akt der Kommunikation, in dem der Autor / die Autorin dem Leser eine Information übermitteln möchte. Die Effizienz dieser Informationsübermittlung muss für dich immer an erster Stelle stehen. Lass dich nicht von (unter Mathematikern recht verbreiteten) Formel-Neurosen unterwerfen! Die Einführung der Symbole $\neg,\land,\to,\forall,\exists$ usw. geschieht \textbf{nicht}, damit wir ab sofort alles in diesen Zeichen aufschreiben. Sondern sie dient uns dazu, die Strukturen mathematischer Aussagen und Argumente analysieren und in aller Allgemeinheit besprechen und reflektieren zu können.
\end{bem}



\begin{comment}
\subsection{Gleichheit}






\begin{de}[Gleichheit]
 Sofern zwei Objekte $x,y$ identisch sind, schreibt man $x=y$ (lies: „$x$ ist gleich $y$“). Sind $x$ und $y$ nicht identisch, so schreibt man $x\neq y$ (lies: „$x$ ist ungleich $y$“). \\
 %Gleichheit ist ein zweistelliges Prädikat, das eine Sonderrolle in der Prädikatenlogik spielt, da es keine Freiheit in der Interpretation einer Aussage der Form „$a=b$“ gibt. Die Formel $a=b$ ist genau dann als wahr zu interpretieren, wenn die beiden Objekte $a$ und $b$ übereinstimmen.
\end{de}





\begin{bem}[Anzahlen]
Es seien $n$ eine natürliche Zahl und $E(x)$ eine Eigenschaft. Dann besagt die Formel
\begin{align*}
\exists x_1,\dots, x_n :\ & ((E(x_1)\land\dots \land E(x_n)) \\
& \land (x_1\neq x_2) \land \ldots \land (x_1\neq x_n)\\
& \land (x_2\neq x_3)\land\ldots\land (x_2\neq x_n) \\
&\ldots \\
& \land (x_{n-1}\neq x_n)) 
\end{align*}
dass es mindestens $n$-viele paarweise verschiedene Objekte gibt, die die Eigenschaft $E$ erfüllen. Ebenso besagt die Formel
\begin{align*}
\forall x_1,\dots, x_n,x_{n+1} :\   ((E(x_1)\land\dots \land E(x_{n+1})) \to &( (x_1= x_2) \lor \ldots \lor (x_1= x_{n+1})\\
& \lor (x_2= x_3)\lor\ldots\lor (x_2= x_{n+1}) \\
&  \ldots \\
& \lor (x_{n}= x_{n+1})) 
\end{align*}
dass es höchstens $n$-viele Objekte gibt, die die Eigenschaft $E$ besitzen (um genau zu sein besagt sie, dass von je $n+1$-vielen Objekten, die die Eigenschaft $E$ besitzen, mindestens zwei identisch sein müssen). Diese furchtbar langen Formeln werden in Logik-Lehrbüchern manchmal abgekürzt durch
\begin{align*}
 \exists_{\geq n} x:\ E(x) && (\text{Es gibt mindestens $n$-viele})\\
 \exists_{\leq n} x:\ E(x) && (\text{Es gibt höchstens $n$-viele})
\end{align*}
Schließlich besagt die Formel
\[ \exists_n x:\ E(x) \qquad\:\Leftrightarrow\qquad  (\exists_{\geq n} x:\ E(x))\land ( \exists_{\leq n} x:\ E(x))  \]
dass es \emph{genau $n$-viele} Objekte gibt, die die Eigenschaft $E$ besitzen.
\end{bem}
Der Sonderfall $n=1$ ist so wichtig, dass er ein eigenes Symbol bekommt:
\begin{de}[Eindeutigkeitsquantor]
Das Symbol $\exists !$ heißt \textbf{Eindeutigkeitsquantor}. Ist $E(x)$ irgendeine Eigenschaft, so wird mit
 \[ \exists ! x:\ E(x) \]
 die Aussage „Es gibt \emph{genau ein} $x$, für das $E(x)$ gilt“ notiert. Mittels Allquantor, Existenzquantor und Gleichheit kann diese Aussage folgendermaßen zusammengesetzt werden:
 \[ (\exists x:\ E(x) )\land (\forall x,y:\ (E(x)\land E(y))\to x=y)\]
\end{de}




\end{comment}


\section{Zweiwertige Interpretationen}



\subsection{Wahrheitswerte}


\begin{bem}[Bivalenzprinzip] \label{bivalenz}
In der klassischen Aussagenlogik ist eine „natürliche“ Struktur, die die Menge der Wahrheitswerte trägt, die Struktur einer sogenannten „boolschen Algebra“. Im Vorkurs beschränken wir uns auf diejenige boolsche Algebra, die ausschließlich aus den beiden Wahrheitswerten „wahr“ und „falsch“ besteht. Diese Einschränkung nennt man auch das „Prinzip der Zweiwertigkeit“ oder „Bivalenzprinzip“\footnote{Philosophen sprechen auch vom „Satz vom ausgeschlossenen Dritten“. Dieser besitzt in der mathematischen Logik aber eine andere Bedeutung als das Bivalenzprinzip, siehe \cref{tnd}}.
\end{bem}


\begin{comment}
\begin{bem}[* „Konstante“ Aussagen]
In der Aussagenlogik kann es bequem sein, Aussagezeichen einzuführen, die für eine Aussage stehen, die stets wahr oder stets falsch sein sollen:
\begin{itemize}
 \item Mit „$\top$“ (wie englisch ``true'') ist eine Aussage gemeint, die in einem absoluten Sinn immer wahr sein soll.
 \item Mit „$\bot$“ ist eine Aussage gemeint, die in einem absoluten Sinn falsch sein soll, unabhängig davon, wie sie interpretiert wird.
 \end{itemize}
 \end{bem}
 \end{comment}
 



\begin{de}[Interpretation] \label{interpretation}
 Eine (zweiwertige) \textbf{Interpretation} einer Aussage $X$ ist die Zuweisung eines der beiden Wahrheitswerte „wahr“ oder „falsch“ zu $X$. Diese Zuweisung darf allerdings nicht vollkommen frei erfolgen, sondern muss den folgenden Regeln gehorchen:
 \begin{itemize}
 % \item Die Aussage „$\top$“ muss stets als wahr und die Aussage „$\bot$“ muss stets als falsch interpretiert werden.
  \item  Ist $X$ eine Aussage, die sich mittels der Junktoren $\neg,\land,\lor,\to,\leftrightarrow$ aus anderen Aussagen $A,B$, denen ebenfalls ein Wahrheitswert zugewiesen wurde, zusammensetzt, so muss sich der Wahrheitswert von $X$ nach den folgenden Regeln aus den Wahrheitswerten von $A$ und $B$ ergeben:
		\[\begin{tabular}{c|c||c|c|c|c|c}
			$A$ & $B$ & $\neg A$ & $A\land B$ & $A\lor B$ & $A\to B$ & $A\leftrightarrow B$ \\
			\hline
			w&w&f & w & w & w & w\\
			w&f&f & f & w & f & f\\
			f&w&w & f & w & w & f\\
			f&f&w & f & f & w & w\\
		\end{tabular}\]
Diese sogenannte \textbf{Wahrheitstafel} ist folgendermaßen zu lesen: In den beiden linken Spalten sind alle möglichen Kombinationen aufgelistet, wie $A$ und $B$ mit Wahrheitswerten belegt sein können. Für jede solche Kombination muss dann der Wahrheitswert von $\neg A$, $A\land B$, $A\lor B$ etc. aus der jeweiligen Zeile übernommen werden. Beispielsweise darf die Aussage „$A\lor B$“ nur dann als falsch interpretiert werden, wenn sowohl $A$ als auch $B$ als falsch interpretiert wurden; in den anderen drei Fällen, also falls mindestens eine der beiden Aussagen $A,B$ als wahr verstanden wird, muss auch $A\lor B$ als wahr interpretiert werden.
\item Ist $E$ eine Eigenschaft, so ist die Allaussage $\forall x:\ E(x)$ als wahr zu interpretieren, falls für jedes Objekt $a$ aus dem Diskursuniversum die Aussage $E(a)$ als wahr interpretiert ist. Wurde dagegen für ein Objekt $a$ aus dem Diskursuniversum die Aussage $E(a)$ als falsch interpretiert, so ist auch „$\forall x:\ E(x)$“ als falsch zu interpretieren.
\item Ist $E$ eine Eigenschaft, so ist die Existenzaussage $\exists x:\ E(x)$ als wahr zu interpretieren, falls es mindestens ein Objekt $a$ aus dem Diskursuniversum gibt, bei dem die Aussage $E(a)$ als wahr interpretiert ist. Wurde dagegen für jedes Objekt $a$ aus dem Diskursuniversum die Aussage $E(a)$ als falsch interpretiert, so ist auch „$\exists x:\ E(x)$“ als falsch zu interpretieren.
 \end{itemize}
\end{de}



\begin{bsp}
 Seien $A,B,C$ drei Aussagen. Um den Wahrheitswert von
 \[ D:= \quad (A\lor \neg B) \to C\quad \land\quad \neg C\]
 für alle möglichen Interpretationen von $A,B,C$ zu ermitteln, stellt man eine Wahrheitstafel auf, die auf der linken Seite mit allen möglichen Wahrheitswerte-Kombinationen für $A,B,C$ startet und in den rechten Spalten in wachsender Komplexität mit den relevanten Verschachtelungen von $A$, $B$ und $C$ fortfährt, bis in der Spalte ganz rechts die gesuchten Wahrheitswerte stehen:
 \[ \begin{tabular}{c|c|c||c|c|c|c|c}
    $A$ & $B$ & $C$ & $\neg B$ & $A\lor \neg B$ & $(A\lor \neg B)\to C$ & $\neg C$ & $((A\lor \neg B) \to C)\land \neg C$ \\
    \hline
    w & w & w & f & w & w & f & f\\
    w & w & f & f & w & f & w & f\\
    w & f & w & w & w & w & f & f\\
    w & f & f & w & w & f & w & f\\
    f & w & w & f & f & w & f & f\\
    f & w & f & f & f & w & w & w\\
    f & f & w & w & w & w & f & f\\
    f & f & f & w & w & f & w & f
    \end{tabular} \]
Also gibt es nur einen Fall, in dem $D$ eine wahre Aussage ist; nämlich wenn $B$ wahr ist und $A,C$ falsch sind. \\
Schon an diesem Beispiel wird vielleicht deutlich, dass das Aufstellen von Wahrheitstafeln eine einfache, aber auch mechanische Tätigkeit ist, die anfällig für Flüchtigkeitsfehler ist und sehr gut einem Computer überlassen werden kann.
\end{bsp}



\begin{bem}
 Die Wahrheitstafel des Implikationspfeils
 		\[\begin{tabular}{c|c||c}
			$A$ & $B$ &  $A\to B$  \\
			\hline
			w&w&w\\
			w&f&f\\
			f&w&w\\
			f&f&w\\
		\end{tabular}\]
 verwirrt manche Anfänger, da sie nicht das umgangssprachliche Verständnis von „$B$ folgt aus $A$“ wiedergibt. Setzt man beispielsweise
 \begin{itemize}
   \item $A:=$ „Der Döner wurde in Deutschland erfunden“
  \item $B:=$ „$529$ ist eine Quadratzahl.“
 \end{itemize}
so ist $B$ eine wahre Aussage. Egal, ob $A$ nun wahr oder falsch ist, ergibt sich aus der Wahrheitstafel des Implikationspfeils, dass „Sofern der Döner in Deutschland erfunden wurde, ist $529$ eine Quadratzahl“ eine wahre Aussage ist, obwohl $B$ mit $A$ ja gar nichts zu tun hat. \\
Der Implikationspfeil braucht in der Mathematik \textbf{nichts} mit einem kausalen Zusammenhang zu tun zu haben. „$A\to B$“ besagt eher soviel wie „Wenn ich $A$ annehme, kann ich $B$ beweisen“. Im nächsten Vortrag wird ausführlich darauf eingegangen, siehe \cref{direkterbeweis} und \cref{modusponens}. %Und insofern ist „Sofern der Döner in Deutschland erfunden wurde, ist $529$ eine Quadratzahl“ auch korrekt, da ich mit Annahme von $A$ beweisen kann, dass $529$ eine Quadratzahl ist. Beweis: $529=23\cdot 23$. In diesem Beweis musste ich sogar gar nicht auf die Annahme $A$ zurückgreifen.
\end{bem}






\subsection{Tautologien}


\begin{de}
 Eine Aussage heißt
 \begin{itemize}
  \item \textbf{Tautologie} oder auch \textbf{allgemeingültig}, falls sie unter jeder möglichen Interpretation eine wahre Aussage ist.
  \item \textbf{erfüllbar}, falls es mindestens eine Interpretation gibt, unter der sie eine wahre Aussage ist.
  \item \textbf{unerfüllbar}, falls sie unter keiner möglichen Interpretation eine wahre Aussage ist.
 \end{itemize}
 \end{de}


\begin{bsp}
 Es gilt:
 \begin{enumerate}[a)]
  \item Die Aussage „Heute ist Mittwoch“ ist erfüllbar, aber keine Tautologie.
  \item Die Aussage „Genau dann ist heute Mittwoch, wenn heute Mittwoch ist“ ist eine Tautologie.
\item Für beliebige Aussagen $A,B$ sind die Aussagen
\[ A\to (A\lor B) \qquad A\to \neg\neg A \qquad \neg(A\land \neg A) \]
jeweils Tautologien, was mithilfe von Wahrheitstafeln überprüft werden kann. Hier ist eine Wahrheitstafel für die Formel $\neg(A\land\neg A)$:
 		\[\begin{tabular}{c||c|c|c}
			$A$ &  $\neg A$ & $A\land \neg A$ & $\neg(A\land \neg A)$ \\
			\hline
			w&f&f&w\\
			f&w&f&w
		\end{tabular}\]
\item Für beliebige Aussagen $A,B$ sind die Aussagen
\[  \neg(A\to (A\lor B)) \qquad A \leftrightarrow \neg A \qquad A\land \neg A\]
unerfüllbar. Hier ist eine Wahrheitstafel für $A\leftrightarrow \neg A$:
 		\[\begin{tabular}{c||c|c}
			$A$ &  $\neg A$ & $A \leftrightarrow \neg A$ \\
			\hline
			w&f&f\\
			f&w&f
		\end{tabular}\]
 \end{enumerate}
 \end{bsp}
 
 
 
 
 \begin{sat}[Nützliche Tatsachen für Tautologien] \label{tauto}
Seien $A,B$ zwei beliebige Aussagen. Dann gilt:
  \begin{enumerate}[a)]
   \item Genau dann ist $A$ unerfüllbar, wenn $\neg A$ eine Tautologie ist.
   \item Genau dann ist $A\leftrightarrow B$ eine Tautologie, wenn die Interpretationen, unter denen $A$ wahr ist, genau dieselben sind, unter denen $B$ wahr ist.
   \item Genau dann ist $A\to B$ eine Tautologie, wenn unter jeder Interpretation, unter der $A$ eine wahre Aussage ist, auch $B$ eine wahre Aussage ist. Diejenigen Interpretationen, unter denen $A$ falsch ist, spielen hierbei keine Rolle.
  \end{enumerate}
 \end{sat}
\begin{bew}
a) Betrachte die Wahrheitstafel der Negation:
\[\begin{tabular}{c||c}
    $A$ &  $\neg A$ \\
			\hline
			w&f\\
			f&w
\end{tabular}\]
Unter einer festen Interpretation ist $\neg A$ genau dann wahr, wenn $A$ falsch ist. Dass $\neg A$ unter allen Interpretationen wahr ist, heißt dann genau, dass $A$ unter allen Interpretationen falsch ist. \\[0.5em]
b) Aus der Wahrheitstafel der Äquivalenz
 		\[\begin{tabular}{c|c||c}
			$A$ &  $B$ & $A \leftrightarrow A$ \\
			\hline
			w&w&w\\
			w&f&f \\
			f & w & f\\
			f & f & w
		\end{tabular}\]
liest man ab, dass $A\leftrightarrow B$ genau dann wahr ist, wenn $A$ und $B$ denselben Wahrheitswert haben. Also ist $A\leftrightarrow B$ genau dann eine Tautologie, wenn $A$ und $B$ unter jeder Interpretation denselben Wahrheitswert haben, was, da wir gemäß dem Bivalenzprinzip nur mit zwei verschiedenen Wahrheitswerten „w“ und „f“ arbeiten, gleichwertig dazu ist, dass $A$ und $B$ unter genau denselben Interpretationen wahr sind. \\[0.5em]
c) Betrachte die Wahrheitstafel der Implikation:
 		\[\begin{tabular}{c|c||c}
			$A$ &  $B$ & $A \to A$ \\
			\hline
			w&w&w\\
			w&f&f \\
			f & w & w\\
			f & f & w
		\end{tabular}\]
Dass $A\to B$ eine Tautologie ist, heißt, dass $A\to B$ unter keiner möglichen Interpretation falsch sein kann. Dies ist äquivalent dazu, dass der Fall, dass $A$ wahr und $B$ falsch ist, niemals auftreten kann. Und das heißt gerade, dass, wann immer $A$ wahr ist, auch $B$ wahr sein muss. \qed
\end{bew}

 
 

\begin{bem}[*]
 Eine große Liste aussagenlogischer Tautologien findest du im Anhang dieses Skripts. Wenn du Lust hast, versuche mal, dir intuitiv für ein paar der Formeln klarzumachen, dass es sich um Tautologien handeln muss. So kannst du ein besseres Verständnis für die Junktoren erwerben.
\end{bem}

 
 

\begin{comment}
\begin{bem}
Besonders erwähnenswerte Tautologien aus dieser Liste sind die folgenden:
\begin{align*}
     A & \leftrightarrow \neg\neg A && (\text{Regel der doppelten Verneinung}) \\[1em]
            \begin{split} ( \neg A \land \neg B) & \leftrightarrow \neg (A \lor B) \\
  (\neg A \lor \neg B) & \leftrightarrow \neg(A \land B) 
  \end{split} && (\text{Regeln von De Morgan}) \\[1em]
    \neg (A\to B) & \leftrightarrow (A \land \neg B) \\
     (A\to B) & \leftrightarrow (\neg A \lor B)
\end{align*}
Die ersten vier Formeln erlauben es, Negationen „in Formeln hineinzuziehen“, ohne ihren Wahrheitswert zu verändern. Au diese Weise kann jede mit Junktoren verschachtelte Aussage bei gleichbleibendem Wahrheitswert so „umgeformt“ werden. Die letzte Formel ermöglicht es Implikationspfeile in Oder-Aussagen „umzuformen“. Hier ist ein Beispiel für eine Umformung der Kontrapositionsformel in eine Formel, die nur noch die Junktoren $\neg$, $\land$ und $\lor$ enthält:
\begin{align*}
 (A\to B)\to (\neg B\to \neg A) & \Leftrightarrow  (\neg A\lor B) \to (\neg\neg B\lor\neg A) \\
 & \Leftrightarrow \neg(\neg A\lor B) \lor (B\lor \neg A) \\
 & \Leftrightarrow (\neg \neg A\land \neg B) \lor (B\lor \neg A) \\
 & \Leftrightarrow (A\land \neg V) \lor B\lor \neg A
\end{align*}
\end{bem}




\begin{bem}[Einige Tautologien der Gleichheit]
Es gilt:
 \begin{itemize}
  \item Für jedes beliebige Objekt $x$ gilt $x=x$. Mit anderen Worten: die Aussage $\forall x:\ x=x$ ist eine Tautologie.
  \item Sei $E$ irgendeine Eigenschaft. Sind $a,b$ zwei Objekte und gilt $a=b$, so gilt genau dann $E(a)$, wenn $E(b)$ gilt. Mit anderen Worten: gleiche Objekte besitzen dieselben Eigenschaften bzw. für jede beliebige Eigenschaft $E$ ist die Formel
  \[ \forall a,b:\ (a=b) \to (E(a)\leftrightarrow E(b)) \]
  eine Tautologie.
  \item Die Idee, diese zweite Tatsache zu einer \emph{Definition} der Gleichheit zu erheben; also zu sagen: Zwei Objekte $a,b$ sind \emph{per Definition} gleich, wenn sie genau dieselben Eigenschaften erfüllen, heißt auf latein „Principium identitatis indiscernibilium“, also soviel wie „Identität des Ununterscheidbaren“.
 \end{itemize}
\end{bem}
\end{comment}

 \begin{bem}[* Entscheidbarkeit der Aussagenlogik] \label{entscheidbar}
  Ist $A$ eine noch so kompliziert verschachtelte Aussage, die keine Prädikate und Quantoren enthält, sondern sich ausschließlich mittels der Junktoren aus anderen Aussagen zusammensetzt, so lässt sich mithilfe von Wahrheitstafeln stets überprüfen ob $A$ eine Tautologie ist. Mit genügend Rechenkapazität kann mir mein Computer also einfach ausrechnen, ob eine Tautologie vorliegt oder nicht. Man sagt, die Aussagenlogik sei (algorithmisch) \textbf{entscheidbar}. Sobald Quantoren ins Spiel kommen, reichen Wahrheitstafeln aber nicht mehr aus: in der mathematischen Logik wird bewiesen, dass es keinen Algorithmus gibt, der für eine beliebige, sich mittels Junktoren und Quantoren aus Prädikaten und Aussagen zusammensetzende Aussage entscheiden kann, ob es sich um eine Tautologie handelt oder nicht. Man spricht von der \textbf{Unentscheidbarkeit der Prädikatenlogik}. Das wäre auch zu schön, denn ein solcher Algorithmus wäre ein „mathematisches Orakel“, das für jede mathematische Aussage ausrechnen könnte, ob sie stets wahr ist oder nicht. Gäbe es so etwas, wären die meisten Mathematiker auf einen Schlag arbeitslos. \\
  Für ein Beispiel seien das Diskursuniversum die natürlichen Zahlen und $E(n)$ das Prädikat „Wenn $n$ eine gerade Zahl und größer als drei ist, dann lässt sich $n$ als Summe zweier Primzahlen schreiben“. Die Aussage $\forall n:\ E(n)$ heißt \href{https://de.wikipedia.org/wiki/Goldbachsche_Vermutung}{\emph{Goldbachsche Vermutung}} und konnte bislang weder bewiesen noch widerlegt werden. Sie lässt sich als eine Art „unendliche Konjunktion“
  \[E(1)\land  E(2)\land E(3)\land E(4)\land E(5) \land \dots \]
  auffassen. Solche „unendlichen Konjunktionen“ entziehen sich aber aussagenlogischen Methoden und sind nicht mehr mit Wahrheitstafeln beherrschbar. Würde man die Konjunktionenkette ab einer bestimmten Zahl abbrechen
  \[ E(1)\land E(2)\land\dots \land E(10^{30}) \]
  so befände sich alles im Rahmen der Aussagenlogik und man müsste nur nacheinander prüfen, ob jede der Zahlen $1,\dots , 10^{30}$ die Eigenschaft $E$ besitzt (was mithilfe von Hochleistungsrechnern tatsächlich verifiziert werden konnte). Aber das reicht ja nicht aus, um die Aussage für \emph{alle} natürlichen Zahlen zu verifizieren. Dafür bräuchte der Rechner unendlich viel Zeit. \\
  Da es kein Patenzrezept gibt, für eine allgemeine prädikatenlogisch aufgebaute Aussage zu entscheiden, ob sie eine Tautologie ist oder nicht, müssen andere Techniken verwendet werden, um zumindest für einige Aussagen beurteilen zu können, ob sie als „wahre“ mathematische Aussagen gehandelt werden können. Dafür sind die Beweismethoden, die im nächsten Vortrag behandelt werden, da.
 \end{bem}










% --------------------------------------------------------------------
% §5 Section <<Übungsaufgaben>>
% --------------------------------------------------------------------
\newpage
\section{Aufgabenvorschläge}

% --------------------------------------------------------------------
% §5.1 Subsection <<Aufgabe 1>>
% --------------------------------------------------------------------



\begin{aufg}[Umgangssprache in Formeln übersetzen]
Zerlegt die folgenden Aussagen mithilfe der im Vortrag behandelten Junktoren und Quantoren in möglichst einfache Grundbausteine.
\begin{enumerate}[a)]
\item Wenn ich entweder alle Prüfungen im ersten Versuch bestehe oder aber durch alle Prüfungen im ersten Versuch durchfalle, werde ich die ganze Nacht hindurch feiern.
 \item Sofern er morgen Abend weder arbeiten muss noch Besuch von seiner Familie kriegt, würde er sich mit mir treffen.
  \item Alle reellen Lösungen der Ungleichung $x^3-3x<3$ sind kleiner als $10$.
 \item Nobody’s perfect.
 \item Wenn es irgendjemand schafft, dann Henrik.
 \item Wenn Henrik es schafft, dann schafft es jeder.
 \item Eine natürliche Zahl $p\neq 1$ ist genau dann eine Primzahl, falls gilt: für alle natürlichen Zahlen $a,b$ mit $p=ab$ ist $a=1$ oder $b=1$.
\end{enumerate}
\end{aufg}





\begin{aufg}[Formeln in Umgangssprache übersetzen]
Übersetzt die folgenden Aussagenformeln in Umgangssprache und beurteilt, ob es sich um wahre oder falsche Aussagen handelt:
	\begin{enumerate}[a)]
	\item $\exists x\in \Rz\ \exists y\in \Zz:\ x<y$
	\item $\forall x\in \Rz\ \forall y\in \Zz:\ x<y$
		\item $\forall x\in \Rz\ \exists y\in \Zz:\ x<y$
		\item $\exists y\in \Zz\ \forall x\in \Rz:\ x<y$
		\item $\forall x\in \Rz:\ (x^2 -x=0\leftrightarrow(x=1\vee x=0))$
		\item $\exists! x\in \Rz\ \exists y\in \Rz:\ (y\neq0\ \land\ x\cdot y=0)$
	\end{enumerate}
\end{aufg}


	
	
\begin{aufg}[Wahrheitstafeln]
Seien $A,B,C$ drei beliebige Aussagen. Entscheidet mithilfe von Wahrheitstafeln, unter welchen Umständen die folgenden Aussagen wahr sind:
\begin{align*}
%\text{b)} && \neg A & \to A\\
\text{a)} && A\to B\quad & \leftrightarrow \quad A\land \neg B \\
\text{b)} && A\to (B\to C) \quad&\leftrightarrow\quad (A\land B)\to C \\
\text{c)} && A\to B \quad&\leftrightarrow\quad \neg A \to \neg B
%\text{d)} && (A\to B)&\lor (B\to C)\\
\end{align*}
\end{aufg}





	
\begin{comment}
\subsection{Aufgabe}
Seien $A,B,C$ drei beliebige Aussagen. Zeigt mithilfe einer Wahrheitstafel, dass die Formel
\[ (A\to B)\lor (B\to C) \]
eine Tautologie ist. Seien nun konkret
\begin{itemize}
 \item $A:=$ „Der Döner wurde in Deutschland erfunden“.
 \item $B:=$ „Heute ist Mittwoch“.
\end{itemize}
Wie muss man „$\to$“ und „$\lor$“ auffassen, damit die Tatsache, dass $(A\to B)\lor (B\to A)$ eine Tautologie ist, kein Nonsens ist? \\%Ergibt so auch die Tatsache, dass $(B\to \neg B)\lor (\neg B\to B)$ eine Tautologie ist, Sinn? \\
Eine Liste weiterer Tautologien des Implikationspfeils, die kontraintuitiv wirken können, findet ihr \href{https://de.wikipedia.org/wiki/Paradoxien_der_materialen_Implikation}{auf der deutschen Wikipedia}. Der nächste Vortrag wird euch dabei helfen, sie besser zu verstehen.
\end{comment}


\begin{aufg}[Seltsame Formeln]
An der Tafel von Captain Chaos stehen die folgenden Ausdrücke:
\begin{align*}
(i)\quad& A\ {\neg}{\to}\ \neg A&(ii)\quad& \exists x\in x:\ x(x)  \\
 (iii)\quad&   \forall n\in \Nz:\ 1<3 & (iv)\quad& \forall E :\ E(x) \\
  (v)\quad& \forall x\in \Rz :\ x &(vi)\quad& \forall x_1\ \exists x_2\ \forall x_3\ \exists x_4\ \forall x_5\ \ldots:\ E(x_1,x_2,x_3,\dots)
\end{align*}
Was haltet ihr davon?
\end{aufg}
	
	
	

\begin{comment}
\subsection{Aufgabe}

Seine A,B,C Aussagen. Beweise mittels Wahrheitstafeln und bekannter Sätze:
\begin{enumerate}
	\item DeMorgan'sche Regeln:
	\begin{enumerate}
		\item $\neg(A\wedge B)\Leftrightarrow(\neg A\vee \neg B)$
		\item $\neg(A\vee B)\Leftrightarrow(\neg A\wedge \neg B)$
	\end{enumerate}
	\item Distributivgesetze:
	\begin{enumerate}
		\item $(A\wedge(B\vee C))\Leftrightarrow((A\wedge B)\vee(A\wedge C))$
	\end{enumerate}
\end{enumerate}

% --------------------------------------------------------------------
% §5.3 Subsection <<Aufgabe 3>>
% --------------------------------------------------------------------
\subsection{Aufgabe}

\begin{enumerate}
	\item Formalisiere die folgende Aussage: (In einzelne Aussagen unterteilen und daraus ein Aussagengebilde mit Operatoren erstellen) \\
	"Wenn ein hartgekochtes Ei nicht mit kaltem Wasser abgeschreckt wird, dann klebt die Schale am Eiweiß und das Ei lasst sich nicht gut schälen."
	\item Andreas, Benedikt, Carolin und Dora sind auf eine Party eingeladen. Folgendes ist bekannt:
	\begin{enumerate}
		\item Wenn Andreas geht, dann geht auch Benedikt
		\item Carolin und Dora gehen nicht beide
		\item Von Andreas und Dora geht mindestens einer
		\item Wenn Benedikt oder Dora geht, dann geht auch Carolin
	\end{enumerate}
	Wer geht auf die Party?
\end{enumerate}

% --------------------------------------------------------------------
% §5.4 Subsection <<Aufgabe 4>>
% --------------------------------------------------------------------
\subsection{Aufgabe}

\begin{enumerate}
	\item Negiere die folgenden Aussagen A:
	\begin{enumerate}
		\item A = $\forall x\in\mathbb{R}:\exists y\in\mathbb{R}:\: x < y^2 $
		\item A = $\forall x,y\in\mathbb{R}:\forall\epsilon>0: \exists\delta>0: (|x-y|<\delta \Rightarrow |g(x)-g(y)|<\epsilon)$
		\item Es gibt eine Universität, an der es keinen Studenten gibt, der Spaß am Negieren von Aussagen hat.
	\end{enumerate}
	\item Entscheide, ob die folgenden Aussagen wahr oder falsch sind oder ob weitere Informationen benötigt werden:
	\begin{enumerate}
		\item $\forall x\in\mathbb{R}:\exists y\in\mathbb{R}: x<y$
		\item $\exists y\in\mathbb{R}: \forall x\in\mathbb{R}: x<y$
		\item $\forall x \in\mathbb{R}:(x^2 -x=0\Rightarrow(x=1\vee x=0))$
		\item $\exists x\in\mathbb{N}: (x\neq0\wedge(\forall y\in \mathbb{N}: (x\cdot y < x+y)))$
	\end{enumerate}
\end{enumerate}
\end{comment}

% --------------------------------------------------------------------
% §6 Section <<Lösungen>>
% --------------------------------------------------------------------
\begin{comment}
\section{Lösungen}

% --------------------------------------------------------------------
% §6.1 Subsection <<Aufgabe 1>>
% --------------------------------------------------------------------
\subsection{Aufgabe}

\begin{enumerate}
	\item \begin{tabular}{c|c||c}
		A & $\neg(\neg$ A) &$ \neg(\neg$ A) $\Leftrightarrow$ A \\
		\hline
		w & w & w\\
		f & f & w \\
	\end{tabular}
	\item \begin{tabular}{c||c|c||c|c}
		A & A$\wedge$ w & (A$\wedge$w)$\Leftrightarrow$ A & A$\wedge$f & (A$\wedge$ f) $\Leftrightarrow$ f \\
		\hline
		w & w & w & f & w \\
		f & f & w & w & w \\
	\end{tabular}
	\item \begin{tabular}{c||c|c||c|c}
		A & A$\vee$ w & (A$\vee$w)$\Leftrightarrow$ w & A$\vee$f & (A$\vee$ f) $\Leftrightarrow$ A \\
		\hline
		w & w & w & w & w \\
		f & w & w & f & w \\
	\end{tabular}
	\item \begin{tabular}{c|c||c|c|c}
		A & B & A$\wedge$B & B$\wedge$A & (A$\wedge$B)$\Leftrightarrow$(B$\wedge$A) \\
		\hline
		w & w & w & w & w \\
		w & f & f & f & w \\
		f & w & f & f & w \\
		f & f & f & f & w \\
	\end{tabular}
	\item \begin{tabular}{c|c|c||c|c|c|c|c}
		A & B & C & B$\wedge$C & A$\wedge$(B$\wedge$C) & A$\wedge$B & (A$\wedge$B)$\wedge$C & A$\wedge$(B$\wedge$C) $\Leftrightarrow$ (A$\wedge$B)$\wedge$C \\
		\hline
		w & w & w & w & w & w & w & w \\
		w & w & f & f & f & w & f & w \\
		w & f & w & f & f & f & f & w \\
		w & f & f & f & f & f & f & w \\
		f & w & w & w & f & f & f & w \\
		f & w & f & f & f & f & f & w \\
		f & f & w & f & f & f & f & w \\
		f & f & f & f & f & f & f & w \\
	\end{tabular}
\end{enumerate}

% --------------------------------------------------------------------
% §6.2 Subsection <<Aufgabe 2>>
% --------------------------------------------------------------------
\subsection{Aufgabe}

\begin{enumerate}
	\item \begin{enumerate}
		\item \begin{tabular}{c|c||c|c|c}
			A & B & A$\wedge$B & $\neg$A$\vee\neg$B & $\neg$(A$\wedge$B) $\Leftrightarrow$ ($\neg$A$\vee\neg$B) \\
			\hline
			w & w & w & f & w \\
			w & f & f & w & w \\
			f & w & f & w & w \\
			f & f & f & w & w \\
		\end{tabular}
		\item \begin{tabular}{c|c||c|c|c}
			A & B & A$\vee$B & $\neg$A$\wedge\neg$B & $\neg$(A$\vee$B) $\Leftrightarrow$ ($\neg$A$\wedge\neg$B) \\
			\hline
			w & w & w & f & w \\
			w & f & w & f & w \\
			f & w & w & f & w \\
			f & f & f & w & w \\
		\end{tabular}
	\end{enumerate}
	\item \begin{tabular}{c|c|c||c|c|c|c|c|c}
		A & B & C & B$\vee$C & A$\wedge$(B$\vee$C) & A$\wedge$B & A$\wedge$C & (A$\wedge$B)$\vee$(A$\wedge$C) & (A$\wedge$(B$\vee$C))$\Leftrightarrow$((A$\wedge$B)$\vee$(A$\wedge$C)) \\
		\hline
		w & w & w & w & w & w & w & w & w \\
		w & w & f & w & w & w & f & w & w \\
		w & f & w & w & w & f & w & w & w \\
		w & f & f & f & f & f & f & f & w \\
		f & w & w & w & f & f & f & f & w \\
		f & w & f & w & f & f & f & f & w \\
		f & f & w & w & f & f & f & f & w\\
		f & f & f & f & f & f & f & f & w \\
	\end{tabular}
\end{enumerate}

% --------------------------------------------------------------------
% §6.3 Subsection <<Aufgabe 3>>
% --------------------------------------------------------------------
\subsection{Aufgabe}

\begin{enumerate}
	\item Definiere die Aussagen:
	\begin{itemize}
		\item A = “Ein hartgekochtes Ei wird mit kaltem Wasser abgeschreckt.”,
		\item B = “Die Schale klebt am Eiweiß.”,
		\item C = “Das Ei laßt sich gut schälen.”
	\end{itemize}
	Erhalte die Formalisierung: $(\neg A)\Rightarrow (B\wedge (\neg C))$
	\item In Grundaussagen einteilen: A: Andreas geht; B: Benedikt geht; C: Carolin geht; D: Dora geht. \\
	Laut Aufgabe muss gelten:
	\begin{enumerate}
		\item $A\Rightarrow B$ muss wahr sein.
		\item $C\wedge D$ muss falsch sein.
		\item $A\vee D$ muss wahr sein.
		\item $B\vee D\Rightarrow C$ muss wahr sein.
	\end{enumerate}
	\textbf{Fall 1:} Andreas geht, d.h. A wahr $\stackrel{a)}{\Rightarrow}$ B wahr $\stackrel{d)}{\Rightarrow}$ C wahr $\stackrel{b)}{\Rightarrow}$ D falsch \\
	D.h. wenn Andreas geht, gehen insgesamt Andreas, Benedikt und Carolin, nur Dora geht dann nicht. \\
	\textbf{Fall 2:} Andreas geht nicht, führt zu Widerspruch: A falsch $\stackrel{c)}{\Rightarrow}$ D wahr $\stackrel{b)}{\Rightarrow}$ C falsch, aus D wahr ergibt sich aber mit (d) auch C wahr, d.h. Widerspruch, d.h. erste Annahme ergibt einzige Lösung. \\
	Insgesamt also: Andreas, Benedikt und Carolin gehen zur Party. Dora geht nicht.
\end{enumerate}

% --------------------------------------------------------------------
% §6.4 Subsection <<Aufgabe 4>>
% --------------------------------------------------------------------
\subsection{Aufgabe}

\begin{enumerate}
	\item \begin{enumerate}
		\item $\neg$A = $\exists x\in\mathbb{R}:\forall y\in\mathbb{R}: x \geq y^2$
		\item Sei B(x,y) = $|x-y|<\delta$, C(x,y) = $|g(x)-g(y)|<\epsilon$. Verneine nun Schritt für Schritt: 
		\begin{align*}
			\neg(\forall x,y\in\mathbb{R}:\forall\epsilon>0: \exists\delta>0: (A(x,y) &\Rightarrow B(x,y)) \\
			\Leftrightarrow \exists x,y\in\mathbb{R}:\neg(\forall\epsilon>0: \exists\delta>0: (A(x,y) &\Rightarrow B(x,y)) \\
			\Leftrightarrow \exists x,y\in\mathbb{R}:\exists\epsilon>0: \neg(\exists\delta>0: (A(x,y) &\Rightarrow B(x,y)) \\
			\Leftrightarrow \exists x,y\in\mathbb{R}: \exists\epsilon>0: \forall\delta>0: \neg((A(x,y) &\Rightarrow B(x,y)) \\
		\end{align*}
		Nun müssen wir noch $\neg( A(x,y)\Rightarrow B(x,y))$ umformulieren: \\
		$\neg( A(x,y)\Rightarrow B(x,y) \Leftrightarrow A(x,y)\wedge\neg B(x,y)$ (Mit Wahrheitstafel beweisen). \\
		$\neg B(x,y) \Leftrightarrow |g(x)-g(y)|\geq\epsilon$ \\
		Also: \[\neg A= \exists x,y\in\mathbb{R}: \exists\epsilon>0: \forall\delta>0: (|x-y|<\delta \wedge |g(x)-g(y)|\geq \epsilon\]
		\item An allen Universitäten gibt es einen Studenten, der keinen Spaß am Negieren von Aussagen hat.
	\end{enumerate}
	\item \begin{enumerate}
		\item wahr.
		\item falsch, denn a) gilt.
		\item wahr, denn: $x^2 - x = x\cdot(x-1) = 0 \Leftrightarrow x=0 \vee x=1$
		\item wahr, $x=1$ erfüllt die Aussage.
	\end{enumerate}
\end{enumerate}


%%% Local Variables:
%%% mode: latex
%%% TeX-master: "Skript"
%%% End:




\begin{comment}
\section{Einige Aussagenlogische Beweistechniken}






\begin{de}[Satz/Beweis]
		Ein \swdde{Satz} ist eine Aussage, die überprüfbar immer wahr ist. Ein \swdde{Beweis} eines Satzes ist eine logische Herleitung dieser Wahrheits-Aussage aus Axiomen und Sätzen.
	\deEnd
	\end{de}
	
	\noindent Andere Bezeichnungen für einen Satz sind auch (je nach Kontext): Lemma, Korollar, Proposition, etc. \\
	Anstatt \anf{Die Aussage ist immer wahr} verwenden wir auch \anf{Die Aussage gilt.} Als Beispiel wollen wir nun den \anf{Satz vom Widerspruch} beweisen. Dies machen wir mit Hilfe von Wahrheitstafeln, d.h. wir gehen alle logischen Möglichkeiten durch und zeigen in jedem Fall, dass die Aussage des Satzes wahr ist.

%------------------
%
	\begin{sat}[Satz vom Widerspruch]
		Für eine beliebige Aussage $A$ gilt:
		\[
			\Leftrightarrow(\wedge(A,\neg(A)),f)=(A\wedge\neg A)\Leftrightarrow f
		\]
		Alternative Formulierung: Es gilt $\neg(A\wedge\neg A)$.
	\end{sat}

%------------------
%
	\begin{bew}
		Wir leiten spaltenweise die Wahrheitswerte der in der Kopfzeile der Tafel stehenden Aussagen her:
		\[\begin{tabular}{c||c|c|c}
			$A$ & $\neg A$ & $A\wedge\neg A$ & $(A\wedge\neg A)\Leftrightarrow f$ \\
			\hline
			w&f&f&w\\
			f&w&f&w\\
		\end{tabular}\] \hfill $\Box$
	\bewEnd
	\end{bew}
	
	\noindent Aussgen mit Äquivalenzzeichen (\anf{$\Leftrightarrow$}), die immer wahr sind, können dazu genutzt werden, andere Aussagen umzuformen, was einer vereinfachten Kommunikation dient. Kommt beispielsweise in einer anderen Aussage $A\wedge\neg A$ vor, so können wir stattdessen einfach $f$ schreiben. \\
	Zum Beispiel gilt für Aussagen $A,B$:
	\[
		B\vee(A\wedge\neg A)\Leftrightarrow B\vee f
	\]
	
	\noindent Nun können wir auch den \anf{modus ponens} beweisen:

%------------------
%
	\begin{bew}[zu \anf{modus ponens}]
		\[\begin{tabular}{c|c||c|c|c}
			A & B & $A\Rightarrow B$ & $A\wedge(A\Rightarrow B)$ & $(A\wedge(A\Rightarrow B))\Rightarrow B$\\
			\hline
			w&w&w&w&w\\
			w&f&f&f&w\\
			f&w&w&f&w\\
			f&f&w&f&w\\
		\end{tabular}\] \hfill $\Box$
	\bewEnd
	\end{bew}
	
	\noindent Auf ähnliche Weise kann der \anf{Satz vom ausgeschlossenen Dritten} bewiesen werden, dies bleibt den Studierenden allerdings als Übungsaufgabe überlassen.

%------------------
%
	\begin{sat}[Satz vom ausgeschlossenen Dritten]
		Für eine beliebige Aussage $A$ gilt:
		\[
			A\vee\neg A
		\]
		Alternative Formulierung: Es gilt $(A\vee\neg A)\Leftrightarrow w$.
	\end{sat}

\end{comment}

% --------------------------------------------------------------------
% §3 Section <<Prädikatenlogik erster Stufe>>
% --------------------------------------------------------------------
\begin{comment}
\section{Prädikatenlogik erster Stufe}

% --------------------------------------------------------------------
% §3.1 Subsection <<Quantoren>>
% --------------------------------------------------------------------
\subsection{Quantoren}

	Im Gegensatz zur Aussagenlogik, welche die Zerlegung von Aussagen in nicht weiter teilbare
	Aussagen (sog. Elementaraussagen) untersucht, beschäftigt sich die Prädikatenlogik mit der
	Struktur dieser Elementaraussagen. \\
	Das Problem in der oben formulierten Aussagenlogik liegt darin, dass wir nur Sätze über Aussagen formulieren können, die uns vollständig bekannt sind.

%------------------
%
	\begin{bsp}
		Wir betrachten die Aussagen $A$ = \anf{Steffen darf Alkohol kaufen} und $B$ = \anf{Steffen ist älter als 10 Jahre}. \\
		\\
		Nehmen wir an, dass wir zeigen konnten, dass wenn $A$ wahr ist, auch $B$ wahr ist. Dann haben wir aber nur eine Aussage über die spezifische Person Steffen. Vielleicht können wir zeigen, dass die Implikation für die Person Claudia auch gilt. Aber das müssten wir dann für jede einzelne Person erneut überprüfen. Unser Ziel ist also, diese Implikation möglichst abstrakt zu formulieren. Vielleicht könnten wir die Implikation für alle Personen in Deutschland zeigen und sie auf diese Menschenmenge verallgemeinern. Also stellt sich die Frage, was die größte Menge an Menschen ist, auf die sich diese Aussage erweitern ließe. Wir wollen so etwas beweisen können wie
		\begin{center}
			Alle Leute, die Alkohol kaufen dürfen, sind älter als 10 Jahre.
		\end{center}
	\end{bsp}
	
	\noindent Die Situation von Beispiel 3.1 tritt auch direkt in der Mathematik auf: Oft kommt es vor, dass mathematische Sätze mit Unbekannten formuliert werden, deren Wert erst später (zum Beispiel bei Anwendung des Satzes) bekannt ist. Wir werden nun Aussagen einführen, die auch Variablen erlauben, so genannte Prädikate.

%------------------
%
	\begin{de}[Prädikat]
		Ein \swdde{Prädikat} $A(X_1,...,X_n)$ ist ein sprachliches Gebilde mit Variablen $X_1,...,X_n$, das zu einer Aussage wird, wenn für jede Variable $X_1,...,X_n$ ein konkreter Wert eingesetzt wird.
	\deEnd
	\end{de}

%------------------
%
	\begin{bem}
		Auf Prädikaten können dieselben Operatoren angewandt werden wie auf Aussagen. Beachte aber, dass das Ergebnis dann immernoch ein Prädikat ist und keine Aussage!
	\end{bem}

%------------------
%
	\begin{bsp}
		Definiere
		\begin{center}
			$A(X)$ := \anf{$X$ darf Alkohol kaufen},
		\end{center}
		und
		\begin{center}
			$B(X)$ := \anf{$X$ ist älter als 10 Jahre}.
		\end{center}
		Die Interpretation des Prädikats $A(\text{Steffen})\Rightarrow B(\text{Steffen})$ lautet: \anf{Wenn Steffen Alkohol kaufen darf, dann ist Steffen älter als 10 Jahre.}
	\end{bsp}

%------------------
%
	\begin{de}[Prädikat Elementrelation]
		Für $X$ ein Objekt und $M$ eine beliebige Menge können wir das Prädikat
		\[
			E(X,M) := X\in M
		\]
		definieren. \\
		Interpretation: \anf{$X$ ist Element von $M$}.
	\deEnd
	\end{de}

%------------------
%
	\begin{bsp}
		Die Aussage $E(2,\{1,2,3\}) =$\anf{2 ist Element von \{1,2,3\}} ist wahr. Also gilt $2\in \{1,2,3\}$ (Infix-Notation). \\
		Die Aussage $E(4,\{1,2,3\})$ ist falsch, also gilt $\neg(4\in\{1,2,3\})$.
	\end{bsp}
	
	\noindent Wie können wir aus Prädikaten wieder Aussagen gewinnen? Eine Möglichkeit ist, konkrete Werte für die Variablen einzusetzen. Die resultierenden Aussagen für sich genommen sind aber sehr schwach, damit erreichen wir keine Aussage wie in Bsp.3.1. In der Prädikatenlogik wurden daher \anf{Quantoren} eingeführt. Diese geben formal richtig an, für wie viele Objekte $C$ ein bestimmtes Prädikat gilt und erleichtern somit auch die Formalisierung von Gedanken.

%------------------
%
	\begin{de}[Quantoren: All-Quantor und Existenz-Quantor]
		Die Aussage
		\[
			\forall X:A(X)
		\]		
		ist wahr, wenn für alle Objekte $X$ die Aussage $A(X)$ wahr ist. $\forall$ heißt \swdde{All-Quantor}. \\
		Die Aussage
		\[
			\exists X:A(X)
		\]
		ist wahr, wenn es (mindestens) ein Objekt $X$ gibt, sodass die Aussage $A(X)$ wahr ist. $\exists$ heißt \swdde{Existenz-Quantor}.
	\deEnd
	\end{de}
	
	\noindent Zum Erreichen einer Aussage wie in Bsp.1.3 können wir also auch Quantoren als Werkzeug verwenden. \\
	Die Quantoren alleine reden von allen möglichen Objekten/Variablen $X$, d.h. mit $X$ können alle möglichen Zahlen, Mengen, etc. gemeint sein. Wollen wir unsere Aussagen auf bestimmte Teilmengen einschränken, brauchen wir dazu zusätzliche Prädikate.

%------------------
%
	\begin{bsp}
		Für zwei Prädikate $A(X),B(X)$ ist
		\[
			\forall X:(A(X)\Rightarrow B(X))
		\]
		eine Aussage. \\
		Interpretation: \anf{Für alle $X$ gilt: Wenn $A(X)$ wahr ist, dann ist auch $B(X)$ wahr}.\\
		Für eine feste Menge $M$ und ein Prädikat $A(X)$ sind
		\begin{align}
			\forall X&: (E(X,M)\Rightarrow A(X)) \\
			\exists X&: (E(X,M)\wedge A(X))
		\end{align}
		Aussagen. \\
		Interpretation: (1): \anf{Für alle $X$ aus der Menge $M$ ist $A(X)$ wahr}, bzw. (2): \anf{Es gibt ein $X$ aus der Menge $M$, für welches $A(X)$ wahr ist}. \\
		Für Aussagen der Form (1) und (2) können wir auch die Infix-Notation als Abkürzungen, bzw. einfachere und übersichtlichere Schreibweise, verwenden, die in der Mathematik allgemein gebräuchlich ist:
		\begin{align*}
			\forall X\in M &: A(X) \\
			\exists X\in M &: A(X)
		\end{align*}
	\end{bsp}

% --------------------------------------------------------------------
% §3.2 Subsection <<Negation von Quantoren>>
% --------------------------------------------------------------------
\subsection{Negation von Quantoren}

%------------------
%
	\begin{bem}[Negation von Quantoren]
		Für ein Prädikat $A(X)$ gilt:
		\[
			\neg(\exists X:A(X)) \Leftrightarrow \forall X:\neg A(X),
		\]
		die Negation von \anf{Es existiert ein $ X $, sodass $A(X)$ wahr} ist also \anf{Für alle $X$, ist $A(X)$ falsch}. \\
		Es gilt weiter:
		\[
			\neg(\forall X:A(X)) \Leftrightarrow \exists X:\neg A(X),
		\]
		die Negation von \anf{Für alle $X$ ist $A(X)$ wahr} ist also \anf{Es existiert ein $X$, für das $A(X)$ falsch ist}.
	\end{bem}
	
	\noindent (Achtung: Die Negation der Aussage $(\forall X:A(X))$ lautet also nicht etwa \anf{Für alle $X$ ist $A(X)$ falsch}! Denn offensichtlich würde dann die Aussage und ihre Negation nicht alle Möglichkeiten abdecken: Der Fall, dass $A(X)$ für genau zwei $X$ falsch ist, wäre weder in der Aussage noch ihrer Negation enthalten.)  

%------------------
%
	\begin{lem}
		Für zwei Aussagen $A,B$ gilt:
		\[
			\neg(A\Rightarrow B) \Leftrightarrow A\wedge \neg B
		\]
	\end{lem}

%------------------
%
	\begin{bew}
		\[\begin{tabular}{c|c||c|c|c|c|c}
			A & B & $A\Rightarrow B$ & $\neg(A\Rightarrow B)$ & $ \neg B $ & $(A\wedge\neg B$ & $ \neg(A\Rightarrow B) \Leftrightarrow A\wedge \neg B $\\
			\hline
			w&w&w&f&f&f&w\\
			w&f&f&w&w&w&w\\
			f&w&w&f&f&f&w\\
			f&f&w&f&w&f&w\\
		\end{tabular}\] \hfill $\Box$
	\bewEnd
	\end{bew}

%------------------
%
	\begin{kor}
		Für eine Menge $M$ und ein Prädikat $A(X)$ gilt:
		\begin{align*}
			\neg(\forall X\in M:A(X)) & \Leftrightarrow \exists X\in M: \neg A(X) \\
			\neg(\exists X\in M:A(X)) & \Leftrightarrow \forall X\in M: \neg A(X)
		\end{align*}
	\end{kor}

%------------------
%
	\begin{bew}
		Es gilt
		\begin{align*}
			\neg(\forall X\in M:A(X)) & \Leftrightarrow \neg(\forall X:(E(X,M)\Rightarrow A(X))) \\
			&\Leftrightarrow \exists X: \neg(E(X,M)\Rightarrow A(X)) \\
			& \stackrel{3.10}{\Leftrightarrow} \exists X: E(X,M) \wedge \neg A(X) \\
			& \Leftrightarrow \exists X\in M: \neg A(X)
		\end{align*}
		Der zweite Teil kann mit einer ähnlichen Rechnung bewiesen werden. \hfill $\Box$
	\bewEnd
	\end{bew}


% --------------------------------------------------------------------
% §4 Section <<Rückblick>>
% --------------------------------------------------------------------
\end{comment}


% ====================================================================
% 2. Vortrag <<Beweismethoden>>
% ====================================================================

\setchapterpreamble[c][.7\textwidth]{%
  \itshape\dgrau\small  In diesem Vortrag werden die grundlegenden logischen Schlussregeln und Beweistechniken erklärt und anhand von Beispielbeweisen vorgestellt. Außerdem werden vorherrschende Normen für das Schreiben schöner und gut lesbarer Beweise besprochen.
  \vspace{24pt}%
}

\chapter{Beweise}


\setchapterpreamble[c][.7\textwidth]{\itshape\dgrau\small
    In diesem Vortrag werden die grundlegenden logischen Schlussregeln und Beweistechniken erklärt und anhand von Beispielbeweisen vorgestellt. Dabei werden auch vorherrschende Normen für das Schreiben schöner und gut lesbarer Beweise besprochen.
\vspace{24pt}}

    
\chapter{Beweise}


\section{Logisches Schließen}


\begin{bem}[Buchtipp]
    Dieses Vorkurs-Kapitel bietet nur einen Crashkurs im Beweisen. Eine freundliche und weit ausführlichere (aber auch sehr seitenstarke) Darstellung bietet das Buch \cite{Vel06}, vor allem dessen Kapitel 3. Über den Link im Literaturverzeichnis kannst du das Buch als Pdf herunterladen.
\end{bem}


\begin{de} \index{Schlussregel} \label{schlussregel}
    Eine \textbf{logische Schlussregel} ist ein Prinzip der Gestalt „Aus den Aussagen $X$ kann auf die Aussage $Y$ geschlossen werden“. Die Aussagen $X$ heißen dabei die \textbf{Prämissen} der Schlussregel und die Aussage $Y$ heißt ihre \textbf{Konklusion}. Die Anwendung einer Schlussregel heißt \textbf{logische Schlussfolgerung} oder auch \emph{deduktiver Schluss}.
\end{de}


\begin{bsp} \label{bsp:schlussregel}
    Seien $A,B$ irgend zwei Aussagen. Die Schlussregel „Modus tollens“ (die später in \cref{reductio} eingeführt wird) geht folgendermaßen:
    \[\begin{tabular}{r}
        $A\to B$ \\
        $\neg B$ \\
        \midrule
        $\neg A$
    \end{tabular}\]
    Sie besagt: „Aus $A\to B$ und $\neg B$ kann auf $\neg A$ geschlossen werden.“ Die Prämissen dieser Schlussregel sind $A\to B$ und $\neg B$ und die Konklusion ist $\neg A$.
    
    Bezeichnet $n$ eine natürliche Zahl, so sind beispielsweise durch 
    \begingroup
    \allowdisplaybreaks
    \begin{align*}
        & \begin{tabular}{l}
            Wäre ich reich, würde ich jeden Tag ins Restaurant gehen. \\
            Ich gehe nicht jeden Tag ins Restaurant. \\
            \midrule
            Also gilt: Ich bin nicht reich.
        \end{tabular} \\[1em]
        & \begin{tabular}{l}
            Ist $n$ eine gerade Zahl, so ist auch $3n$ eine gerade Zahl. \\
            $3n$ ist keine gerade Zahl. \\
            \midrule
            Also gilt: $n$ ist keine gerade Zahl.
        \end{tabular} \\[1em]
        & \begin{tabular}{l}
            Wenn der Fluxkompensator in Unwucht gerät, misslingt der Chronosprung. \\
            Der Chronosprung ist gelungen. \\
            \midrule
            Also gilt: Der Fluxkompensator ist nicht in Unwucht geraten.
        \end{tabular}
    \end{align*}
    \endgroup
    drei \emph{Instanzen} dieser Schlussregel gegeben. Obwohl die drei Schlüsse inhaltlich grundverschieden sind, haben sie alle dieselbe logische Struktur gemeinsam.
\end{bsp}


\begin{de}[Axiom] \index{Axiom}
    In der Regel liegen einer mathematischen Theorie einige Aussagen zugrunde, die nicht bewiesen, sondern schlicht als „gegeben“ vorausgesetzt werden. Sie heißen \textbf{Axiome} und kodieren oftmals Eigenschaften derjenigen Objekte, von denen die Theorie handelt.
\end{de}


\begin{de}[„Es gilt\dots“] \label{def:esgilt}
    Sei $A$ eine Aussage. Wir schreiben „$A$ ist gültig“ oder „Es gilt $A$“ oder „$A$ ist wahr“, falls es möglich ist, die Aussage $A$ mit einer Abfolge logischer Schlussfolgerungen aus den im Kontext angenommenen Axiomen herzuleiten.
\end{de}
 
 
\begin{bem}[* Beweisbarkeit vs. Wahrheit]
    Beachte, dass dies erst einmal nichts mit den Wahrheitswerten aus \cref{def:interpretation} zu tun haben muss. Wahrheit und Herleitbarkeit sind zwei verschiedene Dinge. Zumindest gibt es folgende Zusammenhänge:
    \begin{itemize}
        \item Die logischen Schlussregeln sind so beschaffen, dass sich aus wahren Prämissen auch nur wahre Konklusionen ableiten lassen. Man nennt dies die \emph{Korrektheit} (englisch: ``soundness'') der Schlussregeln. Wenn du aus wahren Prämissen etwas Falsches hergeleitet hast, muss dir zwangsläufig ein Fehlschluss unterlaufen sein.
        \item Aus falschen Prämissen lassen sich dagegen sowohl wahre als auch falsche Aussagen herleiten. Dennoch ändert dies nichts an der Korrektheit. Unabhängig davon, ob $3n$ nun „in Wirklichkeit“ eine gerade Zahl ist oder nicht, ist der Schluss aus \cref{bsp:schlussregel} korrekt, weil er eben nur von einer solchen Situation handelt, in der die Prämissen als wahr „gegeben“ sind.
        \item Lässt sich jede wahre Aussage mittels logischer Schlüsse herleiten, so heißt der Logikkalkül \emph{vollständig}. Die Vollständigkeit ist eine deutlich kompliziertere Angelegenheit als die Korrektheit und in der mathematischen Logik gibt es diverse \href{https://ncatlab.org/nlab/show/completeness+theorem}{Vollständigkeitssätze} und \href{https://ncatlab.org/nlab/show/incompleteness+theorem}{Unvollständigkeitssätze} (deren berühmteste diejenigen von Gödel\footnote{\href{https://de.wikipedia.org/wiki/Kurt_G\%C3\%B6del}{Kurt Gödel (1906-1978)}} sind), die die Vollständigkeit und Unvollständigkeit gewisser Logikkalküle hinsichtlich gewisser Interpretationen beweisen.
    \end{itemize}
\end{bem}


\begin{de}[Satz und Beweis] \index{Beweis}
    Ein \textbf{mathematischer Satz} ist die Feststellung in einem mathematischen Text, dass eine Aussage $A$ „gilt“, d.h. dass sie vermöge der in der Mathematik üblichen logischen Schlussregeln aus denjenigen Aussagen, die im Umfeld des Satzes axiomatisch angenommen werden oder bereits für gültig befunden wurden, hergeleitet werden kann.
    
    Ein \textbf{mathematischer Beweis} für $A$ ist eine (mehr oder weniger ausführliche) Beschreibung einer solchen Herleitung, die dich von der Gültigkeit von $A$ überzeugt.
\end{de}


\begin{bsp}[*]
    In der synthetischen affinen und projektiven Geometrie ist die Aussage
    \begin{itemize}
        \item[(A)] Durch je zwei verschiedene Punkte verläuft genau eine Gerade.
    \end{itemize}
    ein Axiom, das nicht hergeleitet wird, sondern einen Teil unseres Verständnisses von „Punkten“ und „Geraden“ kodiert. Allein aus diesem Axiom kann nun schon die folgende Aussage hergeleitet werden:
    \begin{satz}[*]
        Zwei verschiedene Geraden schneiden sich in höchstens einem gemeinsamen Punkt.
    \end{satz}
    \begin{bew}
        Seien $g,h$ zwei Geraden, die sich in zwei verschiedenen Punkten $P$ und $Q$ schneiden. Nach (A) gibt es \emph{genau eine} Gerade, die durch die beiden Punkte $P,Q$ verläuft. Da $g,h$ beide durch $P,Q$ verlaufen, folgt also aus (A), dass $g=h$ gelten muss. Also sind zwei Geraden, die mindestens zwei verschiedene Schnittpunkte miteinander haben, einander gleich, woraus folgt, dass verschiedene Geraden höchstens einen Schnittpunkt miteinander haben können. \qed
    \end{bew}
\end{bsp}


\begin{bem}[Bedeutung ist prinzipiell verzichtbar]
    Dieser Beweis behält seine Gültigkeit auch dann, wenn das Wort „Gerade“ überall durch das Wort „Bierkrug“ ersetzt wird, d.h. wenn durch je zwei verschiedene Punkte stets genau ein Bierkrug verläuft, so schneiden sich zwei verschiedene Bierkrüge in höchstens einem gemeinsamen Punkt. Dies ist typisch für die Logik: für die Gültigkeit logischer Schlüsse kommt es gar nicht auf den Inhalt der Aussagen an, sondern nur auf ihre Struktur. Beispielsweise ist es auch für die Korrektheit des dritten Schlusses aus \cref{bsp:schlussregel} völlig egal, was eigentlich der „Fluxkompensator“ und der „Chronosprung“ überhaupt sind.
\end{bem}


\begin{bem}[Ausführlichkeit eines Beweises]
    Komplizierte Beweise involvieren dutzende logische Schlussfolgerungen und ein Mathe-Lehrbuch würde, schriebe man alle Beweise in größter Ausführlichkeit auf, den halben Regenwald verschlingen. Daher listen mathematische Beweise selten jede einzelne logische Schlussfolgerung auf, sondern beschreiben mehrere logische Schlüsse auf einmal und führen „Routine-Argumente“, von denen erwartet werden kann, dass sie der Leser mit Leichtigkeit selbst ergänzen kann, gar nicht erst aus. Aus der Schule bist du es ja auch gewohnt, in einer langen Rechnung nicht jeden einzelnen Rechenschritt separat aufzuschreiben, sondern mitunter mehrere Zwischenschritte auf einmal durchzuführen. Im extremsten Fall wird in einem Beweis gar nicht argumentiert, sondern es wird lediglich behauptet, die Aussage gelte „offensichtlicherweise“ oder „sei klar“. In vielen Fällen sind solche Aussagen tatsächlich „offensichtlich“; manchmal kommt es aber auch vor, dass der Prof. selbst nicht weiß, dass die Zwischenschritte, die er gerade überspringt, weil er sie für „trivial“ hält, einer komplizierten Begründung bedürfen. Dann kann es passieren, dass er/sie bei einer Zwischenfrage minutenlang auf dem Schlauch steht. Mit Floskeln wie „gilt offensichtlich“ oder „ist trivial“ solltest du äußerst vorsichtig umgehen. In vielen Fällen ist ihre Verwendung, selbst wenn sie legitim ist, schlechter Stil.
    \begin{figure}[ht]
        \includegraphics[width=10cm]{./_img/Istklar.jpeg}
        \centering \caption{Beispiel aus einer LA1-Vorlesung für einen „Beweis“, in dem gar nichts argumentiert wurde. Vgl. die Begründung von \cref{def:surjektiv}.}
    \end{figure}
\end{bem}


\begin{bem}[Subjektivität des Beweisbegriffs]
    Das Wort „überzeugt“ in meiner Beweisdefinition deutet eine subjektive Komponente an. Wenn dich ein Vorlesungs-„Beweis“ nicht überzeugen kann, dann ist er für dich eben auch kein Beweis. Ein Beweistext kann für den Einen eine befriedigende Begründung sein, während er für den Anderen völlig unverständlich und praktisch wertlos ist. Wenn dir unmittelbar einsichtig ist, dass eine Aussage gilt, kann sogar ein „Ist-klar-Beweis“ überzeugend sein.
    
    Nichtsdestotrotz gibt es gewisse Regeln und Techniken, über deren Zulässigkeit ein Konsens besteht. Beweise, die diesen Regeln unterliegen, muss ein Mathematiker anerkennen. Dass Beweise nicht überzeugend sind, kommt so gut wie nie von Verstößen gegen die Logik her; sondern eher von der Verwendung obskurer Begriffe, dem Mangel an Erläuterung komplizierter Beweisschritte, Schreibfehlern, dem Verschleiern von Beweislücken oder der unbegründeten Verwendung von Aussagen, die irgendwo fünfzig Seiten vorher einmal in einem unscheinbaren Lemma hergeleitet wurden.
\end{bem}


Bereits in diesem Kapitel werde ich Beweise führen, nämlich um zu demonstrieren, wie sich gewisse logische Schlussregeln und Beweistechniken aus anderen ableiten lassen. Setz dich aber nicht unter Druck, die Herleitungen der Beweistechniken lückenlos nachvollziehen zu müssen, sondern begreife sie als Erklärungen, die plausibel machen sollen, warum die Beweistechniken Sinn ergeben. Letztendlich musst du dich selbst davon überzeugen, wie genau ist gar nicht so wichtig. In den Mathevorlesungen (mal abgesehen von Vorlesungen über Logik, wo es genau darum geht) werden die üblichen Beweistechniken größtenteils ohne weitere Begründung verwendet und ab der zweiten Semesterwoche erwartet auch niemand mehr, dass du die Logik, die deinen Beweisen zugrundeliegt, rechtfertigst (solange sie halt nicht „unlogisch“ ist).





\section{Implikationen}


In diesem Abschnitt seien $A,B$ stets zwei beliebige Aussagen.


\begin{axiom}[Der direkte Beweis] \label{direkterbeweis} \index{Direkter Beweis}
    Um die Aussage „$A\to B$“ zu beweisen, kannst du die Technik des \textbf{direkten Beweises} benutzen:
    
    Nimmt an, dass die Aussage $A$ gilt und zeige nun mithilfe dieser Annahme (und aller weiteren Aussagen, die dir zur Verfügung stehen), dass auch $B$ gilt.
\end{axiom}


\begin{bsp}
    Sofern der FC Bayern München nach dem vorletzten Spieltag bereits vier Punkte vor dem Tabellenzweiten steht, wird er Deutscher Meister.
\end{bsp}


\begin{bew}
    Angenommen, der FC Bayern liege nach dem vorletzten Spieltag vier Punkte vor dem Tabellenzweiten. Weil nur noch ein Spieltag verbleibt, kann der Zweite (und jeder weiter unten stehende Verein) nur noch höchstens drei Punkte im letzten Spieltag gewinnen. Also sind dann die Bayern (schon wieder) uneinholbar und werden Deutscher Meister. \qed
\end{bew}


\begin{bem}[* Zusammenhang zur Interpretation von „$\to$“]
    In \cref{tauto} wurde die folgende Aussage T gezeigt:
    \begin{quote}
        (T): Die Implikation $A\to B$ ist genau dann eine Tautologie, wenn unter jeder (zweiwertigen) Interpretation, unter der $A$ eine wahre Aussage ist, auch $B$ eine wahre Aussage ist.
    \end{quote}
    Dies resultierte aus der Beschaffenheit der Wahrheitstafel für den Implikationspfeil:
    \[\begin{tabular}{cc|c}
        $A$ & $B$ & $A\to B$ \\
        \hline
        w & w & w \\
        w & f & f \\
        f & w & w \\
        f & f & w
    \end{tabular}\]
    Vermittels der Aussage (T) kann diese Wahrheitstafel zur Rechtfertigung der Technik des direkten Beweises verwendet werden. Umgekehrt kannst du aber auch die Technik des direkten Beweises als „natürlicher“ ansehen und die Aussage (T) als Rechtfertigung für die Wahrheitstafel des Implikationspfeils ansehen.
\end{bem}
  
  
\begin{bem}[Signalwörter]
    Wenn du die Implikation $A\to B$ direkt beweist, kannst du dies deutlich machen, indem du den Beweis mit „Es gelte $A$“, „Es sei angenommen, dass $A$ gilt“ oder Ähnlichem beginnst.
\end{bem}

  
\begin{satz}[* Jede Aussage impliziert sich selbst] \label{implikationref}
    Es gilt\footnote{vgl. \cref{teilmengeneig}a)}
    \[ A\to A \]
\end{satz}


\begin{bew}
    Für einen direkten Beweis sei angenommen, dass $A$ gilt. Weil unter dieser Annahme ja $A$ gilt, ist schon alles bewiesen. \qed
\end{bew}

  
\begin{satz}[* Wahres folgt aus Beliebigem]\label{wahresausbeliebigem}
    Es gilt
        \[ A \to (B\to A)   \]
    Mit anderen Worten: Sofern $A$ gilt, wird $A$ auch von $B$ impliziert.
\end{satz}


\begin{bew}
    Für einen direkten Beweis sei angenommen, dass $A$ gilt. Nun ist zu zeigen, dass auch $B\to A$ gilt. Dazu sei zusätzlich angenommen, dass $B$ gilt. Nun gilt auch $A$, weil dies ja schon ganz zu Beginn des Beweises angenommen wurde. \qed
\end{bew}


\begin{bem}[„$\to$“ bedeutet keine Kausalität!]
    Die Aussage, dass Wahres aus Beliebigem folgt, mag seltsam erscheinen (und wird unter die \href{https://de.wikipedia.org/wiki/Paradoxien_der_materialen_Implikation}{Paradoxien der materialen Implikation} gezählt), da dann ja auch Wahres aus solchen Aussagen folgt, die gar nichts damit zu tun haben. Beispielsweise ist „Sofern der Döner in Deutschland erfunden wurde, ist $4$ eine Quadratzahl“ eine wahre Aussage. Dass dies „paradox“ erscheint, kommt von einer inadäquaten Interpretation des Implikationspfeils „$\to$“:
    
    In ihrer üblichsten Interpretation besagt die Aussage $A\to B$ nicht, dass es einen kausalen Zusammenhang zwischen $A$ und $B$ geben muss; sondern nur, dass $B$ unter Annahme von $A$ gilt, egal ob diese Annahme in die Herleitung von $B$ mit einfließt oder nicht. Noch ein Beispiel: Da es, sofern sich das Rad einer Windmühle dreht, windig ist, ist „Wenn sich das Windrad dreht, ist es windig“ eine korrekte Aussage. Aber dies besagt natürlich nicht, dass es windig ist, \emph{weil} sich das Windrad dreht, geschweige denn, dass die Windmühle den Wind erzeugen würde.
\end{bem}


\begin{vorschau}[* Relevanzlogiken]
    Logiken, die sich darum bemühen, dass „$\to$“ wirklich die Bedeutung einer kausalen Implikation trägt, heißen „Relevanzlogiken“, da dort für eine Implikation $A\to B$ gefordert wird, dass $A$ in irgendeiner Hinsicht „relevant“ für $B$ ist. In Relevanzlogiken ist die Technik des direkten Beweises nicht mehr uneingeschränkt zulässig.
\end{vorschau}


\begin{axiom}[Modus ponens] \label{modusponens} \index{Modus Ponens}
    Die logische Schlussregel
    \[\begin{tabular}{r}
        $A\to B$ \\
        $A$ \\
        \hline
        $B$
   \end{tabular}\]
    heißt „Modus ponens“. Sie besagt: Wann immer dir gegeben ist, dass sowohl $A\to B$ als auch $A$ gültig sind, kannst du daraus $B$ schlussfolgern.
\end{axiom}


\begin{bsp}
    Ich habe angekündigt, dass ich, sofern ich zum Bürgermeister gewählt werde, Freibier für alle stiften werde. Sofern ich nun tatsächlich zum Bürgermeister gewählt werde, folgt, dass jedem Freibier ausgeschenkt wird.
\end{bsp}


\begin{bem}[Logik-Latein]
    Du brauchst dir nicht merken, dass diese Schlussregel „Modus ponens“ heißt. Ebensowenig brauchst du dir die anderen lateinischen Bezeichnungen in diesem Vortrag zu merken. Ich gebe die Wörter an, um dir das Nachschlagen in Internet und Literatur zu erleichtern.
\end{bem}


\begin{bem}
    \textbf{Vorsicht}: Anfänger machen gelegentlich den Fehler, aus $A\to B$ und $B$ auf die Aussage $A$ zu schließen. Hier ist ein Beispiel für diesen Fehlschluss:
    \begin{quote}
        Wenn ich verschlafe, komme ich zu spät zur Uni. Nun bin ich heute tatsächlich zu spät zur Uni gekommen, also habe ich verschlafen.
    \end{quote}
    Mach dir klar, warum diese Argumentation unzulässig ist.
\end{bem}


\begin{satz}[Direkter Beweis mit Zwischenschritten] \label{implikationtrans}
    Seien $n$ eine natürliche Zahl und $Z_1,\dots , Z_n$ eine Handvoll Aussagen. Um die Implikation $A\to B$ zu beweisen, kannst du die Implikationen
        \[ A\to Z_1,\quad Z_1\to Z_2,\quad \dots ,\quad Z_{n-1}\to Z_n\quad \text{und}\quad Z_n\to B \]
    beweisen.\footnote{vgl. \cref{teilmengeneig}b)} In diesem Fall verwendest du die Aussagen $Z_1,\dots , Z_n$ als \textbf{Zwischenschritte}.
\end{satz}


\begin{bew}
    Es sei angenommen, dass ich alle Implikationen $A\to Z_1$, $Z_1\to Z_2$, \dots , $Z_n\to B$ bewiesen habe. Um zu zeigen, dass dann auch $A\to B$ gilt, sei angenommen, dass $A$ gilt. Wegen $A\to Z_1$ folgt, dass dann auch $Z_1$ gilt. Wegen $Z_1\to Z_2$ folgt, dass auch $Z_2$ gilt. Auf diese Weise können wir schrittweise die $Z$'s durchgehen, sodass am Ende auch $Z_n$ bewiesen ist. Und wegen $Z_n\to B$ gilt dann auch $B$. \qed
\end{bew}


\begin{bsp}
    Falls es nächsten Sommer (schon wieder) zu wenig regnet, wird der Fichtenwald in meiner Heimatstadt gerodet werden.
\end{bsp}


\begin{bew}
    Wenn es nächstes Jahr wieder zu wenig regnet, fehlt es den Fichten an Flüssigkeit, um ausreichend Harz für eine widerstandsfähige Rinde auszubilden. Dies erleichtert es Borkenkäfern, innerhalb der Rinde zu nisten, sodass sich die Borkenkäferpopulation im Wald stark vergrößert und Bäume teilweise absterben werden. Unter diesem Umstand wird die örtliche Forstbehörde beschließen, den Wald zum Schutz vor umstürzenden Bäumen und einer weiteren Ausbreitung der Borkenkäfer zu roden. \qed
\end{bew}

  
\begin{nota}[*]
    Sind $n$ eine natürliche Zahl und $Z_1,\dots , Z_n$ ein paar Aussagen, für die $Z_1\to Z_2, Z_2\to Z_3, \dots, Z_{n-1}\to Z_n$ gilt, so schreibt man auch kurz
        \[ Z_1\to Z_2\to\ \dots\ \to Z_n\]
    Beispielsweise könnte die Argumentation von gerade eben folgendermaßen dargestellt werden:
    \begin{itemize}
        \item[] Nächstes Jahr regnet es zu wenig.
        \item[$\to$] Den Fichten fehlt es an Flüssigkeit, um ausreichend Harz für eine widerstandsfähige Rinde auszubilden.
        \item[$\to$] Borkenkäfern wird es erleichtert, innerhalb der Rinde zu nisten.
        \item[$\to$] Die Borkenkäferpopulation im Wald wird stark vergrößert und Bäume sterben teilweise ab.
        \item[$\to$] Die örtliche Forstbehörde beschließt, den Wald zum Schutz vor umstürzenden Bäumen und einer weiteren Ausbreitung der Borkenkäfer zu roden.   
    \end{itemize}
\end{nota}





\section{Äquivalenzen}


In diesem Abschnitt seien $A,B$ stets zwei beliebige Aussagen.


\begin{axiom}[Hin- und Rückrichtung] \label{hinruck} \index{Hinrichtung} \index{Rueckrichtung@Rückrichtung} \index{Aequivalenzbeweis@Äquivalenzbeweis}
    Aus dem Vorliegen der beiden Implikationen $A\to B$ und $B\to A$ kann auf die Äquivalenz $A\leftrightarrow B$ geschlossen werden.\footnote{vgl. \cref{teilmengeneig}c)}
    \[\begin{tabular}{r}
        $A\to B$ \\
        $B\to A$ \\
        \hline
        $A\leftrightarrow B$
    \end{tabular}\]
    Wenn du die Äquivalenz $A\leftrightarrow B$ beweisen willst, kannst du deinen Beweis also in zwei Teile aufteilen: In der sogenannten \textbf{Hinrichtung} beweist du die Implikation $A\to B$. In der \textbf{Rückrichtung} beweist du die Implikation $B\to A$.
\end{axiom}


\begin{bsp} \label{bsp:hinruck}
    Sei $n$ eine ganze Zahl. Genau dann ist $n$ ein Vielfaches von $6$, wenn es zugleich ein Vielfaches von $2$ und ein Vielfaches von $3$ ist.
\end{bsp}


\begin{bew}
    \begin{enumerate}
        \item[„$\Rightarrow$“:] Sei $n$ ein Vielfaches von $6$. Dies besagt, dass es eine ganze Zahl $k$ mit $n=6\cdot k$ gibt. Es folgt
        \begin{align*}
            n= 2\cdot (3k) \qquad\text{und}\qquad n = 3\cdot (2k)
        \end{align*}
        also ist $n$ sowohl ein Vielfaches von $2$ als auch von $3$.
        \item[„$\Leftarrow$“:] Es sei $n$ sowohl ein Vielfaches von $2$ als auch von $3$. Demzufolge gibt es ganze Zahlen $k,l$ mit
        \begin{align*}
            n = 2k\qquad \text{und}\qquad n  = 3l
        \end{align*}
        Es folgt
        \begin{align*}
            n & = 1\cdot n \\
            & = (3-2)\cdot n \\
            & = 3n - 2n \\
            & = 3\cdot 2k - 2\cdot 3l & (\text{wegen $n=2k$ und $n=3l$})\\
            & = 6k - 6l \\
            & = 6\cdot (k-l)
        \end{align*}
        Also ist $n$ auch ein Vielfaches von $6$. \qed
    \end{enumerate}
\end{bew}


\begin{bem}
    Die Methoden, die in diesen Beweis eingingen, gehören zur „Teilbarkeitstheorie“. Mehr darüber wirst du im zweiten Semester in der Vorlesung „Lineare Algebra 2“ lernen.
\end{bem}


\begin{bem}[Guter Stil]
    Wenn du eine Äquivalenz per Hin- und Rückrichtung beweist, solltest du die jeweiligen Beweisteile mit „$\Rightarrow$“ und „$\Leftarrow$“ beginnen (so wie im Beispiel gerade eben) oder so etwas wie „Ich beweise zuerst die Hinrichtung“ und „Für den Beweis der Rückrichtung sei nun\dots“ schreiben, damit deinem Leser jederzeit klar ist, um welche der beiden Richtungen es gerade geht.
\end{bem}


\begin{axiom}
    Aus der Äquivalenz $A\leftrightarrow B$ kann sowohl auf $A\to B$ als auch auf $B\to A$ geschlossen werden.
    \[\begin{tabular}{r}
        $A\leftrightarrow B$ \\
        \hline 
        $A\to B$ 
    \end{tabular} \qquad\text{und}\qquad \begin{tabular}{r}
        $A\leftrightarrow B$ \\
        \hline 
        $B\to A$ 
    \end{tabular}\]
\end{axiom}


\begin{bsp}
    Es wurde angekündigt, dass man die Prüfung genau dann besteht, wenn man mehr als 50 Punkte erreicht hat. Dann weiß ich einerseits, dass, wenn ich die Prüfung bestanden habe, ich mehr als 50 Punkte erreicht haben muss; andererseits weiß ich, dass ich, sofern ich mindestens 50 Punkte erreicht habe, auf jeden Fall bestanden habe.
\end{bsp}


\begin{satz}[Äquivalenzbeweis mit Zwischenschritten] \label{ifftrans}
    Seien $n$ eine natürliche Zahl und $Z_1,\dots , Z_n$ ein paar Aussagen. Dann kannst du die Äquivalenz $A\leftrightarrow B$ beweisen, indem du die Äquivalenzen
        \[ A\leftrightarrow Z_1,\quad Z_1\leftrightarrow Z_2,\quad \dots,\quad Z_{n-1}\leftrightarrow Z_n \quad\text{und}\quad Z_n\leftrightarrow B \]
    beweist. In diesem Fall fungieren die Aussagen $Z_1,\dots , Z_n$ in deinem Beweis als \textbf{Zwischenschritte}.
\end{satz}


\begin{bew}[*]
    \begin{enumerate}
        \item[„$\Rightarrow$“:] Die Äquivalenzen $A\leftrightarrow Z_1$, $Z_1\leftrightarrow Z_2$, \dots, $Z_n\leftrightarrow B$ beinhalten die Implikationen $A\to Z_1$, $Z_1\to Z_2$, \dots, $Z_n\to B$, woraus sich mittels \cref{implikationtrans} ergibt, dass $A\to B$ gilt.
        \item[„$\Leftarrow$“:] Ebenso beinhalten die Äquivalenzen $A\leftrightarrow Z_1$, $Z_1\leftrightarrow Z_2$, \dots, $Z_n\leftrightarrow B$ auch die Implikationen $B\to Z_n$, $Z_n\to Z_{n-1}$, \dots, $Z_1\to A$, woraus sich mittels \cref{implikationtrans} auch $B\to A$ ergibt. \qed
    \end{enumerate}
\end{bew}


\begin{nota}
    Sind $n$ eine natürliche Zahl und $Z_1,\dots , Z_n$ ein paar Aussagen, für die $Z_1\leftrightarrow Z_2,\dots, Z_{n-1}\leftrightarrow Z_n$ gilt, so schreibt man auch kurz
    \[ Z_1\leftrightarrow Z_2 \leftrightarrow \ldots \leftrightarrow  Z_n\]
\end{nota}


\begin{bsp}
    Sei $x$ eine positive reelle Zahl. Genau dann ist $x$ eine Lösung der Gleichung $x^2-x=1$, wenn $x= \frac{1+\sqrt{5}}{2}$.\footnote{Die Zahl $\frac{1+\sqrt{5}}{2}$ heißt \href{https://de.wikipedia.org/wiki/Goldener_Schnitt}{Goldener Schnitt}.}
\end{bsp}


\begin{bew}
    Es gilt:
    \begin{alignat*}{3}
        x^2-x& =1 \quad&\leftrightarrow\quad&& \left(x-\frac{1}{2}\right)^2 - \frac{1}{4} &= 1 \\
        && \leftrightarrow\quad&& \left(x-\frac{1}{2}\right)^2&=\frac{5}{4} \\
        && \leftrightarrow\quad&& x-\frac{1}{2} &= \pm \sqrt{\frac{5}{4}} \\
        && \leftrightarrow\quad&& x  &= \frac{1\pm \sqrt{5}}{2} 
    \end{alignat*}
    Weil $x$ positiv ist und $\frac{1-\sqrt{5}}{2}<0$ wäre, ist dies wiederum äquivalent zu $x=\frac{1+\sqrt{5}}{2}$. \qed
\end{bew}


\begin{satz}[* Jede Aussage ist äquivalent zu sich selbst]\label{iffref}
    Es gilt
        \[ A\leftrightarrow A \]
\end{satz}


\begin{bew}
    Mit \cref{implikationref} ist zugleich die Hinrichtung und die Rückrichtung bewiesen. \qed
\end{bew}


\begin{satz}[* Kommutativgesetz für $\leftrightarrow$]\label{iffkomm}
    Es gilt
        \[ (A\leftrightarrow B)\leftrightarrow(B\leftrightarrow A) \]
\end{satz}


\begin{bew}
    \begin{enumerate}
        \item[„$\Rightarrow$“:] Es gelte $A\leftrightarrow B$. Daraus folgt, dass sowohl $A\to B$ als auch $B\to A$ gelten. Das kann natürlich auch andersrum gelesen werden: es gelten sowohl $B\to A$ als auch $A\to B$. Hieraus folgt $B\leftrightarrow A$.
        \item[„$\Leftarrow$“:] Die Rückrichtung beweist man ganz analog zur Hinrichtung, dabei müssen lediglich die Rollen von $A$ und $B$ vertauscht werden. \qed
    \end{enumerate}
\end{bew}


\begin{bem}[Substitutionsprinzip] \index{Substitutionsprinzip}
    Seien $A,B$ zwei äquivalente Aussagen. Dann kannst du in Beweisen die Aussagen $A$ und $B$ beliebig miteinander vertauschen. Möchtest du beispielsweise $A$ beweisen, kannst du genausogut $B$ beweisen. Oder ist dir eine Aussage der Gestalt $(A\land C)\to D$ gegeben, so kannst du genausogut auch mit der Aussage $(B\land C)\to D$ arbeiten.
    
    Aus diesem Grund sind Äquivalenzaussagen wertvoll und nützlich. Sie erlauben es, Aussagen von mehreren Blickwinkeln zu beleuchten und dadurch ein „tieferes“ Verständnis für sie zu gewinnen.
\end{bem}


\begin{satz}[* Curry-Paradoxon\footnote{\href{https://de.wikipedia.org/wiki/Haskell_Brooks_Curry}{Haskell Curry (1900-1982)}}] \label{curryparadox}
    Es gilt
        \[ (A\leftrightarrow (A\to B))\to B \]
    Mit anderen Worten: Ist $A$ bereits äquivalent dazu, dass $B$ von $A$ impliziert wird, so gilt $B$.
\end{satz}


\begin{bew}
    Für einen direkten Beweis sei angenommen, dass $A\leftrightarrow (A\to B)$ gilt. Nun ist zu beweisen, dass $B$ gilt.
    \begin{enumerate}[(1)]
        \item Es gilt $A\to B$, denn: Für einen direkten Beweis sei angenommen, dass $A$ gilt. Wegen $A\leftrightarrow (A\to B)$ gilt dann auch $A\to B$. Und weil $A$ als gültig angenommen wurde, folgt daraus, dass $B$ gilt.
        \item Es gilt $B$, denn: Nach Schritt (1) gilt $A\to B$. Wegen $A\leftrightarrow (A\to B)$ gilt dann auch $A$. Weil nach Schritt (1) auch $A\to B$ gilt, folgt nun, dass auch $B$ gilt. \qed
    \end{enumerate}
\end{bew}


\begin{bem}[* Selbstreferenzielle Aussagen]
    Das Curry-Paradoxon wird deshalb als „Paradoxon“ gehandelt, da es zumindest in der Umgangssprache ziemlich leicht ist, Aussagen $A$ zu konstruieren, für die $A\leftrightarrow (A\to B)$ gilt. Zum Beispiel seien
    \begin{itemize}[labelindent=1.5em, leftmargin=!, labelwidth=\widthof{$B:=$}]
        \item[$A:=$] „Wenn diese Aussage wahr ist, gewinne ich morgen im Lotto.“
        \item[$B:=$] „Morgen gewinne ich im Lotto.“
    \end{itemize}
    Dann gilt tatsächlich $A\leftrightarrow (A\to B)$, sodass aus \cref{curryparadox} folgt, dass ich morgen Millionär bin. Lotterien hassen diesen Trick.
    
    Damit sich mit diesem Trick nicht einfach \emph{jede} mathematische Aussage beweisen lassen kann, muss sichergestellt werden, dass sich die Selbstreferenzialität „Wenn \emph{diese Aussage} wahr ist, dann \dots“, die umgangssprachlich problemlos erzeugbar ist, nicht in formaler mathematischer Sprache nachbilden lässt.
\end{bem}





\subsection*{* Drei häufige Anfängerfehler im Umgang mit Äquivalenzumformungen}


\begin{bem}[„Gleichungs-U's“]
    Aus der Schule sind es manche Studienanfänger gewohnt, Gleichungen dadurch zu beweisen, dass sie sie sovielen Äquivalenzumformungen unterziehen, bis am Ende eine „offensichtliche“ Gleichung rauskommt. Hier ein Beispiel für diese Vorgehensweise:
    \begin{bsp}
    Seien $x,y$ zwei reelle Zahlen. Dann gilt:
        \[ x\cdot (y+1)-x = (x+1)\cdot y-y\]
    \end{bsp}
    \begin{bew}[(Mieser Beweis)]
        Es gilt:
        \begin{alignat*}{3}
            && x\cdot (y+1)-x& \quad=\quad (x+1)\cdot y-y \\
            &\leftrightarrow\qquad& xy + x  -x& \quad=\quad xy + y - y \\
            &\leftrightarrow\qquad& xy & \quad=\quad xy
        \end{alignat*}
    \end{bew}
    Diesen „Beweisstil“ solltest du dir auf keinen Fall aneignen bzw. so bald es geht abgewöhnen! Denn bei so einer Äquivalenzenkette geschieht bei jeder Äquivalenzumformung auf jeder der beiden Seiten eine arithmetische Umformung, die der Leser nachvollziehen muss. Und diese arithmetischen Umformungen bilden den eigentlichen Kern des Beweises, letztendlich hat der Leser die Gleichungskette also in einer „U-Form“, deren beide Stränge erst ganz am Schluss zusammenfinden, zu lesen:
    \[\begin{tikzcd}
        && x\cdot (y+1)-x\ar[red, d, dash] & =& (x+1)\cdot y-y \ar[red, d, dash] \\
        &\leftrightarrow& xy + x-x  \ar[red, d, dash] & =& xy + y - y \ar[red, d, dash] \\
        &\leftrightarrow& xy \ar[red, rr, dash, bend right] & =& xy
    \end{tikzcd}\]
    Schöner ist, es diese Kette arithmetischer Umformungen, gar nicht erst als „U“, sondern als die Kette, als die sie letztendlich auch zu lesen ist, hinzuschreiben:
    \begin{bew}[(Schönerer Beweis)]
        Es gilt:
        \begin{align*}
            x\cdot (y+1) -x& = xy + x -x\\
            & = xy  \\
            & = xy + y - y \\
            & = (x+1)\cdot y -y  &&& \qed
        \end{align*}
    \end{bew}
    Unterscheide dabei sorgfältig von der Art und Weise, wie du den Beweis \emph{findest} und der Art und Weise, wie du ihn am Ende \emph{aufschreibst}\footnote{vgl \cref{rumprobieren} und \cref{beweisaufschreiben}}: Während des Beweisfindungsprozesses ist alles erlaubt und du darfst auf dem Schmierblatt soviele Äquivalenzumformungen aufschreiben, wie du willst. Aber am Ende, wenn es darum geht, den Beweis ansprechend aufzuschreiben, solltest du alle Unsauberkeiten tilgen und den Beweis in eine gut lesbare Form bringen. Ein mathematischer Beweis, wie du ihn in einem Lehrbuch findest oder wie ihn ein Dozent in der Vorlesung vorführt, gibt nur selten den Denkprozess, der seiner Entstehung zugrundelag, wieder. (Was in didaktischer Hinsicht manchmal bedauerlich und einer der Hauptgründe dafür ist, dass sich Erstsemester mit dem Verfassen von Beweisen schwertun.)
\end{bem}


\begin{bem}[Unterschied zwischen $=$ und $\leftrightarrow$]
    Bringe nicht $=$ und $\leftrightarrow$ durcheinander. Die Gleichheit „$=$“ ist eine Beziehung, die zwischen beliebigen Objekten desselben Typs Sinn ergibt. Bspw. ergibt es Sinn zu fragen, ob zwei Zahlen gleich sind, zwei Punkte im Raum gleich sind, zwei Funktionen dieselbe Ableitung haben usw. Dagegen ist „$\leftrightarrow$“ eine Beziehung zwischen \emph{Aussagen}. Manche sind es aus der Schule gewohnt, jegliche Art mathematischen Folgerns durch ein Gleichheitszeichen zu notieren. Die (korrekte) Gleichungsumformung
    \begin{align*}
        x & = y-3 \\
        \leftrightarrow\quad  x+3 & = y
    \end{align*}
    würden sie inkorrekterweise als
    \begin{align*}
        x & = y-3 \\
        = \quad x+3 & = y
    \end{align*}
    notieren, was Leser arg verwirren kann (vor allem, wenn es nicht so schön eingerückt wäre, sondern einfach nur $x=y-3=x+3=y$ dastünde) und auch schlicht mathematisch falsch ist.
    
    Auch hier gilt wieder: In der kreativen Phase, in der du nach einem Beweis suchst und Schmierblatt um Schmierblatt mit Ideen vollschreibst, kannst du Gleichheitszeichen nach Belieben spammen und sogar falsch verwenden. Aber bei der Erstellung des Endprodukts, des Beweises, wie ihn Andere lesen sollen, solltest du penibel auf die korrekte Verwendung von „$=$“'s und „$\leftrightarrow$“'s achten.
\end{bem}


\begin{bem}[Beweise „rückwärts“ führen] \label{hintennachvorne}
    Mal angenommen, ich möchte beweisen, dass die Kubikwurzel von $3$ größer ist als die Quadratwurzel von $2$. Auf der Suche nach einem Beweis beginne ich einfach mal mit der zu zeigenden Ungleichung $\sqrt[3]{3}>\sqrt{2}$ und forme ein bisschen um:
    \begin{align*}
        && \sqrt[3]{3}& >\sqrt{2} \\
        & \to& \sqrt[3]{3}^{3\cdot 2} & > \sqrt{2}^{2\cdot 3} & (\text{beide Seiten mit $6$ potenzieren}) \\
        & \to & (\sqrt[3]{3}^3)^2 & > (\sqrt{2}^2)^3 & (\text{Potenzgesetz anwenden})\\
        & \to & 3^2 & > 2^3 \\
        & \to & 9 & > 8
    \end{align*}
    Das sieht schonmal gut aus! Durch ein paar Umformungen bin ich zu einer wahren Aussage gelangt. Mancher Anfänger würde nun denken, dass das Problem damit erledigt ist und die obige Ungleichungskette als Beweis taugt.
    
    Das ist aber falsch. Denn die Ungleichungskette beginnt ja mit der zu beweisenden Aussage. Hier wurde also nur die Aussage „Wenn $\sqrt[3]{3} >\sqrt{2}$ gilt, dann ist $9>8$“ bewiesen, die aber leider nichts darüber aussagt, ob nun tatsächlich $\sqrt[3]{3} >\sqrt{2}$ gilt. Glücklicherweise handelt es sich bei allen Umformungen sogar um Äquivalenzumformungen, sodass die Ungleichungskette auch in umgekehrter Richtung gültig ist:
    \begin{align*}
        && 9 & > 8 \\
        & \to & 3^2 & > 2^3 \\
        & \to & (\sqrt[3]{3}^3)^2 & > (\sqrt{2}^2)^3 \\
        & \to& \sqrt[3]{3}^{3\cdot 2} & > \sqrt{2}^{2\cdot 3} & (\text{Potenzgesetz anwenden}) \\
        &\to &  \sqrt[3]{3}& >\sqrt{2}  & (\text{sechste Wurzel ziehen})
    \end{align*}
    Nun ist die Argumentation zumindest mal nicht mehr mathematisch falsch. Wenn ich jetzt noch das „Ungleichungs-U“ loswerde, kann sich der Beweis sehen lassen. Hier ist der finale Beweis:
    \begin{bew}
        Es gilt
        \begingroup
        \allowdisplaybreaks
        \begin{align*}
            \sqrt[3]{3} & = \sqrt[3]{\sqrt{9}} \\
            & = \sqrt[6]{9} \\
            & > \sqrt[6]{8} & (\text{da $\sqrt[6]{-}$ eine ordnungserhaltende Operation und $9>8$ ist}) \\
            & = \sqrt{\sqrt[3]{8}} \\
            & = \sqrt{2} && \qed
        \end{align*}
        \endgroup
    \end{bew}
    Beachte, dass die Struktur dieses Beweises nicht meinen Denkprozess bei der Beweissuche wiederspiegelt. Das ist aber völlig normal und ok. Sollte dein Beweis sehr kompliziert sein, wäre es natürlich trotzdem nett, wenn du, sofern es deinem Leser hilft, ein paar Meta-Bemerkungen darüber, welche Idee hinter dem aktuellen Beweisschritt steckt, einstreust.
    
    Auch Profis stoßen manchmal auf einen Beweis, indem sie die Argumentation „rückwärts“ ausprobieren, also mit der zu beweisenden Aussage starten und schauen, was sich damit anfangen lässt. Während diese Strategie völlig legitim zur Beweis\emph{findung} ist, ist sie es aber nicht zur Beweis\emph{niederschrift}. Dein zum Schluss aufgeschriebener Beweis muss das Problem sauber von den gegebenen Aussagen auf die zu beweisenden Aussagen durchgehen.
    
    Der Versuch, einen Beweis rückwärts zu führen, kann auch Fehler erzeugen, die du, sofern du den Rückwärts-Gedankengang am Ende nicht kritisch reflektierst, übersiehst. Zum Beispiel könnte man meinen, dass für jede reelle Zahl $x$ gilt, dass $\cos(x)=\sqrt{1-\sin^2(x)}$. Denn man kann ja umformen
    \begin{align*}
        &&\cos(x)& =\sqrt{1-\sin^2(x)} \\
        & \to & \cos^2(x)& = 1-\sin^2(x) & (\text{beide Seiten quadrieren}) \\
        & \to & \cos^2(x) + \sin^2(x) &= 1
    \end{align*}
    und letzteres ist eine wohlbekannte wahre Aussage (die manchmal als „Satz des Pythagoras“\footnote{\href{https://de.wikipedia.org/wiki/Pythagoras}{Pythagoras (6. Jhd. v. Chr.)}} bezeichnet wird). Jedoch ist
        \[ \cos(\pi) = -1 \neq 1 = \sqrt{1-0^2} = \sqrt{1-\sin^2(\pi)} \]
    sodass irgendetwas nicht stimmen kann. Kannst du ausmachen, wie und wo genau sich der Fehler eingeschlichen hat?
\end{bem}


\begin{bem}[Weitere Anfängerfehler]
 Eine lange Liste von sowohl studentischen als auch dozentischen Fehlern, die ihm während seiner Lehrtätigkeit aufgefallen sind, hat Eric Schechter auf \href{https://math.vanderbilt.edu/schectex/commerrs/}{seiner Homepage} zusammengetragen.
\end{bem}





\subsection*{* Mehrfach-Äquivalenzen}


\begin{de}[„Die folgenden Aussagen sind äquivalent“] \label{def:tfae}
    Einige mathematische Sätze haben die Gestalt einer größeren Äquivalenzaussage und sehen etwa folgendermaßen aus:
    \begin{quote}
        Es seien \dots und es gelte \dots. Dann sind die folgenden Aussagen äquivalent:
        \begin{itemize}
            \item[(i)] \dots
            \item[(ii)] \dots
            \item[(iii)] \dots
            \item[(iv)] \dots
            \item[\dots]
        \end{itemize}
    \end{quote}
    Diese Satzstruktur kommt so häufig vor, dass sie im Englischen durch ``tfae'' abgekürzt wird (für ``the following are equivalent''). Sie besagt, dass je zwei der Aussagen (i), (ii), (iii), usw. zueinander äquivalent sind, also dass alle Äquivalenzen
        \[ \text{(i)$\leftrightarrow$(ii),\quad (i)$\leftrightarrow$(iii), \quad (ii)$\leftrightarrow$(iii),\quad (i)$\leftrightarrow$(iv),\quad (ii)$\leftrightarrow$(iv), \quad (iii)$\leftrightarrow$(iv),\quad \dots} \]
    gelten. Man sagt auch, die Aussagen seien „paarweise äquivalent“.
 \end{de}
 
 
\begin{bsp}
    Sei $D$ ein Dreieck in der euklidischen Ebene. Dann sind die folgenden Aussagen äquivalent:
    \begin{enumerate}[(i)]
        \item $D$ ist ein gleichseitiges Dreieck, d.h. alle Seiten von $D$ haben dieselbe Länge.
        \item Alle Innenwinkel von $D$ haben dieselbe Größe.
        \item Der Schwerpunkt von $D$ stimmt mit seinem Umkreismittelpunkt überein.
        \item Der Schwerpunkt von $D$ stimmt mit seinem Inkreismittelpunkt überein.
    \end{enumerate}
Würde man hier die Äquivalenz jedes Aussagenpaars per Hin- und Rückrichtung beweisen, müsste man insgesamt zwölf Implikationen beweisen. Mit der folgenden Beweistechnik lässt sich in solchen Fällen erheblich Arbeit einsparen:
\end{bsp}


\begin{satz}[Ringschluss] \label{ringschluss} \index{Ringschluss}
    Seien $n$ eine natürliche Zahl und $A_1,\dots , A_n$ eine Handvoll Aussagen, von denen du beweisen möchtest, dass sie paarweise äquivalent sind. Dann kannst du dies mit der Technik des \textbf{Ringschlusses} erledigen, indem du lediglich die Implikationen
        \[ A_1\to A_2,\quad A_2\to A_3,\quad \dots ,\quad A_{n-1}\to A_n,\quad A_n\to A_1 \]
    beweist. Auf diese Weise „schließt du einen Ring“ zwischen den Aussagen $A_1,\dots , A_n$.
\end{satz}


\begin{bew}
    Durch den Ringschluss wurden alle Implikationen im folgenden Diagramm bewiesen:
    \[\begin{tikzcd}
        &&& A_1 \ar[rrd] &&& \\
        &A_n\ar[rru]&&&& A_2 \ar[ld]& \\
        && \dots \ar[lu] && A_3 \ar[ll]&& 
    \end{tikzcd} \]
    Man sieht, dass sich nun von jeder Aussage mittels Zwischenschritten zu jeder anderen Aussage gelangen lässt, solange man nur lang genug „im Uhrzeigersinn läuft“. Wegen \cref{implikationtrans} gilt daher für je zwei Aussagen $B,C$ aus den $A_1,\dots , A_n$, dass $B\to C$ und $C\to B$, also insgesamt $B\leftrightarrow C$. Demzufolge sind je zwei beliebige Aussagen von $A_1,\dots , A_n$ zueinander äquivalent. \qed
\end{bew}


\begin{bsp} \label{bsp:ringschluss}
    Sei $n$ eine ganze Zahl. Dann sind die folgenden Aussagen äquivalent:
    \begin{enumerate}[(i)]
        \item Es ist $n\ge 1$.
        \item Für jede ganze Zahl $m$ ist $m+n>m$.
        \item Es gibt mindestens eine ganze Zahl $m$, für die $m+n>m$ gilt.
    \end{enumerate}
\end{bsp}


\begin{bew}
    (i)$\to$(ii): Es gelte (i) und es sei $m$ eine beliebige ganze Zahl. Dann ist
    \begin{align*}
        m+n & \ge m+1 && (\text{wegen (i)}) \\
        & > m
    \end{align*}
    (ii)$\to$(iii) ist trivial, da ganze Zahlen existieren. \\[0.5em]
    (iii)$\to$(i): Es gelte (iii). Dann gibt es eine ganze Zahl $m$ mit $m+n>m$. Subtraktion von $m$ liefert die Ungleichung $n>0$. Und da $n$ eine ganze Zahl ist, muss dann schon $n\ge 1$ gelten. \qed
\end{bew}


\begin{bem}
    \textbf{Achtung}: Ein gelegentlicher Anfängerirrtum besteht darin, zu denken, der Ringschluss müsse \emph{immer} in der Form (i)$\to$(ii), (ii)$\to$(iii), (iii)$\to$(i) durchgeführt werden. Das ist Unsinn und führt dazu, dass sich manche Anfänger einen Haufen unnötige Mehrarbeit aufhalsen.
    
    Genausogut kannst du etwa auch einen Ringschluss über die Implikationen (ii)$\to$(i), (iii)$\to$(ii) und (i)$\to$(iii) durchführen.
    
    Manchmal ist die Ringschluss-Methode auch unangebracht, wenn etwa die beiden Aussagen (ii) und (iii) so widerspenstig gegeneinander sind, dass keine Beweise für (ii)$\to$(iii) und (iii)$\to$(ii) in Sicht sind. In diesem Fall ist es vielleicht einfacher, den scheinbar längeren Weg zu gehen und (i)$\to$(ii), (ii)$\to$(i), (i)$\to$(iii) und (iii)$\to$(i) zu beweisen.
    \[\begin{tikzcd}[column sep=small]
        & (ii) \ar[rd] & \\
        (iii)\ar[ru] && (i)\ar[ll]
    \end{tikzcd}\qquad\text{anstelle von}\qquad\begin{tikzcd}[column sep=small]
        & (i) \ar[rd] & \\
        (iii)\ar[ru] && (ii)\ar[ll]    
    \end{tikzcd}\qquad\text{geht auch.}\]
    \[\text{Manchmal funktioniert aber nur}\qquad \begin{tikzcd}
        & (i) \ar[ld, bend right=20]  \ar[rd, bend right=20] & \\
        (iii)\ar[ru, bend right=20] && (ii) \ar[lu, bend right=20]
    \end{tikzcd}\]
    Entscheidend ist, dass du am Ende so viele Implikationen bewiesen hat, dass man mittels Zwischenschritten von jeder Aussage zu jeder anderen Aussage gelangen kann. \\[0.5em]
    Wenn du eine längere Äquivalenzaussage beweisen möchtest, solltest du immer Ausschau nach Implikationen halten, die „geschenkt“ sind, d.h. deren Beweis besonders naheliegend und einfach ist (in \cref{bsp:ringschluss} war das die Implikation (ii)$\to$(iii)). Daran kannst du dann deine Beweisstrategie orientieren. Ein berüchtigter Äquivalenzbeweis in der LA1-Vorlesung vom Wintersemester 2016/17 verlief so kompliziert, dass der Dozent zu Beginn seines Beweises einen „Plan“ aufgeschrieben hat, um den Studis die Orientierung zu erleichtern.
    \begin{figure}[ht]
        \includegraphics[width=10cm]{./_img/equivbeweis.jpeg}
        \centering \caption{Eine Äquivalenzaussage aus der LA1 vom WS16/17. Hier sind die Implikationen (ii)$\to$(iii) und (ii)$\to$(iv) „geschenkt“.}
    \end{figure}
\end{bem}


\begin{bem}[Guter Stil]
    Wenn du eine Aussage der Gestalt „Folgende Aussagen sind äquivalent\dots“ beweist, solltest du den Beweis jeder einzelnen Implikation mit „(i)$\to$(ii)“, „(iii)$\to$(i)“ oder Ähnlichem betiteln, damit deinem Leser jederzeit klar ist, welche Implikation gerade Thema ist.
\end{bem}





\section{Und und Oder}


In diesem Abschnitt seien $A,B$ stets zwei beliebige Aussagen.


\begin{axiom}[*]\label{undoderaxiome}
    Für je zwei Aussagen $A,B$ gelten:
    \begin{align*}
        (A\land B) & \to A & A & \to (A\lor B) \\
        (A\land B) & \to B & B & \to (A\lor B)
    \end{align*}
    Mit anderen Worten: Aus $A\land B$ folgen sowohl $A$ als auch $B$; und $A$ und $B$ sind wiederum hinreichende Bedingungen für $A\lor B$.
\end{axiom}


\begin{bem}
    Wenn dir die Aussage $A\land B$ gegeben ist, kannst du das also so behandeln, als seien dir zwei Aussagen gegeben, nämlich $A$ und $B$.
    
    Und wenn du $A\lor B$ beweisen möchtest, würde es schon reichen, wenn du $A$ beweist oder wenn du $B$ beweist.
\end{bem}


\begin{axiom}[Und-Aussagen beweisen] \label{undbeweise}
    Um die Aussage $A\land B$ zu beweisen, genügt es, sowohl $A$ als auch $B$ zu beweisen:
    \[\begin{tabular}{r}
        $A$ \\
        $B$ \\
        \hline 
        $A\land B$
    \end{tabular} \]
    Mit anderen Worten: Wenn „$A\land B$“ auf deiner „zu zeigen“-Liste steht, kannst du das als zwei separate Ziele behandeln, nämlich musst du einerseits $A$ und andererseits $B$ beweisen.
\end{axiom}


\begin{bsp}[*]
    Die $25$ ist eine Quadratzahl, die sich als Summe zweier echt kleinerer Quadratzahlen schreiben lässt. Außerdem ist sie die kleinste Quadratzahl mit dieser Eigenschaft.
\end{bsp}


\begin{bew}
    (1) Aus
        \[ 25=5^2 \qquad\text{und}\qquad 25 = 9 + 16 = 3^2 + 4^2 \]
    folgt, dass $25$ eine Quadratzahl ist, die sich als Summe zweier echt kleinerer Quadratzahlen schreiben lässt.
    
    (2) Die einzigen Quadratzahlen $\neq 0$, die noch kleiner als $25$ sind, sind
        \[ 1\qquad 4\qquad 9\qquad 16 \]
    Wäre eine dieser vier Zahlen eine Summe zweier echt kleinerer Quadratzahlen, so müssten auch diese beiden Summanden in der Liste dieser vier Zahlen vorkommen. Aber alle möglichen Summationen
    \begin{align*}
        1+1 & = 2 & 1+4 & = 5 \\
        1+ 9 & = 10 & 1+16 & = 17 \\
        4 + 4 & = 8 & 4+9 & = 13 \\
        4+16 & = 20 & 9+9 & = 18
    \end{align*}
    ergeben keine Quadratzahl. \qed
\end{bew}


\begin{bem}[*]
    Drei natürliche Zahlen $a,b,c$, für die $a^2+b^2=c^2$ gilt, heißen ein \textbf{pythagoräisches Tripel}. Also wurde gerade bewiesen, dass $(2,3,5)$ das kleinste pythagoräische Tripel ist. Nach dem berühmten \href{https://de.wikipedia.org/wiki/Gro\%C3\%9Fer_Fermatscher_Satz}{Großen Satz von Fermat}\footnote{\href{https://de.wikipedia.org/wiki/Pierre_de_Fermat}{Pierre de Fermat (1607-1665)}} besitzt die Gleichung $a^n+b^n=c^n$ für eine natürliche Zahl $n\ge 3$ keine positive ganzzahlige Lösung.
\end{bem}


\begin{axiom}[Beweis mit Fallunterscheidung] \label{fallunterscheidung} \index{Fallunterscheidung (in einem Beweis)}
    Sei $X$ eine Aussage, die du beweisen möchtest. Außerdem sei gegeben, dass $A\lor B$ gilt\footnote{Im Englischen sagt man: ``the two cases $A$ and $B$ are \emph{exhausting}''.}. Dann kannst du $X$ mittels einer \textbf{Fallunterscheidung} beweisen, indem du sowohl zeigst, dass $A\to X$ gilt, als auch, dass $B\to X$ gilt.
\end{axiom}


\begin{bsp} \label{bsp:fallunterscheidung}
    Für jede natürliche Zahl $n$ ist $n\cdot (n+1)$ eine gerade Zahl.
\end{bsp}


\begin{bew}
    Da jede natürliche Zahl entweder gerade oder ungerade ist, führe ich eine Fallunterscheidung durch:
    \begin{enumerate}[label={(Fall \arabic*):}, labelindent=1.5em, leftmargin=!, labelwidth=\widthof{(Fall 2):}]
        \item $n$ ist eine gerade Zahl. In diesem Fall ist $n\cdot (n+1)$ als Vielfaches der geraden Zahl $n$ ebenfalls eine gerade Zahl.
        \item $n$ ist eine ungerade Zahl. In diesem Fall ist $n+1$ eine gerade Zahl, sodass $n\cdot (n+1)$ als Vielfaches der Zahl $n+1$ ebenfalls eine gerade Zahl ist.
    \end{enumerate}
    Also ist $n\cdot(n+1)$ in jedem Fall gerade. \qed
\end{bew}





\section{Quantoren}


In diesem Abschnitt sei $E(x)$ stets ein einstelliges Prädikat.


\begin{axiom}[*] \label{quantorenaxiom}
    Sei $a$ irgendein Objekt. Dann gelten die folgenden beiden Implikationen:
    \begin{align*}
         \forall x: E(x) \quad& \to\quad E(a) \\
         E(a) \quad & \to\quad \exists x: E(x)
    \end{align*}
\end{axiom}

 
\begin{bsp}[*] \quad
    \begin{enumerate}
        \item Weil jede positive reelle Zahl eine Qudaratwurzel besitzt, existiert insbesondere auch eine reelle Quadratwurzel der Zwei.
        \item Es existieren $p,q,x,y\in \N_{\ge 1}$ mit $x^p-y^q=1$, denn es ist $3^2-2^3=1$. (Nach dem \href{https://de.wikipedia.org/wiki/Catalansche_Vermutung}{Satz von Catalan-Mihăilescu} gibt es aber keine weiteren Möglichkeiten)
    \end{enumerate}
\end{bsp}


\begin{satz}[Beweis per Beispiel] \label{beweisperbsp} \index{Beispiel (in einem Beweis)}
    Du kannst die Existenzaussage $\exists x: E(x)$ dadurch beweisen, dass du ein konkretes Objekt $a$ findest, das die Eigenschaft $E$ besitzt. Man nennt dann $a$ ein \textbf{Beispiel} für die Existenzaussage $\exists x: E(x)$.
\end{satz}


\begin{bew}
    Ergibt sich direkt aus der Formel $E(a)\to\exists x:E(x)$. \qed
\end{bew}


\begin{bsp}
    Es gibt eine natürliche Zahl $n\ge 1$, die gleich der Summe ihrer echten Teiler ist.\footnote{Zahlen, die gleich der Summe ihrer echten Teiler sind, heißen auch \href{https://de.wikipedia.org/wiki/Vollkommene_Zahl}{vollkommene Zahlen}.}
\end{bsp}


\begin{bew}
    Ein Beispiel ist die Zahl $28$. Denn die echten Teiler der $28$ sind genau
        \[ 1 \qquad 2 \qquad 4 \qquad 7 \qquad 14 \]
    und es $1+2+4+7+14=28$. \qed
\end{bew}

  
\begin{bem}
    Lässt sich eine Existenzaussage mit einem Beispiel beweisen, so ist es eigentlich schlechter Stil, in einem Buch oder einem Vortrag nur die Existenzaussage anzugeben. Beispielsweise ist ja die Information „Die $28$ ist gleich der Summe ihrer echten Teiler“ umfangreicher als die Information „Es gibt eine natürliche Zahl, die gleich der Summe ihrer echten Teiler ist“. Du solltest dir nicht einfach nur die Existenzaussage merken, sondern, sofern es welche gibt, auch ein paar Beispiele und deren Konstruktion im Hinterkopf behalten. Auch die essenzielle Aussage des Satzes von Euklid \cref{euklid} besteht weniger in der Aussage „Es gibt eine Primzahl, die größer als $n$ ist“ als vielmehr in der trickreichen Art und Weise, wie eine solche Primzahl aufgespürt wird.
    
    Es gibt allerdings auch Situationen, in denen eine Existenzaussage beweisbar ist, obwohl es unmöglich ist, konkrete Beispiele zu geben. Falls in einer Vorlesung keine Beispiele gegeben werden (was bedauerlicherweise recht häufig vorkommt und vom Zeitdruck im Vorlesungsbetrieb verschuldet ist), solltest du beim Prof. nachhaken, ob er/sie vielleicht deshalb keine Beispiele bringt, weil es gar keine gibt oder die wenigen bekannten Beispiele zu kompliziert und zeitaufwendig sind. Gute Bücher und Vorlesungen erkennt man daran, dass sie, wenn sie keine Beispiele geben, auch erklären, warum.
\end{bem}


\begin{axiom}[Allaussagen an einem „beliebigen“ Objekt nachweisen]\label{allbeweis} \index{beliebig}
    Um die Allaussage $\forall x: E(x)$ zu beweisen, kannst du folgendermaßen vorgehen:
    
    Führe eine Variable $a$, die bislang noch nicht im Beweis verwendet wurde und ein \emph{beliebiges} Objekt vom Typ der Variablen $x$ bezeichnen soll, ein und beweise nun, dass $a$ die Eigenschaft $E$ besitzt.
\end{axiom}


\begin{bsp} \label{bsp:allbeweis}
    Für jede reelle Zahl $y\neq 1$ existiert eine reelle Zahl $x$ mit $y=\frac{x+1}{x-2}$.
\end{bsp}


\begin{bew}
    Sei $y$ eine beliebige reelle Zahl $\neq 1$. Wegen $y\neq 1$ ist $y-1\neq 0$, sodass durch $x:= \frac{1+2y}{y-1}$ eine wohldefinierte reelle Zahl gegeben ist. Nun rechnet man nach, dass $x\neq 2$ und $\frac{x+1}{x-2}=y$. \qed
\end{bew}

  
\begin{bem}[Signalwörter]
    Ein Beweis einer Allaussage beginnt meist mit Floskeln wie „Sei $x$ ein beliebiges\dots“ oder „Die Zahl $n$ sei beliebig aber fest“. Viele Texte lassen das Signalwort „beliebig“ auch weg und beginnen schlicht mit sowas wie „Sei $x$ eine reelle Zahl. Dann \dots“. Sie setzen vom Leser voraus, dass er erkennt, dass hier gerade der Beweis einer Allaussage beginnt.
\end{bem}
  
 
\begin{bem}[*]
    Achte darauf, dass die von dir eingeführte Variable, die das „beliebige“ Objekt bezeichnen soll, auch wirklich nirgendwo sonst im bisherigen Beweis aufgetaucht ist, also auch wirklich „beliebig“ ist. Ansonsten könnte dir ein Fehler wie der folgende passieren:
    \begin{bsp}[*]
        Es gibt eine natürliche Zahl $n$, die größergleich jede andere natürliche Zahl ist.
    \end{bsp}
    \begin{bew}
        Setze $n=0$. Es bleibt zu zeigen, dass jede natürliche Zahl kleinergleich $n$ ist. Dazu sei $n$ eine beliebige natürliche Zahl. Weil bekanntlich stets $n\le n$ gilt, ist also jede beliebige Zahl kleinergleich $n$.
    \end{bew}
        Ganz so offensichtliche Fehler passieren natürlich eher selten. Subtilere Fehler dieser Art kommen aber durchaus vor.
\end{bem}


\begin{bsp}[* Satz von Euklid\footnote{\href{https://de.wikipedia.org/wiki/Euklid}{Euklid (ca. 3. Jhd. v. Chr.)}}] \label{euklid}
    Für jede natürliche Zahl $n$ gibt es eine Primzahl $P$, die größer als $n$ ist.
\end{bsp}


\begin{bem}
    Die logische Struktur dieses Satzes ist
        \[ \forall\ (\text{natürliche Zahl $n$})\ \exists\ (\text{Primzahl $P$}):\ P>n \]
    Da es sich insgesamt um eine Allaussage handelt, sollte der Beweis mit „Sei $n$ eine beliebige natürliche Zahl“ beginnen. Da daraufhin die Existenzaussage
        \[ \exists P\in \N:\ \text{$P$ ist eine Primzahl und größer als $n$} \]
    übrig bleibt, fährt der Beweis nun mit der geschickten Konstruktion eines Beispiels fort:
\end{bem}


\begin{bew}
    Sei $n$ eine beliebige natürliche Zahl. Da es nur endlich viele natürliche Zahlen gibt, die $\le n$ sind, gibt es auch nur endlich viele Primzahlen, die $\le n$ sind. Seien $k$ deren Anzahl und $p_1,\dots , p_k$ diese Primzahlen. Betrachte die Zahl\footnote{Im Fall $k=0$ ist $N=2$, weil dann ein „leeres Produkt“ involviert ist, siehe \cref{mehrfachprodukt}.}
        \[ N := p_1\cdot\ldots\cdot p_k + 1 \]
    Dann lässt $N$ bei der Division durch $p_1,\dots , p_k$ jedes Mal den Rest Eins übrig, ist also nicht durch $p_1,\dots , p_k$ teilbar. Wegen $N\ge 2$ muss $N$ gemäß dem Fundamentalsatz der Arithmetik aber mindestens einen Primteiler $P$ besitzen. Da $P$ keines der $p_1,\dots ,p_k$ sein kann, aber die $p_1,\dots , p_k$ alle Primzahlen sind, die $\le n$ sind, muss $P$ größer als $n$ sein. \qed
\end{bew}


\begin{axiom}[* Verwenden von Existenzaussagen] \label{exverwendung}
    Sofern dir in einem Beweis eine Aussage der Gestalt $\exists x: E(x)$ gegeben ist, kannst du eine Variable $a$, die bisher noch nirgends im Beweis aufgetaucht ist, einführen, und die Aussage $E(a)$ als gegeben annehmen.
\end{axiom}
  
  
\begin{bsp} \label{bsp:exverwendung}
    Die Gleichung $x^5=x+1$ besitzt eine reelle Lösung.\footnote{vgl. \cref{zeichendefinieren}}
\end{bsp}


\begin{bew}
    Betrachte die reelle Funktion
        \[ f(x) = x^5-x-1 \]
    Dann gilt $f(1)=-1$ und $f(2)=29$. Somit folgt aus dem Zwischenwertsatz der Analysis, dass $f$ eine Nullstelle irgendwo zwischen $1$ und $2$ haben muss. Sei $\xi$ eine solche Nullstelle (hier wird \cref{exverwendung} genutzt). Dann gilt $\xi^5-\xi-1=0$, also $\xi^5=\xi +1$. \qed
\end{bew}


\begin{satz}[* Vertauschbarkeit von Quantoren derselben Sorte] \label{quantorentausch}
    Sei $R$ ein zweistelliges Prädikat. Dann gilt:
    \begin{align*}
        \forall x\ \forall y: R(x,y) \quad &\leftrightarrow\quad \forall y\ \forall x: R(x,y) \\
        \exists x\ \exists y: R(x,y) \quad &\leftrightarrow\quad  \exists y\ \exists x: R(x,y)
    \end{align*}
\end{satz}


\begin{bew}
    Ich beweise jeweils nur die Hinrichtung „$\to$“. Die Rückrichtung wird, unter Vertauschung der Rollen von $x$ und $y$, analog bewiesen.
    \begin{itemize}
        \item[„$\forall$“:] Seien $a,b$ zwei beliebige Objekte und es gelte $\forall x\ \forall y: R(x,y)$. Mit \cref{quantorenaxiom} folgt durch Einsetzen von $a$, dass $\forall y: R(a,y)$, und durch Einsetzen von $b$, dass $R(a,b)$. Da $a$ beliebig gewählt war, gilt somit sogar $\forall x: R(x,b)$. Und da auch $b$ beliebig gewählt war, folgt hieraus, dass $\forall y\ \forall x: R(x,y)$.
        \item[„$\exists$“:] Es sei angenommen, dass $\exists x\ \exists y: R(x,y)$ gilt. Dann gibt es ein Objekt $a$, für das $\exists y: R(a,y)$ gilt (hier wird \cref{exverwendung} genutzt). Somit gibt es auch ein Objekt $b$, für das $R(a,b)$ gilt. Wegen $R(a,b)$ gilt insbesondere $\exists x: R(x,b)$ und daraus folgt wiederum $\exists y\ \exists x: R(x,y)$. \qed
    \end{itemize}
\end{bew}


\begin{satz}[* Quantoren verschiedener Art sind nicht miteinander vertauschbar!]
    Sei $R$ ein zweistelliges Prädikat. Dann gilt zwar
        \[ \exists x\ \forall y: R(x,y) \quad\to\quad \forall y\ \exists x: R(x,y) \]  
    die umgekehrte Implikation „$\leftarrow$“ ist im Allgemeinen aber falsch, vgl. \cref{quantorreihenfolge}.
\end{satz}


\begin{bew}
    Sei $b$ ein beliebiges Objekt und es gelte $\exists x\ \forall y: R(x,y)$. Dann gibt es ein Objekt $a$, für das $\forall y: R(a,y)$ gilt. Also gilt insbesondere $R(a,b)$. Daraus folgt $\exists x: R(x,b)$ und da das Objekt $b$ beliebig gewählt war, impliziert dies, dass $\forall y\ \exists x: R(x,y)$. \qed
\end{bew}




\subsection*{Eindeutigkeitsbeweise}


Im Logikkapitel wurde thematisiert, wie sich der Eindeutgikeitsquantor „$\exists !$“ aus dem Allquantor und dem Existenzquantor zusammensetzt:
    \[ \underbrace{\exists x:\ E(x)}_{\text{Es gibt mindestens ein\dots}}\quad \land\quad \underbrace{\forall y,z:\ (E(y)\land E(z)) \to y=z}_{\text{Es gibt höchstens ein\dots}}\]
Diese Und-Aussage kann gemäß \cref{undbeweise} als zwei separate Aussagen behandelt werden:


\begin{satz}[Existenz- und Eindeutigkeitsbeweis] \label{eindbeweis} \index{Eindeutigkeitsbeweis}
    Wenn du eine Aussage der Form $\exists ! x: E(x)$ beweisen möchtest, kannst du deinen Beweis in einen Existenz-Teil und einen Eindeutigkeit-Teil aufteilen:
    \begin{itemize}
        \item Im Existenz-Teil beweist du, dass es mindestens ein Objekt gibt, das die Eigenschaft $E$ besitzt (z.B. durch Angabe eines Beispiels).
        \item Im Eindeutigkeit-Teil beweist du, dass je zwei Objekte, die die Eigenschaft $E$ besitzen, identisch sind.
    \end{itemize}
    Dabei spielt es keine Rolle, ob du erst den Existenz- und dann den Eindeutigkeit-Teil aufschreibst oder umgekehrt.
\end{satz}


\begin{bsp} \label{bsp:eindbeweis}
    Es gibt genau eine reelle Zahl $a$, die für jede reelle Zahl $x$ die Gleichung $a\cdot x=a$ erfüllt.
\end{bsp}


\begin{bew}
    (Eindeutigkeit): Seien $a,b$ zwei reelle Zahlen derart, dass für jede reelle Zahl $x$ gilt, dass $ax=a$ und $bx=b$. Nun ist
    \begin{align*}
        a & = a\cdot b & (\text{wegen der besonderen Eigenschaft von $a$}) \\
        & = b \cdot a  \\
        & = b & (\text{wegen der besonderen Eigenschaft von $b$})
    \end{align*}
    (Existenz): Da für jede reelle Zahl $x$ gilt, dass $0\cdot x=0$, erfüllt die Zahl $0$ die gewünschte Eigenschaft. \qed
\end{bew}


\begin{bem}[Guter Stil]
    Du solltest den Existenz-Teil und den Eindeutigkeit-Teil deines Beweises immer auch als solchen betiteln, so wie es gerade im Beispiel geschah.
\end{bem}


\begin{bem}[Wechselspiel zwischen Formeln und Umgangssprache]
    Anfänger neigen dazu, in ihren Beweisen möglichst alle Sachverhalte in Formelsprache auszudrücken und logische Schritte möglichst rechnerisch, als symbolische Manipulation gewisser Formelterme, durchzuführen. Versuche stattdessen, in deinen Beweisen ein Gleichgewicht aus Formeln und Umgangssprache herzustellen. Wo ein kurzer deutscher Satz dasselbe sagt wie eine Formel, ziehe in Erwägung, den deutschen Satz hinzuschreiben. Gedruckte Beweise (wie etwa in diesem Skript) enthalten oft mehr Fließtext als handschriftliche Beweise (wie sie etwa dein Prof. an die Tafel schreibt).
\end{bem}





\begin{comment}
\section{Induktionsbeweise}


Im Sonderfall, dass sich Allaussagen auf natürliche Zahlen beziehen, stehen Beweistechniken zur Verfügung, die zur Gattung der „Induktionsbeweise“ gehören.


\begin{bem}[Erklärung des Induktionsbeweises]
    Die natürlichen Zahlen lassen sich „abzählen“. Das heißt, wenn ich mit der Null starte und dann anfange zu zählen „Null, Eins, Zwei, Drei,\dots“, so werde ich zwar nie fertig, da es ja unendlich viele Zahlen gibt -- ich werde aber jede beliebige Zahl nach hinreichend langer Zeit abgezählt haben. Mit anderen Worten: Der Zählprozess schöpft die Gesamtheit aller natürlichen Zahlen vollständig aus. Oder: Die Gesamtheit aller natürlichen Zahlen kann durch den Zählprozess „vollständig abgetragen“ werden. \\
    Zählen heißt, mit der Null zu starten und nach jeder genannten Zahl mit der kleinsten noch nicht genannten Zahl fortzufahren. Die natürlichen Zahlen lassen sich also restlos durch die beiden Operationen
    \begin{itemize}
        \item Mit der Null beginnen.
        \item Sofern man schon ein paar Zahlen abgezählt hat, mit der kleinsten Zahl fortfahren, die noch nicht drankam.
    \end{itemize}
    „abtragen“. Diese Eigenschaft der natürlichen Zahlen liegt dem Prinzip des Induktionsbeweises zugrunde.
\end{bem}


\begin{axiom}[Induktionsbeweis]
    Möchtest du beweisen, dass jede natürliche Zahl die Eigenschaft $E$ besitzt, so kannst du dies folgendermaßen erledigen:
    \begin{itemize}
        \item Im sogenannten \textbf{Induktionsanfang} beweist du, dass $E(0)$ gilt. (Je nach Kontext kann der Induktionsanfang auch bei $n=1$ oder sogar noch höher stattfinden)
        \item Im sogenannten \textbf{Induktionsschritt} fixierst du eine beliebige natürliche Zahl $n\ge 1$ und nimmst an, dass jede Zahl, die kleiner als $n$ ist, die Eigenschaft $E$ besitzt (diese Annahme heißt \textbf{Induktionsannahme} oder auch \textbf{Induktionsvoraussetzung}). Mithilfe dieser Annahme beweist du nun, dass auch $n$ die Eigenschaft $E$ besitzt.
        %\[ \forall n \ge 1:\ (\forall k\le n: E(k)) \to E(n) \]
    \end{itemize}
\end{axiom}


\begin{bsp}[Division mit Rest]
    Seien $a,b$ zwei natürliche Zahlen und $b\neq 0$. Dann gibt es eindeutig bestimmte natürliche Zahlen $q,r$, für die gilt
        \[ a=qb+r \qquad\text{und}\qquad r< b \]
\end{bsp}


\begin{bew}
    (Existenz): Im Fall $a<b$ kann man einfach $q=0$ und $r=a$ setzen. Deshalb sei ab sofort angenommen, dass $a\ge b$ gilt. Der Beweis geschieht nun per Induktion über $a$. \\[0.5em]
    (Induktionsanfang) Im Fall $a=0$ folgt aus $b\neq 0$, dass $a<b$, sodass man einfach $q=0$ und $r=a$ wählen kann. \\
    (Induktionsschritt) Wegen $a\ge b$ ist auch $a-b$ eine natürliche Zahl. Wegen $a-b< a$ gibt es nach Induktionsvoraussetzung eine natürliche Zahl $p$ und eine Zahl $r<b$, für die
        \[ a-b = pb + r \]
    gilt. Setzt man $q:=p+1$, so folgt
        \[ a = (p+1)b + r = qb+r\]
    (Eindeutigkeit): Seien $q,q,r,r'$ natürliche Zahlen mit $r,r'<b$ und $a=qb+r=q'b+r'$. \TODO
\end{bew}


\begin{bem}[Guter Stil]
    An diesem Beweis werden eine Reihe wichtiger Punkte deutlich.
    \begin{itemize}
        \item Wenn im Beweis mehrere Variablen vom Typ „natürliche Zahl“ auftreten, musst du deutlich machen, über welche Variable dein Induktionsbeweis verläuft.
        \item Induktionsanfang und Induktionsschritt musst du klar und deutlich kennzeichnen. Deinem Leser muss zu jedem Zeitpunkt klar sein, ob er sich gerade im Induktionsschritt befindet.
        \item Wenn du im Induktionsschritt von der Induktionsannahme Gebrauch musst, musst du das deutlich hervorheben. Diese Stelle ist meist das Herzstück des ganzen Beweises!
    \end{itemize}
\end{bem}


\begin{bem}
    Es gibt einen Haufen Varianten dieser Induktionstechnik. Oftmals wird gar nicht benötigt, dass \emph{alle} kleineren Zahlen als $n$ die Eigenschaft $E$ besitzen und man kann beweisen, dass $E(n)$ allein schon aus $E(n-1)$ folgt. Manchmal führt man den Induktionsanfang auch für $n=1$ oder $n=2$ durch, sofern man etwa sowieso nur beweisen möchte, dass die Eigenschaft für alle Zahlen $\ge 2$ gilt.
    
    Die allgemeine Technik des Induktionsbeweises beschränkt sich nicht nur auf natürliche Zahlen, sondern auf alle möglichen Objekte, die in gewisser Weise „induktiv“ definiert sind. Beispielsweise besagt das \emph{Fundierungsaxiom} der Mengenlehre, dass auch Aussagen über alle Mengen mit einer Induktionstechnik bewiesen werden können. Mehr darüber kannst du in Büchern und Vorlesungen über mathematische Logik und Mengenlehre lernen.
\end{bem}
\end{comment}


\section{Widerlegen}


Alle bisher besprochenen Beweistechniken zielten darauf ab, die „Wahrheit“ von Aussagen zu etablieren. Nun soll es darum gehen, wie man von einer Aussage nachweisen kann, dass sie „falsch“ ist.

In diesem Abschnitt seien $A,B$ stets zwei Aussagen.


\begin{de}[Widerlegung] \index{Widerlegung}
    Eine \textbf{Widerlegung} der Aussage $A$ ist ein Beweis ihrer Negation $\neg A$.
    
    Anstelle von „Es gilt $\neg A$“ schreiben wir auch „$A$ ist falsch“\footnote{Beachte, dass dies erst einmal nichts mit den Wahrheitswerten aus \cref{def:interpretation} zu tun haben muss, vgl. \cref{def:esgilt}}.
\end{de}


\subsection*{Indirekt Argumentieren}


\begin{axiom}[Indirekte Widerlegung] \label{reductio}
    Es gilt:
    \[\begin{tabular}{r}
        $A\to B$ \\
        $\neg B$ \\ \hline
        $\neg A$
    \end{tabular} \]
    Mit anderen Worten: Wenn aus $A$ etwas Falsches folgt, muss $A$ selbst falsch sein.
    
    Du kannst die Aussage $A$ also dadurch widerlegen, dass du eine falsche Aussage $B$ findest, die aus $A$ folgen würde. Man nennt diese Technik eine \textbf{indirekte Widerlegung} oder auch \textbf{Reductio ad absurdum} (latein für „Rückführung auf das Widersinnige“).
\end{axiom}


\begin{bsp} \label{bsp:reductio}
    $198$ ist nicht durch $17$ teilbar.
\end{bsp}


\begin{bew}
    Es ist $187=11\cdot 17$. Wäre $198$ durch $17$ teilbar, so auch die Differenz $198-187 = 11$. Aber $11$ ist kein Vielfaches von $17$. \qed
\end{bew}


\begin{satz}[Widerlegung einer Allaussage per Gegenbeispiel] \label{gegenbeispiel} \index{Gegenbeispiel}
    Wenn du eine Aussage der Gestalt $\forall x: E(x)$ widerlegen möchtest, genügt es, irgendein Objekt $a$ zu finden, für das du $\neg E(a)$ beweisen kannst. Man nennt dann das Objekt $a$ ein \textbf{Gegenbeispiel} zur Allaussage $\forall x: E(x)$.
\end{satz}


\begin{bew}
    Angenommen, es wurde $\neg E(a)$ bewiesen. Wegen $(\forall x: E(x)) \to E(a)$ würde dann aus $\forall x:E(x)$ eine falsche Aussage folgen, sodass $\forall x:E(x)$ falsch sein muss. \qed
\end{bew}

 
\begin{bsp}
    Nicht jeder Mensch findet im Leben die große Liebe.
\end{bsp}


\begin{bew}
    Schauen wir uns Franz Schubert an. Mit Mitte Zwanzig an der Syphilis erkrankt, mit 31 Jahren gestorben, war es dem Armen nicht leicht gemacht, einen Herzenspartner zu finden. Mehr als kurzzeitige Liebschaften, die er nicht frei ausleben konnte, waren dem Wiener Komponisten zu Lebzeiten nicht vergönnt. Ich meine, hör dir seine \href{https://youtu.be/F6I6Y1LhMKo?t=1665}{Winterreise} nur mal an! -- \qed
\end{bew}


\begin{satz}[* Implikationen widerlegen]
    Du kannst die Implikation $A\to B$ dadurch widerlegen, dass du beweist, dass $A$ und $\neg B$ gelten.
\end{satz}


\begin{bew}
    Angenommen, es wurden $A$ und $\neg B$ bewiesen. Da $A$ gilt, würde dann aus $A\to B$ die falsche Aussage $B$ folgen. Gemäß \cref{reductio} ist dadurch $A\to B$ widerlegt. \qed
\end{bew}


\begin{bem}[*]
    Die indirekte Widerlegung basiert darauf, dass eine Aussage, aus der etwas Falsches folgt, nicht stimmen kann. Dass aus einer Aussage etwas Wahres folgt, lässt dagegen keinen Rückschluss auf ihren Wahrheitsgehalt zu, da nach \cref{wahresausbeliebigem} ja Wahres aus Beliebigem folgt.
    
    Betrachte z.B. die (falsche) Aussage $A=$ „Jedes Kind weiß, dass die Summe der Zahlen $1$ bis $100$ gleich $5050$ ist“. Dann folgt aus $A$, dass auch der neunjährige Gauß\footnote{\href{https://de.wikipedia.org/wiki/Carl_Friedrich_Gau\%C3\%9F}{Carl Friedrich Gauß (1777-1855)}} dies wusste, was der \href{https://de.wikipedia.org/wiki/Gau\%C3\%9Fsche_Summenformel#Geschichte_der_Bezeichnung}{Anekdote} zufolge sogar eine wahre Aussage ist. Das ändert allerdings nichts daran, dass $A$ wohl trotzdem falsch ist.
\end{bem}



\begin{de} \index{Kontraposition}
 Die Implikation $\neg B \to \neg A$ heißt die \textbf{Kontraposition} der Implikation $A\to B$. 
\end{de}


\begin{bsp}
    Die Kontraposition der Aussage „Wenn ich krank bin, bleibe ich zuhause“ ist „Wenn ich nicht zuhause bleibe, bin ich nicht krank“.
\end{bsp}


\begin{satz}[Der indirekte Beweis] \label{indirekterbeweis} \index{Indirekter Beweis}
    Die Implikation $\neg A\to \neg B$ kannst du dadurch beweisen, dass du stattdessen die Implikation $B\to A$ beweist. Diese Technik heißt \textbf{indirekter Beweis} oder auch \textbf{Beweis per Kontraposition}.
\end{satz}


\begin{bew}
    Angenommen, es wurde $B\to A$ bewiesen. Dass nun $\neg A\to \neg B$ gilt, zeige ich per direktem Beweis. Dazu sei angenommen, dass $\neg A$ gilt. Wegen $B\to A$ würde dann aus $B$ die falsche Aussage $A$ folgen. Gemäß \cref{reductio} muss also $B$ falsch sein. \qed
\end{bew}


\begin{bsp}
    Sei $n$ eine natürliche Zahl. Sofern $n$ keine Quadratzahl ist, ist auch $n^3$ keine Quadratzahl.
\end{bsp}


\begin{bew}
    Beweis per Kontraposition. Sei $n^3$ eine Quadratzahl, d.h. es gebe eine natürliche Zahl $m$ mit $m^2=n^3$. Es folgt $n=n^3/n^2=m^2/n^2 = (m/n)^2$. Nach dem sogenannten „Satz von der rationalen Nullstelle“ ist jede rationale Zahl, deren Quadrat eine ganze Zahl ist, selbst bereits eine ganze Zahl. Also ist $n=(m/n)^2$ eine Quadratzahl. \qed
\end{bew}
  
  
\begin{bem}[Guter Stil]
    Wenn du einen indirekten Beweis führst, solltest du dies ankündigen, beispielsweise mit „Ich führe einen indirekten Beweis“, „Der Beweis geschieht indirekt“ oder „Beweis per Kontraposition:“.
\end{bem}





\subsection*{Widersprüche}


\begin{de} \index{Widerspruch}
    Ein \textbf{Widerspruch} ist eine Aussage der Gestalt $A\land \neg A$.
\end{de}


\begin{bsp}
    Falls $A$ die Aussage „Schrödingers Katze geht es gut“ ist, so besagt $A\land \neg A$, dass es der Katze sowohl gut geht als auch nicht gut geht.
\end{bsp}


\begin{axiom}[Satz vom Widerspruch] \index{Satz vom Widerspruch}
    Für jede Aussage $A$ gilt
        \[ \neg(A\land \neg A) \]
    Mit anderen Worten: Jeder Widerspruch ist eine falsche Aussage.
\end{axiom}


\begin{satz}[Der Widerspruchsbeweis] \label{widerspruchsbeweis} \index{Widerspruchsbeweis}
    Du kannst die Aussage $A$ dadurch wiederlegen, dass du aus ihr einen Widerspruch herleitest. Diese Beweistechnik heißt \textbf{Widerspruchsbeweis}.
\end{satz}


\begin{bew}
    Es sei angenommen, dass aus $A$ ein Widerspruch der Gestalt $B\land \neg B$ folgt. Nach dem Satz vom Widerspruch gilt $\neg (B\land \neg B)$, sodass aus $A$ eine falsche Aussage folgt. Wegen \cref{reductio} ist $A$ somit falsch. \qed
\end{bew}

  
\begin{bsp} \label{bsp:widerspruchsbeweis}
    Unter den positiven reellen Zahlen gibt es keine kleinste.
\end{bsp}


\begin{bew}
    Für einen Widerspruchsbeweis sei angenommen, es gäbe eine kleinste positive reelle Zahl $x$. Da $x$ positiv ist, wäre auch $\frac{x}{2}$ eine positive reelle Zahl. Ferner wäre $\frac{x}{2}<x$. Aber dies widerspräche der Annahme, dass $x$ die kleinste positive reelle Zahl sei. \qed
\end{bew}
  
  
\begin{bem}[* Indirekte Widerlegung vs. Widerspruchsbeweis]
    Meist wird auch die indirekte Widerlegung einer Aussage „Widerspruchsbeweis“ genannt. Aufgrund des Satzes vom Widerspruch ist tatsächlich jeder Widerspruchsbeweis auch eine indirekte Widerlegung (siehe Herleitung von \cref{widerspruchsbeweis}); andererseits kann jede indirekte Widerlegung auch (umständlicherweise) als ein Widerspruchsbeweis formuliert werden, denn mit $A\to B$ und $\neg B$ gilt, da Wahres aus Beliebigem folgt, auch $A\to \neg B$ und somit insgesamt $A\to (B\land \neg B)$.
\end{bem}

  
\begin{bem}[Guter Stil]
    Wenn du dich in einem Widerspruchsbeweis befindest, kannst du, um deinem Leser zu signalisieren, dass du gerade mit \emph{falschen} Aussagen arbeitest, den Konjunktiv II verwenden („dann wäre“, „nun gälte“). Außerdem solltest du die Annahme einer falschen Aussage stets mit „Angenommen, dass\dots“ oder Ähnlichem beginnen. Für den Leser ist es äußerst wichtig zu wissen, zu welchem Zeitpunkt im Beweis es gerade um die Herleitung wahrer Aussagen geht und zu welchem es (um eines Widerspruchsbeweises willen) um die Herleitung falscher Aussagen geht.
    
    Die Stelle im Beweis, an der ein Widerspruch erreicht wird, wird handschriftlich oft mit einem Blitz $\lightning$ markiert. In gedruckten Texten ist der Blitz weniger gängig. Egal wie du es handhabst: du solltest den Moment, an dem du bei einem Widerspruch angelangt bist, stets sprachlich hervorheben.
\end{bem}


\begin{satz}[*] \label{paradox}
    Es gilt
        \[ \neg (A\leftrightarrow \neg A) \]
\end{satz}


\begin{bew}
    Für einen Widerspruchsbeweis sei angenommen, dass $A\leftrightarrow \neg A$ gilt. Daraus folgte $A\to \neg A$ und wegen $A\to A$ gälte dann insgesamt $A\to (A\land \neg A)$. Wegen \cref{widerspruchsbeweis} müsste $A$ demnach falsch sein, d.h. es müsste $\neg A$ gelten. Wegen $A\leftrightarrow \neg A$ folgte aus $\neg A$, dass auch $A$ gälte. Insgesamt läge nun der Widerspruch $A\land \neg A$ vor. \qed
\end{bew}



\begin{bsp}[*] \label{bsp:paradox}
    Aus diesem Grund werden auch Aussagen der Gestalt $A\leftrightarrow \neg A$ gelegentlich als „Widerspruch“ bezeichnet.  Ich persönlich bevorzuge es, Aussagen, die zu ihrer eigenen Negation äquivalent sind, „Paradoxa“ zu nennen.
    \begin{enumerate}
        \item Ein berühmtes Beispiel für eine Aussage, die äquivalent zu ihrer Negation ist, ist das (selbstreferenzielle) \textbf{Lügner-Paradoxon}:
        \begin{quote}
            $A:=$ „Diese Aussage ist falsch.“
        \end{quote}
        Hier gilt tatsächlich $A\leftrightarrow \neg A$.
        \item Eine weitere berühmte Situation, in der eine Aussage äquivalent zu ihrer Negation ist, ist das „Barbier-Paradoxon“, das wiederum ein Spezialfall der Russellschen Antinomie \cref{russell} ist:
        \begin{satz}
            In Sevilla lebt kein Mann, der genau denjenigen Männern Sevillas den Bart rasiert, die sich nicht selbst den Bart rasieren.
        \end{satz}
        \begin{bew}
            Für einen Widerspruchsbeweis sei einmal angenommen, dass es doch einen solchen Mann gäbe. Aus der Beschreibung leitet man ab, dass sich dieser Mann genau dann selbst den Bart rasierte, wenn er ihn sich nicht selbst rasierte. Aber das ist unmöglich. \qed
        \end{bew}
    \end{enumerate}
    
\end{bsp}

  
\begin{satz}[Existenzaussagen widerlegen] \label{existenzwiderleg}
    Sei $E$ eine Eigenschaft. Dann kannst du $\nexists x: E(x)$ dadurch beweisen, dass du $\forall x: \neg E(x)$ beweist.
\end{satz}


\begin{bew}
    Es sei bewiesen, dass $\forall x: \neg E(x)$ gilt. Für einen Widerspruchsbeweis sei nun angenommen, dass dennoch $\exists x: E(x)$ gälte. Dann gäbe es ein Objekt $a$, für das $E(a)$ gälte. Aber wegen $\forall x: \neg E(x)$ gälte auch $\neg E(a)$ und dies ist ein Widerspruch. \qed
\end{bew}


\begin{satz}[* Russellsche Antinomie\footnote{\href{https://de.wikipedia.org/wiki/Bertrand_Russell}{Bertrand Russell 1872-1970}}] \label{russell} \index{Russellsche Antinomie}
    Sei $R$ ein zweistelliges Prädikat, dessen beide Variablen vom selben Typ sind. Dann gilt:
        \[ \nexists x\ \forall y:\ (R(x,y) \leftrightarrow \neg R(y,y))\]
    Mit anderen Worten: Es gibt kein Objekt $x$, sodass jedes Objekt $y$ genau dann in Relation zu $x$ stünde, wenn es nicht in Relation zu sich selbst stünde.
\end{satz}


\begin{bew}
    Für einen Widerspruchsbeweis sei einmal angenommen, es gäbe ein Objekt $a$, für das
        \[ \forall y:\ R(a,y) \leftrightarrow \neg R(y,y) \]
    gälte. Weil es sich hierbei um eine Allaussage handelt, könnten wir für $y$ das Objekt $a$ einsetzen und erhielten die Äquivalenz
        \[ R(a,a) \leftrightarrow \neg R(a,a) \]
    Aber dies mündet mit \cref{paradox} in einen Widerspruch. \qed
\end{bew}
 

\begin{vorschau}[*]
    Definiert man hierbei $R(x,y):\Leftrightarrow$ „$x$ rasiert $y$ den Bart“, so ergibt sich genau das Barbier-Paradoxon aus \cref{bsp:paradox}. Bezeichnen andererseits $x,y$ zwei Mengen und $R(x,y):\Leftrightarrow x\in y$, so erhält man die Aussage, dass es keine Menge gibt, deren Elemente genau diejenigen Mengen sind, die kein Element von sich selbst sind. Diese Aussage war Auslöser der \href{https://de.wikipedia.org/wiki/Grundlagenkrise_der_Mathematik}{Grundlagenkrise} zu Beginn des 20. Jahrhunderts.
\end{vorschau}
  
  

  
    
\section{Aus Falschem folgt Beliebiges}


In diesem Abschnitt seien $A,B$ stets zwei Aussagen.


\begin{axiom}[Modus tollendo ponens] \label{modustp}
    Aus $A\lor B$ und $\neg A$ kannst du schlussfolgern, dass $B$ gilt:
    \[\begin{tabular}{r}
        $A\lor B$ \\
        $\neg A$ \\
        \hline 
        $B$
    \end{tabular} \]
\end{axiom}


\begin{bem}
    Diese Schlussregel kommt bei der Entscheidungsfindung durch Ausschlusskriterien zum Einsatz: Wenn ich weiß, dass von einer Handvoll Aussagen mindestens eine gelten muss, kann ich die wahre Aussage finden, falls ich alle anderen Aussagen ausschließen kann.
\end{bem}


\begin{bsp}
    Ich habe beschlossen, meiner Freundin einen Erdbeerkuchen oder einen Käsekuchen zum Geburtstag zu backen. Falls ich morgen keine Erdbeeren mehr auftreiben kann, werde ich ihr also einen Käsekuchen backen.
\end{bsp}


\begin{satz}[Aus Falschem folgt Beliebiges] \label{exfalso} \index{ex falso quodlibet}
    Es gelten die folgenden beiden Implikationen:
    \begin{align*}
        \neg A & \to (A\to B) \\
        (A\land \neg A) & \to B
    \end{align*}
    Mit anderen Worten: Aus einer falschen Aussage oder einem Widerspruch lässt sich jede beliebige Aussage ableiten. Auf Latein heißt dieses Prinzip \textbf{ex falso quodlibet}.
\end{satz}


\begin{bew}
    Für einen direkten Beweis sei angenommen, dass $\neg A$ und $A$ gelten. Aus $A$ folgt, dass $A\lor B$ gilt, und wegen $\neg A$ muss gemäß \cref{modustp} dann schon $B$ gelten. \qed
\end{bew}


\begin{vorschau}[``principle of explosion''] \label{explosion}
    Da aus Widersprüchen Beliebiges folgt, ist in die Logik eine Art „Bombe“ eingebaut (in der englischen Literatur spricht man sogar vom ``principle of explosion''). Denn wenn es uns gelingen sollte, auch nur eine einzige Aussage sowohl zu beweisen als auch zu widerlegen, folgt aus \cref{exfalso}, dass \emph{jede} beliebige mathematische Aussage beweisbar ist, so unsinnig sie auch sei. Das Angeben von Beweisen wäre dann nicht mehr dazu geeignet, die „Wahrheit“ irgendwelcher Aussagen zu begründen.
    
    Logiken, in denen das ex falso quodlibet nicht gilt, heißen \href{https://ncatlab.org/nlab/show/paraconsistent+logic}{parakonsistente Logiken}. In solchen Logiken hält sich der von Widersprüchen verursachte Schaden in Grenzen und wird teils sogar absichtlich in Kauf genommen; dafür ist dort die Schlussregel aus \cref{modustp} nicht uneingeschränkt anwendbar.
\end{vorschau}





\section{Der Satz vom ausgeschlossenen Dritten}


In diesem Abschnitt seien $A,B$ stets zwei Aussagen.

Die bisherigen Axiome bilden zusammen die sogenannte „intuitionistische“ oder auch „konstruktive“ Logik. Zur sogenannten „klassischen Logik“, die der Mainstream-Mathematik zugrundeliegt, fehlt nur noch das folgende Axiom:


\begin{axiom}[Satz vom ausgeschlossenen Dritten] \label{excludedmiddle} \index{tertium non datur} \index{Satz vom ausgeschlossenen Dritten}
    Es gilt:
        \[ A\lor \neg A \]
    Dieses Prinzip heißt auch \textbf{tertium non datur}, was Latein für „ein Drittes kommt nicht vor“ ist. Im Englischen spricht man vom ``principle of excluded middle''.
\end{axiom}


\begin{bsp}
    Sei $A:=$ „Heute ist Mittwoch“. Dann ergibt sich aus dem tertium non datur die Aussage „Heute ist Mittwoch oder heute ist nicht Mittwoch“.
\end{bsp}


\begin{bem}
    Der Satz vom ausgeschlossenen Dritten ist nicht zu verwechseln mit dem Bivalenzprinzip aus \cref{bivalenz}. Das tertium non datur schließt (entgegen seinem Namen -- die Terminologie ist unglücklich kontraintuitiv) nicht aus, dass es mehr als nur zwei Wahrheitswerte gibt; es besagt lediglich, dass sich die beiden Wahrheitswerte von $A$ und $\neg A$ „in der Summe immer zu `absolut wahr' kombinieren“.
    
    In der philosophischen Logik wird dagegen der Satz vom ausgeschlossenen Dritten gelegentlich mit dem Bivalenzprinzip gleichgesetzt.
\end{bem}


\begin{bem}[Konsequenz für Fallunterscheidungsbeweise]
    Mit dem tertium non datur stehen uns bedingungslos Oder-Aussagen der Gestalt $A\lor \neg A$ zur Verfügung, die wir für Fallunterscheidungsbeweise einsetzen können. Möchten wir eine Aussage $X$ beweisen und ist $A$ irgendeine beliebige weitere Aussage, so genügt es, $X$ einmal unter der Annahme, dass $A$ gilt, zu beweisen, und andererseits unter der Annahme, dass $A$ falsch ist.
    
    Eine der berühmtesten bisher unentschiedenen Aussagen der Mathematik ist die sogenannte \href{https://de.wikipedia.org/wiki/Riemannsche_Vermutung}{Riemannsche Vermutung}\footnote{\href{https://de.wikipedia.org/wiki/Bernhard_Riemann}{Bernhard Riemann (1826 - 1866)}}. Obwohl bislang unbekannt ist, ob die Vermutung zutrifft oder nicht, konnten \href{https://en.wikipedia.org/wiki/Riemann_hypothesis#Excluded_middle}{diverse Aussagen} dadurch bewiesen werden, dass sie sowohl im Fall, dass die Vermutung zutrifft, gelten, als auch im Fall, dass die Vermutung falsch wäre. Hierbei wird essenziell auf das tertium non datur zurückgegriffen.
\end{bem}


\begin{bsp}
    Es existieren zwei positive irrationale Zahlen $a,b$, für die $a^b$ eine rationale Zahl ist.
\end{bsp}


\begin{bew}
    Ich setze als bekannt voraus, dass $\sqrt{2}$ eine irrationale Zahl ist. Betrachte nun die Aussage
    \begin{quote}
        $A:=$ „$\sqrt{2}^{\sqrt{2}}$ ist eine rationale Zahl.“
    \end{quote}
    Gemäß dem tertium non datur gilt entweder $A$ oder $\neg A$, sodass eine Fallunterscheidung durchgeführt werden kann:
    \begin{itemize}
        \item Falls $A$ gilt, setze einfach $a=b=\sqrt{2}$.
        \item Falls $A$ falsch ist, setze $a=\sqrt{2}^{\sqrt{2}}$ und $b=\sqrt{2}$. Weil $A$ falsch ist, ist $a$ eine irrationale Zahl und es ist
            \[ a^b = (\sqrt{2}^{\sqrt{2}})^{\sqrt{2}} = \sqrt{2}^{\sqrt{2}\cdot\sqrt{2}} = \sqrt{2}^2 = 2 \]
        eine rationale Zahl. \qed
    \end{itemize}
\end{bew}

   
\begin{vorschau}[* Nichtkonstruktivität des tertium non daturs] \label{nichtkonstruktiv} \index{nichtkonstruktiver Beweis}
    Der vorige Beweis ist ein berühmtes Beispiel für einen „nichtkonstruktiven Beweis“. In ihm wird die Existenz zweier Zahlen $a,b$ mit besonderen Eigenschaften bewiesen, ohne dass am Ende des Beweises ein konkretes Beispiel für ein solches Zahlenpaar vorliegt. Zwar wird aus dem Beweis deutlich, dass mindestens eine der beiden Wahlen
    \begin{align*}
        b=\sqrt{2}\ \text{und}\ a=\sqrt{2} \qquad \text{oder}\qquad  b=\sqrt{2}\ \text{und}\ a=\sqrt{2}^{\sqrt{2}}
    \end{align*}
    funktioniert -- welche genau dies ist, bleibt aber im Dunkeln. (Tatsächlich ist $\sqrt{2}^{\sqrt{2}}$ eine irrationale Zahl, sodass $a=\sqrt{2}^{\sqrt{2}}$ gewählt werden muss -- aber dies ist deutlich schwieriger zu beweisen.)
    
    Mit dem tertium non datur kommt ein weiteres seltsames Phänomen auf: In der Mathematik gibt es Aussagen, die, sofern keine Widersprüche herleitbar sind (denn dann wäre nach \cref{exfalso} ja \emph{alles} beweisbar), weder beweisbar noch widerlegbar sind. Die berühmteste dieser „unentscheidbaren“ Aussagen ist vielleicht die \href{https://de.wikipedia.org/wiki/Kontinuumshypothese}{Kontinuumshypothese}. Ist nun $A$ eine unentscheidbare Aussage und ist die Theorie konsistent, so sind weder $A$ noch $\neg A$ beweisbar, obwohl dem tertium non datur gemäß $A\lor \neg A$ gilt. Es liegt also die Situation vor, dass eine Aussage der Gestalt „$X$ oder $Y$“ beweisbar ist, obwohl weder $X$ noch $Y$ beweisbar sind.
    
    In der \emph{intuitionistischen Logik}, die aus allen bisherigen Axiomen mit Ausnahme des tertium non datur besteht, kann dies nicht vorkommen. Mithilfe fortgeschrittener semantischer Methoden lässt sich zeigen, dass sich, sofern sich eine Aussage der Gestalt „$X$ oder $Y$“ ohne Rückgriff aufs tertium non datur beweisen lässt, stets auch schon eine der beiden Aussagen $X$ oder $Y$ beweisen lässt\footnote{Auf Englisch nennt man dies die \href{https://en.wikipedia.org/wiki/Disjunction_and_existence_properties}{``Disjunction property''} der intuitionistischen Logik.}. Allerdings lassen sich ohne Rückgriff aufs tertium non datur eben auch weniger Aussagen überhaupt beweisen.
\end{vorschau}

   
\begin{satz}[Regel der doppelten Verneinung] \label{doppelneg}
    Es gilt
    \[ A\leftrightarrow \neg \neg A \]
\end{satz}


\begin{bew}[(*)]
    \begin{itemize}
        \item[„$\Rightarrow$“:] Es gelte $A$. Dann würde aus $\neg A$ sofort der Widerspruch $A\land \neg A$ folgen, sodass $\neg A$ falsch sein muss. Also gilt $\neg\neg A$.
        \item[„$\Leftarrow$“:] Es gelte $\neg \neg A$. Gemäß tertium non datur gilt $A\lor \neg A$. Da $\neg A$ falsch ist, folgt vermöge \cref{modustp}, dass $A$ gelten muss. \qed
    \end{itemize}
\end{bew}


\begin{bem}
    Aufgrund der Regel der doppelten Verneinung kann jede Aussage $A$ mit der verneinenden Aussage $\neg\neg A$ identifiziert werden. Die Aussage $A$ zu beweisen, ist dann gleichwertig dazu, die Aussage $\neg A$ zu widerlegen. Diese Verlagerung erlaubt es, für den Beweis von $A$ auch alle Widerlegetechniken einsetzen zu können.
\end{bem}


\begin{satz} \label{implikationchar}
    Es gilt:
        \[ A\to B\quad \leftrightarrow\quad \neg A\lor B \]
\end{satz}


\begin{bew}[(*)]
    \begin{enumerate}
        \item[„$\Rightarrow$“:] Es gelte $A\to B$. Um nun „$\neg A\lor B$“ zu beweisen, mache ich mir das tertium non datur zunutze, demzufolge $A$ oder $\neg A$ gelten muss, und führe eine Fallunterscheidung durch:
        \begin{enumerate}[(1)]
            \item Der Fall, dass $A$ wahr ist. In diesem Fall folgt aus der Annahme $A\to B$, dass auch $B$ gilt. Aber dann ist insbesondere auch $\neg A\lor B$ korrekt.
            \item Der Fall, dass $A$ falsch ist. In diesem Fall gilt $\neg A\lor B$ ebenfalls, da ja schon $\neg A$ gilt.
        \end{enumerate}
        Also gilt in jedem Fall $\neg A\lor B$.
        \item[„$\Leftarrow$“:] Es gelte $\neg A\lor B$. Ich werde $A\to B$ direkt beweisen, weshalb einmal angenommen sei, dass $A$ gelte. Aber dann ist $\neg A$ falsch, sodass wegen $\neg A\lor B$ nur noch die Option übrig bleibt, dass $B$ gilt.\footnote{Hier wurde \cref{modustp} benutzt.} \qed
    \end{enumerate}
\end{bew}


\begin{satz}[*] \label{oderperimplikation}
    Möchtest du eine Aussage der Gestalt $A\lor B$ beweisen, so kannst du stattdessen auch $\neg A\to B$ beweisen.
\end{satz}


\begin{bew}
    Es ist
    \begin{align*}
        \neg A\to B \qquad &\leftrightarrow\qquad \neg\neg A\lor B && (\text{wegen \cref{implikationchar}})\\
        & \leftrightarrow\qquad A\lor B && (\text{Regel der doppelten Verneinung}) \qed
    \end{align*}
\end{bew}


\begin{bsp}[*]
    In der Ebene sind zwei Geraden zueinander parallel oder sie schneiden sich in einem Punkt.
\end{bsp}


\begin{bew}
    Wenn zwei Geraden nicht parallel sind, so laufen sie in einer Richtung aufeinander zu. Weil in dieser Richtung der Abstand beider Geraden mit konstanter Rate abnimmt, müssen sich die Geraden in dieser Richtung irgendwann schneiden. \qed
\end{bew}


\begin{satz}[Regeln von De Morgan\footnote{\href{https://de.wikipedia.org/wiki/Augustus_De_Morgan}{Augustus De Morgan (1806 - 1871)}}] \label{demorgan} \index{De Morgansche Regeln}
Es gilt:
    \begin{align*}
        \neg (A\lor B) \quad& \leftrightarrow\quad \neg A\land \neg B \\
        \neg(A\land B) \quad& \leftrightarrow\quad \neg A\lor \neg B
    \end{align*}
\end{satz}


\begin{bew}[(*)]
    Es müssen insgesamt vier Implikationen
    \begin{align}
    \neg (A\lor B) \quad& \to\quad \neg A\land \neg B \tag{1}\\
    \neg A\land \neg B\quad& \to\quad  \neg (A\lor B) \tag{2}\\
    \neg(A\land B) \quad& \to \quad \neg A\lor \neg B \tag{3}\\
    \neg A\lor \neg B \quad& \to \quad \neg(A\land B) \tag{4}
    \end{align}
    bewiesen werden.
    \begin{enumerate}
        \item[(1)] Wegen $A\to (A\lor B)$ und $B\to (A\lor B)$ folgt per Kontraposition, dass $\neg (A\lor B)\to \neg A$ und $\neg(A\lor B) \to \neg B$ gelten muss. Insgesamt folgt nun $\neg(A\lor B)\to( \neg A\land\neg B)$.
        \item[(4)] Wegen $(A\land B)\to A$ und $(A\land B) \to B$ folgt per Kontraposition, dass $\neg A\to \neg(A\land B)$ und $\neg B\to \neg(A\land B)$ gelten muss. Insgesamt folgt nun $(\neg A\lor \neg B)\to \neg(A\land B)$.
        \item[(2)] Sowohl im Fall $A$ als auch im Fall $B$ würde aus $\neg A\land \neg B$ ein Widerspruch folgen. Somit gilt $(A\lor B)\to \neg (\neg A\land \neg B)$. Per Kontraposition folgt daraus $(\neg A\land \neg B) \to \neg(A\lor B)$.
        \item[(3)] Für einen indirekten Beweis sei angenommen, dass $\neg (\neg A\lor\neg B)$ gilt. Mit der (bereits bewiesenen) Implikation (1) folgt daraus, dass $\neg\neg A\land\neg\neg B$ gilt, was mit der Regel der doppelten Verneinung zu $A\land B$ vereinfacht werden kann. \qed
    \end{enumerate}
\end{bew}

 
\begin{satz}[Implikationen per Widerspruch beweisen]
    Um die Implikation $A\to B$ zu beweisen, kannst du folgendermaßen vorgehen: Nimm an, dass sowohl $A$ als auch $\neg B$ gelten und versuche nun, einen Widerspruch herzuleiten.
\end{satz}


\begin{bew}
    Folgt aus der Annahme $A\land \neg B$ ein Widerspruch, so ist damit $\neg(A\land \neg B)$ bewiesen. Wegen
    \begin{align*}
        \neg (A\land \neg B) \quad&\leftrightarrow\quad \neg A\lor \neg\neg B && (\text{Regel von De Morgan}) \\
        &\leftrightarrow\quad \neg A\lor B && (\text{Regel der doppelten Verneinung}) \\
        & \leftrightarrow\quad A\to B && (\text{nach \cref{implikationchar}})
    \end{align*}
    gilt dann auch $A\to B$. \qed
\end{bew}

 
\begin{bem}[Tücken des Widerspruchsbeweises]
    Im Vergleich zum direkten Beweis ist ein Widerspruchsbeweis für $A\to B$ oft leichter zu finden, weil ja eine Annahme mehr als beim direkten Beweis zur Verfügung steht (nämlich $\neg B$). Andererseits sind Widerspruchsbeweise dort, wo auch ein direkter oder indirekter Beweis möglich ist, oft unnötig kompliziert, weshalb du, wenn du einen validen Widerspruchsbeweis gefunden hast, immer noch einmal überprüfen solltest, ob er sich nicht leicht in einen direkten Beweis umformulieren lässt.
    
    Einige Anfänger machen den „Fehler“, ihre Tutoren mit Widerspruchsbeweisen zu überhäufen, wo schon einfachste Modifikationen einen direkten Beweis lieferten (vielleicht weil sie Widerspruchsbeweise mysteriös und cool finden). Allerdings gibt es Situationen, in denen mithilfe metamathematischer Methoden nachgewiesen werden kann, dass ein direkter Beweis unmöglich ist. In solchen Fällen sind Widerspruchsbeweise unvermeidlich.
\end{bem}
  
  
\begin{bsp}[*]
    Ein weiterer Grund dafür, dass du direkte Beweise Widerspruchsbeweisen vorziehen solltest, besteht darin, dass sich Fehler in Widerspruchsbeweisen (wo ja ohnehin mit falschen Aussagen jongliert wird) oft schwieriger ausmachen lassen als in direkten Beweisen. Ein Beispiel:
    \begin{satz}
        Es ist $1=0$.
    \end{satz}
    \begin{bew}[(Unnötig komplizierter Widerspruchs„beweis“)]
        Ich führe einen Widerspruchsbeweis. Dazu nehme ich an, dass $1\neq 0$ ist, und leite daraus einen Widerspruch her.
        
        Wegen $1\neq 0$ ist auch
            \[ 0 = 2\cdot 0 \neq 2\cdot 1 = 2 \]
        und aus $0\neq 2$ folgt $-1\neq 1$. Durch weiteres Umformen kommt man auf
        \begin{align*}
            -1 & \neq 1 \\
            & = \sqrt{1} \\
            & = \sqrt{(-1)\cdot (-1)} \\
            & = \sqrt{-1} \cdot \sqrt{-1} \\
            & = \sqrt{-1}^2 \\
            & = -1
        \end{align*}
        also insgesamt $-1\neq -1$. Aber dies ist eine falsche Aussage. Also muss die Annahme zu Beginn des Beweises, nämlich dass $0\neq 1$ ist, falsch sein. 
    \end{bew}
    Kannst du den Fehler finden? Denk darüber eine Zeitlang nach und versuche daraufhin, den Fehler im folgenden direkten Beweis  zu finden:
    \begin{bew}[(Diesmal als direkter „Beweis“)]
        Es gilt:
        \begin{align*}
            1 & = \sqrt{1} \\
            & = \sqrt{(-1)\cdot (-1)} \\
            & = \sqrt{-1}\cdot \sqrt{-1} \\
            & = \sqrt{-1}^2 \\
            & = -1
        \end{align*}
        also insgesamt $1=-1$. Weitere Äquivalenzumformungen ergeben:
        \begin{align*}
            1 & = -1 \quad\xrightarrow{+1}\quad 2=0 \quad\xrightarrow{:2}\quad 1=0 
        \end{align*}
    \end{bew}
    Erstsemestern passiert es öfters, dass sie ihre eigenen Denkfehler nur deshalb übersehen, weil ihr Beweis einer unnötig komplizierten Logik unterliegt, mit der sie sich am Ende selbst verwirren. Versuche deshalb, immer den Überblick über deine Beweisstruktur zu behalten und sie so simpel wie möglich zu halten.
\end{bsp}


\begin{satz}[Quantoren negieren] \label{quantorennegieren} \index{Negation von Quantoren}
    Sei $E$ eine Eigenschaft. Dann gilt:
    \begin{align*}
        \nexists x: E(x) \quad& \leftrightarrow\quad \forall x: \neg E(x) \\
        \neg (\forall x: E(x)) \quad& \leftrightarrow\quad \exists x: \neg E(x)
    \end{align*}
\end{satz}


\begin{bew}[(*)]
    Nach Aufteilung in Hin- und Rückrichtungen bleiben insgesamt vier Implikationen übrig, die bewiesen werden müssen:
    \begin{align}
        \neg (\exists x:\ E(x)) \quad& \to\quad (\forall x:\ \neg E(x)) \tag{1} \\
        (\forall x:\ \neg E(x)) \quad& \to\quad \neg (\exists x:\ E(x)) \tag{2} \\
        \neg (\forall x:\ E(x)) \quad& \to\quad (\exists x:\ \neg E(x)) \tag{3} \\
        (\exists x:\ \neg E(x)) \quad& \to\quad \neg (\forall x:\ E(x)) \tag{4}
    \end{align}
    \begin{enumerate}
        \item[(1)] Es gelte $\nexists x : E(x)$. Sei nun $t$ ein beliebiges Objekt. Wegen
            \[ E(t) \quad \to \quad \exists x: E(x) \]
        folgt per Kontraposition, dass $\neg E(t)$ gelten muss. Weil das Objekt $t$ beliebig gewählt wurde, ist damit bewiesen, dass $\forall x: \neg E(x)$ gilt.
        \item[(4)] Ergibt sich aus \cref{gegenbeispiel}.
        \item[(2)] Ergibt sich aus \cref{existenzwiderleg}.
        \item[(3)] Für einen indirekten Beweis sei angenommen, dass $\nexists x: \neg E(x)$ gilt. Mit der (bereits bewiesenen) Implikation (1) folgt daraus, dass $\forall x: \neg\neg E(x)$ gilt, was mit der Regel der doppelten Verneinung zu $\forall x: E(x)$ vereinfacht werden kann. \qed
    \end{enumerate}
\end{bew}


\begin{vorschau}[* Vollständigkeit der erststufigen Prädikatenlogik]
    Jede Formel, die bisher als Axiom gewählt oder hergeleitet wurde, ist eine Tautologie, d.h. wahr unter jeder möglichen (zweiwertigen) Interpretation im Sinne von \cref{def:interpretation}. Der \emph{Gödelsche Vollständigkeitssatz} besagt auch das Umgekehrte: Jede Tautologie (im Sinne zweiwertiger Interpretationen) lässt sich mithilfe der Axiome und Beweistechniken aus diesem Kapitel herleiten.
    
    Für eine allgemeine Aussage mit Junktoren und Quantoren ist das nicht allzu hilfreich, denn die Beweistechniken dienen uns ja dazu, herauszufinden, welche Aussagen wahr sind -- und nicht etwa dient uns umgekehrt irgendein Wahrheitsorakel dazu, herauszufinden, welche Aussagen beweisbar sind.
    
    Sofern wir es aber mit einer Aussage zu tun haben, die keine Prädikate und Quantoren enthält, sieht die Sache anders aus. Denn für eine solche Aussage, die allein aus den aussagenlogischen Junktoren aufgebaut ist, können wir brute force überprüfen, ob es sich um eine Tautologie handelt, indem wir einen Computer eine Wahrheitstafel ausrechnen lassen und schauen, ob am Ende überall nur w's stehen, siehe \cref{entscheidbar}. Bei komplizierten Aussagen steigt die dafür nötige Rechenkapazität allerdings ins Unermessliche.
\end{vorschau}


\begin{bem}
    Eine Liste mit vielen aussagenlogischen und prädikatenlogischen Tautologien findest du in \cref{formelsammlung}. Wenn du möchtest, kannst du dir einmal zwei oder drei davon raussuchen und versuchen, sie zu beweisen.
\end{bem}





\section{Entstehungsprozess eines Beweises}


Das Schreiben eines mathematischen Beweises lässt sich grob in drei Phasen gliedern, die in diesem Abschnitt einmal durchgegangen werden sollen. Parallel zu deren abstrakter Beschreibung wird ein ganz konkretes Beispiel aus der Linearen Algebra entwickelt:


\begin{aufg}
    Gegeben seien die drei Vektoren des $\R^3$:
        \[ v_1:= \begin{pmatrix} 1 \\ 1 \\ 2 \end{pmatrix},\qquad  v_2:= \begin{pmatrix} 2 \\ 1 \\ 0 \end{pmatrix},\qquad v_3:= \begin{pmatrix} 1 \\ 1 \\ 1 \end{pmatrix}\]
    Man beweise, dass $(v_1,v_2,v_3)$ eine Basis des $\R^3$ ist.
\end{aufg}


Möglicherweise wirst du mit diesem Beispiel erst in einigen Wochen etwas anfangen können. Komm dann nochmal hierher zurück.

 
\begin{de}[Phase 1: Recherche]
    In diesem Schritt stellt ihr sicher, euch auf dem Stand der Dinge zu befinden:
    \begin{itemize}
        \item Sofern ihr nicht die Bedeutung aller in der Aufgabenstellung vorkommenden Begriffe kennt, müsst ihr sie im Vorlesungsskript, in eurem Aufschrieb, in einem Lehrbuch oder im Internet nachschlagen. Solange ihr nicht genau wisst, was die Aufgabe besagt, könnt ihr sie nicht lösen.
        \item Mit den Definitionen allein kommt ihr meist aber noch nicht weit. Denn in der Regel wurden in der Vorlesung bereits ein paar praktische Aussagen bewiesen, die ihr euch für eure Lösung zunutze machen könnt. Auf diese Weise bekommt ihr Werkzeuge in die Hand, die ihr in eurem Beweis einsetzen könnt, um euch die Arbeit zu erleichtern. Erstsemestern passiert es nicht selten, dass sie keinen genauen Überblick darüber, was genau in der Vorlesung bewiesen wurde und was nicht, haben, und deshalb unnötige Mehrarbeit verrichten, indem sie unbeabsichtigt versuchen, bereits in der Vorlesung bewiesene Sätze noch einmal selbst zu beweisen.
        \item Ein Problem, das vor allem das erste Studiensemester betrifft, ist, dass gewisse „offensichtliche“ oder bereits aus der Schule bekannte Aussagen für die Lösung der Übungszettel nicht verwendet werden sollen, weil sie noch nicht in der Vorlesung bewiesen wurden. Leider ist im ersten Semester manchmal nicht ganz klar, was man denn nun alles für bekannt voraussetzen darf und wobei es sich um „nichttriviale“ Aussagen, die eines Beweises bedürfen, handelt. Im Zweifelsfall solltet ihr bei eurem Tutor / eurer Tutorin nachfragen. Glücklicherweise hört diese Problematik spätestens im dritten Semester auf.
    \end{itemize}
    Zu dem Beispiel mit den Basisvektoren: Solange ich nicht genau weiß, was eine „Basis des $\R^3$“ ist, kann ich die Aufgabe nicht lösen. Ein Blick in den Vorlesungsaufschrieb verrät mir:
    \begin{quote}
        Das Tripel $(v_1,v_2,v_3)$ ist genau dann eine Basis des $\R^3$, wenn es für jeden Vektor $v\in \R^3$ eindeutig bestimmte reelle Zahlen $a,b,c\in \R$ gibt, für die $v=av_1+bv_2+cv_3$ gilt.
    \end{quote}
    An dieser Definition könnte ich nun meinen Beweisansatz orientieren. Damit würde ich aber unnötige Beweisarbeit verrichten, die bereits in der Vorlesung erledigt wurde. Denn dort wurde die folgende Aussage bewiesen:
    \begin{quote}
        Da der $\R^3$ dreidimensional ist, ist das Tripel $(v_1,v_2,v_3)$ schon dann eine Basis, wenn es linear unabhängig ist.
    \end{quote}
    Dies führt mich auf den Begriff „linear unabhängig“, dessen Bedeutung ich, sofern sie mir nicht absolut klar ist, ebenfalls nachschlagen muss:
    \begin{quote}
        Die Vektoren $(v_1,v_2,v_3)$ heißen \emph{linear unabhängig}, falls für alle $a,b,c\in \R$ mit $av_1+bv_2+cv_3=0$ bereits gelten muss, dass $a=b=c=0$.
    \end{quote}
    Damit habe ich jetzt alle Definitionen beisammen und hoffe, dass ich keinen weiteren Satz aus der Vorlesung, der mir noch mehr Arbeit abnehmen könnte, übersehen habe\footnote{Später in der LA1-Vorlesung wird meist ein „Determinantenkriterium“ bewiesen, mit dessen Hilfe die Aufgabe nochmal erheblich vereinfacht würde.}. Beachte auch, wie mir das Nachschlagen des Vorlesungssatzes Arbeit abgenommen hat. Anfangs hätte ich beweisen müssen, dass es für jeden beliebigen Vektor $v\in \R^3$ eindeutig bestimmte Zahlen $a,b,c\in \R$ mit $v=av_1+bv_2+cv_3$ gibt. Nun muss ich nur noch beweisen, dass diese Zahlen im Fall $v=0$ eindeutig bestimmt sind. Das ist eine erhebliche, unmittelbare Vereinfachung der Aufgabe!
\end{de}


\begin{bem}
    Die Recherchephase ist auch für die Spitzenforschung nicht zu unterschätzen. Probleme im Wissensaustausch sind vielleicht das größte Hindernis mathematischen Fortschritts. So schreibt Peter Johnstone in der Einführung seines Buchs \emph{Stone Spaces} (1982), einem Klassiker der punktfreien Topologie:
    \begin{quote}
        The enormous increase in the number of practising mathematicians since the 1930s has inevitably produced a corresponding decrease in the range of mathematical knowledge that each one possesses on average, and the effect of this is easy to see: theorems and techniques which are commonplace in one field are laboriously and imperfectly rediscovered in adjacent ones.
    \end{quote}
    Gehöre also später nicht zu den Mathematikern, die ihre Zeit damit vergeuden, solche Sätze, die in Fachkreisen längst bewiesen wurden, unter großen Mühen und auch nur halbgar neu zu beweisen, nur weil sie zu schlecht informiert sind; oder zu den Programmierern, die eine Prozedur furchtbar kompliziert und ineffizient programmieren, nur weil ihnen die frei zugänglichen Bibliotheken unbekannt sind, in denen die Algorithmen bereits hocheffizient implementiert sind! Von diesen Leuten gibt es leider eine ganze Menge.
\end{bem}


\begin{de}[Phase 2: Rumprobieren] \label{rumprobieren}
    Der zweite Schritt ist die kreativste Phase. Nachdem ihr euch möglichst alle Hilfsmittel, die euch die Vorlesung zum Thema bereitstellt, vergegenwärtigt habt, müsst ihr nun irgendwie einen Beweis aus dem Hut zaubern. Oft werden eure Überlegungen auch dazu führen, dass ihr nochmal zu Phase 1 zurückgeht und weitere Definitionen und Sätze nachschlagt.
    \begin{itemize}
        \item Beleuchtet das Problem von mehreren Seiten. Wenn eine Implikation $A\to B$ zu beweisen ist: seht euch die Kontraposition $\neg B\to \neg A$ an und schaut, ob ihr dadurch eher auf eine Beweisidee kommt. Oder nehmt an, dass sowohl $A$ als auch $\neg B$ gelten und schaut, ob daran irgendetwas faul ist.
        \item In dieser Phase ist wirklich \emph{alles} erlaubt. Ihr könnt völlig ungerechtfertigt irgendwelche Vermutungen aufstellen und mit Hypothesen arbeiten, die euch zwar plausibel erschienen, über die ihr euch aber gar nicht hundertprozentig sicher seid. Ihr braucht euch hier an keinerlei Logikregeln halten und könnt jeden noch so fernliegenden Bullshit ausprobieren. In dieser Phase betreibt ihr „experimentelle Mathematik“, die nicht logisch fundiert sein muss.
        \item Manchmal kann der Beweis auch „von hinten nach vorne“ durchgeführt werden. D.h. ihr beginnt mit der zu zeigenden Aussage und sucht nach Prämissen, unter deren Annahme ihr die Aussage beweisen könnt\footnote{Für ein Beispiel siehe \cref{hintennachvorne}.}. Dann versucht ihr, diese Prämissen zu beweisen, bis ihr irgendwann bei einer Aussage angekommen seid, die ihr schon an und für sich beweisen könnt. Ich markiere auf meinem Schmierblatt solche Gleichungen, die ich als Hypothesen verwende und die noch „zu zeigen“ sind, mit einem Ausrufezeichen wie etwa
        \begin{align*}
            a,b,c & \stackrel{!}{=} 0 && (\text{lies: „$a,b,c$ sollen gleich $0$ sein“})
        \end{align*}
        Beachtet aber: Während ihr in Phase 2 von hinten nach vorne arbeiten könnt, müsst ihr in Phase 3, wenn es um das Aufschreiben des Beweises geht, so gut es geht „von vorne nach hinten“ arbeiten.
        \item Haltet im Vorlesungsmaterial nach Aussagen ähnlicher Art wie die Aufgabenstellung Ausschau. Möglicherweise könnt ihr Beweistechniken aus der Vorlesung imitieren.
        \item In dieser Phase werdet ihr möglicherweise mehrere Schmierblätter mit für andere Leute völlig unsinnigen, unlesbaren Skizzen vollschreiben. Egal! Es geht hier um \emph{eure} Ideenfindung und erst in der nächsten Phase werdet ihr eure Gedanken für Andere verständlich machen müssen.
        \item Werd im ersten Semester nicht zum Einzelkämpfer! Tausche dich mit deinen Zettelpartnern oder anderen StudentInnen aus! In dieser Phase geht es darum, einen möglichst großen Vorrat an Ideen anzuhäufen, aus dem sich früher oder später die Lösung formen muss. Manchmal hat dein Zettelpartner den entscheidenden Gedanken, der noch fehlt, um deine Strategie aufgehen zu lassen und es wäre dumm und schade, wenn er ihn dir nicht mitteilte. Außerdem kann sich die Gedankenwelt deiner Partner fundamental von deiner eigenen unterscheiden und nur durch Austausch mit Anderen (einschließlich Lehrbücher und Internetseiten) kannst du ein vielseitiges Verständnis für mathematische Objekte gewinnen.
    \end{itemize}
    Diese Phase endet, sobald ihr einen Ansatz gefunden und weiterentwickelt habt, der sich als erfolgreich herausstellt, d.h. von dem ihr euch sicher seid, dass er sich in einen wasserdichten Beweis formulieren lässt. \\[0.5em]
    Zu dem Beispiel mit der Basis im $\R^3$: Die Recherche hat ergeben, dass ich nur noch beweisen muss: Sind $a,b,c\in \R$ drei beliebige reelle Zahlen mit
        \[ a\begin{pmatrix} 1 \\ 1 \\ 2 \end{pmatrix} + b \begin{pmatrix} 2 \\ 1 \\ 0 \end{pmatrix}+c \begin{pmatrix} 1 \\ 1 \\ 1 \end{pmatrix} = 0  \]
    so muss bereits $a=b=c=0$ gelten. Da es sich um eine Gleichung im $\R^3$ handelt, kann ich sie in ein System dreier Gleichungen zerlegen:
    \[\begin{array}{ccccccc}
        a &+& 2b &+& c &=& 0 \\
        a &+& b &+& c & =& 0 \\
        2a && &+ &c & =& 0
    \end{array}\]
    Dieses lineare Gleichungssystem kann ich nun einerseits mit Schulwissen, andererseits mit dem in der LA-Vorlesung präsentierten „Gauß-Algorithmus“ lösen. \\[0.5em]
    Damit ist eine vielversprechende Beweisstrategie gefunden. Auf dem Schmierblatt vergewissere ich mich nun durch ein paar Umformungen, die für nicht-eingeweihte Leser keinen Sinn ergeben müssen, dass das Gleichunssystem tatsächlich auf $a=b=c=0$ führt:
    \begingroup
    \allowdisplaybreaks
    \begin{align*}
        \leadsto&&   c & = -2a \\
        \leadsto&& -a + b & = 0 \\
        \leadsto&& b& = a \\
        \leadsto&& a +2a-2a & = 0 \\
        \leadsto&& a &= 0 \\
        \leadsto && c & = -2a=0
    \end{align*}
    \endgroup
    Damit ist die Aufgabe im Prinzip gelöst. Jetzt muss die Lösung nur noch ordentlich aufgeschrieben werden.
\end{de}
 
 
\begin{de}[Phase 3: Aufschreiben]\label{beweisaufschreiben}
    In dieser Phase geht es darum, einen gut lesbaren Beweistext zu formulieren, der allen Regeln der Kunst genügt. Oftmals werdet ihr in dieser Phase auf Schwächen im Beweis stoßen, die euch dazu zwingen, nochmal in Phase 2 zurückzugehen, um Reparaturen am Beweis vorzunehmen.
    \begin{itemize}
        \item Macht euch die logische Struktur eures Beweises deutlich und überlegt euch eine Gliederung der Beweisschritte. Schreibt diese in der Reihenfolge der logischen Argumentationskette auf, nicht in der Reihenfolge der kreativen Ideenkette, die euch auf den Beweis geführt hat, auf. Sollte der Beweis kompliziert sein, solltet ihr aber, sofern es eurem Leser hilft, auch ein paar Meta-Bemerkungen darüber, welche Idee hinter dem aktuellen Beweisschritt steckt, einstreuen.
        \item Sorgt für ein ausgeglichenes Wechselspiel zwischen Formeln und Umgangssprache.
        \item Stellt sicher, dass ihr jede Aussage, die ihr im Beweis verwendet und die nicht völlig naheliegend ist, begründet.
        \item Wenn ihr euch spezielle Aussagen aus der Vorlesung zunutze macht, schreibe so etwas wie „Aus der Vorlesung ist bekannt“ oder „Nach Vorlesung gilt\dots“.
        \item Definitionen aus der Vorlesung braucht ihr nicht noch einmal ausformulieren, sondern ihr könnt voraussetzen, dass euer Tutor / eure Tutorin alle Definitionen, die bislang in der Vorlesung drankamen, kennt. Ebenso könnt ihr alle Variablen, die in der Aufgabenstellung bereits eingeführt wurden, im Beweis verwenden ohne noch einmal neu definieren zu müssen, was sie bedeuten.
        \item Zeige deinen Beweis deinen Zettelpartnern. Wenn sie ihn ohne Zusatzerklärungen nicht verstehen, muss er verbessert werden. Der Beweis muss am Ende selbsterklärend für deinen Tutor / deine Tutorin sein.
        \item Sollte sich herausstellen, dass eure Lösung eine Lücke besitzt, für die euch einfach keine Lösung einfällt -- lasst sie stehen. Ein häufiger Erstsemesterfehler besteht darin, viel zu viel Zeit in die Bearbeitung der Übungszettel zu investieren -- Zeit, die deutlich nützlicher und nachhaltiger darin angelegt wäre, die Vorlesung zu reflektieren, Literatur zu lesen und Freundschaften aufzubauen.
        \item Sollte euer Beweis, wie gerade beschrieben, am Ende noch Lücken enthalten: Seid ehrlich zu euch selbst und eurem Tutor / eurer Tutorin! Ihr werdet mehr Anerkennung und Respekt dafür bekommen, dass ihr in einer kurzen Anmerkung darauf aufmerksam macht, an welcher Stelle noch eine Lücke besteht, und erläutert, woran genau es scheitert, als wenn ihr versucht, eure Lücke mit wirren Formulierungen und komplizierten Formeln zu verschleiern.
    \end{itemize}
    Der letzte Punkt ist vielleicht der allerwichtigste: Niemand erwartet, dass ihr von Anfang an perfekte Zettelabgaben produziert! Es ist der Job eures Profs,  eures Tutors und euer selbst, euch mathematisches Arbeiten beizubringen. Falls ihr methodisches oder inhaltliches Unverständnis habt, dürft ihr das nicht verschleiern oder gar euch dafür schämen! Vielmehr solltet ihr versuchen, dieses Unverständnis zu reflektieren und klar auszuformulieren. Das allein ist schon ein schwieriges Unterfangen, für dessen Gelingen ihr Stolz und Anerkennung verdient! \\[0.5em]
    Zu dem Beispiel mit der Basis im $\R^3$: Hier ist ein Beweis, den ich am Ende aufschreiben würde.
    \begin{bew}
        Aus der Vorlesung ist bekannt, dass, da der $\R^3$ ein dreidimensionaler Vektorraum ist, eine Familie dreier Vektoren im $\R^3$ genau dann eine Basis ist, wenn sie linear unabhängig ist. Demnach genügt es zu zeigen, dass $v_1,v_2,v_3$ linear unabhängig sind. \\[0.5em]
        Dazu seien $a,b,c\in \R$ drei beliebige reelle Zahlen mit
            \[ a\begin{pmatrix} 1 \\ 1 \\ 2 \end{pmatrix} + b \begin{pmatrix} 2 \\ 1 \\ 0 \end{pmatrix}+c \begin{pmatrix} 1 \\ 1 \\ 1 \end{pmatrix} = 0  \]
        Man erhält ein lineares Gleichungssystem
        \[\begin{array}{rcccccccc}
            \text{I} &&    a &+& 2b &+& c &=& 0 \\
            \text{II}&& a &+& b &+& c & =& 0 \\
            \text{III} &&  2a && &+ &c & =& 0
        \end{array}\]
        Nun gilt:
        \begin{align*}
            2a+c&=0 \quad \to\quad c = -2a \\
            \xrightarrow[\text{in II einsetzen}]{c=-2a}\quad -a+b &=0 \quad\to\quad b=a \\[0.5em]
            \xrightarrow[\text{in I einsetzen}]{b=a,\ c=-2a} \quad a+2a-2a &= 0 \quad\to\quad a= 0 \\
            \xrightarrow{b=a}\quad b &= 0 \\
            \xrightarrow[c=-2a]{a=0,} \quad c &= 0
        \end{align*}
        Also muss $a=b=c=0$ gelten. Da $a,b,c\in \R$ beliebig gewählt waren, ist damit gezeigt, dass $v_1,v_2,v_3$ linear unabhängig und somit eine Basis sind. \qed
    \end{bew}
\end{de}


Zugegebenermaßen würde ich in einem handschriftlichen Beweisaufschrieb weniger lange Sätze schreiben und mehr mit Einrückungen und Ellipsen arbeiten. Dies lässt sich aber nicht gut in einem gedruckten Text wiedergeben. Beweise in Büchern sehen anders aus als handschriftlich aufgeschriebene Beweise.





\clearpage
\section{Aufgabenvorschläge}


\begin{aufg}[(Un)Logische Schlussfolgerungen]
    Seien $A,B,C$ drei beliebige Aussagen und $E,F,G$ drei Eigenschaften. Beurteilt die nachfolgenden Schlussfolgerungen danach, ob sie logisch haltbar sind:
    \begin{align*}
        \text{a)}\quad &\begin{tabular}{l}
            Entweder ist der Butler oder der Koch der Mörder. \\
            Entweder ist der Koch oder der Gärtner der Mörder. \\ \hline
            Also gilt: Entweder ist der Butler oder der Gärtner der Mörder.
        \end{tabular} \\[1em]
        \text{b)}\quad &\begin{tabular}{l}
            Im Sterben sagen alle Menschen die Wahrheit. \\
            Im Sterben sagte Siddhartha: „Alles Geschaffene ist vergänglich.“ \\ \hline
            Also gilt: Alles Geschaffene ist vergänglich.
        \end{tabular} \\[1em]
        \text{c)}\quad &\begin{tabular}{l}
            Wenn sich niemand von uns ins Zeug legt, kriegen wir das Projekt nie \\
            \qquad im September fertig. \\
            Wir haben das Projekt im September fertig gekriegt. \\ \hline
            Also gilt: Jeder von uns hat sich ins Zeug gelegt.
        \end{tabular} \\[1em]
        \text{d)}\quad &\begin{tabular}{r}
            $A\to B$ \\
            $C\to \neg B$ \\ \hline
            $A\leftrightarrow \neg C$
        \end{tabular} \qquad\qquad\qquad \text{e)}\quad \begin{tabular}{r}
            $\nexists x:\ F(x)\land G(x)$ \\
            $\forall x:\ E(x)\to F(x)$ \\ \hline 
            $\nexists x:\ E(x)\land G(x)$
        \end{tabular}
    \end{align*}
\end{aufg}



\begin{aufg}[Fehlersuche I]
    Betrachtet den folgenden falschen Satz samt fehlerhaftem Beweis:
    \begin{satz}[Fehlerhafter Satz]
        Seien $x,y$ zwei reelle Zahlen und $x\neq 3$. Gilt dann auch noch $x^2y=9y$, so ist $y=0$
    \end{satz}
    \begin{bew}
        Es gelte $x^2y=9y$. Dann folgt $(x^2-9)y=0$. Wegen $x\neq 3$ ist $x^2\neq 9$, sodass $x^2-9\neq 0$. Daher können wir bei der Gleichung $(x^2-9)y=0$ beide Seiten durch $x^2-9$ teilen und erhalten $y=0$.
    \end{bew}
    \begin{enumerate}
        \item Findet den Fehler im Beweis.
        \item Widerlegt den Satz mit einem Gegenbeispiel.
    \end{enumerate}
\end{aufg}


\begin{aufg}[Fehlersuche II]
    Analysiert den folgenden Satz samt Beweis. Enthält der Beweis Fehler oder Lücken? Stimmt der Satz überhaupt?
    \begin{satz}
        Für jede natürliche Zahl $n$ existiert eine natürliche Zahl $m$ derart, dass keine der Zahlen $m+1,m+2,\dots , m+n$ eine Primzahl ist. Mit anderen Worten: Es gibt beliebig große Primzahllücken.
    \end{satz}
    \begin{bew}
        Es sei $n$ eine beliebige natürliche Zahl. Setze 
            \[ m:= (1 \cdot 2 \cdot 3 \cdot \ldots \cdot n \cdot(n+1)) +1 \]
        Dann ist $m+1$ durch $2$ teilbar, $m+2$ durch $3$ teilbar und allgemein $m+k$ durch $k+1$ teilbar für jedes $k\le n$. Also ist keine der Zahlen $m+1,m+2,\dots , m+n$ eine Primzahl. \qed
    \end{bew}
\end{aufg}


\begin{aufg}[Das Dorf der Lügner]
    % --- Der Aufgabentext enthält keine Tippfehler! ---
    Im beschaulichen Dörfchen Lügenscheid im Sauerland leben genau siebzig Menschen. Jeder dieser Einwohner 
    \begin{itemize}
        \item sagt entweder immer die Wahrheit
        \item oder aber alles was er sagt ist immer gelogen.
    \end{itemize}
    Fasziniert von dieser Begebenheit reist Beate Weiß, eine Mathematik-Postdoktorandin aus Wahrschau, deren Habilitation im Bereich der angewandten Logik seit Monaten von ihrer Fakultät ausgebremst wird, um sie solange es geht mit prekären befristeten Anstellungen bei der Stange zu halten, in das Dorf, um endlich einen Aufhänger für ihre Arbeit zu finden. Nachdem sie sich im Gasthaus eingerichtet hat, lädt sie alle siebzig Einwohner nacheinander zu einer Befragung ein, bei der sie nur eine einzige Frage stellt: „Wieviele Einwohner dieses Dorfs lügen immer?“
    
    Die erste Person, die zur Befragung erscheint, ein gutmütig aussehender, älterer Herr im Pullunder, sagt ihr, sie müsse sich keine Sorgen machen: im Dorf gebe es nur einen einzigen Lügner -- alle anderen Einwohner sagten dagegen immer die Wahrheit.
    
    Die zweite Person in der Befragung, eine großgewachsene, elegant gekleidete Dame mittleren Alters, sagt ihr, es gebe genau zwei Lügner im Ort.
    
    So geht es immer weiter: Die dritte Person meint, es gebe genau drei Lügner im Dorf, die vierte Person sagt, es gebe genau vier Lügner usw.
    
    Als bereits der Abend angebrochen ist, erscheint endlich auch Einwohner Nummer siebzig, eine greise, etwas orientierungslos wirkende Dame, zur Befragung und krächzt: „Der Teufel soll uns holen. In diesem Dorf gibt es keinen einzigen ehrlichen Menschen. Wir alle sind dazu verflucht, tagein, tagaus zu lügen!“
    
    Frau Weiß ist zufrieden: innerhalb eines Tages ist es ihr gelungen, jeden einzelnen Einwohner des Dorfs zu befragen. Bei einem Bierchen in der Ortskneipe knobelt sie über dem Ergebnis ihrer Studie und versucht herauszufinden, wieviele Lügner es denn nun tatsächlich in Lügenscheid gibt.
    \begin{enumerate}
        \item Findet heraus, wieviele Lügner es in Lügenscheid gibt, und formuliert einen Beweis für eure Behauptung.
        \item Analysiert eure Argumentation. Welche Techniken (z.B. direkter Beweis, indirekter Beweis, Widerspruchsbeweis, Fallunterscheidung) kommen darin zum Einsatz? Kann der Beweis vielleicht noch vereinfacht werden?
    \end{enumerate}
\end{aufg}



% ====================================================================
% 3. Vortrag  <<Mengen, natürliche Zahlen, Induktion>>
% ====================================================================

\setchapterpreamble[c][.7\textwidth]{%
  \itshape\dgrau\small Die Menge ist der Grundbaustein so gut wie aller mathematischer Strukturen. Die aussagenlogischen Junktoren geben Anlass zu Mengenoperationen, die mithilfe des Begriffs der Familie verallgemeinert werden können.
  \vspace{24pt}%
}

\chapter{Mengen und Familien}


\setchapterpreamble[c][.7\textwidth]{\itshape\dgrau\small
    In diesem Vortrag werden grundlegende Eigenschaften und Operationen von Mengen vorgestellt. Ebenso werden Familien und Tupel eingeführt, da sie grundlegend für die Konzepte „Relation“, „Folge“ und „Verknüpfung“ aus den späteren Vorträgen sind.
\vspace{24pt}}

    
\chapter{Mengen und Familien}


\section{Mengen und Elemente}


Der Begriff der Menge wurde bereits im Vortrag über Logik in \cref{mengenimlogikkapitel} angesprochen. Seine Einführung in die Mathematik erfolgte durch Georg Cantor\footnote{\href{https://de.wikipedia.org/wiki/Georg_Cantor}{Georg Cantor (1845-1918)}} in den 1870er Jahren. Cantor beschreibt seine Idee in \cite{Can95} mit den folgenden Worten:


\begin{de}[Mengendefinition von Cantor] \label{def:menge} \index{Menge}
    „Unter einer \textbf{Menge} $M$ verstehen wir jede Zusammenfassung von bestimmten wohlunterschiedenen Objekten $m$ unserer Anschauung oder unseres Denkens (welche die \textbf{Elemente} von $M$ genannt werden) zu einem Ganzen.“
\end{de}


\begin{bem}
    Eine (möglicherweise fiktive) Anekdote aus \cite{Ded32}, S. 449 beschreibt folgende „Veranschaulichungen“ des Mengenbegriffs:
    \begin{quote}
        Dedekind\footnote{\href{https://de.wikipedia.org/wiki/Richard_Dedekind}{Richard Dedekind (1831-1916)}} äußerte, hinsichtlich des Begriffes der Menge: er stelle sich eine Menge vor wie einen geschlossenen Sack, der ganz bestimmte Dinge enthalte, die man aber nicht sähe, und von denen man nichts wisse, außer dass sie vorhanden und bestimmt seien. Einige Zeit später gab Cantor seine Vorstellung einer Menge zu erkennen: Er richtete seine kolossale Figur hoch auf, beschrieb mit erhobenem Arm eine großartige Geste und sagte mit einem ins Unbestimmte gerichteten Blick: „Eine Menge stelle ich mir vor wie einen Abgrund.“
    \end{quote}
\end{bem}


\begin{nota}[Elementzeichen]
    Seien $M$ eine Menge und $x$ irgendein Objekt. Man schreibt
    \begin{align*}
        x  \in M \qquad&:\Leftrightarrow\qquad \text{$x$ ist ein Element von $M$} && (\text{lies: „$x$ in $M$“})\\
        x\notin M \qquad &:\Leftrightarrow\qquad \text{$x$ ist kein Element von $M$} && (\text{lies: „$x$ nicht in $M$“})
    \end{align*}
    Das Symbol „$\in$“ bezeichnet ein zweistelliges Prädikat im Sinne von \cref{def:praedikat}.
\end{nota}


\begin{axiom}[Extensionalitätsaxiom] \label{mengengleich} \index{Extensionalitätsaxiom}
    Zwei Mengen $M,N$ stimmen genau dann überein, wenn sie dieselben Elemente haben. Als Formel:
        \[ (\forall x:\quad x\in M\ \leftrightarrow\ x\in N)\qquad\leftrightarrow\qquad M=N\]
    Dementsprechend sind zwei Mengen \emph{verschieden}, wenn mindestens eine der beiden ein Element enthält, das die andere nicht enthält.
\end{axiom}


\begin{de}[Extension einer Eigenschaft] \label{def:extension} \index{Extension (einer Eigenschaft)}
    Sei $E(x)$ eine Eigenschaft. Die Menge aller Objekte mit der Eigenschaft $E$ ist wegen des Extensionalitätsaxioms \cref{mengengleich} eindeutig bestimmt. Sie heißt die \textbf{Extension}\footnote{vgl. \cref{extensionimlogikkapitel}} von $E$ und wird notiert durch
    \begin{align*}
        \{ x\mid E(x) \} && (\textnormal{lies: „Menge aller $x$, für die gilt: $E(x)$“})
    \end{align*}
    Per Definition gilt:
        \[ \forall a:\qquad a\in \{x\mid E(x)\} \ \leftrightarrow\ E(a) \]
    Für eine Menge $M$ von Objekten vom Typ der Variablen $x$ wird durch
        \[ \{ x\in M\mid E(x) \} := \{ x\mid x\in M\ \text{und}\ E(x)\} \]
    die Menge aller Elemente von $M$, die die Eigenschaft $E$ besitzen, notiert.
\end{de}


\begin{bsp} \label{bsp:extension}
    Das Konzept „Extension einer Eigenschaft“ erlaubt es, Mengen mittels Eigenschaften zu definieren. Zum Beispiel:
    \begin{enumerate}
        \item Es ist $\bbP := \{p\mid p\ \text{ist eine Primzahl}\}$ die Menge aller Primzahlen. Beispielsweise sind $23\in \bbP$ und $97\in\bbP$, denn $23$ und $97$ sind Primzahlen. Dagegen ist $91\notin \bbP$, denn $91$ ist keine Primzahl wegen $91=7\cdot 13$.
        \item Es ist $\{ x\mid x\ \text{liest dieses Vorkurs-Skript} \}$ die Menge aller Leser dieses Skripts. Zum Beispiel bist DU ein Element dieser Menge.
        \item Es ist $[0,1] := \{ x\in \R \mid 0\le x\le 1\}$ die Menge aller reellen Zahlen zwischen Null und Eins, das sogenannte (abgeschlossene) \emph{Einheitsintervall}.
        \item Es ist $Q:=\{ n\in \N \mid \exists m\in \N:\ n=m^2 \}$ die Menge aller Quadratzahlen. Beispielsweise sind $100\in Q$ und $529\in Q$, aber $12\notin Q$ und $1000\notin Q$.
    \end{enumerate}
\end{bsp}


\begin{nota}[Definition einer Menge durch Auflistung] \label{auflistung}
    Seien $n\in \N$ und $a_1,\dots , a_n$ eine Handvoll Objekte. Dann bezeichnet
        \[ \{a_1,\dots , a_n\} := \{ x\mid x=a_1\ \text{oder}\ x=a_2\ \text{oder}\ \ldots\ \text{oder}\ x=a_n \} \]
    diejenige Menge, die genau aus $a_1,\dots , a_n$ besteht. Auf diese Weise kannst du Mengen mit „wenigen“ Elementen schlicht durch Auflistung aller ihrer Elemente definieren.
\end{nota}


\begin{bsp}
    Zum Beispiel ist
        \[ M:=\{2,3,5,7,11,13,17,19,23,29,31,37,41,43,47,53,59,61,67,71,73,79,83,89,97\} \]
    diejenige Menge, die genau aus den Zahlen $2,3,5,\dots , 97$ besteht. Es gilt
        \[ M = \{p\mid p\ \text{ist eine Primzahl und}\ 1< p <100 \} \]
    d.h. $M$ ist die Menge aller Primzahlen zwischen $1$ und $100$. Sie enthält genau fünfundzwanzig Elemente.
\end{bsp}


\begin{bem}
    Gelegentlich werden auch Mengen mit unendlich vielen Elementen über eine Art „Auflistung“ definiert, zum Beispiel
        \[ M := \{1, 3, 5, 7, 9,\ldots \} \]
    Diese Art von „Definition“ setzt beim Leser das Erkennen eines intendierten Musters voraus und kann zu Missverständnissen führen. Im Zweifelsfall solltest du eine Definition durch eine Eigenschaft bevorzugen, in diesem Fall etwa
        \[ M := \{n \in \N \mid n\ \text{ist ungerade} \} \]
\end{bem}


\begin{nota}[*] \label{beschraenktquant}
    Seien $M$ eine Menge und $E$ eine Eigenschaft. Für Aussagen über die Elemente von $M$ gibt es folgende Schreibweisen\footnote{vgl. \cref{def:allquant}, \cref{def:existquant} und \cref{def:eindquant}}
    \begin{align*}
        \forall x\in M:\ E(x) \qquad :& \Leftrightarrow\qquad \forall x:\ (x\in M\ \to\ E(x))  \\
        (\text{lies: „Für jedes $x$ aus $M$ gilt $E(x)$“}) & \\[0.5em]
        \exists x\in M:\ E(x) \qquad :& \Leftrightarrow\qquad \exists x:\ (x\in M\ \land\ E(x))  \\
        (\text{lies: „Es gibt ein $x$ in $M$, für das $E(x)$ gilt“}) & \\[0.5em]
        \nexists x\in M:\ E(x) \qquad :& \Leftrightarrow\qquad \nexists x:\ (x\in M\ \land\ E(x))  \\
        (\text{lies: „Es gibt kein $x$ in $M$, für das $E(x)$ gilt“}) & \\[0.5em]
        \exists ! x\in M :\ E(x) \qquad :& \Leftrightarrow\qquad \exists ! x:\ (x\in M\ \land\ E(x)) \\ (\text{lies: „Es gibt genau ein $x$ in $M$, für das $E(x)$ gilt“})
    \end{align*}
    um auszudrücken, dass jedes/mindestens eines/keines/genau eines der Elemente von $M$ die Eigenschaft $E$ besitzt.
\end{nota}


\begin{bsp}[*]
    Es gilt:
    \begin{align*}
        \forall n\in \N &:\ n\cdot (n+1)\ \text{ist eine gerade Zahl} && (\text{vgl. \cref{bsp:fallunterscheidung}})\\
        \exists x\in \R &:\ x=x^5-1 && (\text{vgl. \cref{bsp:exverwendung}})\\
        \nexists n\in \N &:\ 17\cdot n = 198 && (\text{vgl. \cref{bsp:reductio}})\\ 
        \exists ! x\in \R \ \forall y\in \R & :\ xy=x &&(\text{vgl. \cref{bsp:eindbeweis}})
    \end{align*}
\end{bsp}





\section{Teilmengen}


\begin{de}[Teilmenge] \label{def:teilmenge} \index{Teilmenge}
    Seien $M$ und $N$ zwei Mengen.
    \begin{itemize}
        \item $M$ heißt eine \textbf{Teilmenge} (manchmal auch: \textbf{Untermenge}) von $N$ (und $N$ eine \textbf{Obermenge} von $M$), falls jedes Element von $M$ auch ein Element von $N$ ist. Notation:
        \begin{align*}
            M \subseteq N \qquad &:\Leftrightarrow\qquad M\ \text{ist eine Teilmenge von $N$} \\
            M \nsubseteq N \qquad &:\Leftrightarrow\qquad M\ \text{ist keine Teilmenge von $N$}
        \end{align*}
        \item $M$ heißt eine \textbf{echte Teilmenge} von $N$, falls $M$ eine Teilmenge von $N$ ist, aber von $N$ verschieden ist. Notation:
        \begin{align*}
            M\subsetneq N \qquad:\Leftrightarrow\qquad M\subseteq N\ \text{und}\ M\neq N
        \end{align*}
        Mit anderen Worten: Jedes Element von $M$ ist auch eines von $N$, wohingegen es mindestens ein Element von $N$ gibt, das keines von $M$ ist.
        \begin{figure}[ht]
            \begin{tikzpicture}
                \filldraw[fill=blue!10!white, draw=black] (0,0) ellipse (2cm and 1.3cm);
                \filldraw[fill=blue!25!white, draw=black] (.5,0) ellipse (1cm and .7cm);
                \path (-1.6,0) node {$N$} (-.1,0) node {$M$};
            \end{tikzpicture}
            \centering \caption{Die Menge $M$ ist Teilmenge der Menge $N$.}
        \end{figure}
    \end{itemize}
\end{de}


\begin{bsp} \quad
    \begin{enumerate}
        \item Es ist $\{m\mid m\ \text{ist ein Mensch}\}\subseteq \{t\mid t\ \text{ist ein Säugetier}\}$, da alle Menschen Säugetiere sind.
        \item Es ist $\Q\subsetneq \R$, da zwar jede rationale Zahl eine reelle Zahl ist, es aber auch irrationale reelle Zahlen, zum Beispiel $\sqrt{2}$, gibt.
    \end{enumerate}
\end{bsp}


\begin{nota}[Alternative Schreibweisen]
    Bisweilen werden in der Literatur auch abweichende Notationen benutzt: etwa $M\subseteq N$ für beliebige und $M\subset N$ für echte Teilmengen; oder aber $N\subseteq M$ für beliebige und $M\subsetneqq N$ für echte Teilemengen. Hier ist eine Liste verschiedener Konventionen:
    \[\begin{tabular}{c|ccc}
        & Teilmenge & echte Teilmenge \\ \hline
        Dieser Text & $\subseteq$ & $\subsetneq$ \\
        Alternative 1 & $\subseteq$ & $\subset$ \\
        Alternative 2 & $\subset$ & $\subsetneq$ oder $\subsetneqq$
    \end{tabular}\]
\end{nota}


\begin{bem}[Teilmengeninklusionen beweisen/widerlegen] \label{teilmengebeweisen}
    Seien $M,N$ zwei Mengen. Da es sich bei der Aussage „$M\subseteq N$“ um eine Allaussage handelt (\emph{jedes} Element von $M$ ist ein Element von $N$), kannst du sie mit der Technik aus \cref{allbeweis} beweisen: fixiere ein beliebiges Element von $M$ und zeige irgendwie, dass dies auch ein Element von $N$ ist.
    
    Wenn du meinst, dass „$M\subseteq N$” eine falsche Aussage ist, kannst du sie mit der Technik aus \cref{gegenbeispiel} widerlegen: finde ein Element von $M$, das kein Element von $N$ ist.
\end{bem}


\begin{satz}[* Eigenschaften von $\subseteq$]\label{teilmengeneig}
    Seien $L, M, N$ drei Mengen. Dann gilt:
    \begin{enumerate}
        \item $M\subseteq M$ (Reflexivität).
        \item Aus $L\subseteq M$ und $M\subseteq N$ folgt $L\subseteq N$ (Transitivität).
        \item Aus $M \subseteq N$ und $N\subseteq M$ folgt $M=N$ (Antisymmetrie).
    \end{enumerate}
    \begin{figure}[ht]
        \begin{tikzpicture}
            \filldraw[fill=blue!10!white, draw=black] (0,0) ellipse (2cm and 1.3cm);
            \filldraw[fill=blue!20!white, draw=black] (.5,0) ellipse (1.3cm and .9cm);
            \filldraw[fill=blue!30!white, draw=black] (.8,0) ellipse (.7cm and .5cm);
            \path (-1.6,0) node {$N$} (-.4,0) node {$M$} (.5,0) node {$L$};
        \end{tikzpicture}
        \centering \caption{Illustration der Transitivität von „$\subseteq$“}
    \end{figure}
\end{satz}


\begin{bew}
    \begin{enumerate}
        \item Für alle Elemente $m\in M$ gilt per se schon $m\in M$, also ist $M$ in $M$ enthalten.\footnote{vgl. \cref{implikationref}: Jede Aussage impliziert sich selbst.}
        \item Sei $x\in L$ beliebig. Wegen $L\subseteq M$ gilt dann $x\in M$. Aus $x\in M$ folgt wegen $M\subseteq N$, dass auch $x\in N$. Da das Element $x\in L$ beliebig gewählt war, ist damit $L\subseteq N$ bewiesen.\footnote{Das war ein direkter Beweis mit Zwischenschritten, vgl. \cref{implikationtrans}:
            \[ x\in L\xrightarrow{L\subseteq N} x\in N \xrightarrow{N\subseteq M} x\in M \]}
        \item Wegen $M\subseteq N$ und $N\subseteq M$ ist jedes Element von $M$ auch eines von $N$ und umgekehrt. Mit anderen Worten: $M$ und $N$ haben genau dieselben Elemente. Aus dem Extensionalitätsaxiom \cref{mengengleich} folgt nun, dass $M=N$. \qed
    \end{enumerate}
\end{bew}


\begin{bem}[* Gleichheit von Mengen beweisen] \label{mengengleichbeweis}
    Kombinieren von \cref{teilmengebeweisen} und \cref{teilmengeneig}(c) liefert eine Methode, die Gleichheit zweier Mengen $M,N$ zu zeigen, indem nacheinander die Inklusion $M\subseteq N$ und die Inklusion $N\subseteq M$ gezeigt werden.
    
    Aus beiden Inklusionen folgt dann mit \cref{teilmengeneig}(c), dass $M=N$. In Beweisen solltest du die Abschnitte, die den einzelnen Inklusionen $M\subseteq N$ und $N\subseteq M$ gewidmet sind, durch Markierungen deutlich machen.
    
    Diese Beweistechnik ist auch Inhalt von \cref{aufg:capcupgesetze} .
\end{bem}


\begin{bsp}[*] \label{bsp:mengengleichbeweis}
    Es ist $\{1,2,3\}=\{1,1,3,2\}$.
\end{bsp}


\begin{bew}
    \begin{itemize}
        \item[„$\subseteq$“] Sei $x\in\{1,2,3\}$, also $x=1$, $x=2$ oder $x=3$. In jedem dieser Fälle\footnote{vgl. \cref{auflistung} und \cref{fallunterscheidung}} gilt auch $x\in\{1,1,3,2\}$.
        \item[„$\supseteq$“] Sei umgekehrt $x\in\{1,1,3,2\}$,  also $x=1$, $x=1$, $x=3$ oder $x=2$. In jedem dieser Fälle gilt auch $x\in\{1,2,3\}$. \qed
    \end{itemize}
\end{bew}


\begin{bem} \label{mengenstrukturlos}
    Im letzten Beispiel sehen wir ganz deutlich, dass die Elemente einer Menge keiner Reihenfolge oder „Zählung“ unterliegen: Eine Menge trägt von sich aus bis auf ihre Elemente keine weitere Struktur; Fragen wie „Wie genau / Wie oft / An welcher Stelle ist $x$ in $M$ enthalten?“ kann sie also nicht beantworten, sondern nur „Ist $m$ in $M$ enthalten (oder nicht)?“.
    
    Möchte man den Elementen dennoch verschiedene „Plätze“ zuweisen, bietet sich der Begriff der „Familie“ an, der später in \cref{def:familie} eingeführt wird.
\end{bem}





\section{Die leere Menge}


\begin{de} \index{Leere Menge} \index{Einermenge}
    Eine Menge heißt
    \begin{itemize}
        \item \textbf{leer}, wenn sie gar keine Elemente enthält.
        \item \textbf{nichtleer}, wenn sie nicht leer ist.\footnote{Im Englischen nennt man eine Menge auch ``\href{https://en.wikipedia.org/wiki/Inhabited_set}{inhabited}'', wenn sie mindestens ein Element enthält.}
        \item eine \textbf{Einermenge} (englisch: ``singleton''), wenn sie genau ein Element enthält, also von der Form $\{x\}$ für irgendein Objekt $x$ ist.
    \end{itemize}
\end{de}


\begin{satz} \label{leeremengeeind}
    Es gibt genau eine leere Menge.
\end{satz}


\begin{bew}
    (Existenz): Sei $E(x)$ irgendeine Eigenschaft, die kein einziges Objekt besitzt (z.B. „$x\neq x$“). Dann ist $\{ x\mid E(x) \}$ eine leere Menge. \\[0.5em]
    (Eindeutigkeit): Seien $M,N$ zwei leere Mengen. Dann enthalten $M$ und $N$ genau dieselben Elemente (nämlich gar keine), sodass aus \cref{mengengleich} folgt, dass $M=N$.
\end{bew}


\begin{nota}[\textbf{Die} leere Menge]
    Somit ergibt es Sinn, von \emph{der} leeren Menge zu sprechen. Die leere Menge wird meist notiert durch
        \[ \emptyset \]
    Angelehnt an \cref{auflistung} notieren manche Autoren die leere Menge auch mit $\{\}$.
\end{nota}


\begin{bsp}
    Es gilt:
    \begin{enumerate}
        \item Die Menge $M:=\{p\mid p\ \text{ist eine gerade Primzahl}\}$ ist nichtleer, weil die $2$ eine gerade Primzahl ist. Weil $2$ auch die einzige gerade Primzahl ist, ist sogar $M=\{2\}$, d.h. $M$ ist eine Einermenge.
        \item Die Menge $M:= \{ q\in \Q\mid q^2=2\}$ ist leer, d.h. $M=\emptyset$, da die $2$ keine rationale Quadratwurzel besitzt.
    \end{enumerate}
\end{bsp}


\begin{bem}[Vacuous Truth] \label{vacuoustruth} \index{vacuous truth}
    Eine weitere Feinheit im Umgang mit leeren Mengen sind so genannte ``vacuous truths''. Das sind Aussagen wie
    \begin{itemize}
        \item „Alle rosa Elefanten können fliegen.“
        \item „Alle meine Geschwister sind Gürteltiere“, gesprochen von einem Einzelkind.
    \end{itemize}	
    Formal sind diese Aussagen von der Form
        \[ \forall x\in M:\ E(x) \]
    wobei $E$ in unserem Fall die Eigenschaft „kann fliegen“ bzw. „ist ein Gürteltier“, war, und $M$ die Menge der rosa Elefanten bzw. der Geschwister des Einzelkindes war. Diese Mengen haben keine Elemente, sind also gleich der leeren Menge:
        \[ \forall x\in \emptyset:\ E(x) \]
    Gemäß \cref{beschraenktquant} ist diese Formel gleichbedeutend zu
        \[ \forall x:\qquad x\in \emptyset\ \to\ E(x) \]
    Die Aussage $x\in\emptyset$ ist aber für alle Objekte $x$ falsch, denn die leere Menge hat keine Elemente. Nach dem Prinzip \emph{ex falso quodlibet} aus \cref{exfalso} ist die Implikation $x\in \emptyset\ \to\ E(x)$ also für jedes $x$ wahr, und somit auch $\forall x: \left(x\in \emptyset\ \to\ E(x)\right)$ eine wahre Aussage.
	
    Fazit: Aussagen, die den Elementen einer leeren Menge (wie der Menge der rosa Elefanten oder den Geschwistern eines Einzelkindes) eine gewisse Eigenschaft zuordnen, sind immer wahr, egal was die Eigenschaft ist.
\end{bem}


\begin{satz} \label{leeremengeimmerdrin}
    Die leere Menge $\emptyset$ ist Teilmenge jeder beliebigen Menge.
\end{satz}


\begin{bew}
    Sei $M$ eine beliebige Menge. Zu zeigen ist $\emptyset\subseteq M$, also
        \[ \forall x:\qquad x\in \emptyset\ \to\ x\in M \]
    was nach \emph{ex falso quodlibet} eine wahre Aussage ist. \qed
\end{bew}





\section{Die Potenzmenge}


\begin{de}[Potenzmenge] \label{def:potenzmenge} \index{Potenzmenge}
    Sei $M$ eine Menge. Die Menge aller Teilmengen von $M$ heißt die \textbf{Potenzmenge} von $M$ und wird mit „$\calP(M)$“ notiert:
    \begin{align*}
        \calP(M) := \{N \mid N \ \text{ist eine Teilmenge von $M$} \}
    \end{align*}
\end{de}


\begin{bem}
    Sei $M$ eine Menge. Nach \cref{leeremengeimmerdrin} und \cref{teilmengeneig}a) gilt auf jeden Fall $\emptyset\in \calP(M)$ und $M\in \calP(M)$.
\end{bem}


\begin{bsp} \label{bsp:potenzmenge}
    Sei $M:=\{1,2,3\}$. Dann ist
        \[ \calP(M)=\left\{\emptyset,\{1\},\{2\},\{3\},\{1,2\},\{1,3\},\{2,3\},\{1,2,3\}\right\}\,. \]
\end{bsp}


\noindent In alten Büchern wird die Potenzmenge einer Menge $X$ auch mit „$2^X$“ notiert, woher der Name „Potenzmenge“ stammt. Es lässt sich zeigen: ist $n\in \N$ und enthält die Menge $M$ genau $n$-viele Elemente, so enthält $\calP(M)$ genau $2^n$-viele Elemente.


\begin{bem}
    Die Konzepte „Einermenge“ und „Potenzmenge“ erlauben es, Elementaussagen und Teilmengenaussagen ineinander zu überführen. Denn für Mengen $M,N$ und Objekte $x$ gilt:
    \begin{align*}
        x\in M\ & \leftrightarrow\ \{x\}\subseteq M & M\subseteq N\ &\leftrightarrow\ M\in \calP(N)
    \end{align*}
\end{bem}


\begin{vorschau}[* Mengen von Mengen von Mengen von \dots]
    Das Konzept der Menge ist iterativ. Eine Menge kann wiederum Element einer anderen Menge sein, die wiederum in einer Menge enthalten sein kann usw.. Zum Beispiel ist $\N$ ein Element der Einermenge $\{\N\}$, die wiederum ein Element von $\{ \{\N \}\}$ ist.

    Ist $M$ eine Menge, so solltest du dir den Unterschied zwischen $M$ und $\{M\}$ klarmachen. Betrachte etwa $\emptyset$ im Vergleich mit $\{\emptyset\}$. Auch wenn es zunächst scheinen mag, dass in beiden Mengen letztlich „nichts drin ist“ , so liegen doch zwei verschiedene Mengen vor. Mit $\emptyset$ haben wir nämlich eine Menge mit gar keinen Elementen, mit $\{\emptyset\}$ dagegen eine Einermenge, deren einziges Element die leere Menge $\emptyset$ ist.

    Das \href{https://de.wikipedia.org/wiki/Nat\%C3\%BCrliche_Zahl#Von_Neumanns_Modell_der_nat\%C3\%BCrlichen_Zahlen}{von Neumannsche Modell der natürlichen Zahlen} macht sich genau diesen Unterschied zunutze. Es ermöglicht die Simulation natürlicher Zahlen durch gewisse Mengen und demonstriert dadurch, dass sich der Begriff der natürlichen Zahl, wenn man es denn will, zurückführen lässt auf den Begriff der Menge.
\end{vorschau}


\begin{de}[Mengensystem] \index{System von Mengen}
    Eine Menge $\calM$ heißt ein \textbf{Mengensystem}, wenn jedes ihrer Elemente ebenfalls eine Menge ist.

    Ist $M$ irgendeine Menge, so ist beispielsweise jede Teilmenge $\calT\subseteq \calP(M)$ ein Mengensystem. Man nennt dann $\calT$ auch ein \textbf{System von Teilmengen von $M$}.
\end{de}


\begin{nota}[Sprechweisen bei $\in$ und $\subseteq$]
    Seien $M,N$ zwei Mengen und $x$ irgendein Objekt. Lesarten für die Elementaussage „$x\in M$“ sind
    \begin{itemize}
        \item „$x$ ist ein Element von $M$“.
        \item „$x$ liegt in $M$“.
        \item „$x$ ist enthalten in $M$“.
        \item „$x$ wird von $M$ umfasst“.
    \end{itemize}
    Lesarten für die Teilmengenaussage „$N\subseteq M$“ sind:
    \begin{itemize}
        \item „$N$ ist eine Teilmenge von $M$“.
        \item „$N$ liegt in $M$”.
        \item „$N$ ist enthalten in $M$“.
        \item „$N$ wird von $M$ umfasst“.
    \end{itemize}
    Die Ausdrücke „liegt in“, „enthält“, „umfasst“ sind also mehrdeutig und können sowohl ein Elementverhältnis als auch ein Teilmengenverhältnis beschreiben. Ohne Weiteres ist nicht klar, ob „$M$ liegt in $N$“ besagen soll, dass $M\in N$ oder $M\subseteq N$. Beides ist möglich; zum Beispiel ist
    \begin{align*}
        \N \subseteq \Z \qquad & \text{aber}\qquad \N \notin \Z \\
        \N \in \{ \N,\Z,\Q,\R,\C\} \qquad & \text{aber}\qquad \N \nsubseteq  \{ \N,\Z,\Q,\R,\C\}
    \end{align*}
    Meistens ist die Bedeutung der Wörter „enthält“, „umfasst“ etc. aus dem Kontext unmittelbar ersichtlich, weshalb deren Mehrdeutigkeit unproblematisch ist. Besteht dennoch die Gefahr eines Missverständnisses, solltest du explizit „ist Teilmenge von“ oder „ist Element von“ schreiben.
\end{nota}





\section{Familien}


Mit dem Begriff der Familie möchten wir einer Sammlung von Objekten eine zusätzliche Struktur verleihen: Statt uns nur zu merken, welche Objekte in unserer Sammlung enthalten sind (wie dies bei Mengen der Fall ist), unterscheiden wir nun verschiedene „Stellen“, an denen sich Objekte in der Sammlung befinden können. Realisiert wird das, indem wir die Elemente unserer Sammlung mit Indizes versehen und dann von dem „Objekt am Index $i$/an der Stelle $i$“ sprechen:


\begin{de}[Familien] \label{def:familie} \index{Familie} \index{Indexmenge} \index{Einträge einer Familie}
    Eine \textbf{Familie} $a$ besteht aus
    \begin{itemize}
        \item Einer beliebigen Menge $I$, welche die \textbf{Indexmenge} der Familie genannt wird und deren Elemente die \textbf{Indizes} der Familie heißen.
        \item Für jeden Index $i\in I$ ein Objekt „$a_i$“, welches der \textbf{Eintrag an der $i$-ten Stelle} oder die \textbf{$i$-te Komponente} der Familie $a$ genannt wird.
    \end{itemize}
    Die Familie $a$ wird dann notiert als
    \begin{align*}
        (a_i)_{i\in I} && (\text{lies: „$a_i$, $i$ aus $I$“})
    \end{align*}
    Man spricht auch von einer \emph{Familie mit Indexmenge $I$} oder einer \textbf{durch $I$ indizierten Familie}.
\end{de}


\begin{axiom}[Gleichheit von Familien] \label{familiengleich}
    Seien $I$ irgendeine Menge und $(a_i)_{i\in I}$ und $(b_i)_{i\in I}$ zwei durch $I$ indizierte Familien, so sind diese beiden Familien genau dann gleich, wenn sie an jedem Index denselben Eintrag besitzen. Als Formel:
    \begin{align*}
        (a_i)_{i\in I}=(b_i)_{i\in I} \qquad\leftrightarrow\qquad \forall i\in I:\ a_i=b_i
    \end{align*}
    Dementsprechend sind zwei solche Familien \emph{verschieden}, wenn sie sich an mindestens einem Index unterscheiden.
    
    Zwei Familien mit verschiedenen Indexmengen sind von vornherein voneinander verschieden.
\end{axiom}


\begin{nota} \label{mengeeinerfamilie} \quad
    \begin{itemize}
        \item Wenn ich, ohne vorher die Menge $I$ definiert zu haben, so etwas wie „Sei $(a_i)_{i\in I}$ eine Familie“ schreibe, so meine ich damit „Seien $I$ eine beliebige Menge und $(a_i)_{i\in I}$ eine Familie mit Indexmenge $I$“.
        \item Sei $a=(a_i)_{i\in I}$ eine Familie. Dann bezeichnet
            \[ \{a_i \mid i\in I\} := \{x\mid \exists i\in I:\ x=a_i \} \]
        die Menge der Einträge der Familie $a$. Beim Übergang von $(a_i)_{i\in I}$ zu $\{a_i\mid i\in I\}$ wird die Indizierung „vergessen“.
    \end{itemize}
\end{nota}


\begin{bsp} \quad
    \begin{enumerate}
	\item Nehmen wir als Indexmenge $I$ die Menge der belegten Stühle im Raum und als das Objekt $a_i$ die Person, die auf Stuhl $i$ sitzt, so erhalten wir eine Familie der hier im Raum sitzenden Leute $(a_i)_{i\in I}$. Sie unterscheidet sich von der \textit{Menge} der hier im Raum sitzenden Leute dadurch, dass sie auch die Information darüber, wer auf welchem Stuhl sitzt, enthält.
	
	Sind $i,j\in I$ zwei belegte Stühle und tauschen die Personen $a_i,a_j$ auf diesen beiden Stühlen ihren Platz, so wird aus der Familie $(a_i)_{i\in I}$ der im Raum sitzende Leute eine neue Familie $(b_i)_{i\in I}$. Diese beiden Familien sind nicht gleich, denn es gilt $b_i=a_j\neq a_i$ und $b_j=a_i\neq a_j$.
	
	Dagegen hat sich die \emph{Menge} der im Raum sitzenden Leute durch den Sitzplatzwechsel nicht verändert, da immer noch dieselben Leute im Raum sitzen (nur eben an anderen Stellen). Bei $\{a_i\mid i\in I\}$ und $\{b_i \mid i\in I\}$ handelt es sich also um dieselbe Menge, während $(a_i)_{i\in I}$ und $(b_i)_{i\in I}$ zwei verschiedene Familien sind.
	
        Hieran wird der Unterschied zwischen den Begriffen „Familie“ und „Menge“ deutlich. Zwei Mengen, in denen die gleichen Elemente vorkommen, sind nach \cref{mengengleich} schon gleich; für eine Gleichheit von Familien müssen dagegen die Elemente sich auch noch an denselben „Stellen“ befinden. Im Gegensatz zu Mengen haben Familien also eine gewisse durch die Indizierung gegebene Zusatzstruktur.\footnote{vgl. \cref{mengenstrukturlos}}
	\item Bei Matrizen und den ``arrays'' aus aus der Informatik handelt es sich um Familien, siehe \cref{bsp:matrizen}.
    \end{enumerate}
\end{bsp}


\begin{de}[Menge der Familien] \label{def:mengenpotenz}
    Seien $I,M$ zwei beliebige Mengen und $(a_i)_{i\in I}$ eine durch $I$ indizierte Familie. Ist jedes der $a_i$'s ein Element von $M$, so nennt man die Familie $(a_i)_{i\in I}$ eine \textbf{(durch $I$ indizierte) Familie mit Einträgen aus $M$} oder auch eine \textbf{$M$-wertige Familie}.
    
    Die Menge aller Familien mit Indexmenge $I$ und Einträgen aus $M$ wird mit $M^I$ notiert:
        \[ M^I := \{ (a_i)_{i\in I} \mid \forall i\in I:\ a_i \in I \} \]
    $M^I$ heißt auch die „$I$-te Potenz von $M$“, was aber nicht mit der Potenzmenge aus \cref{def:potenzmenge} verwechselt werden sollte.
\end{de}


\begin{bsp}[Folgen]
    Familien, deren Indexmenge die Menge $\N$ der natürlichen Zahlen ist, heißen \textbf{Folgen} und sind ein zentrales Thema im sechsten Vortrag.\footnote{siehe \cref{def:folge}} Beispielsweise ist $\R^\N$ die Menge derjenigen Folgen, deren Einträge allesamt reelle Zahlen sind (man spricht von \emph{reellen Folgen}). Eine solche Folge ist beispielsweise die Folge $(a_n)_{n\in\N}$ mit $a_n:=\frac{1}{n}$
    \[\begin{tabular}{c|cccc}
            $n$ & 1 & 2 & 3 & $\dots$ \\ \midrule
            $a_n$ & 1 & 1/2 & 1/3 & $\dots$
    \end{tabular}\]
    oder die Folge $(n^2)_{n\in \N}$ der Quadratzahlen. Analog sind $\Z^\N$, $\Q^\N$, $\C^\N$ die Mengen aller Folgen mit ganzzahligen/rationalen/komplexen Einträgen. $(\Z^\N)^\N$ ist die Menge aller Folgen von Folgen von ganzen Zahlen.
\end{bsp}


\begin{de}[Tupel] \label{def:tupel} \index{Tupel}
    Sei $n\in \N$ eine natürliche Zahl. Familien $(a_i)_{i\in I}$ mit der Indexmenge $I=\{1,\dots,n\}$ heißen \textbf{$n$-Tupel} und werden notiert in der Gestalt
        \[ (a_1,\dots,a_n)  \]
    2-Tupel, also Objekte der Form $(a,b)$, heißen \textbf{(geordnete) Paare}, 3-Tupel werden auch \textbf{Tripel} genannt.
    
    Ist $A$ eine Menge, so bezeichnet „$A^n$“ die Menge aller $n$-Tupel mit Einträgen aus $A$:
        \[ A^n := \{ (a_1,\dots , a_n) \mid a_1,\dots , a_n \in A \} \]
    Gemäß \cref{familiengleich} stimmen zwei $n$-Tupel genau dann überein, wenn sie komponentenweise übereinstimmen:
	\[ (a_1,\dots , a_n)=(b_1,\dots , b_n) \qquad\leftrightarrow\qquad a_1=b_1,\ \ldots\ ,\ a_n=b_n \]
\end{de}


\begin{bsp}
    Es ist $\R^3$ die Menge aller Tripel reeller Zahlen:
        \[ \R^3 = \{(x,y,z) \mid x,y,z\in \R \} \]
    Diese Menge ist dir vielleicht schon aus der Schule bekannt. In der analytischen Geometrie repräsentieren die Elemente des $\R^3$ Punkte im Raum und die drei Einträge eines solchen Tripels repräsentieren seine Koordinaten in einem Koordinatensystem. In der Matrizenrechnung werden die Komponenten in einer senkrechten Anordnung notiert
    \[\begin{pmatrix}
        x \\ y \\ z 
    \end{pmatrix} \]
    man spricht von „Vektoren“. Hier wird auch die Signifikanz des Tupelbegriffs deutlich: Beispielsweise sind $\begin{pmatrix} 2 \\ 0 \\ 3 \end{pmatrix}$ und $\begin{pmatrix} 2 \\ 3 \\ 0 \end{pmatrix}$ zwei verschiedene Vektoren, denn sie unterscheiden sich in ihrer zweiten und dritten Komponente. Dagegen sind die \emph{Mengen} $\{2,0,3\}$ und $\{0,3,2\}$ identisch\footnote{vgl. \cref{bsp:mengengleichbeweis}(b)}. Bei einer bloßen Menge gibt es keinen „zweiten Eintrag“ oder dergleichen.
\end{bsp}


\begin{de}[Leere Familie] \label{def:leerefam}
    Es gibt genau eine Familie, deren Indexmenge die leere Menge ist. Weil diese Familie keine Indizes hat, besitzt sie auch keine Einträge. Man spricht von der \textbf{leeren Familie} und notiert sie, ebenso wie die leere Menge, mit dem Zeichen $\emptyset$.
\end{de}


\begin{bem}[* „Niedrige“ Potenzen]
    Sei $M$ eine beliebige Menge.
    \begin{itemize}
        \item(Nullte Potenz) Es ist eine ``vacuous truth'', dass alle Einträge der leeren Familie Elemente von $M$ sind. Daher ist $M^\emptyset=M^0 = \{\emptyset\}$ eine Einermenge, die als einziges Element die leere Familie enthält. Dies entspricht der Rechenregel, dass $x^0=1$ für jede Zahl $x$.
        \item(Erste Potenz) Es ist $M^1 = \{(x)\mid x\in M\}$ die Menge aller „Eins-Tupel“ von Elementen aus $M$. Deren Elemente entsprechen letztendlich genau den Elementen von $M$.\footnote{In der Sprache von \cref{def:bijektiv} heißt das: Man hat eine „natürliche Bijektion“ zwischen $M$ und $M^1$.} Oftmals wird zwischen $M$ und $M^1$ gar nicht unterschieden, was der Rechenregel $x^1=x$ entspricht.
    \end{itemize}
\end{bem}





\section{Operationen mit Mengen}


\subsection*{Durchschnitt und Vereinigung}


\begin{de} \index{Schnittmenge} \index{Durchschnitt von Mengen} \index{Vereinigungsmenge} \index{Differenzmenge} \index{Komplement} \label{def:capcup}
    Seien $M,N$ zwei Mengen.
    \begin{itemize}
        \item Der \textbf{Schnitt}\footnote{Die Bezeichnung kommt daher, dass der \emph{Schnitt}punkt zweier Geraden genau derjenige Punkt ist, der auf beiden Geraden zugleich liegt} (oder auch \textbf{Durchschnitt} oder die \textbf{Schnittmenge}) von $M$ und $N$ 
        \begin{align*}
            M \cap N & := \{x \mid x \in M\ \text{und}\ x \in N\} && (\text{lies: „$M$ geschnitten $N$“})
        \end{align*}
        ist die Menge aller Objekte, die sowohl in $M$ als auch in $N$ enthalten sind.
        \item Die \textbf{Vereinigung} von $M$ und $N$
        \begin{align*}
            M \cup N & := \{x \mid x \in M\ \text{oder}\ x \in N\} && (\text{lies: „$M$ vereinigt $N$“})
        \end{align*}
        besteht aus denjenigen Objekten, die in mindestens einer der beiden Mengen $M,N$ enthalten sind.
        \item Die \textbf{Differenzmenge} von $M$ und $N$:
        \begin{align*}
            M \setminus N & := \{ x \mid x \in M \ \text{und}\ x \notin N \}  && (\text{lies: „$M$ ohne $N$“})
        \end{align*}
        ist die Menge aller Elemente von $M$, die nicht in $N$ enthalten sind.

        Manche Autoren schreiben auch „$M-N$“ für die Differenzmenge. Ist $N$ eine Teilmenge von $M$, so schreibt man auch
        \begin{align*}
            N^c & := M\setminus N && (\text{sofern $N\subseteq M$})
        \end{align*}
        und spricht vom \textbf{Komplement von $N$ (in $M$)}. Manche Autoren schreiben auch „$\calC_M(N)$“ für das Komplement von $N$ in $M$. Beachte, dass diese Schreibweise nur Sinn ergibt, wenn aus dem Kontext heraus klar ist, dass $N$ als Teilmenge von $M$ verstanden wird.
	\end{itemize}
    \begin{figure}[ht]
        \begin{minipage}{.48\textwidth}
            \centering
            \begin{tikzpicture}[scale=.75]
                \begin{scope}
                \clip (-1,0) circle (1.5cm);
                \fill[blue!30!white] (1,0) circle (1.5cm);
                \end{scope}
                \draw (-1,0) circle (1.5cm) (-1,-1) node {$M$} (1,0) circle (1.5cm) (1,-1) node {$N$};
            \end{tikzpicture}
            \caption{Schnitt $M\cap N$}
        \end{minipage}
        \quad
        \begin{minipage}{.48\textwidth}
            \centering
            \begin{tikzpicture}[scale=.75]
                \filldraw[fill=blue!30!white, draw=black] (-1,0) circle (1.5cm) (-1,-1) node {$M$} (1,0) circle (1.5cm) (1,-1) node {$N$};
            \end{tikzpicture}
            \caption{Vereinigung $M\cup N$}
        \end{minipage}
        \quad\\[1em]
        \begin{minipage}{.48\textwidth}
            \centering
            \begin{tikzpicture}[scale=.75]
                \fill[blue!30!white] (-1,0) circle (1.5cm);
                \begin{scope}
                \clip (-1,0) circle (1.5cm);
                \fill[white] (1,0) circle (1.5cm);
                \end{scope}
                \draw (-1,0) circle (1.5cm) (-1,-1) node {$M$} (1,0) circle (1.5cm) (1,-1) node {$N$};
            \end{tikzpicture}
            \caption{Differenz $M\setminus N$}
        \end{minipage}
        \quad
        \begin{minipage}{.48\textwidth}
            \centering
            \begin{tikzpicture}[scale=.75]
                \filldraw[fill=blue!25!white, draw=black] (0,0) ellipse (2cm and 1.3cm);
                \filldraw[fill=white!25!white, draw=black] (.5,0) ellipse (1cm and .7cm);
                \path (-1.2,0) node {$N^c$} (0,-1) node {$M$} (.5,0) node {$N$};
            \end{tikzpicture}
            \caption{Komplement von $N$ in $M$}
        \end{minipage}
    \end{figure}
\end{de}


\begin{bsp} \quad
    \begin{enumerate}
        \item Seien $ M = \{1, 2, 4\}$ und $N = \{1, 4, 7\}$. Dann gilt: 
	\begin{align*}
            M \cap N &= \{1, 4\} & M \cup N &= \{1, 2, 4, 7\} & M \setminus N &= \{2\}
	\end{align*}
        \item Für eine natürliche Zahl $n\in \N$ bezeichne
            \[ n\Z := \{ m\in \Z \mid \exists k\in \Z:\ m=kn \} \]
        die Menge aller ganzzahligen Vielfachen von $n$, d.h. aller ganzen Zahlen, die durch $n$ teilbar sind. Dann wurde in \cref{bsp:hinruck} bewiesen, dass
            \[ 2\Z \cap 3\Z = 6\Z \]
        Allgemein gilt für natürliche Zahlen $n,m\in \N$, dass $n\Z\cap m\Z = \kgV(n,m)\Z$, wobei „$\kgV(n,m)$“ das kleinste gemeinsame Vielfache von $n$ und $m$ bezeichnet.
    \end{enumerate}
\end{bsp}


\begin{de}[Mengenfamilie] \index{Mengenfamilie}
    Eine \textbf{Mengenfamilie} (oder auch: \emph{Familie von Mengen}) ist eine Familie, deren Einträge allesamt Mengen sind.
\end{de}


\begin{bsp}
    Beispielsweise ist $(\{1,\dots , n\})_{n\in \N}$ eine Familie von Teilmengen von $\N$, deren Einträge genau die „Anfangsabschnitte“ von $\N$ sind.
\end{bsp}


\begin{de}[Schnitte und Vereinigungen beliebig vieler Mengen] \label{def:mehrfachcapcup}
    Sei $(M_i)_{i\in I}$ eine Familie von Mengen. Es heißen
    \begin{align*}
        \bigcap_{i\in I} M_i &:= \{x \mid \forall i\in I:\ x\in M_i\} & \bigcup_{i\in I}M_i &:= \{x \mid \exists i\in I:\ x\in M_i\}
    \end{align*}
    der \textbf{Durchschnitt} und die \textbf{Vereinigung} der $M_i$'s.
    
    Ist $M$ eine Menge, deren Elemente ebenfalls allesamt Mengen sind, so heißen
    \begin{align*}
        \bigcap M & := \{ x \mid \forall m\in M:\ x\in m \} & \bigcup M & := \{x\mid \exists m\in M:\ x\in m \}
    \end{align*}
    der \emph{Durchschnitt} und die \emph{Vereinigung} von $M$.
\end{de}


\begin{nota} \label{alternativmehrfachcapcup}
    Ist die Indexmenge von der Form $I=\{1,2,\dots,n\}$ für eine natürliche Zahl $n\in \N$, so schreibt man alternativ:
    \begin{align*}
        M_1\cap M_2\cap\ldots\cap M_n & := \bigcap_{i=1}^n M_i := \bigcap_{i\in \{1,\dots , n\}} M_i\\
        M_1\cup M_2\cup\ldots\cup M_n & := \bigcup_{i=1}^n M_i := \bigcup_{i\in \{1,\dots , n\}} M_i
    \end{align*}
\end{nota}


\begin{bsp}
    Sei $H$ die Menge aller Lustigen Taschenbücher und für $h\in H$ sei $L_h$ die Menge aller Menschen, die das Comicbuch $h$ gelesen haben. Dann ist $\bigcap_{h\in H} L_h$ die Menge aller Leute, die \emph{jedes} LTB gelesen haben. Dagegen ist $\bigcup_{h\in H} L_h$ die Menge aller Leute, die mindestens einmal ein LTB gelesen haben.
\end{bsp}


\begin{bem}[Mengen-Operatoren vs. Aussagen-Operatoren]
    Die Operationen $\cap$ und $\cup$ sind in gewisser Weise mengentheoretische „Realisierungen“ der aussagenlogischen Junktoren $\land$ und $\lor$. Deshalb sehen sich auch die Zeichen so ähnlich. Trotzdem solltest du $\cap$ und $\cup$ sorgfältig von $\land$ und $\lor$ unterscheiden: Der Ausdruck „$M\cap N$“ ergibt nur Sinn, wenn $M$ und $N$ Mengen sind; der Ausdruck „$A\wedge B$“ dagegen nur, wenn $A$ und $B$ Aussagen sind. Für zwei Mengen $M,N$ ist „$M\land N$“ kein sinnbehafteter Ausdruck!
    \[\begin{tabular}{c|c|c|c|c|c}
        Mengenoperation & $M\cap N$ & $M\cup N$ & $N^c$ & $\bigcap_{i\in I} M_i$ & $\bigcup_{i\in I} M_i$ \\ \midrule
        Logikoperation & $A\land B$ & $A \lor B$ & $\neg B$ & $\forall x: E(x)$ & $\exists x: E(x)$
    \end{tabular}\]
\end{bem}


\subsection*{Produkt}


\begin{de}[Produkte von Mengen] \index{Produkt von Mengen} \index{Kartesisches Produkt}
    Sei $(M_i)_{i\in I}$ eine Familie von Mengen. Die Menge
        \[ \prod_{i\in I} M_i := \{ (a_i)_{i\in I} \mid \forall i\in I:\ a_i\in M_i \} \]
    aller durch $I$ indizierten Familien, an deren $i$-ter Stelle jeweils ein Element aus $M_i$ steht, heißt das \textbf{(kartesische) Produkt} der $M_i$'s.

    Ist die Indexmenge von der Gestalt $I=\{1,\dots , n\}$ für eine natürliche Zahl $n$, so schreibt man auch
        \[ M_1\times\ldots\times M_n := \prod_{i=1}^n M_i := \prod_{i\in \{1,\dots , n\}} M_i   \]
    Per Definition ist
        \[ M_1\times\ldots\times M_n = \{ (x_1,\dots , x_n) \mid x_1\in M_1,\ \ldots,\ x_n\in M_n \} \]
    d.h. bei $M_1\times\ldots\times M_n$ handelt sich um die Menge aller $n$-Tupel, deren $i$-ter Eintrag ($i\in \{1,\dots , n\}$) jeweils ein Element von $M_i$ ist.
    Insbesondere ist für zwei Mengen $M$ und $N$
    \begin{align*}
        M \times N & = \left\{ (x,y) \mid x\in M\ \text{und}\ y\in N \right\}  && (\text{lies: „$M$ kreuz $N$“})
    \end{align*}
    die Menge aller geordneten Paare, deren erster Eintrag aus $M$ und deren zweiter Eintrag aus $N$ stammen. Man spricht vom \emph{(kartesischen) Produkt} von $M$ und $N$.
\end{de}


\begin{bsp} \quad
    \begin{enumerate}
        \item Es ist
            \[ \{2,3,4\}\times \{ \clubsuit,\heartsuit\} = \{ (2,\clubsuit), (3,\clubsuit), (4,\clubsuit), (2,\heartsuit), (3,\heartsuit), (4,\heartsuit) \}\]
        \item Es ist $\R\times \R=\R^2$ und die Elemente von $\R\times \R$ können als Koordinaten von Punkten in der Ebene interpretiert werden. Die Idee, Punkte mithilfe von Koordinatensystemen und geometrische Figuren als Lösungsmengen von Gleichungen zu beschreiben, wird traditionell Descartes\footnote{\href{https://de.wikipedia.org/wiki/Rene_Descartes}{René Descartes (1596-1650)}} (latinisiert: Cartesius) zugeschrieben. Daher spricht man auch vom \emph{kartesischen} Produkt.
        \item Die Elemente von $\N\times \N$ kannst du dir als „Gitterpunkte“ vorstellen.
        \begin{figure}[ht]
            \begin{tikzpicture}
                \draw[style=help lines,step=1 cm] (-2.3,-2.3) grid (1.5,1.5);
                \begin{scope}
                \draw[->, thick] (-2.5,-2) -- (2,-2) node[right] {$\N$};
                \draw[->, thick] (-2,-2.5) -- (-2,2) node[above] {$\N$};
                \end{scope}
            \end{tikzpicture}
        \centering \caption{Visualisierung von $\N\times \N$}
        \end{figure}
    \end{enumerate}
\end{bsp}


\begin{bsp}[* Matrizen und arrays] \label{bsp:matrizen} \index{Matrix} \quad
    \begin{enumerate}
        \item Seien $m,n\in \N$ und $M=\{1,\dots , m\}$, $N=\{1,\dots,n\}$. Dann ist
            \[ M\times N = \{ (i,j) \in \N\times \N \mid 1\le i\le m \ \text{und}\ 1\le j \le n \} \]
        die Menge aller Paare natürlicher Zahlen, deren erster Eintrag zwischen $1$ und $m$ und deren zweiter Eintrag zwischen $1$ und $n$ liegt.

        Familien mit Indexmenge $M\times N$ werden auch \textbf{($m\times n$)-Matrizen} genannt. Die Einträge einer Matrix werden in einem zweidimensionales Schema (in „Matrixgestalt“) aufgelistet, wobei der erste Index die Zeile und der zweite Index die Spalte markiert:
        \begin{align*}
            \begin{pmatrix}
                a_{11} & \cdots & a_{1n} \\
                \vdots && \vdots \\
                a_{m1} & \cdots & a_{mn}
            \end{pmatrix}
        \end{align*}
        Ist $M$ irgendeine Menge, so wird mit
        \begin{align*}
            M^{m\times n} := M^{\{1,\dots , m\}\times \{1,\dots , n\}}
        \end{align*}
        die Menge aller $(m\times n)$-Matrizen mit Einträgen aus $M$ notiert. Beispielsweise ist $\R^{m\times n}$ die Menge aller $(m\times n)$-Matrizen, deren Einträge reelle Zahlen sind. Solche Matrizen sind ein zentrales Studienobjekt der Linearen Algebra.
        \item Seien $n\in \N$, $m_1,\dots , m_n\in \N$ und $M_i:=\{1,\dots , m_i\}$ für $i\in \{1,\dots , n\}$. Eine Familie mit Indexmenge $M_1\times\ldots\times M_n$ kannst du dir als „$n$-dimensionale Matrix“ vorstellen. In der Informatik werden solche Objekte „$n$-dimensionale arrays“ genannt. In der Physik und der multilinearen Algebra werden sie zur Koordinatendarstellung sogenannter „Tensoren“ verwendet.
    \end{enumerate}
\end{bsp}


\begin{bem}[Potenz als Produkt mit sich selbst]
    Seien $I,M$ zwei Mengen und $(M)_{i\in I}$ die „konstante Familie“, die an jedem Eintrag die Menge $M$ stehen hat. Dann ist $\prod_{i\in I} M$ schlicht die Menge $M^I$ aus \cref{def:mengenpotenz}. Ebenso gilt für jede natürliche Zahl $n$:
        \[ M^n = \underbrace{M\times\ldots\times M}_{\text{$n$-mal}} \]
    sodass beispielsweise $\R\times \R\times \R= \R^3$.

    Also ist $M^I$ eine Art „$I$-faches Produkt von $M$ mit sich selbst“, genauso wie für zwei natürliche Zahlen $m,n\in \N$ die Potenz
        \[ m^n := \underbrace{m\cdot\ldots\cdot m}_{\text{$n$-mal}}\]
    das $n$-fache Produkt von $m$ mit sich selbst ist. Aus diesem Grund wird $M^I$ die „$I$-te Potenz von $M$“ genannt.
\end{bem}


\begin{vorschau}[* Auswahlaxiom] \label{auswahlaxiom} \index{Auswahlaxiom} \index{intangible}
    Sind $M,N$ zwei nichtleere Mengen, etwa mit Elementen $x\in M$ und $y\in N$, so ist auch das kartesische Produkt $M\times N$ nichtleer, weil $(x,y)\in M\times N$. Ist dagegen $(M_i)_{i\in I}$ eine Familie unendlich vieler nichtleerer Mengen, so ist es, sofern keine weiteren Informationen über die Mengen $M_i$ bekannt sind, unmöglich, zu zeigen, dass auch das Produkt $\prod_{i\in I} M_i$ nichtleer ist.

    Sei zum Beispiel $I:= \calP(\R)\setminus \{\emptyset\}$ die Menge aller nichtleeren Teilmengen von $\R$. Du kannst ja mal versuchen, ein konkretes Element von $\prod_{M\in I} M$ hinzuschreiben, was darauf hinausläuft, irgendwie aus jeder nichtleeren Teilmenge von $\R$ ein Element „auszuwählen“.

    Einige (nichtkonstruktive) Sätze der modernen Mathematik sind nun allerdings darauf angewiesen, dass Produkte beliebig vieler nichtleerer Mengen stets nichtleer sind. Daher bildet genau diese Aussage ein eigenes Axiom, das sogenannte \href{https://en.wikipedia.org/wiki/Axiom_of_choice}{Auswahlaxiom}.

    Es lässt sich zeigen, dass es selbst mit Auswahlaxiom unmöglich ist, ein konkretes Element des Produkts $\prod_{M\in \calP(\R)\setminus \{\emptyset\}} M$ hinzuschreiben. Daher handelt es sich um ein \emph{nichtkonstruktives} Axiom: es behauptet die Existenz eines Elements, ohne eine Anleitung für dessen Konstruktion an die Hand zu geben.\footnote{In der sogenannten „konstruktiven Mathematik“ (vgl. \cref{nichtkonstruktiv}) wird das Auswahlaxiom daher nicht akzeptiert. Im mathematischen Mainstream, dem auch die Vorlesungen folgen, wird es dagegen bedenkenlos eingesetzt.} Mathematische Objekte, deren Existenz zwar abstrakt bewiesen werden kann, die sich aber nicht konkret beschreiben oder konstruieren lassen, werden im Englischen auch ``intangibles'' genannt. Beispielsweise handelt es sich bei den Elementen von $\prod_{M\in \calP(\R)\setminus \{\emptyset\}} M$ um intangibles.
\end{vorschau}





\subsection*{* Disjunkte Vereinigung}


\begin{de}[Disjunktheit] \label{def:disjunkt} \index{disjunkte Mengen}
    Zwei Mengen $M,N$ heißen \textbf{disjunkt}, wenn sie keine gemeinsamen Elemente haben, also wenn $M\cap N=\emptyset$.

    Eine Familie von Mengen $(M_i)_{i\in I}$ heißt \textbf{paarweise disjunkt}, wenn je zwei der $M_i$'s disjunkt sind, d.h. wenn gilt:
    \begin{align*}
        M_i\cap M_j \neq  \emptyset \quad& \to\quad i=j &&\text{für alle $i,j\in I$}
    \end{align*}
    Sind $M,N$ zwei disjunkte Mengen, so wird deren Vereinigung auch eine \textbf{disjunkte Vereinigung} genannt und mit
    \begin{align*}
        M\ \dot\cup\ N & := M\cup N && (\text{sofern $M,N$ disjunkt sind})
    \end{align*}
    notiert. Beachte, dass „$\dot\cup$“, anders als es der Vergleich mit \cref{entwederoder} vielleicht nahelegt, keine neue Mengenoperation darstellt, sondern schlicht die Vereinigung aus \cref{def:capcup} gemeint ist. Der Punkt dient lediglich als Mitteilung darüber, dass die beiden Mengen, von deren Vereinigung die Rede ist, disjunkt sind.

    Ebenso wird die Vereinigung einer Familie $(M_i)_{i\in I}$ paarweiser disjunkter Mengen deren \emph{disjunkte Vereinigung} genannt. Man schreibt:
    \begin{align*}
        \dot{\bigcup_{i\in I}}\ M_i & := \bigcup_{i\in I} M_i && (\text{sofern die $M_i$'s paarweise disjunkt sind})
    \end{align*}
\end{de}


\begin{bsp} \label{bsp:disjunkt} \quad
    \begin{enumerate}
        \item Die Mengen $\{\text{Spieler der TSG Hoffenheim}\}$ und $\{\text{Spieler des FSV Mainz 05}\}$ sind disjunkt, weil kein Spieler zugleich bei beiden Vereinen verpflichtet ist.
        \item Die Mengen $\{\text{Informatikstudenten}\}$ und $\{\text{Mathestudenten}\}$ sind nicht disjunkt, weil manche Leute auch beide Fächer zugleich studieren.
        \item Es ist
        \begin{align*}
        \{1,2,3,4,5\} & = \{1,3,4\}\ \dot\cup\ \{2,5\} \\
        \Z & = \{n\in \Z\mid n\ge 0\} \ \dot\cup\ \{n\in \Z\mid n<0\}
        \end{align*}
    \end{enumerate}
\end{bsp}


\begin{bem}
    Die Vereinigung einer paarweise disjunkten Mengenfamilie kannst du dir als eine Art „Addition“ vorstellen. Beispielsweise lässt sich zeigen: sind $n,m\in \N$ und $M$ eine Menge mit $m$-vielen Elementen, $N$ eine zu $M$ disjunkte Menge mit $n$-vielen Mengen, so enthält die disjunkte Vereinigung $M\ \dot\cup\ N$ genau $(m+n)$-viele Elemente.

    Für nicht-disjunkte Vereinigungen gilt das aber nicht. Beispielsweise ist
        \[\{1,3,4\} \cup \{2,3\}= \{1,2,3,4\} \]
    d.h. die Vereinigung einer 2-elementigen Menge mit einer dreielementigen Menge enthält nur vier Elemente.

    Aber auch nicht-disjunkte Mengen können aufeinander „addiert“ werden mithilfe des folgenden Tricks: Ist $(M_i)_{i\in I}$ eine beliebige (nicht notwendig paarweise disjunkte) Mengenfamilie, so sind die Mengen
    \begin{align*}
        \{i\}\times M_i &= \{(i,x) \mid i\in I,\ x\in M_i\} && i\in I
    \end{align*}
    paarweise disjunkt. Die Elemente von $\{i\}\times M_i$ kannst du dir vorstellen als die gleichen Elemente wie von $M_i$, nun aber mit einem „Marker“ versehen, der ihre Zugehörigkeit zu $M_i$ kennzeichnet und sie von den Elementen von $\{j\}\times M_j$ für $j\in I\setminus \{i\}$ unterscheidet. Durch den Übergang von $M_i$ zu den $\{i\}\times M_i$ wurden die $M_i$'s „künstlich disjunkt gemacht”. 
\end{bem}


\begin{de}[Disjunkte Vereinigung] \label{def:disjunktcup} \index{Disjunkte Vereinigung}
    Sei $(M_i)_{i\in I}$ eine Familie von Mengen. Die Menge
        \[ \bigsqcup_{i\in I} M_i := \bigcup_{i\in I} (\{i\}\times M_i) = \{(i,x) \mid i\in I,\ x\in M_i \}\]
    heißt die (äußere) \textbf{disjunkte Vereinigung} der $M_i$'s.
    
    Ist die Indexmenge $I$ von der Gestalt $I=\{1,\dots , n\}$ für eine natürliche Zahl $n\in \N$, so schreibt man auch
        \[ M_1\sqcup\ldots\sqcup M_n := \bigsqcup_{i=1}^n M_i := \bigsqcup_{i\in \{1,\dots , n\}} M_i \]
    Insbesondere ist für zwei Mengen $M$ und $N$:
        \[ M\sqcup N = \{ (i,x)\mid i=1\ \text{und}\ x\in M\quad\text{oder}\quad i=2\ \text{und}\ x\in N\} \]
    Man spricht von der \emph{disjunkten Vereinigung von $M$ und $N$}.
\end{de}


\begin{bsp}
    Es ist
        \[\N \sqcup\N=\{(1,1),(1,2),(1,3),(1,4)\dots,(2,1),(2,2),(2,3),(2,4)\dots\} \]
    d.h. die Menge $\N\sqcup \N$ besteht aus zwei „Kopien“ von $\N$. Für jede natürliche Zahl $n$ enthält sie jeweils zwei „Klone“ von $n$, realisiert als $(1,n)$ und $(2,n)$.
    \begin{figure}[ht]
    \begin{tikzpicture}
        \begin{scope}[xshift=-5cm,scale=1.5]
            \draw[step=.5cm,style=help lines] (-1,-.1) grid (1.3,.1);
            \draw[->,thick] (-1,0) -- (1.5,0) node[right] {$\N$};
            \foreach \x in {1,2,3,4,5}
            \draw[xshift=.5*\x cm] (-1.5,-.1) node[below] {\x};
        \end{scope}
        \begin{scope}[scale=1.5, xshift= 0.4cm]
            \draw[style=help lines,step=.5 cm] (-1.3,-.1) grid (1.3,1.3);
            \begin{scope}
                \draw[->, thick] (-1.3,0) -- (1.5,0) node[right] {$\N$};
                \draw[thick] (-1.2,-.1) -- (-1.2,1.4) node[above] {$\{1,2\}$} coordinate(y axis);
                \foreach \x in {1,2,3,4,5}
                \draw[xshift=.5*\x cm] (-1.5,-.1) node[below] {\x};
                \foreach \y in {1,2}
                \draw[yshift=.5*\y cm] (-1.3,0) node[left] {\y};
            \end{scope}
        \end{scope}
        \begin{scope}[scale=1.5,xshift=4cm]
            \draw[style=help lines,step=.5 cm] (-1.3,-.1) grid (1.3,2.3);
            \begin{scope}
                \draw[->, thick] (-1.3,0) -- (1.5,0) node[right] {$\N$};
                \draw[->, thick] (-1.2,-.1) -- (-1.2,2.5) node[above] {$\N$} coordinate(y axis);
                \foreach \x in {1,2,3,4,5}
                \draw[xshift=.5*\x cm] (-1.5,-.1) node[below] {\x};
                \foreach \y in {1,2,3,4}
                \draw[yshift=.5*\y cm] (-1.3,0) node[left] {\y};
            \end{scope}
        \end{scope}
    \end{tikzpicture}
    \centering \caption{Vergleich von $\N\cup\N\ (=\N)$, $\N\sqcup\N$ und $\N\times\N$.}
\end{figure}
\end{bsp}
    
    
\begin{bem}[Produkt als Summe mit sich selbst]
    Seien $I,M$ zwei Mengen und $(M)_{i\in I}$ die „konstante Familie“, die an jedem Eintrag die Menge $M$ stehen hat. Dann ist $\bigsqcup_{i\in I} M_i$ genau das kartesische Produkt $I\times M$. Ebenso gilt für jede natürliche Zahl $n$:
        \[ \{1,\dots , n\} \times M = \underbrace{M\sqcup\ldots\sqcup M}_{\text{$n$-mal}} \]
    Also ist $I\times M$ die „$I$-fache Summe von $M$ mit sich selbst“, genauso wie für zwei natürliche Zahlen $m,n\in \N$ das Produkt
        \[ n\cdot m := \underbrace{m +\ldots + m}_{\text{$n$-mal}}\]
    die $n$-fache Summe von $m$ mit sich selbst ist.
\end{bem}


\begin{vorschau}
    Der Ausdruck „disjunkte Vereinigung“ ist nun mit mehreren Bedeutungen „überladen“.
    \begin{enumerate}[(1)]
        \item Zum einen bezeichnet er die disjunkte Vereinigung aus \cref{def:disjunktcup}, also die Konstruktion, mit der Mengen zuerst „künstlich disjunkt“ gemacht und daraufhin „addiert“ werden. Wir haben dafür folgende Notation verwendet:
            \[ M\sqcup N \qquad\text{bzw.}\qquad \bigsqcup_{i\in I} M_i \]
        \item Zum anderen bezeichnet er schlicht die gewöhnliche Vereinigung einer Familie paarweiser disjunkter Mengen, siehe \cref{def:disjunkt}, was wir durch
            \[ M\ \dot\cup\ N \qquad\text{bzw.}\qquad   \dot{\bigcup_{i\in I}}\ M_i \]
        notiert haben.
    \end{enumerate}
    Hinzu kommt, dass in der Literatur beide Notationen durcheinander verwendet werden, d.h. sowohl die Objekte (1) als auch die Objekte (2) werden mal mit „$\dot\cup$“, mal mit „$\sqcup$“ bezeichnet
    
    Tatsächlich ist es in den meisten Fällen auch gar nicht wichtig, wie genau die disjunkte Vereinigung „$M \sqcup N$“ definiert ist, d.h. wie ganau man die Mengen $M,N$ künstlich disjunkt gemacht hat. Die disjunkte Vereinigung bezieht ihre Signifikanz vor allem aus der kategorientheoretischen Eigenschaft, ein sogenanntes \emph{Koprodukt} zu sein. In dieser Hinsicht ist sie ein genaues Gegenstück zum kartesischen Produkt. Mehr darüber wirst du in Vorlesungen und Büchern über Kategorientheorie lernen.
\end{vorschau}





\clearpage
\section{Aufgabenvorschläge}


\begin{aufg}[Kennenlernen]
    Es sei $T$ die Menge aller Leute, die sich gerade in diesem Tutorium befinden.
    Welche der folgenden Mengen sind Teilmengen voneinander? Welche Mengen sind sogar gleich?
    \begin{align*}
        S&:= \{ x\in T \mid x\ \text{ist sportlich} \}  && \emptyset  \\
        N&:=  \{ x\in T \mid x\ \text{geht gern in die Natur} \} && T \\
        W & := \{ x\in T \mid x\ \text{kennt sich auf einem Spezialgebiet richtig gut aus} \} &&  N\setminus S  \\
        M &:= \{ x \in T \mid x\ \text{spielt ein Musikinstrument} \} && W \cup S
    \end{align*}
\end{aufg}


\begin{aufg}[Mengen vs. Familien]
    Es sei $I:=\{1,2,3,4,5\}$ und es sei $a=(a_i)_{i\in I}$ diejenige Familie mit $a_i=i+2$ für jedes $i\in I$. Welche der folgenden Objekte sind einander gleich und welche sind voneinander verschieden?
    \begin{center}
        {\renewcommand{\arraystretch}{1.6}
        \begin{tabular}{ccccccccc}
            $I$ && $(a_i)_{i\in I}$&& $\{a_i\mid i\in I\}$ && $(4,3,5,6,7)$ && $(2,1,3,4,5,5)$ \\
            $a$ && $\{1,2,3,4,5\}$ && $(3,4,5,6,7)$ && $\{4,3,5,6,7\}$ && $\{2,1,3,4,5,5\}$
	\end{tabular}}
    \end{center}
\end{aufg}


\begin{aufg}[Elemente und Teilmengen]
    Beurteilt für jede der folgenden Aussagen, ob sie wahr oder falsch ist:
    \begin{align*}
        \N&  \in \N & \N & \subseteq \N & \N & \in \calP(\N) &  \N & \subseteq \calP(\N) \\
        \N & \in \{ \N\} & \N & \subseteq \{ \N\} & \{ \N\} & \in \{ \N\} & \{\N\} & \subseteq \{ \N\} \\
        \{\N\} & \in \calP(\N) & \{ \N\} & \subseteq \calP(\N) & \calP(\N) & \in \calP(\N) & \calP(\N) & \subseteq \calP(\N) \\
        \emptyset & \in \emptyset & \emptyset &\subseteq \emptyset & \emptyset & \in \calP(\emptyset) & \{\emptyset\} & = \calP(\emptyset)
    \end{align*}
\end{aufg}


\begin{aufg}[Produkte konkret]
    Wir betrachten die Mengen $T := \{ \text{Dr.} \}, A := \{ \text{Herr, Frau} \}$, $V:= \{ \text{Anne, John} \}$ und $N := \{ \text{Hathaway, Sinclair, Wayne} \}$. Listet die Elemente der folgenden Mengen auf:
    \begin{enumerate}
	\item $A \times N$
	\item $A\times T \times V\times N$
	\item $(T \cup V) \times N$
	\item $(T\times N)\cup (V\times N)$
    \end{enumerate}
\end{aufg}


\begin{aufg}[Einige Rechenregeln für $\cap$ und $\cup$] \label{aufg:capcupgesetze}
    Seien $X,Y,Z$ drei beliebige Mengen. Vollzieht die folgenden Gleichungen nach:
    \begin{align*}
        \begin{split}
            (X \cap Y) \cap Z & = X \cap (Y \cap Z) \\
            (X \cup Y) \cup Z & = X \cup (Y \cup Z)
        \end{split}
        && (\text{Assoziativgesetze}) \\[1em]
        \begin{split}
            X \cap (Y \cup Z) & = (X\cap Y) \cup (X\cap Z) \\
            X \cup (Y\cap Z) & = (X\cup Y) \cap (X\cup Z)
        \end{split}
        && ( \text{Distributivgesetze}) 
    \end{align*}
    (Eine lange Liste mit weiteren Rechenregeln für Mengen befindet sich in \cref{formelsammlung}.)
\end{aufg}




% ====================================================================
% 4. Vortrag <<Abbildungen>>
% ====================================================================

\setchapterpreamble[c][.7\textwidth]{%
  \itshape\dgrau\small Der Begriff der Abbildung verallgemeinert den Funktionenbegriff aus der Schule. In diesem Vortrag werden grundlegende Werkzeuge im Umgang mit Abbildungen anhand von Beispielen besprochen.
  \vspace{24pt}%
}

\chapter{Abbildungen}


\setchapterpreamble[c][.7\textwidth]{\itshape\color{gray}\small
    Der Begriff der Abbildung verallgemeinert den Funktionenbegriff aus der Schule. In diesem Vortrag werden grundlegende Begriffe, Sprech- und Schreibweisen im Umfeld von Abbildungen besprochen und mit Beispielen illustriert.
\vspace{24pt}}


\chapter{Abbildungen} \label{chap:abbildungen}


\section{Grundlegendes}


Ich werde das Kapitel sogleich mit einer Definition und einigen Allgemeinheiten beginnen. Falls dir das zu abstrakt anmutet, wirf schonmal einen Blick auf die Liste in \cref{bsp:abbildung}.


\begin{defin}[Abbildung] \label{def:abbildung} \index{Abbildung} \index{Funktion} \index{Definitionsbereich} \index{Wertebereich} \index{Graph (einer Abbildung)}
    Seien $X,Y$ zwei Mengen. Eine \textbf{Abbildung von $X$ nach $Y$} (oder auch \textbf{$Y$-wertige Funktion auf $X$}\footnote{Manche Leute bevorzugen das Wort „Abbildung“ im Umfeld abstrakter Mengen und das Wort „Funktion“ im Umfeld von $\R$ oder Vektorbündeln. Ich werde aber beide Wörter völlig synonym verwenden.}) ist ein mathematisches Objekt, das \emph{jedem} Element von $X$ \emph{genau ein} Element von $Y$ zuordnet.

    Sind $f$ eine Abbildung von $X$ nach $Y$ und $x\in X$, so heißt das eindeutig bestimmte Element von $Y$, das $f$ dem Element $x$ zuordnet, der \textbf{Funktionswert von $f$ an der Stelle $x$} oder auch das \textbf{Bild von $x$ unter der Abbildung $f$} und wird notiert mit
    \begin{align*}
        f(x) && (\text{lies: „$f$ von $x$“})
    \end{align*}
    Ferner heißen
    \begin{itemize}
        \item $X$ der \textbf{Definitionsbereich} (oder auch die \textbf{Quelle}) von $f$ (englisch: ``domain'' oder ``source''),
        \item $Y$ der \textbf{Wertebereich} (oder auch das \textbf{Ziel}) von $f$ (englisch: ``codomain'' oder ``target''),
        \item die Menge $\{(x,y)\in X\times Y\mid y=f(x)\}$ der \textbf{Graph} von $f$.
    \end{itemize}
\end{defin}


\begin{axiom}[Gleichheit von Abbildungen] \label{abbgleich}
    Seien $X,Y$ zwei Mengen. Zwei Abbildungen $f,g$ von $X$ nach $Y$ stimmen genau dann überein, wenn sie an jeder Stelle denselben Funktionswert haben. Als Formel:
        \[ f=g \qquad\Leftrightarrow\qquad \forall x\in X:\ f(x)=g(x) \]
    Dementsprechend sind $f$ und $g$ voneinander verschieden, wenn es mindestens ein $x\in X$ gibt mit $f(x)\neq g(x)$.
    
    Zwei Abbildungen, die sich in ihrem Definitionsbereich oder ihrem Wertebereich unterscheiden, denkt man sich als „Objekte verschiedenen Typs“. Sie seien von vornherein voneinander verschieden.
\end{axiom}


\begin{nota}
    Seien $X,Y$ zwei Mengen.
    Anstelle von „$f$ ist eine Abbildung von $X$ nach $Y$“ schreiben wir kurz und bündig:
    \begin{align*}
        f:X \to Y \qquad & \text{oder auch}\qquad X\xrightarrow{f} Y && (\text{lies: „$f$ von $X$ nach $Y$“})
    \end{align*}
    Definitionsbereich, Wertebereich und Graph einer Abbildung $f$ werden manchmal notiert mit
        \[ \dom(f)\ , \qquad \codom(f) \qquad\text{und}\qquad \graph(f)\]
    Die Menge aller Abbildungen von $X$ nach $Y$ wird mit $\text{Abb}(X,Y)$ notiert:
        \[ \text{Abb}(X,Y) := \{ f \mid f\ \text{ist eine Abbildung von $X$ nach $Y$} \} \]
\end{nota}


\begin{nota}[Eine Abbildung über eine Zuordnungsvorschrift definieren] \label{def:zuordnung} \index{Zuordnungsvorschrift} \index{Abbildungsvorschrift} \index{Funktionsterm}
    Sei $t(a)$ ein Term in der Variablen $a$. Sind $X,Y$ zwei Mengen dergestalt, dass, setzt man für die Variable $a$ ein beliebiges Element aus $X$ ein, der Term $t$ stets ein Element von $Y$ ergibt, so heißt der Ausdruck
    \begin{align*}
        X \to Y \ &,\ x\mapsto t(x) && (\text{lies: „von $X$ nach $Y$, $x$ geht auf $t(x)$“})
    \end{align*}
    eine \textbf{Zuordnungsvorschrift} oder auch \textbf{Abbildungsvorschrift} von $X$ nach $Y$. Sie ist zu interpretieren als „Zuordnung“, die jedem Element $x\in X$ das Element $t(x)\in Y$, das durch Einsetzen des Objekts $x$ anstelle der Variablen $a$ zustandekommt, „zuordnet“.
    
    Jede solche Zuordnungsvorschrift definiert eine Abbildung $X\to Y$, deren Funktionswerte an Elementen $x\in X$ genau mit den Elementen $t(x)$ übereinstimmt. Möchte man diese Abbildung mit einer Variable bezeichnen, so schreibt man:
    \begin{align*}
        f : X\to Y \ & ,\ x\mapsto t(x) && (\text{lies: „$f$ von $X$ nach $Y$, $x$ geht auf $t(x)$“})
    \end{align*}
    Mit dieser Notation wird festgelegt, dass das Zeichen „$f$“ fortan diejenige Abbildung $X\to Y$ bezeichnen soll, die durch die Zuordnung $x\mapsto t(x)$ gegeben ist. Für jedes $x\in X$ gilt also $f(x)=t(x)$. Nach \cref{abbgleich} ist $f$ durch diese Eigenschaft eindeutig bestimmt.

    Der Term $t$ fungiert hier als sogenannter \textbf{Funktionsterm}.\footnote{Im für die Informatik relevanten \emph{Lambda-Kalkül} sagt man: „Die Variable $x$ wird durch $\lambda$-Abstraktion gebunden“, und schreibt „$\lambda x.t(x)$“ anstelle von $x\mapsto t(x)$.} Im Ausdruck „$x\mapsto t(x)$“ ist das Zeichen „$x$“ eine gebundene Variable im Sinne von \cref{gebundenevariable}.
\end{nota}


\begin{bsp} \label{bsp:abbildung} \quad
    \begin{enumerate}
        \item Reelle Polynome ergeben Zuordnungen reeller Zahlen, die in der Schulmathematik auch „ganzrationale Funktionen“ genannt werden. Zum Beispiel:
        \begin{align*}
            f: \R \to \R \ & ,\ x  \mapsto x^2-1 \\
            g: \R \to \R \ & ,\  x  \mapsto 3x^2-2x+4
        \end{align*}
        Deren Definitions- und Wertebereich sind hier jeweils die Menge $\R$ der reellen Zahlen. Die Funktion $g$ ordnet jeder Zahl $x\in\R$ die Zahl $3x^2-2x+4$ zu, beispielsweise ist $g(4) = 3\cdot 16 - 2\cdot 4 + 4 = 44$.
        \begin{comment}
        \begin{figure}[ht]
            \includegraphics[width=9cm]{./_img/Polynom-Plot.jpeg}
            \centering \caption{Visualisierung des Graphen der Funktion $\R\to \R \ ,\ x\mapsto 3x^2-2x+4$}
        \end{figure}
        \end{comment}
        \item Seien $\mathtt{cat}$ die Menge aller Katzen und $\mathtt{colour}$ die Menge aller Farben. Wir haben eine Abbildung
            \[ f: \mathtt{cat} \to \mathtt{colour} \ ,\ C \mapsto (\text{Die Fellfarbe von $C$}) \]
        deren Definitionsbereich genau $\mathtt{cat}$ und deren Wertebereich genau $\mathtt{colour}$ ist. Für eine Katze $C$ ist $f(C)$ genau die Fellfarbe von $C$.
        \item Sei $X$ eine beliebige Menge. Die Abbildung
            \[ \{-\}: X \to \calP(X) \ ,\ x \mapsto \{x\} \]
        hat als Definitionsbereich $X$, als Wertebereich die Potenzmenge von $X$ und sie ordnet jedem Element $x \in X$ diejenige Einermenge zu, die nur aus $x$ besteht.

        Der senkrechte Strich im Ausdruck „$\{-\}$“ ist eine suggestive Schreibweise, die kommunizieren soll, dass sich der Funktionswert von $x$ genau durch Einsetzen an der mit dem Strich markierten Stelle ergibt.
        \item Sei $\calA:=\{A\mid A\ \text{ist eine Aussage}\}$ die Menge aller mathematischen Aussagen. Die Junktoren $\land$ und $\neg$ ergeben Abbildungen
        \begin{align*}
            \calA\times \calA & \to \calA & \calA & \to \calA \\
            (A,B) & \mapsto A\land B & A& \mapsto \neg A
        \end{align*}
        d.h. im ersten Fall wird jedem Aussagenpaar $(A,B)$ die Konjunktion „$A$ und $B$“ zugeordnet und im zweiten Fall jeder Aussage ihre Negation. Ebenso liefern auch die anderen Junktoren $\lor,\to,\leftrightarrow$ jeweils Abbildungen $\calA\times\calA\to \calA$.
        \item Es bezeichne $\Set$ die Gesamtheit aller Mengen. Dann haben wir eine Abbildung
            \[ \Abb(-,-) : \Set \times \Set \to \Set \ ,\ (X,Y) \mapsto \Abb(X,Y) \]
        die jedem Mengenpaar $(X,Y)$ die Menge $\Abb(X,Y)$ aller Abbildungen $X\to Y$ zuordnet.
    \end{enumerate}
    Aus der Schule bist du es vielleicht gewohnt, dass der Graph einer Funktion eine Art „Kurve“ ist. Beachte, dass unsere allgemeine Graphendefinition deutlich abstrakter ist und der Graph einer Abbildung in der Regel keine „geometrische“ Bedeutung besitzt. Zumindest im Fall einer Abbildung $\R\to \R$ ist der Graph eine Teilmenge des $\R^2$, im Allgemeinen kann er jedoch alles andere als „kurvig“ aussehen.
\end{bsp}


\begin{vorschau}[* mengentheoretische Abbildungsdefinition]
    Die obige Definition einer Abbildung ist, ebenso wie die Mengendefinition \cref{def:menge}, rein axiomatisch. In der im 20. Jahrhundert vorherrschenden Praxis kann der Abbildungsbegriff formal präzise auf dem Mengen- und Elementbegriff aufgebaut werden: Eine Analyse ergibt, dass für zwei Abbildungen $f,g$ äquivalent sind:
    \begin{enumerate}[(i)]
        \item $f=g$.
        \item $\dom(f)=\dom(g)$, $\codom(f)=\codom(g)$ und $\graph(f)=\graph(g)$.
    \end{enumerate}
    \emph{Definiert} man nun eine Abbildung $f$ als ein Tripel $(X,G,Y)$ bestehend aus drei Mengen $X,G,Y$, wobei $G$ eine Teilmenge von $X\times Y$ mit der Eigenschaft, dass es für jedes $x\in X$ genau ein $y\in Y$ mit $(x,y)\in G$ gibt, sein soll, und definiert man $f(x):=(\text{das eindeutige $y\in Y$ mit $(x,y)\in G$})$ für $x\in X$, so wird die Äquivalenz (i)$\leftrightarrow$(ii) zu einer beweisbaren Aussage. Dagegen wurde sie vermittels \cref{abbgleich} einfach per Axiom festgelegt.
    
    In der \href{https://en.wikipedia.org/wiki/Topos#Elementary_topoi_(topoi_in_logic)}{Topostheorie} wird umgekehrt verfahren: hier sind die „Abbildungen“ das fundamentale Konzept, das nicht definiert und nur durch Axiome beschrieben wird, worauf dann formal präzise die Begriffe von Element und Teilmenge aufgebaut werden.
    
    Weil ich keiner Herangehensweise den Vorzug geben will, werde ich es bei \cref{def:abbildung} belassen und nicht weiter auf eine formale Abbildungsdefinition eingehen.
\end{vorschau}


\begin{bem}[Beweisen, dass zwei Abbildungen gleich oder ungleich sind]
    Seien $X,Y$ zwei Mengen und $f,g:X\to Y$ zwei Abbildungen. Da das Gleichheitskriterium aus \cref{abbgleich} eine Allaussage ist (für \emph{jedes} $x\in X$ ist $f(x)=g(x)$), kann die Gleichheit $f=g$ nach \cref{allbeweis} dadurch bewiesen werden, dass ein „beliebiges“ Element von $X$ fixiert wird und dann bewiesen wird, dass dessen Funktionswerte unter $f$ und $g$ übereinstimmen.

    Bevor du überhaupt versuchst, die Gleichheit zweier Abbildungen zu beweisen, solltest du dich natürlich erstmal vergewissern, dass sie überhaupt denselben Definitionsbereich und denselben Wertebereich haben. Meistens ist das aber „offensichtlich“ und muss im Beweis nicht nochmal erwähnt werden.

    Um hingegen zu zeigen, dass die Abbildungen $f,g$ verschieden sind, genügt es nach \cref{gegenbeispiel}, ein Gegenbeispiel anzugeben, d.h. in diesem Fall, ein Element $x\in X$ zu finden, für das $f(x)\neq g(x)$.
\end{bem}


\begin{bsp} \label{bsp:abbgleich} \quad
    \begin{enumerate}
        \item Betrachten wir die Abbildungen: 
        \begin{align*}
            f: \R \to \R\ &,\ x \mapsto x^2 \\
            g: \R \to \R\ &,\ x \mapsto 2x+1
        \end{align*}
        so ist $f \neq g$, denn es ist $f(1)=1$ und $g(1)=3$.
        \item Betrachten wir dagegen die Abbildungen
        \begin{align*}
            \alpha: \R \to \R\ &,\ x \mapsto (x+1)\cdot (x-1) \\
            \beta: \R \to \R\ &,\ x \mapsto  x^2-1
        \end{align*}
        so gilt für jede beliebige reelle Zahl $x\in \R$:
            \[ \alpha(x) = (x+1)(x-1) = x^2-1 = \beta(x) \]
        Demnach ist $\alpha=\beta$.
    \end{enumerate}
\end{bsp}


\begin{bem}[Zuordnungsvorschriften vs. Abbildungen] \label{zuordvsabb}
    Jede Zuordnungsvorschrift kann zur Definition einer Abbildung verwendet werden. Beachte, dass Zuordnungsvorschriften und Abbildungen nicht ganz dasselbe sind. Eine Zuordnungsvorschrift ist ein sprachliches Gebilde, eine konkrete Vorschrift, wie Gegenständen vom Typ $X$ Gegenstände vom Typ $Y$ zuzuordnen seien, wohingegen der Abbildungsbegriff eine mathematische Abstraktion darstellt. Das Verhältnis zwischen Termen und Abbildungen ist, genau wie das zwischen Eigenschaften und Mengen (vgl. \cref{mengenvseig}), ein Aspekt der Dualität zwischen Syntax und Semantik (siehe \cref{syntaxvssemantik}).
    \begin{itemize}
        \item Dieselbe Abbildung kann durch verschiedene Zuordnungsvorschriften zustandekommen. Beispielsweise sind
        \begin{align*}
            \R \to \R \ &,\ x\mapsto x^2-1 \\
            \R \to \R \ &,\ x\mapsto (x+1)(x-1)
        \end{align*}
        zwei verschiedene Zuordnungsvorschriften, die aber nach \cref{bsp:abbgleich} dieselbe Abbildung induzieren.
        \item Nicht jede Abbildung muss durch eine einfache Zuordnungsvorschrift definierbar sein. Abbildungsvorschriften können beliebig kompliziert sein und manche Abbildungen sind so chaotisch, dass sie gar keiner Abbildungsvorschrift gehorchen. Beispielsweise beweist man in der Berechenbarkeitstheorie, dass es Abbildungen $\N \to \{0,1\}$ (also Abfolgen von Nullen und Einsen) geben muss, die nicht Turing-berechenbar\footnote{\href{https://de.wikipedia.org/wiki/Alan_Turing}{Alan Turing (1912-1954)}} sind, d.h. für die es keinen konventionellen Algorithmus gibt, der nacheinander alle Folgenglieder auflistet.
        \item Definitions- und Wertebereich sind grundlegende Bestandteile einer Abbildung. Beispielsweise kann der Ausdruck
            \[ x\mapsto x^2 \]
        in verschiedenfacher Weise als Zuordnungsvorschrift interpretiert werden, z.B. als Zuordnung $\R\to \R$, $\Z\to \N_0$ oder $\C\to \C$. Die drei Abbildungen
        \begin{align*}
            \R \to \R \ & ,\ x\mapsto x^2 \\
            \Z \to \N_0 \ & ,\ x\mapsto x^2 \\
            \C \to \C \ &,\ x\mapsto x^2 
        \end{align*}
        sind aber allesamt voneinander verschieden, da sie verschiedene Definitions- und Wertebereiche haben. Siehe auch \cref{bsp:injsur}.
    \end{itemize}
\end{bem}


\begin{bem}[„Funktionen in mehreren Veränderlichen“]
    Nach unserer Abbildungsdefinition kann in eine Abbildung $f:X\to Y$ genau ein Element $x$ aus $X$ „eingesetzt“ werden, um genau ein Element $f(x)$ von $Y$ zu erhalten. Dies erweckt vielleicht den Anschein, dass unsere Abbildungsdefinition nicht in der Lage wäre, „Funktionen in mehreren Veränderlichen“ wie zum Beispiel
        \[ x,y \quad\mapsto\quad 2xy + xy^2 \]
    zu modellieren. -- Sie ist es aber, und der Trick besteht in der Verwendung des kartesischen Produkts als Definitionsbereich. Beispielsweise haben wir eine Abbildung
        \[ \R \times \R \to \R \ ,\ (x,y) \mapsto 2xy +xy^2 \]
    Abbildungen dieser Art sind das Hauptthema von \cref{chap:verknuepfungen}. Verwenden wir noch allgemeinere Produkte $\prod_{i\in I} M_i$ als Definitionsbereich, können wir auch „Funktionen in unendlich vielen Variablen“ definieren.
\end{bem}





\section{Verketten von Abbildungen}


\begin{defin}[Verketten von Abbildungen]\label{def:verkettung} \index{Verkettung von Abbildungen}
    Seien $X,Y,Z$ drei Mengen und $X\xrightarrow{f} Y$, $Y \xrightarrow{g} Z$ zwei Abbildungen. Mit
    \begin{align*}
        g\circ f && (\text{lies: „$g$ nach $f$“, „$g$ verkettet $f$“ oder „$g$ kringel $f$“})
    \end{align*}
    wird diejenige Abbildung $X\to Z$ bezeichnet, die gegeben ist durch die Zuordnungsvorschrift
    \begin{align*}
        X \to Z \ ,\ x \mapsto g(f(x))
    \end{align*}
    Die Abbildung $g\circ f$ heißt die \textbf{Verkettung}, \textbf{Komposition} oder \textbf{Hintereinanderausführung} von $f$ und $g$.
    
    Insgesamt erhalten wir eine Abbildung, die ein Paar von Abbildungen auf eine Abbildung abbildet:
        \[ \circ : \Abb(Y,Z)\times \Abb(X,Y) \to \Abb(X,Z) \ ,\ (g,f) \mapsto g\circ f\]
\end{defin}


\begin{bem} \quad
    \begin{itemize}
        \item Sind zwei Abbildungen durch Funktionsterme gegeben, so erhält man ihre Verkettung durch Einsetzen des einen Terms in den anderen im Sinne von \cref{def:substitution}, siehe dazu die nachfolgenden Beispiele. Dies ist ein Aspekt der Dualität zwischen Syntax und Semantik (siehe \cref{syntaxvssemantik}).
        \item Beachte, dass im Term „$g\circ f$“ diejenige Abbildung, die „zuerst“ ausgeführt wird, an rechter Stelle steht. Für ein Element $x \in X$ berechnet man $(g\circ f)(x)$, indem man zuerst $f(x)$ ausrechnet und darauf nun die Abbildung $g$ anwendet.
    \end{itemize}
\end{bem}


\begin{bsp} \label{bsp:verkettung} \quad
    \begin{enumerate}
        \item Es seien $S:= \{ \text{Schauspieler in Filmen}\}$ die Menge aller Filmdarsteller und $F:=\{\text{Filme}\}$ die Menge aller Filme. Wir haben Abbildungen
        \begin{align*}
            f : S \to F \ & ,\ s\mapsto (\text{Der erste Film, in dem $s$ mitspielte}) \\
            g : F \to \N \ & ,\ x \mapsto (\text{Das Jahr, in dem die Dreharbeiten an $x$ begannen})
        \end{align*}
        Dann ist die Verkettung $g\circ f$ diejenige Abbildung $S\to \N$, die jedem Schauspieler das Jahr zuordnet, in dem der erste Film, bei dem er mitgespielt hat, gedreht wurde.
        \item Betrachte die Abbildungen
        \begin{align*}
            f : \R \to \R \ &,\ x\mapsto x^2 \\
            g : \R \to \R \ &,\ x \mapsto x-1
        \end{align*}
        Dann gilt für alle $x\in \R$:\footnote{vgl. \cref{bsp:substitution}}
        \begin{align*}
            (g\circ f)(x) & = g(f(x)) = g(x^2) = x^2-1 \\
            (f\circ g)(x) & = f(g(x)) = f(x-1) = (x-1)^2
        \end{align*}
        Wegen $(g\circ f)(4)\neq (f\circ g)(4)$ ist $g\circ f\neq f\circ g$.
        \item Die Technik, einer Abbildung $g$ eine Abbildung $f$ „vorzuschalten“, kennst du vielleicht schon aus der Schule als „Variablensubstitution“, vgl. \cref{bsp:substitution}. Bei der \emph{Integration durch Substitution} spielt dies eine maßgebliche Rolle. Sind $f,g:\R\to \R$ zwei hinreichend „glatte“ Funktionen und $a,b\in \R$, so besagt die Substitutionsregel, dass
        \[ \int_{f(a)}^{f(b)} g = \int_a^b (g\circ f) \cdot f' \]
    \end{enumerate}
\end{bsp}


\begin{satz}[Verketten von Abbildungen ist assoziativ] \label{abbassoziativ}
    Seien $A,B,C,D$ vier beliebige Mengen und
    \[\begin{tikzcd}
        A \ar[r, "f"] & B \ar[r, "g"] & C \ar[r, "h"] & D
    \end{tikzcd}\]
    drei Abbildungen. Dann gilt:
        \[ (h\circ g)\circ f = h\circ (g\circ f)\]
\end{satz}
\begin{proof}
    Für jedes $a \in A$ ist
    \begin{align*}
        \bigl((h \circ g) \circ f\bigr) (a)	& = \bigl(h \circ g\bigr) \bigl(f(a)\bigr) \\
        & = h\bigl(g(f(a))\bigr) \\
        & = h\bigl((g \circ f)(a)\bigr) \\
        & = \bigl(h \circ (g \circ f)\bigr)(a)
    \end{align*}
    Da das Element $a\in A$ beliebig gewählt war, folgt $(h\circ g)\circ f = h\circ (g\circ f)$.
\end{proof}


\begin{bem}[Klammern sparen] \label{abbklammerfrei}
    Aufgrund von \cref{abbassoziativ} wird in dessen Situation einfach
        \[ h\circ g\circ f \]
    geschrieben, da es egal ist, ob erst $g$ mit $f$ verkettet wird und dann $h$ mit $g\circ f$ -- oder ob erst $h$ mit $g$ verkettet wird und dann $h\circ g$ mit $f$. Mit fortgeschrittenen Techniken lässt sich zeigen, dass es auch bei Verkettungen von mehr als drei Abbildungen egal ist, wie man Klammern setzt. Daher werden Klammerungen meist ganz weggelassen.\footnote{vgl. dazu auch \cref{klammerfrei}} Sind etwa $A,B,C,D,E,F$ sechs Mengen und sind Abbildungen
    \[\begin{tikzcd}
        A \ar[r, "a"] & B \ar[r, "b"] & C \ar[r, "c"] & D \ar[r, "d"] & E \ar[r, "e"] & F
    \end{tikzcd}\]
    gegeben, so wird mit
        \[ e\circ d\circ c\circ b\circ a \]
    diejenige Abbildung notiert, die durch Verkettung dieser fünf Abbildungen entsteht, die sich also dadurch zusammensetzt, dass erst $a$ durchlaufen wird, dann $b$, dann $c$, dann $d$ und zuletzt $e$.
\end{bem}





\section{Identität und Inklusion}


Sei $X$ eine beliebige Menge. Gibt es überhaupt eine Abbildung $X\to X$? Solange wir keine Informationen über die Elemente von $X$ haben, können wir ja schwerlich eine besondere Abbildungsvorschrift angeben. Dennoch gibt es stets eine Abbildung $X\to X$, deren Zuordnungsvorschrift so simpel ist, dass man sie schon wieder übersehen könnte:


\begin{defin}[Identitätsabbildung] \index{Identität}
    Sei $X$ eine Menge. Die Abbildung
    \begin{align*}
        \id_X: X \to X \ ,\ x \mapsto x
    \end{align*}
    heißt die \textbf{Identität auf $X$}.
\end{defin}


\begin{bem}
    Die Identität auf der Menge $X$ bildet jedes Element auf sich selbst ab und „tut gar nichts“. Die Identität auf $\R$ kennst du bereits aus der Schule: die Polynomfunktion „$f(x)=x$“ ist genau die Identitätsabbildung $\id_\R$ auf der Menge der reellen Zahlen.
\end{bem}


Der nächste Satz zeigt, dass sich die Identität hinsichtlich der Verkettung von Abbildungen verhält, wie die $0$ bei der Addition oder die $1$ bei der Multiplikation, vgl. \cref{def:neutrales}.


\begin{satz}[Neutralität der Identität] \label{idneutral}
    Seien $X,Y$ zwei beliebige Mengen und $X\xrightarrow{f} Y$ eine Abbildung. Dann gilt:
        \[ \id_Y \circ f = f \qquad\textnormal{und}\qquad f\circ \id_X = f \]
\end{satz}
\begin{proof}
    Für ein beliebiges Element $x \in X$ gilt:
    \begin{align*}
        (\id_Y \circ f)(x) & =\id_Y(f(x)) \\
        & = f(x)&& (\text{per Definition von $\id_Y$}) \\[0.5em]
        (f\circ \id_X)(x) & = f(\id_X(x)) \\
        & =f(x) && (\text{per Definition von $\id_X$})
    \end{align*}
    Da das Element $x\in X$ beliebig gewählt war, folgt $\id_Y\circ f=f$ und $f\circ \id_X=f$.
\end{proof}


\begin{vorschau}[* Die Kategorie der Mengen] \label{kategorien}
    \cref{idneutral} und \cref{abbassoziativ} lassen sich zusammenfassen in der Aussage, dass Mengen und Abbildungen die Struktur einer \href{https://ncatlab.org/nlab/show/category}{Kategorie} bilden. Die Sprache der Kategorien ist von fundamentaler Bedeutung für die gesamte Mathematik und du wirst, falls du in die reine Mathematik gehst, in deinem Studium noch haufenweise weitere Kategorien kennenlernen. Zum Beispiel:
    \begin{longtable}{lccr}
    Fachgebiet & \phantom{Platzhalter} & Kürzel & \phantom{Platzhalterhalter} Kategorie \\
    \midrule
    Grundlagen && $\Set$ & Mengen \\ 
    && $\Pos$ & Geordnete Mengen \\[0.5em]
    Lineare Algebra && ${}_K\mathsf{Vec}$ & $K$-Vektorräume \\
    && ${}_R\mathsf{Mod}$ & $R$-Moduln \\
    && ${}_R\mathsf{Mat}$ & Matrizen über $R$ \\[0.5em]
    Analysis && $\mathsf{Top}$ & Topologische Räume \\
    && $\mathsf{Met}$ & Metrische Räume \\
    && $\mathsf{Man}$ & Mannigfaltigkeiten \\
    && $\sigma\mathsf{\text{-}Alg}$ & Sigma-Algebren \\[0.5em]
    Algebra && $\Mon$ & Monoide \\
    && $\Grp$ & Gruppen \\
    && ${}_G\Set$ & $G$-Mengen \\
    && $\mathsf{Ring}$ & Ringe \\
    && ${}_R\mathsf{Alg}$ & $R$-Algebren \\[0.5em]
    Funktionalanalysis && $\mathsf{Unif}$ & Uniforme Räume \\
    && $\mathsf{Ban}$ & Banachräume \\
    && $\mathsf{Hilb}$ & Hilberträume \\[0.5em]
    Fortgeschrittene Vorlesungen && $\mathsf{Cat}$ & Kategorien \\
    && $\mathrm{Sh}(\calC)$ & Garben auf $\calC$ \\
    && $\mathsf{sSet}$ & Simpliziale Mengen \\
    && $\mathrm{Ch}(\calA)$ & Komplexe in $\calA$ \\
    && $\mathsf{Sch}$ & Schemata \\
    \end{longtable}
\end{vorschau}


\begin{defin}[Inklusionsabbildung] \label{def:inklusion} \index{Inklusion (einer Teilmenge)}
    Seien $X$ eine Menge und $U\subseteq X$ eine Teilmenge. Die \textbf{Inklusionsabbildung} (oder auch kurz: \textbf{Inklusion}) von $U$ in $X$ ist die Abbildung
    \begin{align*}
        U \to X \ ,\ x \mapsto x
    \end{align*}
    Sie wird bevorzugt mit dem griechischen Buchstaben $\iota$ (iota) und einem Pfeil mit Haken „$\hookrightarrow$“ notiert:
        \[ \iota_U : U \hookrightarrow X \]
    Gelegentlich spricht man auch von der \emph{natürlichen Inklusion} oder der \emph{kanonischen Inklusion}.
\end{defin}


\begin{bem}
    Die natürliche Inklusion ordnet einem Element aus $U$ genau das gleiche Element in $X$ zu. Wie bei der Identitätsabbildung werden auch von der Inklusion keine Elemente verändert. Trotz ihres trivialen Abbildungsverhaltens erweist sich die Inklusionsabbildung für einige theoretische Überlegungen als nützlich, siehe z.B. \cref{bsp:quelleschrank} und \cref{pullbackrel}.
    
    Der „behakte Pfeil $\hookrightarrow$“ ist schlicht eine Variante zum gewöhnlichen Abbildungspfeil, die bei Abbildungen, die sich ähnlich wie die natürliche Inklusion verhalten, verwendet wird, um ihren „Inklusionscharakter“ zu betonen.
\end{bem}


\begin{bsp}
    Wegen $\Z\subseteq\R$ hat man eine Inklusionsabbildung $\Z \hookrightarrow \R$. Ein Pendant in der Informatik stellt die Typumwandlung eines \emph{integers} zu einem \emph{float} dar. Der Vergleich hinkt jedoch, insofern Typumwandlungen, im Gegensatz zu Inklusionsabbildungen, nicht immer ohne Informationsverlust vonstatten gehen.
\end{bsp}


\begin{defin}[* Leere Abbildung] \index{Leere Abbildung}
    Sei $X$ eine beliebige Menge. Dann gibt es genau eine Abbildung $\emptyset\to X$, die sogenannte \textbf{leere Abbildung}. Es handelt sich genau um die natürliche Inklusion der leeren Teilmenge $\emptyset\subseteq X$.\footnote{vgl. \cref{leeremengeimmerdrin}}
\end{defin}





\section{Bilder und Urbilder von Teilmengen}


\begin{defin}[Transfer von Teilmengen] \label{def:bildmenge} \index{Faser} \index{Bild} \index{Urbild}
    Seien $X,Y$ Mengen und $X \xrightarrow{f} Y$ eine Abbildung.
    \begin{itemize}
        \item Die Menge
            \[ \im(f) := \{y\in Y\mid \exists x\in X:\ f(x)=y  \} \]
        heißt das \textbf{Bild von $f$} (englisch: ``image'').
        \item Für eine Teilmenge $A\subseteq X$ heißt
            \[ f(A) := \{ y \in Y \mid \exists a\in A:\ f(a)=y \} \]
        die \textbf{Bildmenge von $A$} oder schlicht das \textbf{Bild von $A$} unter $f$.
        \item Für eine Teilmenge $B\subseteq Y$ heißt
            \[ f^{-1}(B) := \{ x \in X \mid f(x)\in B \} \]
        die \textbf{Urbildmenge von $B$} oder schlicht das \textbf{Urbild} von $B$ unter $f$.
        \item Für ein Element $y\in Y$ heißt die Menge
            \[ f^{-1}(y) := \{ x\in X \mid f(x)=y \} \]
        die \textbf{Faser von $y$} unter der Abbildung $f$. Man spricht auch vom \textbf{Urbild von $y$}. Die Elemente von $f^{-1}(y)$ werden ebenfalls \textbf{Urbilder von $y$} genannt.
    \end{itemize}
\end{defin}


\begin{bsp}
    Für die Abbildung
    \begin{align*}
        f: \Z \to \Z \ ,\ n \mapsto n^2
    \end{align*}
    gilt beispielsweise:
    \begin{align*}
        \im(f) & = \{n\in \N_0 \mid n\ \text{ist eine Quadratzahl} \} & f^{-1}(9) & = \{-3,3\} \\
        f(\Z_{\le 0}) & = f(\Z_{\ge 0}) = \im(f) & f^{-1}(0) & = \{0\} \\
        f(\{7\}) & = \{49\} & f^{-1}(-9) & = \emptyset \\
        f(\{2,6,-2,-5\}) & = \{4,25,36\} & f^{-1}(\{6,-1\}) & = \emptyset \\
        f^{-1}(\im(f)) & = \Z & f^{-1}(\{0,9,12\}) & = \{0,3,-3\}
    \end{align*}
\end{bsp}


\begin{bsp}[*] \quad
    \begin{enumerate}
        \item Für die Abbildung
            \[ f : \R \to \R \ ,\ x\mapsto x^5-x-1 \]
        wurde in \cref{bsp:exverwendung} gezeigt, dass $f^{-1}(0)\neq\emptyset$. Tatsächlich ist diese Faser sogar einelementig, ihr Element, also die eindeutige reelle Lösung der Gleichung $x^5-x-1=0$, lässt sich aber nicht mit herkömmlichen arithmetischen Operationen hinschreiben, vgl. \cref{zeichendefinieren}.
        \item In \cref{bsp:fallunterscheidung} wurde für die Abbildung
            \[ f : \N \to \N \ ,\ n \mapsto n\cdot (n+1) \]
        bewiesen, dass $\im(f)\subseteq \{n\in \N\mid n\ \text{ist gerade}\}$.
    \end{enumerate}
\end{bsp}


\begin{defin}[Konstante Abbildung] \index{konstante Abbildung}
    Seien $X,Y$ zwei Mengen. Eine Abbildung $X\xrightarrow{f} Y$ heißt \textbf{konstant}, wenn $\im(f)$ eine Einermenge ist, d.h. wenn es ein $y\in Y$ gibt derart, dass $f(x)=y$ für alle $x\in X$.
\end{defin}


\begin{bsp} \quad
    \begin{enumerate}
        \item Zum Beispiel ist
            \[ \R \to \R \ ,\ x\mapsto 4 \]
        die konstante Abbildung, die alles auf $4$ schickt. In der Analysis wird gezeigt, dass eine Funktion $f : \R\to \R$ genau dann konstant ist, wenn sie differenzierbar ist mit $f'=0$.
        \item Allgemein geben Terme, die von gar keinen Variablen abhängen (siehe \cref{bsp:term}), Anlass zu konstanten Abbildungen.
        %\item Für eine Menge $X$ ist die leere Abbildung $\emptyset\to X$ keine konstante Abbildung, da ihr Bild leer ist und damit nicht einelementig. Die Handhabung ist in der Literatur aber nicht eindeutig: bei manchen sind auch leere Abbilungen konstant und die meisten verlieren hierüber gar kein Wort.
    \end{enumerate}
\end{bsp}





\section{Einschränkung von Definitions- und Wertebereich}


In diesem Abschnitt seien stets $X,Y$ zwei Mengen und $f:X\to Y$ eine Abbildung.


\begin{defin}[Einschränken des Definitionsbereichs] \label{def:quelleschrank} \index{Restriktion}
    Sei $A \subseteq X$ eine Teilmenge von $X$. Dann ist die \textbf{Einschränkung von $f$ auf $A$} oder auch \textbf{Restriktion von $f$ auf $A$} diejenige Abbildung $A\to Y$, die durch die Abbildungsvorschrift
    \begin{align*}
        A \to Y \ &,\ a \mapsto f(a)
    \end{align*}
    gegeben ist. Notation:
    \begin{align*}
        f\vert_A && (\text{lies: „$f$ eingeschränkt auf $A$“})
    \end{align*}
    Die Abbildung $f\vert_{A}$ funktioniert also genau wie $f$, mit dem einzigen Unterschied, dass der Definitionsbereich verkleinert wurde und $f\vert_{A}$ nur noch den Elementen aus $A$ etwas zuordnet. Insgesamt erhalten wir eine sogenannte \emph{Restriktionsabbildung}:
        \[ \res : \Abb(X,Y) \to \Abb(A,Y) \ ,\ f \mapsto f\vert_A \]
    Beachte: Im Fall $A\subsetneq X$ sind $f$ und $f\vert_A$ zwei verschiedene Abbildungen, da sie verschiedene Definitionsbereiche haben.
\end{defin}


\begin{bsp} \label{bsp:quelleschrank} \quad
    \begin{enumerate}
    \item Betrachte die Abbildung
        \[ f : \R \to \R \ ,\ x \mapsto x^2 \]
    Deren Einschränkung auf die Teilmenge $\Q\subseteq \R$ ist dann gegeben durch
        \[ f\vert_{\Q} : \Q \to \R \ ,\ x\mapsto x^2 \]
    Obwohl die Zuordnungsvorschrift ($x\mapsto x^2)$ dieselbe ist, handelt es sich um eine andere Funktion! Beispielsweise ist $7\in \im(f)$ wegen $f(\sqrt{7})=7$; dagegen ist $7\notin \im(f\vert_\Q)$, denn es gibt keine rationale Zahl $q\in \Q$, für die $q^2=7$ gälte.
    \item Bezeichnet $\iota_A$ die Inklusion einer Teilmenge $A\subseteq X$, so gilt $f\vert_A = f\circ\iota_A$, d.h. die Restriktion ergibt sich durch Verketten mit der Inklusionsabbildung $A\hookrightarrow X$. Insbesondere ist die Inklusionsabbildung selbst $\iota_A = \id_X\circ \iota_A = \id_X \vert_A$ genau die Einschränkung der Identitätsabbildung auf $A$.
    \end{enumerate}
\end{bsp}
 
 
\begin{defin}[* Einschränkung und Fortsetzung] \label{def:fortsetzung} \index{Einschränkung einer Abbildung} \index{Fortsetzung}
    Seien $X,Y$ zwei Mengen, $A\subseteq X$ eine Teilmenge und $X\xrightarrow{F} Y$, $A\xrightarrow{f} Y$ zwei Abbildungen. $F$ heißt eine \textbf{Fortsetzung} von $f$ (und $f$ eine \textbf{Einschränkung} von $F$), wenn $F\vert_A=f$.
\end{defin}


\begin{bsp}[*] \quad
    \begin{enumerate}
        \item Die \emph{komplexe Betragsfunktion}
            \[ \C \to \R_{\ge 0} \ ,\ x+iy \mapsto \sqrt{x^2+y^2} \]
        ist eine Fortsetzung der reellen Betragsfunktion $\R\to \R_{\ge 0} \ ,\ x\mapsto \vert x\vert$.
        \item In der Analysis wird die Sprache der „Reihen“ entwickelt und bewiesen, dass durch
            \[ \zeta : \R_{>1} \to \C \ ,\ s \mapsto \sum_{n=1}^\infty \frac{1}{n^s} \quad =\  \frac{1}{1^s} + \frac{1}{2^s} + \frac{1}{3^s} + \frac{1}{4^s} + \frac{1}{5^s} + \ldots\]
        eine wohldefinierte Abbildung gegeben ist. Beispielsweise bewies Euler 1735, dass
            \[ 1 + \frac{1}{4} + \frac{1}{9} + \frac{1}{16} + \frac{1}{25} + \ldots = \zeta(2) = \frac{\pi^2}{6} \]
        und nach einem Satz von Apéry\footnote{\href{https://de.wikipedia.org/wiki/Roger Apery}{Roger Apéry (1916-1994)}} ist über die Zahl $\zeta(3)$ zumindest bekannt, dass sie irrational ist.

        Die Abbildung $\zeta$ besitzt unendlich viele verschiedene Fortsetzungen auf die Obermenge $\C \supseteq \R$, von denen die meisten aber völlig uninteressant sind. Allerdings wird in der Funktionentheorie gezeigt, dass $\zeta$ \emph{genau eine} Fortsetzung auf $\C\setminus \{1\}$ besitzt, die \emph{holomorph}, d.h. komplex differenzierbar, ist. Diese Fortsetzung wird ebenfalls mit dem Buchstaben „$\zeta$“ notiert und heißt die \emph{Riemannsche Zeta-Funktion}. Beispielsweise ist $\zeta(-1)=-\frac{1}{12}$. In der Unterhaltungsmathematik werden manchmal verblüffende „Beweise“ präsentiert für die Gleichung $1+2+3+4+5+\ldots = -\frac{1}{12}$.
    \end{enumerate}
\end{bsp}


\begin{defin}[* Einschränken des Wertebereichs] \label{def:zielschrank}
    Sei $B\subseteq Y$ eine Teilmenge mit $\im(f) \subseteq B$. Dann ist die \textbf{Einschränkung von $f$ auf $B$} diejenige Abbildung $X \to B$, die durch die Abbildungsvorschrift
    \begin{align*}
        X \to B \ ,\ x \mapsto f(x)
    \end{align*}
    gegeben ist. Notation:
    \begin{align*}
        f\vert^B && (\text{lies: „$f$ eingeschränkt auf $B$“})
    \end{align*}
    Die Abbildung $f\vert^B$ funktioniert also genau wie $f$, mit dem einzigen Unterschied, dass der Wertebereich verkleinert wurde.
\end{defin}


\begin{bsp}[* Auf- und Abrunden]
    Für $x\in \R$ gibt es zur Notation der nächstkleineren und nächstgrößeren ganzen Zahl die \textbf{Gauß-Klammern}:
    \begin{align*}
        \lfloor x\rfloor & := (\text{Die nächstkleinere ganze Zahl}) & \lceil x \rceil & := (\text{Die nächstgrößere ganze Zahl})
    \end{align*}
    Beispielsweise gilt:
    \begin{align*}
    \biggl\lfloor \frac{23}{6} \biggr\rfloor & = \lfloor 3{,}8\bar 3\rfloor = 3 & \lfloor \pi \rfloor & = 3 & \lfloor -\pi \rfloor & = -4 & \lfloor 3 \rfloor & = 3 \\
    \biggl\lceil \frac{23}{6} \biggr\rceil & = \lfloor 3{,}8\bar 3\rfloor = 4 & \lceil \pi \rceil & = 4 & \lceil -\pi \rceil & = -3 & \lceil 3 \rceil & = 3
    \end{align*}
    Man erhält zwei Abbildungen $\begin{tikzcd}[cramped] \R \ar[r, "\lfloor - \rfloor", shift left] \ar[r, "\lceil - \rceil"', shift right] & \R \end{tikzcd}$. Wegen $\lfloor x\rfloor,\lceil x\rceil\in \Z$ für alle $x\in \R$ lassen sich diese beiden Funktionen im Wertebereich auf $\Z$ einschränken zu zwei Abbildungen
    \begin{align*}
        \lfloor - \rfloor : \R \to \Z \ & ,\ x\mapsto \lfloor x\rfloor & \lceil - \rceil : \R \to \Z \ & ,\ x\mapsto \lceil x\rceil
    \end{align*}
    der \textbf{Abrundungsfunktion} und der \textbf{Aufrundungsfunktion}. Sie besitzen folgende \emph{universelle Eigenschaften}:
        \[ n \le x \iff n\le \lfloor x\rfloor \quad\text{und}\quad \lceil x \rceil \le n \iff x\le n \qquad \text{für alle}\ n\in \Z,\ x\in \R \]
\end{bsp}


\begin{bem}[*] \label{einschraenkbarkeit}
    Beachte, dass sich der Wertebereich einer Funktion $f$ nur auf solche Teilmengen einschränken lässt, die $\im(f)$ umfassen. Ist beispielsweise
        \[ f : \Z \to \Z \ ,\ n \mapsto 2n-n^2 \]
    so ergäbe der Ausdruck „$f\vert^{\N}$“ keinen Sinn. Denn die Zuordnung $n\mapsto 2n-n^2$ definiert keine Abbildung $\Z\to \N$, weil beispielsweise $2\cdot 5 - 5^2$ gar kein Element von $\N$ wäre.

    Beim Definitionsbereich besteht keine solche Anforderung. Er lässt sich auf beliebige Teilmengen einschränken.
\end{bem}





\section{Injektiv, surjektiv, bijektiv}


In diesem Abschnitt seien $X,Y$ stets zwei Mengen.


\begin{defin}[Injektive Abbildung] \label{def:injektiv} \index{injektiv}
    Eine Abbildung $X \xrightarrow{f} Y$ heißt \textbf{injektiv} (oder auch: eine \textbf{Injektion}), wenn sie eine (und damit jede) der folgenden äquivalenten Bedingungen erfüllt:
    \begin{enumerate}[(i)]
        \item Für alle $a,b\in X$ gilt:
            \[ f(a)=f(b) \quad\implies\quad a=b \]
        \item Für je zwei verschiedene Elemente $a,b \in X$ sind auch $f(a)$ und $f(b)$ voneinander verschieden.
        \item Für jedes $y\in Y$ gibt es höchstens ein $x\in X$ mit $f(x)=y$.
    \end{enumerate}
\end{defin}
\begin{proof}
    (i)$\leftrightarrow$(ii): Es ist (ii) genau die Kontraposition von (i) und damit zu (i) äquivalent.

    (i)$\rightarrow$(iii): Es gelte (i) und es seien $a,b\in X$ mit $f(a)=y$ und $f(b)=y$. Wegen $f(a)=f(b)$ folgt aus (i), dass $a=b$. Also hat $y$ höchstens ein Urbild.

    (iii)$\rightarrow$(i): Seien $a,b\in X$ mit $f(a)=f(b)$. Da das Element $y:=f(a)\in Y$ nach (iii) höchstens ein Urbild hat, muss dann $a=b$ sein.
\end{proof}


\begin{bsp} \quad
    \begin{enumerate}
        \item Die Abbildung
            \[ f:\Z \to \Z \ ,\ n \mapsto n^2-2n \]
        ist nicht injektiv, denn es ist $f(0)=f(2)$.
        \item Für eine beliebige Menge $X$ ist die Abbildung
            \[ X \to \calP(X) \ ,\ x \mapsto \{x\} \]
        injektiv. Denn sind $x,y\in X$ mit $\{x\}=\{y\}$, so ist bereits $x=y$.
        \item Für jede Teilmenge $U\subseteq X$ ist die Inklusionsabbildung $U\hookrightarrow X$ injektiv.
    \end{enumerate}
\end{bsp}
	

\begin{defin}[Surjektive Abbildung] \label{def:surjektiv} \index{surjektiv}
    Eine Abbildung $X \xrightarrow{f} Y$ heißt \textbf{surjektiv} (oder auch: eine \textbf{Surjektion}), wenn sie eine (und damit jede) der folgenden äquivalenten Bedingungen erfüllt:
    \begin{enumerate}[(i)]
        \item Für jedes $y\in Y$ gibt es mindestens ein $x\in X$ mit $f(x)=y$.
        \item Es gilt $\im(f) = Y$, d.h. das Bild von $f$ stimmt mit dem Wertebereich von $f$ überein.
    \end{enumerate}
\end{defin}
\begin{proof}
    (i)$\leftrightarrow$(ii) ergibt sich direkt aus der Definition von $\im(f)$, siehe \cref{def:bildmenge}.
\end{proof}


\begin{bsp} \label{bsp:surjektiv} \quad
    \begin{enumerate}
        \item Die Abbildung
            \[ \R \setminus \{2\} \to \R \setminus \{1\} \ ,\ x \mapsto \frac{x+1}{x-2} \]
        ist surjektiv. Dies wurde in \cref{bsp:allbeweis} bewiesen.
        \item Die Abbildung
            \[ f:\N_0 \to \N_0 \ ,\ n \mapsto n+1 \]
        ist zwar injektiv, jedoch nicht surjektiv. Denn es gibt keine natürliche Zahl $n\in \N_0$, für die $n+1=0$ gälte. Daher ist $0\notin \im(f)$ und $f$ ist nicht surjektiv. Dedekind\footnote{\href{https://de.wikipedia.org/wiki/Richard_Dedekind}{Richard Dedekind (1831-1916)}} erhob dieses Phänomen sogar zur Definition von Unendlichkeit: Eine Menge $X$ heißt \emph{Dedekind-unendlich}, wenn es eine injektive Abbildung $X\to X$ gibt, die nicht surjektiv ist.
        \item Die Abbildung
            \[ g: \N_0 \to \N_0\ ,\ n \mapsto \begin{cases}
                n-1 & n\ge 1 \\
                0 & n=0
            \end{cases}\]
        hingegen ist surjektiv (aber nicht injektiv), denn für jedes $n \in \N_0$ ist $n=g(n+1)$.
        \item Es sei $\bbP$ die Menge aller ungeraden Primzahlen. Goldbach\footnote{\href{https://de.wikipedia.org/wiki/Christian_Goldbach}{Christian Goldbach (1690-1764)}} vermutete 1742 in einem Briefwechsel mit Euler, dass sich jede gerade Zahl $\ge 4$ als Summe zweier Primzahlen schreiben lässt. Diese \href{https://de.wikipedia.org/wiki/Goldbachsche_Vermutung}{Goldbachsche Vermutung} ist äquivalent dazu, dass die Abbildung
            \[ \bbP \times \bbP \to \{n\in \N \mid n\ \text{ist gerade und}\ n\ge 6\} \ ,\ (p,q) \mapsto p+q \]
        surjektiv ist. Sie konnte bislang weder bewiesen noch widerlegt werden. Die nächstschwächere Vermutung, dass jede ungerade Zahl $\ge 5$ eine Summe \emph{dreier} Primzahlen ist, konnte wohl 2013 von Harald Helfgott\footnote{\href{https://de.wikipedia.org/wiki/Harald_Helfgott}{Harald Helfgott (*1977)}} bewiesen werden.
    \end{enumerate}
\end{bsp}


\begin{defin}[Bijektive Abbildung] \label{def:bijektiv} \index{bijektiv}
    Eine Abbildung $X \xrightarrow{f} Y$ heißt \textbf{bijektiv} (oder auch: eine \textbf{Bijektion}), wenn sie eine (und damit jede) der folgenden äquivalenten Bedingungen erfüllt:
    \begin{enumerate}[(i)]
        \item Für jedes $y\in Y$ gibt es genau ein $x\in X$ mit $f(x)=y$.
        \item $f$ ist injektiv und surjektiv.
    \end{enumerate}
\end{defin}
\begin{proof}
    $f$ ist genau dann injektiv, wenn es für jedes $y\in Y$ \emph{höchstens ein} $x\in X$ mit $f(x)=y$ gibt; und genau dann surjektiv, wenn es für jedes $y\in Y$ \emph{mindestens ein} $x\in X$ mit $f(x)=y$ gibt. Kombination beider Aussagen ergibt die Äquivalenz (i)$\leftrightarrow$(ii).
\end{proof}


\begin{figure}[ht]
    \begin{minipage}{.48\textwidth}
        \includegraphics[width=7cm]{./_img/injsur4.jpeg}
        \centering \caption{injektiv, aber nicht surjektiv}
    \end{minipage}
    \quad
    \begin{minipage}{.48\textwidth}
        \includegraphics[width=7cm]{./_img/injsur2.jpeg}
        \centering \caption{surjektiv, aber nicht injektiv}
    \end{minipage}
    \quad\\[1em]
    \begin{minipage}{.48\textwidth}
        \includegraphics[width=7cm]{./_img/injsur1.jpeg}
        \centering \caption{weder injektiv noch surjektiv}
    \end{minipage}
    \quad
    \begin{minipage}{.48\textwidth}
        \includegraphics[width=7cm]{./_img/injsur3.jpeg}
        \centering \caption{bijektiv}
    \end{minipage}
\end{figure}


\begin{bsp} \label{bsp:injsur}
    Die vier Abbildungen
    \begin{align*}
        f_1 : \R \to \R \ &,\ x\mapsto x^2 & f_3 :\R\to \R_{\ge 0} \ &,\ x\mapsto x^2 \\
        f_2 : \R_{\ge 0} \to \R \ &,\ x\mapsto x^2 & f_4 :\R_{\ge 0} \to \R_{\ge 0} \ &,\ x\mapsto x^2
    \end{align*}
    sind, obwohl sie dieselbe Zuordnungsvorschrift haben, alle voneinander verschieden, weil sie verschiedene Definitions- oder Wertebereiche haben. Es gilt:
    \begin{enumerate}
        \item $f_1$ ist weder injektiv (wegen $f_1(1)=f_1(-1)$) noch surjektiv (wegen $-1\notin \im(f)$).
        \item $f_2$ ist injektiv (da jede reelle Zahl höchstens eines nichtnegative Quadratwurzel besitzt), aber nicht surjektiv.
        \item $f_3$ ist surjektiv (da jede nichtnegative reelle Zahl eine reelle Quadratwurzel besitzt), aber nicht injektiv.
        \item $f_4$ ist bijektiv. Denn jede nichtnegative reelle Zahl besitzt genau eine nichtnegative reelle Quadratwurzel.
    \end{enumerate}
\end{bsp}


\begin{vorschau}[* Abbildungen künstlich surjektiv machen] \label{surjektivmachen}
    Seien $X,Y$ zwei Mengen und $f:X\to Y$ eine Abbildung. Gemäß \cref{def:zielschrank} lässt sich der Wertebereich von $f$ auf die Menge $\im(f)$ einschränken. Die dadurch erhaltene Abbildung
        \[ f\vert^{\im(f)} : X \to \im(f) \ ,\ x\mapsto f(x) \]
    ist automatisch surjektiv, da es ja für jedes $y\in \im(f)$ mindestens ein $x\in X$ mit $f(x)=y$ gibt.

    Auf diese Weise ist es uns gelungen, $f$ „künstlich surjektiv“ zu machen. Ist uns daran gelegen, mit surjektiven Abbildungen zu arbeiten, so stellt das also kein Problem dar, da wir den Wertebereich einer Abbildung stets auf ihr Bild einschränken können.

    Ebenso ist es möglich, die Abbildung $f$ „künstlich injektiv“ zu machen. Dabei besteht der Trick darin, zwischen solchen Elementen von $X$, die unter $f$ denselben Funktionswert haben, „nicht mehr zu unterscheiden“. Die Technik „ähnliche Elemente nicht mehr voneinander zu unterscheiden“ wird in der LA1 eine prominente Rolle im Umfeld des \href{https://de.wikipedia.org/wiki/Homomorphiesatz}{Homomorphiesatzes} spielen und kann mithilfe der in \cref{sec:aequirel} thematisierten \emph{Äquivalenzrelationen} formalisiert werden, siehe \cref{teilmengenvsfaktormengen}.
\end{vorschau}





\section{Invertierbare Abbildungen}


In diesem Abschnitt seien stets $X,Y$ zwei Mengen und $f:X\to Y$ eine Abbildung.


\begin{defin}[Umkehrabbildung] \label{def:umkehrabb} \index{Umkehrabbildung} \index{inverse Funktion} \index{invertierbare Abbildung} \index{Retraktion}
    Eine Abbildung $g:Y \to X$ heißt 
    \begin{itemize}
        \item \textbf{linksinvers zu $f$} (oder auch: eine \textbf{Retraktion von $f$}), falls $g\circ f=\id_X$.
        \item \textbf{rechtsinvers zu $f$} (in manchem Kontext auch: ein \textbf{Schnitt von $f$}), falls $f\circ g=\id_Y$.
        \item \textbf{invers zu $f$} (oder auch: eine \textbf{Umkehrabbildung von $f$} oder eine \textbf{Inverse zu $f$}), falls sie sowohl links- als auch rechtsinvers zu $f$ ist.
    \end{itemize}
    $f$ heißt eine \textbf{invertierbare Abbildung}, wenn sie eine Umkehrabbildung besitzt.
\end{defin}


\begin{bsp} \label{bsp:umkehrabb} \qquad
    \begin{enumerate}
        \item Die Abbildung
            \[ \Z\to \Z \ ,\ n\mapsto n+1 \]
        ist invertierbar, eine Inverse ist gegeben durch die Abbildung $n\mapsto n-1$.
        \item Subtraktion, Division, Wurzel und Logarithmus sind jeweils invers zu Addition, Multiplikation und Potenz:
        \begin{center}
        \begin{tabular}{clll}
            Def.- und Wertebereich & Zuordnung & Inverse Zuordnung & \phantom{Platzhalter}\\
            \cmidrule{1-3}
            $\R\to \R$ & $x\mapsto x+a$ & $x\mapsto x-a$ & (für $a\in \R$)\\
            $\R\to \R$ & $x\mapsto ax$ & $x\mapsto \frac{x}{a}$ & (für $a\in \R\setminus \{0\}$) \\
            $\R_{>0} \to \R_{>0}$ & $x\mapsto x^a$ & $x\mapsto \sqrt[a]{x}$ & (für $a\in \R\setminus \{0\}$) \\
            $\phantom{\R_{>0} \to \R_{>0}}\makebox[0pt][r]{$\R \to \R_{>0}$}$ & $x \mapsto a^x$ & $x \mapsto \log_a(x)$ & (für $a\in \R_{>0}$)
        \end{tabular}
        \end{center}
        Du bist es ja aus der Schule gewohnt, Gleichungen, die Summen, Produkte und Potenzen enthalten, aufzulösen mittels Subtrahieren, Dividieren, Wurzelziehen und Logarithmieren.
        \item Die „Spiegelung an der $x$-Achse“
        \[ \R^2 \to \R^2 \ ,\ (x,y) \mapsto (x,-y) \]
        ist zu sich selbst invers. Selbstinverse Abbildungen heißen auch \textbf{Involutionen}.
        \item Sei $X$ eine beliebige Menge. Aus \cref{idneutral} folgt $\id_X\circ \id_X=\id_X$, sodass $\id_X$ eine invertierbare, selbstinverse Abbildung ist.\footnote{vgl. \cref{regelnfuerinv}a)}
        \item Für $k\in \N_0$ betrachte die beiden Abbildungen
        \begin{align*}
            f:\N_0 \to \N_0 \ &,\ n \mapsto n+1 \\
            g_k: \N_0 \to \N_0\ &,\ n \mapsto \begin{cases}
                n-1 & n\ge 1 \\
                k & n=0
            \end{cases}
        \end{align*}
        Dann gilt
        \begin{align*}
            (g_k\circ f)(n)&=g_k(n+1)=n && n\in \N_0 \\
            (f\circ g_k)(0) & = f(k) = k+1 \neq 0
        \end{align*}
        sodass $g_k\circ f=\id_{\N_0}$ aber $f\circ g_k\neq \id_{\N_0}$. Also ist zwar $g_k$ linksinvers zu $f$ (und dementsprechend $f$ rechtsinvers zu $g_k$), aber nicht rechtsinvers zu $f$.
        \[\begin{tikzcd}
            f\vcentcolon & 0 \ar[r, bend left]& 1 \ar[r, bend left]& 2 \ar[r, bend left]& 3 \ar[r, bend left]& 4 \ar[r, bend left]& 5 \ar[r, bend left]& \dots \\
            g_0\vcentcolon & 0 \ar[loop, out=145, in=215, looseness=5] & 1 \ar[l, bend right] & 2 \ar[l, bend right] & 3 \ar[l, bend right] & 4  \ar[l, bend right] & 5  \ar[l, bend right] &  \ar[l, bend right]  \dots
        \end{tikzcd}\]
        Von weiteren Abbildungen dieser Art handelt die Geschichte vom \href{https://en.wikipedia.org/wiki/Hilbert\%27s_paradox_of_the_Grand_Hotel}{Hilbert-Hotel}.
    \end{enumerate}
\end{bsp}


\begin{satz}[Eindeutigkeit der Inversen]\label{umkehreind}
    Sei $f$ eine invertierbare Abbildung. Dann ist die Umkehrabbildung von $f$ eindeutig bestimmt.\footnote{vgl. \cref{inveind}}
\end{satz}
\begin{proof}
    Seien $g,h:Y\to X$ zwei Umkehrabbildungen von $f$. Dann gilt:
    \begin{align*}
        g & = g\circ \id_Y && (\text{nach \cref{idneutral}})\\
        & = g\circ (f\circ h) && (\text{da $h$ invers zu $f$ ist}) \\
        & = (g\circ f)\circ h && (\text{nach \cref{abbassoziativ}}) \\
        & = \id_X \circ h && (\text{da $g$ invers zu $f$ ist}) \\
        & = h && (\text{nach \cref{idneutral}}) \qedhere
    \end{align*}
\end{proof}


\begin{bem}[\textbf{Die} Umkehrabbildung] \label{dieumkehrabb}
    Im Sinne von \cref{kennzeichnung} berechtigt uns dieser Eindeutigkeitssatz dazu, anstelle von „einer“ Umkehrabbildung von $f$ von \emph{der} Umkehrabbildung von $f$ zu sprechen.
    
    Die zu einer invertierbaren Abbildung $f$ inverse Abbildung wird notiert mit
        \[ f^{-1} \]
    Beachte, dass es nur nur dann Sinn ergibt, von „der Umkehrabbildung $f^{-1}$“ zu sprechen, wenn $f$ invertierbar ist. Im Allgemeinen sind Abbildungen nicht invertierbar und wir werden in \cref{invwiderleg} Techniken besprechen, mit denen sich dies beweisen lässt.

    Eine bloße Links- oder Rechtsinverse braucht dagegen nicht eindeutig sein, wie \cref{bsp:umkehrabb} zeigt.
\end{bem}


\begin{bem}
    Das Zeichen „$f^{-1}$“ ist bereits in \cref{def:bildmenge} aufgetaucht und bezeichnete dort Fasern und Urbildmengen. Ist $f$ invertierbar, so trägt es nun drei verschiedene Bedeutungen zugleich:
    \begin{center}
    \begin{tabular}{cl}
        Zeichen & Bedeutung \\
        \midrule
        $f$ & Die Abbildung $X \to Y \ ,\ x \mapsto f(x)$ \\
        $f$ & Die (mengenwertige) Abbildung $\calP(X) \to \calP(Y) \ ,\ A\mapsto f(A)$  \\
        $f^{-1}$ & Die (mengenwertige) Abbildung $\calP(Y) \to \calP(X) \ ,\ B\mapsto f^{-1}(B)$ \\
        $f^{-1}$ & Die (mengenwertige) Abbildung $Y \to \calP(X) \ ,\ y\mapsto f^{-1}(y)$ \\
        \midrule
        $f^{-1}$ & Die Umkehrabbildung $f^{-1} : Y\to X \ ,\ y \mapsto f^{-1}(y)$ \\
        & (existiert nur, wenn $f$ invertierbar ist)
    \end{tabular}
    \end{center}
    Beachte, dass die ersten vier Ausdrücke immer Sinn ergeben; der fünfte aber nur dann, wenn $f$ invertierbar ist.
\end{bem}


\begin{satz} \label{bijektiviso}
    Seien $X,Y$ zwei Mengen und $X\xrightarrow{f} Y$ eine Abbildung. Dann sind äquivalent:
    \begin{enumerate}[(i)]
        \item $f$ ist invertierbar.
        \item $f$ ist bijektiv.
    \end{enumerate}
\end{satz}
\begin{proof}
    (ii)$\to$(i): Sei $f$ bijektiv. Dann gibt es für jedes $y\in Y$ genau ein $x\in X$ mit $f(x)=y$, sodass durch
        \[ g : Y\to X \ ,\ y \mapsto (\text{Das eindeutig bestimmte $x\in X$ mit $f(x)=y$}) \]
    eine Abbildung definiert ist. Nun ist $g$ invers zu $f$, denn:
    \begin{labeling}
        \item[($g\circ f=\id$):] Für ein beliebiges $a\in X$ ist
            \[ (g\circ f)(a) = g(f(a)) = (\text{Das eindeutig bestimmte $x\in X$ mit $f(x)=f(a)$}) = a \]
        Weil $a\in X$ beliebig war, folgt die Gleichheit von Abbildungen $g\circ f=\id_X$.
        \item[($f\circ g=\id$):] Für ein beliebiges $b\in Y$ ist
            \[ (f\circ g)(b) = f(\text{Das eindeutig bestimmte $x\in X$ mit $f(x)=b$}) = b \]
        Also ist $f\circ g=\id_Y$.
    \end{labeling}
    Insgesamt ist damit gezeigt, dass $g$ invers zu $f$ ist.

    Der Beweis der Implikation (i)$\to$(ii) ist Inhalt von \cref{aufg:bijektiviso}.
\end{proof}


\begin{bem}[Bijektivität beweisen]
    Ist $f$ eine Abbildung, für die du beweisen möchtest, dass sie bjektiv ist, so stehen dir nun zwei Wege offen:
    \begin{enumerate}
        \item Du arbeitest mit \cref{def:bijektiv}, d.h. du beweist sowohl, dass $f$ injektiv ist, als auch, dass $f$ surjektiv ist.
        \item Du arbeitest mit \cref{bijektiviso}, d.h. du schreibst einen Kandidaten für die Umkehrabbildung hin und beweist daraufhin, dass er tatsächlich invers zu $f$ ist.
    \end{enumerate}
    In manchen Situationen ist es schwer bis unmöglich, eine Abbildungsvorschrift für die Umkehrabbildung zu finden. In diesem Fall ist der erste Weg leichter.
    
    Es gibt aber auch Situationen wie z.B. in \cref{bsp:umkehrabb}, in denen du bei genauem „Hinsehen“ bereits einen Kandidaten für die Umkehrabbildung ausfindig machen kannst. In diesem Fall bietet sich der zweite Weg an. Beachte auch, dass ein Beweis über den zweiten Weg stets informativer ist, da er dem Leser bereits die Gestalt der inversen Abbildung mitteilt.
\end{bem}


\begin{bem}[Invertierbarkeit widerlegen] \label{invwiderleg}
    Möchtest du beweisen, dass eine Abbildung keine Umkehrabbildung besitzt, so genügt es nach \cref{bijektiviso} bereits, wenn du beweist, dass sie nicht injektiv oder nicht surjektiv ist.
\end{bem}


\begin{bsp}
    Beispielsweise gilt:
    \begin{enumerate}
        \item Die Abbildung $f : \R \to \R \ ,\ x\mapsto x^2$ ist nicht invertierbar, weil sie nicht injektiv ist. Denn es ist zum Beispiel $f(-5)=f(5)$.
        \item Die Abbildung $f : \N \to \calP(\N) \ ,\ n \mapsto \{n\}$ ist nicht invertierbar, weil sie nicht surjektiv ist. Denn es ist $\emptyset \notin \im(f)$.
    \end{enumerate}
\end{bsp}





\clearpage
\section{Aufgabenvorschläge}


\begin{aufg}[Zerlegen von Abbildungen]
    Realisiert die folgenden Abbildungen als Verkettung möglichst elementarer Bausteine:
    \begin{enumerate}
        \item $\R\to \R \ ,\ x\mapsto (3x+1)^3$
        \item Sei $A$ die Menge aller Leute im Tutorium. Betrachte die Abbildung $f:A\to \N \ ,\ a\mapsto$ (die Anzahl der Wörter im Erstlingswerk des Literaturnobelpreisträgers im Geburtsjahr von $a$).
        \item $\calP(\R) \to \calP(\Q) \ ,\ X \mapsto \Q \setminus (X^c\cap \Z)$
    \end{enumerate}
\end{aufg}


\begin{aufg}[Wohldefiniertheit] \label{aufg:wohldef}
    An der Tafel von Captain Chaos stehen die folgenden Ausdrücke:
    \begin{align*}
        & (i) & \N \to \N\ &,\ n \mapsto n-1 \\[0.5em]
        & (ii) & \R\times \R \to \R \ &,\ (x,y) \mapsto \frac{x}{y} \\[0.5em]
        & (iii) & \R \times \R \to \R \ &,\ (x,y) \mapsto x^y \\[0.5em]
        & (iv) & \Q \to \Z \ &,\ \frac{p}{q} \mapsto p-q \\
        & (v) & \R \to \R \ &,\ x \mapsto \begin{cases}
            x^2+1\ , & \text{falls}\ x \le 0 \\
            x^2-1\ , & \text{falls}\ x\ge 0
        \end{cases}
    \end{align*}
    Was haltet ihr davon?
\end{aufg}


\begin{aufg}[injektiv, surjektiv, bijektiv (L)] \label{aufg:insurbi}
    Untersucht die folgenden Abbildungen darauf, ob sie injektiv, surjektiv oder bijektiv sind:
    \begin{align*}
        & \text{a)} & \Z \to \Z\ &,\ n\mapsto 2n \\
        & \text{b)} & \Z \times \N_{\ge 1} \to \Q\ &,\ (z,n) \mapsto \frac{z}{n} \\
        & \text{c)} & \R \to \R\ &,\ x\mapsto 3x+1 \\
        & \text{d)} & \calP(\R) \to \calP(\R) \ &,\ A \mapsto A^c \\
        & \text{e)} & X\xrightarrow{\id} X\ & && (\text{für eine beliebige Menge $X$}) \\
        & \text{f)} & X\times Y \to X \ &,\ (x,y) \mapsto x && (\text{für beliebige Mengen $X,Y$})
    \end{align*}
\end{aufg}


\begin{aufg}[Vererbung von Injektivität und Surjektivität] \label{aufg:bijektiviso}
    Seien $X,Y,Z$ drei beliebige Mengen und $X \xrightarrow{f} Y \xrightarrow{g} Z$ zwei Abbildungen. Beweist die folgenden Implikationen:
    \begin{enumerate}
        \item $f,g$ injektiv $\implies$ $g\circ f$ injektiv $\implies$ $f$ injektiv.
        \item $f,g$ surjektiv $\implies$ $g\circ f$ surjektiv $\implies$ $g$ surjektiv.
        \item Invertierbare Abbildungen sind bijektiv (vgl. \cref{bijektiviso}).
    \end{enumerate}
\end{aufg}



% ====================================================================
% 5. Vortrag <<Relationen>>
% ====================================================================

\setchapterpreamble[c][.7\textwidth]{%
  \itshape\dgrau\small Relationen drücken Beziehungen zwischen Elementen einer Menge aus. Von besonderer Bedeutung sind Ordnungs- und Äquivalenzrelationen, deren grundlegende Eigenschaften in diesem Vortrag besprochen werden.
  \vspace{24pt}%
}

\chapter{Relationen}
% ====================================================================
%                                                                 
%
%       §01
%
%
% %%ts latex start%%[2019-03-07 Thu 14:45]%%ts latex end%%
% ====================================================================


% --------------------------------------------------------------------
% §1.1 section <<Begriff der Relation>>
% --------------------------------------------------------------------
\section{Begriff der Relation}

Im letzten Kapitel haben wir Ausdrücke der Form $f(x)=x^3$ in \textit{Abbildungen} $f:A\to B$ verallgemeinert, die eine Beziehung zwischen den Elementen einer Menge $A$ und einer Menge $B$ herstellen. Heute möchten wir Ausdrücke wie $2\leq 3$ oder $\sqrt{2}\neq \sqrt{3}$, die Beziehungen zwischen je zwei Elementen der selben Menge angeben, als \textit{Relationen} realisieren. Relevant sind vor allem \textit{Ordnungsrelationen} und \textit{Äquivalenzrelationen}.

% Definition Relation
Abstrakt kodieren wir für den Relationenbegriff die Elemente, die in Beziehung
stehen sollen, in einem Paar, also einem Element des kartesischen
Produkts.
\begin{de}
	Sei $X$ eine Menge. Eine \textbf{Relation} $\sim$ auf $X$ ist verkörpert durch eine Teilmenge $R\subseteq X\times X$ mit
\begin{align*}
 x\sim y\quad :&\Leftrightarrow\quad (x,y)\in R && x,y\in X
\end{align*}
\end{de}

%\begin{bsp}\label{bsp:gatsby}
%	Auf der Menge $X=\{\text{Nick Carraway, Jay Gatsby, Daisy Buchanan, Thomas Buchanan}\}$ betrachten wir die Relation \glqq\,$\_$ liebt $\_$\grqq\,. Wir haben $N$ liebt $J$, $D$ liebt $J$, $J$ liebt $D$ und $T$ liebt $D$.
%		\[ \begin{tikzcd}
%			N \ar[r] & J \ar[d,leftrightarrow]\\ T \ar[r] & D
%		\end{tikzcd} \]
%	Die kodierende Menge $R\subset X\times X$ ist also
%		\[ R = \left\{(N,J),(J,D),(D,J),(T,D)\right\} \,. \]
%\end{bsp}

\begin{bsp}\label{bsp:starwars}
	Auf der Menge $X=\{$Luke Skywalker, Anakin Skywalker, Yoda, Obi-Wan Kenobi, Qui-Gon Jinn, Dooku, Palpatine, Darth Maul, Darth Plagueis$\}$ betrachten wir die Relation
		\[ x \sim y :\Longleftrightarrow \text{ $x$ war Padawan von $y$. } \]
	Wir haben
		\[\begin{tikzcd}
			&		& \text{Darth Plagueis} & \\
			& \text{Yoda} & \text{Palpatine} \ar[u] & \\
			& 		& \text{Dooku}\ar[u]\ar[ul] & \text{Darth Maul}\ar[ul] \\
			&		& \text{Qui-Gon Jinn} \ar[u] &\\
			&		& \text{Obi-Wan Kenobi} \ar[u] & \\
			& \text{Luke Skywalker}\ar[uuuu]\ar[ur]	& &\text{Anakin Skywalker}\ar[ul]\ar[uuuul]
		\end{tikzcd}\]
	Ein Pfeil von $a$ nach $b$ soll dabei dafür stehen, dass $a$ mit $b$ in Beziehung $a\sim b$ steht, also $a$ ein Padawan von $b$ war. Eine solche Darstellung nennt man einen \textbf{Graphen} der Relation.
	
	Die kodierende Menge $R\subset X\times X$ ist also $R = \big\{$(Luke,Yoda), (Luke,Obi-Wan), (Anakin,Obi-Wan), (Anakin, Palpatine), (Obi-Wan,Qui-Gon), (Qui-Gon, Dooku), (Dooku, Yoda), (Dooku, Palpatine) (Darth Maul, Palpatine), (Palpatine, Darth Plagueis)$\big\}$.
\end{bsp}

\begin{bsp}\label{bsp:leq}
	Auf $\Zz$ definieren die Relation \glqq$\leq$\grqq\, durch
\begin{align*}
 n\leq m \quad &:\Leftrightarrow\quad \exists k\in\mathbb{N}_0: m=n+k  && n,m\in \Zz
\end{align*}
	Durch welche Menge $R\subset\Zz\times\Zz$ wird \glqq$\leq$\grqq\, repräsentiert? Für $n,m\in \Zz$ gilt: 
		\[ (n,m)\in R\quad \overset{!}{\Leftrightarrow}\quad n\leq m \quad\Leftrightarrow\quad \exists k\in\mathbb{N}_0: m=n+k \,. \]
	Eine mögliche Formulierung wäre also \( R = \left\{(n,m)\in\Zz\times\Zz\mid\exists k\in\mathbb{N}_0: m=n+k\right\} \). Letztlich haben wir hier einfach nur die Definition der Relation abgeschrieben und nicht wirklich etwas gelernt; wenn man tatsächlich mit Relationen arbeitet, ist es ganz allgemein meist ergiebiger, die Menge $R$ als technische Umsetzung zu vergessen und sich mit der direkten Definition der Relation auseinanderzusetzen. Wie das gemeint ist, sehen wir gleich bei den Eigenschaften von Relationen.
\end{bsp}

\begin{de}\label{bsp:trivialrelationen}
	Vorher noch drei Extrembeispiele: Sei $X$ eine beliebige Menge.
\begin{itemize}
 \item Die \textbf{Allrelation} auf $X$ ist verkörpert durch die Menge $X\times X$, setzt also alle Elemente miteinander in Beziehung:
		\[ \begin{tikzcd}[column sep = huge]
			x \ar[loop] \ar[r,<->] \ar[rr, out=320, in=220, <->] &
			y \ar[loop] \ar[r,<->]  &
			z \ar[loop] 
		\end{tikzcd} \]
	Die Doppelpfeile stehen dabei für eine Beziehung in beide Richtungen, etwa gilt $x\sim y$ sowie $y\sim x$.
\item Die \textbf{Nullrelation} ist verkörpert durch die leere Menge $\emptyset$, setzt also keine Elemente in Beziehung:
		\[ \begin{tikzcd}[column sep = huge]
			 x &y &z &  
		\end{tikzcd} \]
\item Die \textbf{Gleichheitsrelation} ist definiert durch $x\sim y:\Leftrightarrow x=y$, also
		 \[ \begin{tikzcd}[column sep = huge]
		 	x \ar[loop] &
		 	y \ar[loop] &
		 	z \ar[loop] 
		 \end{tikzcd} \]
	Sie ist verkörpert durch die so genannte \emph{Diagonale}
		\[ D_X :=\left\{ (x,x)\in X\times X\mid x\in X \right\} \,. \]
	(In der Darstellung des kartesischen Produkts als Ebene mit Koordinatenachsen ist das gerade die Winkelhalbierende, daher der Name.)
\end{itemize}
\end{de}

\begin{de}
	Sei $X$ eine Menge und $\sim$ eine Relation auf $X$. Dann heißt die Relation $\sim$
	\begin{itemize}
		\item \textbf{reflexiv}, falls für alle $x\in X$ die Relation $x\sim x$ gilt. Im Graphen muss jedes Element also einen Pfeil von sich auf sich selbst tragen.
		\item \textbf{symmetrisch}, falls für alle $x,y\in X$ aus $x\sim y$ schon $y\sim x$ folgt. Im Graphen müssen also alle Pfeile zwischen verschiedenen Elementen Doppelpfeile sein.
		\item \textbf{antisymmetrisch}, falls für alle $x,y\in X$ aus $x\sim y$ und $y\sim x$ schon $x=y$ folgt.	Im Graphen darf es also zwischen verschiedenen Elementen keine Doppelpfeile geben.
		\item \textbf{transistiv}, falls für alle $x,y,z\in X$ aus $x\sim y$ und $y\sim z$ schon $x\sim z$ folgt. Im Graphen gibt es für einen indirekten Weg entlang der Pfeile auch immer den direkten.
	\end{itemize}
	\begin{figure}[h]
		\begin{tikzpicture}[commutative diagrams/every diagram]
			\begin{scope}[xshift=-6cm]
				\node (P0) at (0,0) {$x$};
				\path[commutative diagrams/.cd, every arrow, every label]
				(P0) edge[commutative diagrams/loop] (P0);
				\node[below] at (0,-.5) {reflexiv};
			\end{scope}
			\begin{scope}[xshift=-2cm]
				\node (P0) at (-.7,0) {$x$};
				\node (P1) at (.7,0) {$y$};
				\path[commutative diagrams/.cd, every arrow, every label]
				(P0) edge[commutative diagrams/<->] (P1);
				\node[below] at (0,-.5) {symmetrisch};
			\end{scope}
			\begin{scope}[xshift=2cm]
				\node (P0) at (-.7,0) {$x$};
				\node (P1) at (.7,0) {$y$};
				\path[commutative diagrams/.cd, every arrow, every label]
				(P0) edge[commutative diagrams/<->] (P1);
				\draw (-.3,-.3) -- (.3,.3);
				\node[below] at (0,-.5) {antisymmetrisch};
			\end{scope}
			\begin{scope}[xshift=6cm]
				\node (P0) at (-1.3,.5) {$x$};
				\node (P1) at (0,0) {$y$};
				\node (P2) at (1.3,.5) {$z$};
				\path[commutative diagrams/.cd, every arrow, every label]
				(P0) edge (P1)
				(P1) edge (P2)
				(P0) edge[commutative diagrams/dotted, out=20, in=160] (P2);
				\node[below] at (0,-.5) {transitiv};
			\end{scope}
 		\end{tikzpicture}
	\end{figure}
\end{de}



\begin{bem}
 Die Definition einer symmetrischen Relation kann als die Formel
 \begin{align*}
  x\sim y\quad & \implies\quad y\sim x && x,y\in X
 \end{align*}
formuliert werden. Dies impliziert bereits die Äquivalenz
 \begin{align*}
  x\sim y\quad & \Leftrightarrow\quad y\sim x && x,y\in X
 \end{align*}
 da „$x$“ und „$y$“ ja austauschbare Bezeichnungen für beliebige Elemente von $X$ sind.
\end{bem}


\begin{bsp}\label{bsp:relationseigenschaften}
	Wir prüfen die Padawan-Relation aus \cref{bsp:starwars} auf diese Eigenschaften:
		\begin{itemize}
			\item Sie ist nicht reflexiv, etwa war Obi-Wan nicht sein eigener Padawan (tatsächlich gilt das für kein $x\in X$).
			\item Sie ist nicht symmetrisch, etwa war Luke Yodas Padawan, aber Yoda nicht Lukes.
			\item Sie ist antisymmetrisch, denn es gibt keine $x,y\in X$ mit $x\sim y$ und $y\sim x$; daher ist die Implikation
				\[ \forall x,y\in X : \big(\underbrace{x\sim y \text{ und }y\sim x}_\text{falsch}\implies x=y\big) \]
			trivialerweise eine wahre Aussage (\textit{ex falso quod libet}, siehe auch \cref{vacuoustruth}).
			\item Sie ist nicht transitiv, denn es war Darth Maul Padawan von Palpatine, und Palpatin Padawan von Darth Plagueis, aber Darth Maul nicht Padawan von Darth Plagueis.
		\end{itemize}
	Für die $\leq$-Relation aus \cref{bsp:leq} gilt:
		\begin{itemize}
			\item Sie ist reflexiv, denn für alle $n\in\Zz$ gilt $n\leq n$ (setze $k=0$).
			\item Sie ist nicht symmetrisch, so ist etwa $2\leq3$ aber $3\not\leq2$.
			\item Sie ist antisymmetrisch, denn aus $n\leq m$ und $m\leq n$ folgt $n=m+k_1$ und $m=n+k_2$. Setzen wir ein, erhalten wir $n=n+k_1+k_2$, also $k_1+k_2=0$. Wegen $k_1,k_2\geq0$ folgt $k_1=k_2=0$ und daher $n=m$.
			\item Sie ist transitiv, denn: Sei $n\leq m$ und $m\leq k$, etwa $m=n+k_1$ und $k=m+k_2$. Dann gilt
				\[ k=m+k_2=n+\underbrace{k_1+k_2}_{=:k_3} \]
				und wegen $k_3=k_1+k_2\in\mathbb{N}_0$ folgt daraus gerade $n\leq k$.
		\end{itemize}
\end{bsp}



\begin{bem}[Die tiefere Bedeutung der Transitivität] \label{kettenfalten}
 Seien $X$ eine Menge mit einer Relation „$\sim$“ und $a,b,c,d\in X$ ein paar Elemente von $X$. Anstelle von „Es gilt $a\sim b$, $b\sim c$ und $c\sim d$“ schreibt man kurz einfach
 \[ a\sim b\sim c\sim d \]
 Beispielsweise schreibt man
 \[ \Nz \subseteq \Zz\subseteq\Qz\subseteq\Rz\subseteq \Cz \]
 oder
 \[ 2 < 4 < 8 < 12 < 15 \]
 Mit dieser Notation lässt sich die Definition der Transitivität auch durch folgende Formel ausdrücken:
 \begin{align*}
  x\sim y \sim z \quad&\implies\quad x\sim z && x,y,z\in X
 \end{align*}
Transitivität heißt also, dass sich jede aus drei Elementen bestehende „Relationskette“ „zusammenfalten“ lässt. Es lässt sich zeigen, dass dies bei transitiven Relationen auch für beliebig lange „Ketten“ gilt. Mit anderen Worten: Sind $n\in \Nz$ und $x_1,\dots , x_n\in X$ ein paar Elemente mit
\[ x_1 \sim x_2\sim \ldots \sim x_n \]
so folgt, sofern ${\sim}$ eine transitive Relation ist, daraus auch schon $x_1\sim x_n$. Im Spezialfall, dass $\sim$ die Gleichheitsrelation ist, hast du das auch schon in der Schule andauernd ausgenutzt: Sind $x_1,\dots , x_n$ irgendwelche Zahlen/Vektoren/Funktionen und gilt
\[ x_1=x_2\ldots = x_n \]
so folgt daraus, dass auch $x_1=x_n$ ist. \\[0.5em] 
 Beachte, dass diese Schlussfolgerung bei nicht-transitiven Relationen fehlschlägt. Beispielsweise gilt
 \[ 2\cdot 6 \neq 3\cdot 6\neq 3\cdot 4 \]
 aber es ist ja dennoch $2\cdot 6=3\cdot 4$.
\end{bem}

% --------------------------------------------------------------------
% §1.2 section <<Ordnungsrelationen>>
% --------------------------------------------------------------------
\section{Ordnungsrelationen}

Ordnungsrelationen sind Verallgemeinerungen der bekannten \glqq$\leq$\grqq-Relation auf den Zahlenräumen $\mathbb{N},\mathbb{Z},\mathbb{Q},\mathbb{R}$ und werden daher häufig mit dem Symbol $\leq$ anstelle von $\sim$ notiert, auch wenn es nicht zwingend um Zahlenräume geht.

\begin{de}[Ordnungsrelationen]
	Eine \textbf{Ordnungsrelation} (auch: \swdde{partielle Ordnung} oder \emph{Halbordnung}) auf einer Menge $X$ ist eine Relation $\leq$ auf $X$, die reflexiv, antisymmetrisch und transitiv ist. Konkret:
\begin{align*}
 & (\text{O1}) & \forall x\in X:&\quad x\leq x && (\text{Reflexivität}) \\
& (\text{O2}) & \forall x,y\in X:&\quad  x\leq y\ \wedge\  y\leq x \quad\implies\quad x=y && (\text{Antisymmetrie}) \\
& (\text{O3}) & \forall x,y,z\in X:&\quad x\leq y\ \wedge\ y\leq z \quad\implies\quad x\leq z && (\text{Transitivität})
 \end{align*}
In diesem Fall nennt man das Paar $(X,\leq)$ eine \textbf{geordnete Menge}.

Besitzt die Ordnungsrelation „$\leq$“ sogar die Eigenschaft
\begin{align*}
 & (\text{TO}) & \forall x,y \in X:&\quad x\leq y\ \vee\ y\leq z 
\end{align*}
so nennt man sie eine \textbf{Totalordnung}. Eine Totalordnung ist also eine Halbordnung, bei der je zwei Elemente miteinander \glqq vergleichbar\grqq\, sind.
\end{de}

\begin{bsp} Es gilt:
\begin{itemize}
 \item Die Padawan-Relation aus \cref{bsp:starwars} ist keine Halbordnung, denn in \cref{bsp:relationseigenschaften} haben wir gesehen, dass sie nicht transitiv (und auch nicht reflexiv) ist.
 \item Die $\leq$-Relation aus \cref{bsp:leq} dagegen ist nach \cref{bsp:relationseigenschaften} tatsächlich eine Halbordnung. Es liegt sogar eine Totalordnung vor, denn: Seien $m,n\in\Zz$. Dann gilt entweder $m-n\geq0$ oder $m-n<0$. Im ersten Fall setze $k=m-n\in\mathbb{N}_0$ und erhalte aus $m=n+k$ schon $n\leq m$. Im zweiten Fall setze $k=n-m=-(m-n)\in\mathbb{N}_0$ und erhalte aus $n=m+k$ schon $m\leq n$.
\end{itemize}
\end{bsp}

\begin{bsp}[Teilmengenrelation]
	Sei $M$ eine Menge. Auf $\cP(M)$ betrachten wir die Inklusionsrelation \glqq$\subseteq$\grqq\,. In \cref{sat:Mengenrelationen} haben wir genau die für eine Halbordnung geforderten Eigenschaften nachgewiesen. Somit ist $(\mathcal{P}(M),\subseteq)$ eine geordnete Menge. \\[0.5em]
Sofern $M$ mindestens zwei verschiedene Elemente enthält, ist sie jedoch nicht totalgeordnet. Denn sind $a,b\in M$ zwei verschiedene Elemente, so sind die Einermengen $\{a\}$ und $\{b\}$ jeweils Teilmengen von $M$ aber wegen $a\neq b$ ist weder $\{a\}\subseteq \{b\}$ noch $\{b\}\subseteq \{a\}$.
\end{bsp}





\subsection{Schranken}
Besonders interessant sind häufig Elemente, die als sogenannte Schranken fungieren:
\begin{de}[Schranken] \label{schranken}
	Sei $X$ eine halbgeordnete Menge und $T$ eine Teilmenge von $X$. Dann heißt ein Element $x\in X$
		\begin{enumerate}
			\item eine \textbf{obere Schranke} von $T$, wenn für alle $t\in T$ gilt $t\leq x$.
			\item eine \textbf{untere Schranke} von $T$, wenn für alle $t\in T$ gilt $x\leq t$.
		\end{enumerate}
	Hat $T$ eine obere (bzw. untere) Schranke, so heißt $T$ \textbf{nach oben (bzw. unten) beschränkt}.
\end{de}

\begin{bsp}\label{SchrankenZ}
	Die Teilmenge $\Nz_0\subset\Zz$ ist bezüglich der $\leq$-Relation nach unten, aber nicht nach oben beschränkt.
\end{bsp}
\begin{bew}
Wir behaupten (etwas verschärft) zwei Dinge: 
	\begin{enumerate}[(i)]
		\item Die Null ist eine untere Schranke von $\Nz_0$.
		\item Es gibt keine obere Schranke von $\Nz_0$.
	\end{enumerate}
	
	Zu (i): Sei $n\in\Nz_0$. Dann gilt $n=0+n$ und wegen $n\in\Nz_0$ ist die Definition von $0\leq n$ damit erfüllt.
	
	Zu (ii): Angenommen, $k\in\Zz$ wäre eine obere Schranke von $\Nz_0$. Insbesondere gälte dann $0\leq k$, also wäre $k\in\Nz_0$ eine natürliche Zahl. Dann wäre aber auch $k+1\in\Nz_0$. Da wir $k$ aber als obere Schranke von $\Nz_0$ angenommen haben, müsste dann $k+1\leq k$ gelten, was nicht sein kann. \qed
\end{bew}

\begin{bsp}
	Ist $M$ eine Menge, so ist in der geordneten Menge $(\mathcal{P}(M),\subseteq)$ jede Teilmenge nach oben und nach unten beschränkt.
\end{bsp}
\begin{bew}
 Für alle $A\in\cP(M)$ gilt $\emptyset\subseteq A$ und $A\subset M$. Ist $T\subset\cP(M)$ eine Teilmenge, so gilt dies natürlich immer noch auch für alle $A\in T$. Für jedes $T\subset\cP(M)$ ist also $\emptyset$ eine untere und $M$ eine obere Schranke. \qed
\end{bew}

\begin{bem}[mehrere Schranken]
	Eben in \cref{SchrankenZ} hatten wir bewiesen, dass die Null eine untere Schranke für $\Nz_0$ ist. Genauso gut hätten wir aber auch $-1$ (oder jede andere negative ganze Zahl) nehmen können, und der Beweis hätte nur leicht modifiziert werden müssen. Die Null ist jedoch eine natürlichere Wahl in dem Sinne, dass sie $\Nz_0$ am engst möglichen eingrenzt. Solche Schranken erhalten einen besonderen Namen:
\end{bem}

\begin{de}[Supremum/Infimum]
	Sei $X$ eine halbgeordenete Menge und $T\subset X$ eine Teilmenge.
		\begin{itemize}
			\item Ein Element $x\in X$ heißt \textbf{Supremum} von $T$, wenn $x$ eine kleinste obere Schranke von $T$ ist. Das heißt, $x$ ist eine obere Schranke von $T$ und für jede weitere obere Schranke $y$ von $T$ gilt $x\leq y$.
			\item Ein Element $x\in X$ heißt \textbf{Infimum} von $T$, wenn $x$ eine größte untere Schranke von $T$ ist. Das heißt, $x$ ist eine untere Schranke von $T$ und für jede weitere untere Schranke $y$ von $T$ gilt $y\leq x$.
		\end{itemize}
\end{de}

\begin{bsp}
	Die Null ist das Infimum von $\Nz_0$, denn: Wir haben bereits gesehen, dass Null eine untere Schranke ist. Ist $n\in\Zz$ eine weitere untere Schranke von $\Nz_0$, so gilt definitionsgemäß $n\leq 0$, denn $0\in\Nz_0$. Damit ist alles gezeigt. Unser Infimum war hier selbst Element der zu beschränkenden Menge, das ist jedoch nicht immer so (vgl. gleich \cref{verschiedeneSuprema}).
\end{bsp}

\begin{figure}[h]
	\begin{tikzpicture}[commutative diagrams/every diagram]
		\foreach \x in {-2,-1,0,1,2,3}
			{\node (\x) at (\x*2,0) {$\x$} ;};
		\node (-3) at (-6,0) {$\dots$};
		\node (4) at (8,0) {$\dots$};
		\path[commutative diagrams/.cd, every arrow, every label]
		\foreach \x in {1,2,3,4}
			{(0) edge[out=30, in=150,blue] (\x)}
		(0) edge[commutative diagrams/loop above,blue] (0)
		\foreach \x in {-3,-2,-1}
			{(\x) edge[out=30,in=159,green] (0) };
		\draw[help lines] (4,0) ellipse (4.8cm and 1.5cm);
		\node at (3.8,-.8) {$\Nz_0$};
	\end{tikzpicture}
	\centering \caption{Die Null ist das Infimum von $\Nz_0$ in $\Zz$ bezüglich \glqq$\leq$\grqq\,: Null ist eine untere Schranke (blau) und jede weitere untere Schranke ist kleiner (grün).}
\end{figure}

Wir können tatsächlich von \textit{dem} Infimum sprechen:

\begin{sat}[Eindeutigkeitssatz für Infimuma und Suprema]
	Sei $X$ eine halbgeordnete Menge und $T\subset X$ eine Teilmenge. Existiert dann ein Supremum oder Infimum von $T$, so sind diese jeweils eindeutig bestimmt.
\end{sat}

\begin{bew}
	Angenommen, $x$ und $y$ seien beides Infima von $T$. Dann ist $x$ eine untere Schranke von $T$, und da $y$ eine größte untere Schranke von $T$ ist, gilt $x\leq y$. Umgekehrt ist aber auch $y$ eine untere Schranke von $T$, und da $x$ ebenfalls eine größte untere Schranke ist, gilt $y\leq x$. Aus der Antisymmetrie folgt $y=x$.
	
	Der Beweis für Suprema funktioniert sehr ähnlich und ist daher Übungsaufgabe. \qed
\end{bew}

\begin{bem}\label{verschiedeneSuprema}
	Damit ist nur gesagt, dass falls ein Supremum oder Infimum existiert, es eindeutig ist; die Existenz allein ist jedoch nicht unbedingt gesichert. Wenn eine Teilmenge gar keine obere Schranke besitzt, so besitzt sie erst recht auch kein Supremum. Beispielsweise ist die Teilmenge $\Nz_0 \subseteq \Zz$ nach oben unbeschränkt und besitzt daher kein Supremum. \\[0.5em]
Aber selbst, wenn eine Teilmenge nach oben beschränkt ist, braucht sie kein Supremum besitzen. Ein Beispiel in der geordneten Menge $(\Qz,\leq)$ ist die Teilmenge $T:=\{x\in \Qz \mid x^2 < 2\}$. Dann ist $T$ sowohl nach oben als auch nach unten beschränkt in $\Qz$, besitzt aber dennoch weder ein Supremum noch ein Infimum in $\Qz$.
\end{bem}
\begin{bem}[*]
Ein weiteres Beispiel einer beschränkten Teilmenge, die kein Supremum besitzt, liefert die folgende Ordnungsrelation, die wir durch ein \href{https://de.wikipedia.org/wiki/Hasse-Diagramm}{Hasse-Diagramm} darstellen:
\[ \begin{tikzcd}
    \bullet \ar[rrdd, dash] && \bullet \ar[lldd, dash, crossing over]\\
    && \\
    \bullet \ar[uu, dash]  && \bullet \ar[uu, dash] 
   \end{tikzcd} \]
   Dabei sollen die Punkte die Elemente der geordneten Menge repräsentieren und ein Punkt soll genau dann kleiner als ein anderer Punkt sein, wenn er am unteren Ende eines Verbindungsstrichs zum anderen Punkt liegt. \\
   Man sieht, dass die Menge der unteren beiden Punkte beschränkt ist und dass die beiden oberen Punkte jeweils Schranken darstellen. Allerdings ist keine dieser beiden Schranken ein Supremum. \\[0.5em]
   Konkret können wir diese  Ordnungsrelation z.B. folgendermaßen realisieren. Seien $a,b,c,d$ vier paarweise verschiedene Objekte und $X=\left\{\{a\},\{b\},\{a,b,c\},\{a,b,d\}\right\}$. Dann ist $(X,\subseteq)$ eine geordnete Menge, die genau so aussieht, wie das abstrakte Bild von gerade eben. Die Teilmenge $T=\left\{\{a\},\{b\}\right\}$ hat genau die oberen Schranken $\{a,b,c\}$ und $\{a,b,d\}$. Aber keine dieser Schranken ist ein Supremum.
	\begin{figure}[H]
    \begin{tikzpicture}[commutative diagrams/every diagram]
				\node (a) at (-2,0) {$\{a\}$};
				\node (b) at (2,0) {$\{b\}$};
				\node (abc) at (-2.3,2) {$\{a,b,c\}$};
				\node (abd) at (2.3,2) {$\{a,b,d\}$};
				\path[commutative diagrams/.cd, every arrow, every label]
				(a) edge[commutative diagrams/loop left] (a)
				(b) edge[commutative diagrams/loop right] (b)
				(abc) edge[commutative diagrams/loop left] (abc)
				(abd) edge[commutative diagrams/loop right] (abd)
				(a) edge (abc)
				(a) edge (abd)
				(b) edge (abc)
				(b) edge (abd)
				;
				\draw[help lines] (0,0) ellipse (4cm and 1cm);
				\node at (3.8,-.8) {$T$};
		\end{tikzpicture}
		\centering \caption{Die beiden oberen Schranken von $T$ sind nicht vergleichbar mittels der Inklusionsrelation, also gibt es keine kleinste obere Schranke und $T$ hat kein Supremum.}
	\end{figure}
Durch künstliches Hinzufügen eines weiteren Punkts ist es aber möglich, ein Supremum zu bekommen. Wir wollen das hier nicht im Detail ausführen, sondern nur mit einem Bild visualisieren:
	\begin{figure}[H]
$\begin{tikzcd}
    \bullet && \bullet \\
    & \color{red}{\bullet} \ar[lu, dash] \ar[ru, dash] & \\
    \bullet \ar[uu, dash] \ar[ru, dash] && \bullet \ar[uu, dash] \ar[lu, dash]
   \end{tikzcd}$ $\begin{tikzpicture}[commutative diagrams/every diagram]
			\node (a) at (-2,0) {$\{a\}$};
			\node (b) at (2,0) {$\{b\}$};
			\node (ab) at (0,1.2) {$\{a,b\}$};
			\node (abc) at (-2.3,2) {$\{a,b,c\}$};
			\node (abd) at (2.3,2) {$\{a,b,d\}$};
			\path[commutative diagrams/.cd, every arrow, every label]
			(a) edge (abc)
			(a) edge[out=150,in=150, looseness=2] (abd)
			(b) edge[out=30,in=30, looseness=2] (abc)
			(b) edge (abd)
			(a) edge (ab)
			(b) edge (ab)
			(ab) edge (abc)
			(ab) edge (abd)
			;
			\draw[help lines] (0,0) ellipse (4cm and .9cm);
			\node at (3.8,-.8) {$T$};
		\end{tikzpicture}$
    	\centering \caption{Plötzlich besitzt die Menge $T$ ein Supremum. (Im rechten Diagramm wurden die Reflexivitätspfeile aus Platzgründen weggelassen. Im Hasse-Diagramm auf der linken Seite werden auch diejenigen Verbindungslinien, die sich aus der Transitivität und den anderen Linien ergeben, weggelassen.)}
	\end{figure}
	% \\[0.5em]
%	Wir könnten die Menge $X$ jedoch erweitern, sodass es ein Supremum gibt. Fügen wir $\{a\}\cup\{b\}=\{a,b\}$ der Menge $X$ hinzu, so ist dieses Element das Supremum der Menge $T$ in $X$, denn: Sowohl $\{a\}\subseteq\{a,b\}$ und $\{b\}\subseteq\{a,b\}$, also ist $\{a,b\}$ eine obere Schranke von $T$. Ist $A$ eine weitere obere Schranke von $T$, so muss $\{a\}\subseteq A$ sowie $\{b\}\subseteq A$ gelten. Daraus folgt aber $a\in A$ und $b\in A$, woraus wiederum $\{a,b\}\subseteq A$ folgt. Also ist $\{a,b\}$ die kleinste obere Schranke von $T=\left\{\{a\},\{b\}\right\}$.
\begin{comment}
	\begin{figure}[h]
		\begin{tikzpicture}[commutative diagrams/every diagram]
			\node (a) at (-2,0) {$\{a\}$};
			\node (b) at (2,0) {$\{b\}$};
			\node (ab) at (0,1.2) {$\{a,b\}$};
			\node (abc) at (-2.3,2) {$\{a,b,c\}$};
			\node (abd) at (2.3,2) {$\{a,b,d\}$};
			\path[commutative diagrams/.cd, every arrow, every label]
			(a) edge (abc)
			(a) edge[out=150,in=150, looseness=2] (abd)
			(b) edge[out=30,in=30, looseness=2] (abc)
			(b) edge (abd)
			(a) edge (ab)
			(b) edge (ab)
			(ab) edge (abc)
			(ab) edge (abd)
			;
			\draw[help lines] (0,0) ellipse (4cm and .9cm);
			\node at (3.8,-.8) {$T$};
		\end{tikzpicture}
		\centering \caption{Plötzlich gibt es ein Supremum. (Die Reflexivitätspfeile wurden aus Platzgründen weggelassen.)}
	\end{figure}
\end{comment}
\end{bem}

% --------------------------------------------------------------------
% §1.3 section <<Äquivalenzrelationen>>
% --------------------------------------------------------------------
\section{Äquivalenzrelationen}
Sei $X$ eine Menge. Äquivalenzrelationen auf $X$ sind die Relationen, anhand derer wir die Objekte von $X$ in gewisse \glqq Klassen\grqq\, einteilen können.

% Definition Aequivalenzrelation
\begin{de}
	Sei $X$ eine Menge. Eine \textbf{Äquivalenzrelation} auf $X$ ist eine Relation $\sim$, die reflexiv, symmetrisch und transitiv ist. Konkret:
\begin{align*}
 & (\text{ÄR1}) & \forall x\in X:&\quad x\sim x && (\text{Reflexivität}) \\
& (\text{ÄR2}) & \forall x,y\in X:&\quad  x\sim y\quad \implies\quad y\sim x && (\text{Symmetrie}) \\
& (\text{ÄR3}) & \forall x,y,z\in X:&\quad x\sim y\ \wedge\ y\sim z \quad\implies\quad x\sim z && (\text{Transitivität})
 \end{align*}
\end{de}

\begin{bsp}[Wohngemeinschaften]
	Auf der Menge aller Menschen in Deutschland betrachten wir die Relation \glqq \_ wohnt zusammen mit \_\grqq\,. Diese ist reflexiv, denn jede Person wohnt mit sich selbst zusammen (so komisch das klingt); sie ist symmetrisch, denn wenn Alice mit Bob zusammen wohnt, wohnt Bob auch mit Alice zusammen; und sie ist transitiv, denn wenn Alice mit Bob zusammen wohnt und Bob mit Charlie, so wohnt auch Alice mit Charlie zusammen (wobei wir vereinfachend annehmen, dass jede Person nur einen Wohnsitz hat). Es liegt also eine Äquivalenzrelation vor.
\end{bsp}

\begin{bsp}[Äquivalenz von Aussagen] \label{aussagenequiv}
Es sei $\mathcal{A}:= \{A\mid A\ \text{ist eine Aussage} \}$ die Menge aller Aussagen. Dann liefert die Äquivalenz von Aussagen
\begin{align*}
 A\sim B \quad:\Leftrightarrow\quad \text{Es lässt sich beweisen, dass $A\leftrightarrow B$ gilt} && A,B \in \mathcal{A}
\end{align*}
eine Äquivalenzrelation auf $\mathcal{A}$ (was ja auch schon dem Namen nach Sinn ergibt). Denn:
\begin{itemize}
 \item Die Reflexivität ist erfüllt nach \cref{AtotoA}.
 \item Die Symmetrie ergibt sich aus \cref{komm:toto}.
 \item Die Transitivität ist eine Konsequenz aus \cref{trans:toto}.
\end{itemize}
\end{bsp}


\begin{comment}
\begin{bsp}[Gleichheit]
 Auf jeder beliebigen Menge $X$ ist die Gleichheitsrelation „$=$“ eine Äquivalenzrelation, denn für alle $x,y,z\in X$ gilt:
 \begin{align*}
  x &= x \\
  x=y\quad&\Rightarrow\quad y=x \\
  x=y,y=z \quad&\Rightarrow\quad x=z
 \end{align*}
 Diese Aussagen lassen sich nicht wirklich beweisen, sondern sind unmittelbare Konsequenzen der Logik.
\end{bsp}
\end{comment}




\subsection{Äquivalenzklassen}
\begin{de}
	Seien $X$ eine Menge, $\sim$ eine Äquivalenzrelation und $x\in X$ ein Element in $X$. Dann nennen wir
		\[[ x ] := \{ y\in X \mid y\sim x \}\]
	die \textbf{Äquivalenzklasse von $x$}. Sie besteht aus allen Elementen, die zu $x$ „äquivalent“ sind. Ist $y\in[x]$ ein Element, so nennt man $y$ einen \textbf{Vertreter} oder auch \textbf{Repräsentanten} von $[x]$. \\[0.5em]
Da „$\sim$“ reflexiv ist, gilt $x\sim x$, also $x\in [x]$. Mit anderen Worten: Jedes Element ist ein Vertreter seiner Äquivalenzklasse.
\end{de}
%------------------
% Beispiel
\begin{bsp}
	In unserem Beispiel der \glqq\_ wohnt zusammen mit \_\grqq-Relation ist $[
	\text{Bob} ]$ die Menge aller Mitbewohner von Bob (und auch Bob). Wenn Alice mit Bob zusammenwohnt, so ist diese
	Menge aber identisch mit $[ \text{Alice} ]$, da dann Alice und Bob
	natürlich die selben Mitbewohner haben.
\end{bsp}




Beim nächsten Satz handelt es sich um eine Mehrfach-Äquivalenz, siehe \cref{tfaw}.
\begin{sat}[Gleichheit von Äquivalenzklassen]
	\label{lem:vertreter}
	Seien $X$ eine Menge, $\sim$ eine Äquivalenzrelation auf $X$ und $x,y\in
	X$. Dann sind die folgenden Aussagen äquivalent:
\begin{enumerate}[(i)]
 \item Es ist $[x]=[y]$, d.h. $x$ und $y$ bestimmen dieselbe Äquivalenzklasse.
 \item Es gilt $x\sim y$.
\item Es ist $x\in [y]$.
\item Es ist $[x]\cap [y]\neq \emptyset$, d.h. $[x]$ und $[y]$ haben mindestens ein gemeinsames Element.
\end{enumerate}
\end{sat}
\begin{bem}
Unsere Beweisstrategie besteht in der Etablierung der folgenden Beziehungen:
\[ \begin{tikzcd}
(i) \ar[d] & (ii) \ar[l]  \ar[dl, leftrightarrow]  \\
(iii) \ar[r] & (iv) \ar[u]
   \end{tikzcd} \]
   Daraus ergibt sich dann, dass alle vier Aussagen zueinander äquivalent sind, siehe \cref{ringschluss}.
   \end{bem}
\begin{bew}
Die Äquivalenz (ii)$\leftrightarrow$(iii) ergibt sich sofort aus der Definition von $[y]$. \\[0.5em]
(iii)$\to$(iv): Es gelte (iii), also $x\in [y]$. Da auch $x\in [x]$, ist insgesamt $x\in [x]\cap [y]$. Also gilt (iv). \\[0.5em]
(iv)$\to$(ii): Es gelte (iv), d.h. es gebe ein Element $a\in [x]\cap [y]$. Dann ist $a\sim x$ und $a\sim y$. Wegen der Symmetrie gilt dann auch $x\sim a$ und dank der Transitivität folgt aus $x\sim a$ und $a\sim y$, dass auch $x\sim y$ gilt. \\[0.5em]
(ii)$\to$(i): Es gelte $x\sim y$. Da „$\sim$“ eine symmetrische Relation ist, gilt dann auch $y\sim x$. Wir zeigen nun die Gleichheit von Mengen $[x]=[y]$: \\
„$\subseteq$“: Sei $z\in [x]$ ein beliebiges Element. Dann ist $z\sim x$ und zusammen mit $x\sim y$ folgt aus der Transitivität, dass auch $z\sim y$. Also ist $z\in [y]$. \\
„$\supseteq$“: Sei $z\in [y]$ ein beliebiges Element. Dann ist $z\sim y$ und zusammen mit $y\sim x$ folgt aus der Transitivität, dass auch $z\sim x$. Also ist $z\in [x]$. \\[0.5em]
(i)$\to$(iii): Es gelte (i), also $[x]=[y]$. Wegen $x\in [x]$ ist dann auch $x\in [y]$. \qed
\end{bew}



\begin{comment}
%------------------
% Bemerkung
\begin{lem}
	\label{bem:disjunkt}
	
	Seien $X$ eine Menge, $\sim$ eine Äquivalenzrelation auf $X$ und $x,y\in
	X$. Dann gilt genau eine der folgenden Aussagen:
	\begin{resListeN}
		\item $[x] = [y]$
		\item $[x] \cap [y] = \emptyset$
	\end{resListeN}
\end{lem}

%------------------
% Beweis
\begin{bew}
Wegen $[x]\neq \emptyset$ können schonmal nicht beide Aussagen gleichzeitig eintreten. Daher müssen wir nur noch beweisen dass mindestens eine Aussage gilt, wozu wir die Beweistechnik aus \cref{oderperimplikation} verwenden. Es sei also angenommen, dass (ii) falsch ist. 
	Wir unterscheiden die Fälle $x\in [y]$ und $x\notin [y]$.
	\begin{resListeN}
		
		\item[\tiny{$\big(x\in {[y]}\big)$}] Gelte $x\in [y]$, dann folgt wegen
		\cref{lem:vertreter} $[x]=[y]$. Also $x\in [x] \cap [y]
		\neq \emptyset$.
		
		\item[\tiny{$\big(x\notin {[y]}\big)$}] Gelte $x\notin [y]$, also
		$[x]\neq [y]$. Angenommen es existiert ein $z\in [x]\cap [y]$,
		d.h. $z \sim x$ und $z \sim y$. Da $\sim$ symmetrisch ist, folgt
		$x\sim z$ und wegen der Transitivität von $\sim$ auch $x\sim y$. Dann
		gilt aber $x\in [y]$, ein Widerspruch. Die Annahme war falsch und es
		gilt $[x] \cap [y] =\emptyset$.
		
	\end{resListeN}
	\bewEnd
\end{bew}
\end{comment}


\begin{bsp}
	Oft sind die Äquivalenzklassen interessanter als die ursprünglichen Objekte der Menge $X$: Auf der Menge der Laptops ist \glqq\_ ist das gleiche Modell wie \_\grqq\, eine Äquivalenzrelation (siehe \cref{laptopaufg}). Eine Äquivalenzklasse besteht dann aus allen Laptops desselben Modells. Wer sich einen neuen Laptop besorgen möchte, geht aber nicht in den Laden und lässt sich jeden Computer einzeln vorführen (also: betrachtet jedes Objekt), sondern entscheidet sich nur für ein Modell (also: sucht eine Äquivalenzklasse aus) und welches konkrete Objekt in der Äquivalenzklasse schließlich mitgenommen wird, ist relativ egal.
	
	Tatsächlich ist üblicherweise zu jedem Modell nur ein Vorführgerät (also: ein Vertreter der Äquivalenzklasse) ausgestellt, denn die Unterschiede, die die Laptops desselben Modells aufweisen, sind für den Käufer irrelevant.
	
	Diese Idee findet Ausdruck in der nächsten Definition.
\end{bsp}





\subsection{Die Faktormenge modulo einer Äquivalenzrelation}
\begin{de}[Faktormenge]
	Seien $X$ eine Menge und $\sim$ eine Äquivalenzrelation auf $X$. Die Menge aller Äquivalenzklassen bezüglich $\sim$ heißt die \textbf{Faktormenge $X$ modulo $\sim$} und wird mit $\sfrac{X}{\sim}$ notiert:
	\begin{align*}
		\sfrac{X}{\sim} & := \big\{ [x] \mid x\in X \big\} 
	\end{align*}
\end{de}

\begin{bsp}
	In dem Laptop-Beispiel kann die Faktormenge $\sfrac{X}{\sim}$ als Menge aller Laptop-Modelle aufgefasst werden. Sie ist für den Käufer sehr viel interessanter als die Menge $X$ aller Laptops.
\end{bsp}

\begin{bsp}
	Bezüglich der \glqq\_ wohnt zusammen mit \_\grqq-Relation besteht eine Äquivalenzklasse aus allen Mitgliedern einer Wohngemeinschaft. Die Faktormenge $\sfrac{X}{\sim}$ kann dann als die Menge aller Haushalte in Deutschland verstanden werden.
\end{bsp}






\begin{de}[Kanonische Projektion]
	Seien $X$ eine Menge und $\sim$ eine Äquivalenzrelation auf $X$. Die Abbildung
	\[ \pi: X \to \sfrac{X}{\sim}, x \mapsto [x] \,,\]
die jedem Element von $X$ seine Äquivalenzklasse zuordnet, heißt die \textbf{(kanonische) Projektion von $X$ auf $\sfrac{X}{\sim}$}.
\end{de}


\begin{bem}
Seien $X$ eine Menge und $\sim$ eine Äquivalenzrelation auf $X$. Dann ist die kanonische Projektion $\pi : X \to  \sfrac{X}{\sim}$ eine surjektive Abbildung. 
\end{bem}
\begin{bew}
Jedes Element von $\sfrac{X}{\sim}$ ist von der Gestalt $[x]$ für irgendein $x\in X$. Und für dieses Element gilt dann $\pi(x)=[x]$, also ist $x$ ein Urbild von $[x]$. \qed
\end{bew}

\begin{bsp}
	In dem Beispiel mit den Laptops ordnet $\pi$ jedem Gerät sein Modell zu. In dem Beispiel der Wohnungen ordnet $\pi$ jedem Menschen seinen Haushalt zu. Dass $\pi$ surjektiv ist, spiegelt den intuitiven Fakt wieder, dass es zu jedem Modell mindesten ein Gerät dieses Modells gibt und an jedem Haushalt mindestens eine Person teilhat.
\end{bsp}




\begin{bem}[Philosophie hinter der Faktormenge] \label{simphilo}
 Der Übergang von $X$ zur Faktormenge $\sfrac{X}{\sim}$ beinhaltet einen Abstraktionsprozess, der dazu dient, gewisse Details über die Elemente von $X$ zu „vergessen“ und nur noch „modulo die Relation $\sim$ zu rechnen“. Im Laptop-Beispiel vergessen wir die individuellen Eigenschaften der einzelnen Laptops und „identifizieren“ solche Laptops, die demselben Modell angehören. Und im WG-Beispiel „identifizieren“ wir diejenigen Leute, die in derselben Wohnung leben. \\
 Im Rahmen von $\sfrac{X}{\sim}$ führt diese „Identifikation“ tatsächlich zu einer Gleichheit im streng mathematischen Sinn. Denn nach \cref{lem:vertreter} gilt für solche Individuen $x,y\in X$ mit $x\sim y$, dass $[x]=[y]$. Unter der kanonischen Projektion $x\mapsto [x]$ werden also „äquivalente“ Elemente künstlich \emph{identisch} „gemacht“, sodass sie in $\sfrac{X}{\sim}$ nicht mehr unterschieden werden. \\
 Während die Elemente von $\sfrac{X}{\sim}$ zwar ihrer formalen Definition nach gewisse Teilmengen von $X$ sind, können sie ihrer „Philosophie“ nach oft wiederum als eigenständige Objekte aufgefasst werden oder als „Elemente von $X$, die aber weniger streng unterschieden werden als die echten Elemente von $X$“.
\end{bem}




\begin{sat}
	Seien $X$ eine Menge und $\sim$ eine Äquivalenzrelation auf $X$. Dann gilt:
\begin{enumerate}[a)]
 \item Es ist $X = \bigcup_{x\in X} [x]$.
 \item Für alle $x,y \in X$ mit $[x]\neq [y]$ gilt $[x] \cap [y] = \emptyset$.
\end{enumerate}
Mit anderen Worten: Die Menge $X$ „zerfällt“ in paarweise disjunkte Äquivalenzklassen.
\end{sat}



%------------------
% Beweis
\begin{bew}
a): Wir müssen eine Gleichheit von Mengen beweisen: \\
„$\subseteq$“: Sei $a\in X$ ein beliebiges Element. Dann ist $a\in [a]$ und somit erst recht auch in $a\in\bigcup_{x\in X} [x]$. \\
„$\supseteq$“ Für jedes $x\in X$ ist $[x]$ eine Teilmenge von $X$. Also ist auch $\bigcup_{x\in X} [x]$ eine Teilmenge von $X$. \\[0.5em]
b) ergibt sich sofort aus der Äquivalenz (i)$\leftrightarrow$(iv) in \cref{lem:vertreter}. \qed
\end{bew}

\begin{figure}[h]
	\centering
\includegraphics[width=12cm]{./_img/equivalence.jpeg}
	\caption{Die Menge $X$ zerfällt in disjunkte Äquivalenzklassen.}
\end{figure}


% --------------------------------------------------------------------
% §2 Section <<Aufgabenvorschlaege>>
% --------------------------------------------------------------------


\clearpage
\section{Aufgabenvorschläge}





\begin{aufg}[Konkrete Relationen]
Untersuche die folgenden Relationen darauf, ob sie reflexiv, symmetrisch, antisymmetrisch oder transitiv sind. Welche Relationen sind Äquivalenzrelationen, welche Halbordnungen, welche Totalordnungen? Sofern eine Äquivalenzrelation vorliegt, beschreibe die Äquivalenzklassen.
\begin{enumerate}[a)]
\item Es bezeichne $S$ die Menge aller Städte auf der Erde. Betrachte darauf die Relation
\begin{align*}
 A \sim B \quad& :\Leftrightarrow\quad \text{$A$ und $B$ sind höchstens 100km voneinander entfernt} && A,B\in S
\end{align*}
\item Auf der Menge $\Rz$ die Ungleichheitsrelation „$\neq$“.
\item Auf der Menge $\Rz$ die übliche „$<$“-Relation.
\item Es sei $M$ die Menge aller Menschen. Betrachte darauf die Relation
\begin{align*}
 a \sim b \quad& :\Leftrightarrow\quad \text{$a$ und $b$ haben (bzw. hatten) ein gemeinsames Kind} && a,b\in M
\end{align*}
\item Auf der Menge $\Zz$ die Teilbarkeitsrelation
\begin{align*}
 m\mid n \quad& :\Leftrightarrow\quad \exists c\in \Zz:\  c\cdot m=n && m,n \in \Zz
\end{align*}
\end{enumerate}
\end{aufg}


 
 
 
 
 \begin{aufg}[Abstrakte Relationen] \label{abstraktrel}
Untersuche die folgenden Relationen darauf, ob sie reflexiv, symmetrisch, antisymmetrisch oder transitiv sind. Welche Relationen sind Äquivalenzrelationen, welche Halbordnungen, welche Totalordnungen? Sofern eine Äquivalenzrelation vorliegt, beschreibe die Äquivalenzklassen.
\begin{enumerate}[a)]
	\item Auf einer beliebigen Menge $X$ die Gleichheitsrelation „$=$“.
	\item Für zwei Mengen $X,Y$ und eine beliebige Abbildung $f:X\to Y$ die Relation
	\begin{align*}
	 a\sim b \quad &:\Leftrightarrow\quad f(a)=f(b) && a,b\in X
	\end{align*}
          \item Auf einer beliebigen Menge $X$ die „Allrelation“, bezüglich der für alle $x,y\in X$ gilt, dass $x\sim y$. Verkörpert wird diese Relation durch die Teilmenge $X\times X\subseteq X\times X$.
        \item Auf einer beliebigen Menge $X$ die „leere Relation“, bezüglich der gar keine Elemente von $X$ zueinander in Beziehung stehen und die durch die Teilmenge $\emptyset\subseteq X\times X$ verkörpert wird.
\end{enumerate}
 \end{aufg}

 


\begin{aufg}[Graphen]
Bestimme, ob die Relationen, die durch die folgenden Graphen repräsentiert werden, reflexiv, symmetrisch, antisymmetrisch und/oder transitiv sind. Liegt sogar eine Äquivalenzrelation oder eine Ordnungsrelation vor?
\begin{figure}[H]
	\begin{tikzpicture}
		\begin{scope}[xshift=-6cm]
			\node (P0) at (-1,1) {$a$};
			\node (P1) at (-1,-1) {$b$};
			\node (P2) at (1,-1) {$c$};
			\node (P3) at (1,1) {$d$};
			\path[commutative diagrams/.cd, every arrow, every label]
			(P3) edge[commutative diagrams/loop] (P3)
			(P2) edge (P0)
			(P0) edge (P3)
			(P1) edge[<->] (P2);
		\end{scope}
		\begin{scope}[xshift=-2cm]
			\node (P0) at (-1,1) {$a$};
			\node (P1) at (-1,-1) {$b$};
			\node (P2) at (1,-1) {$c$};
			\node (P3) at (1,1) {$d$};
			\path[commutative diagrams/.cd, every arrow, every label]
			(P0) edge (P2)
			(P1) edge (P2)
			(P0) edge[<->] (P1);
		\end{scope}
		\begin{scope}[xshift=2cm]
			\node (P0) at (-1,1) {$a$};
			\node (P1) at (-1,-1) {$b$};
			\node (P2) at (1,-1) {$c$};
			\node (P3) at (1,1) {$d$};
			\path[commutative diagrams/.cd, every arrow, every label]
			(P1) edge[commutative diagrams/loop] (P1)
			(P2) edge[commutative diagrams/loop] (P2)
			(P3) edge[commutative diagrams/loop] (P3)
			(P0) edge[commutative diagrams/loop] (P0)
			(P1) edge[<->] (P2);
		\end{scope}
		\begin{scope}[xshift=6cm]
			\node (P0) at (-1,1) {$a$};
			\node (P1) at (-1,-1) {$b$};
			\node (P2) at (1,-1) {$c$};
			\node (P3) at (1,1) {$d$};
			\path[commutative diagrams/.cd, every arrow, every label]
			(P0) edge (P1)
			(P0) edge (P2)
			(P0) edge (P3)
			(P1) edge (P2)
			(P3) edge (P2);
		\end{scope}
	\end{tikzpicture}
\end{figure}
 \end{aufg}
 




\begin{aufg}[Äquivalenzklassen]
 Betrachte die beiden Abbildungen
 \begin{align*}
   f : \Rz^2 \to \Rz \ &,\ (x,y) \mapsto x+y \\
   g : \Rz^2 \to \Rz \ &,\ (x,y) \mapsto x^2+y^2
 \end{align*}
 Nach \cref{abstraktrel}b) sind durch
 	\begin{align*}
	 p\sim_f q \quad &:\Leftrightarrow\quad f(p)=f(q) && p,q\in \Rz^2 \\
	 p\sim_g q \quad &:\Leftrightarrow\quad g(p)=g(q) && p,q\in \Rz^2
	\end{align*}
zwei Äquivalenzrelationen $\sim_f$, $\sim_g$ auf $\Rz^2$ gegeben. Visualisiere die Äquivalenzklassen jeweils durch eine Zeichnung.
\end{aufg}



 \begin{aufg}[Laptops] \label{laptopaufg}
Zeige, dass auf der Menge $X$ aller Laptops die Relation \glqq\_ ist das gleiche Modell wie \_ \grqq eine Äquivalenzrelation ist. Verdeutliche Dir anhand des Beispiels noch einmal den Unterschied zwischen $X$ und $\sfrac{X}{\sim}$.
\end{aufg}

 
 \begin{comment}
\begin{aufg}[Parallelität]
Eine \textit{Gerade} $g$ in der Ebene $\Rz^2$ ist eine Menge der Form
\[ g= \left\{ x+ty \mid t\in\Rz \right\} \,, \]
wobei $x\in\Rz^2$ ein \textit{Ortsvektor} und $y\in\Rz^2$ ein \textit{Richtungsvektor} der Geraden genannt werden.

Auf der Menge aller Geraden in der Ebene $\Rz^2$ betrachten wir die Relation $g\parallel h :\Leftrightarrow$ $g$ ist parallel zu $h$, oder genauer: ein möglicher Richtungsvektor von $g$ ist ein Vielfaches eines möglichen Richtungsvektors von $h$. Ist etwa
\begin{align*}
	g&=\left\{x+ty\mid t\in\Rz \right\} & y&=\left\{u+tv\mid t\in\Rz\right\}
\end{align*}
für gewisse Vektoren $x,y,u,v\in\Rz^2$, so kannst Du annehmen
\[ g\parallel h \Leftrightarrow \exists r\in\Rz: y=rv \,. \]

Zeige, dass Parallelität auf der Menge aller Geraden in $\Rz^2$ eine Äquivalenzrelation bildet. Was sind die Äquivalenzklassen?
\end{aufg}
\end{comment}


%%% Local Variables:
%%% mode: latex
%%% TeX-master: "Skript"
%%% End:


% ====================================================================
% 6. Vortrag <<Folgen>>
% ====================================================================
\setchapterpreamble[c][.7\textwidth]{%
  \itshape\dgrau\small In diesem Vortrag werden einige Eigenschaften und Beispiele für Folgen vorgestellt. In Form von Abstand und Folgenkonvergenz werden grundlegende Begriffe der Analysis thematisiert.
  \vspace{24pt}%
}
\chapter{Folgen, Abstand und Grenzwerte}
% ====================================================================
%                                                                 
%
%       §06
%
%
% %%ts latex start%%[2019-03-07 Thu 14:45]%%ts latex end%%
% ====================================================================
% --------------------------------------------------------------------
% §1 Section <<Motivation>>
% --------------------------------------------------------------------



\begin{bem}[das Zeichen „$\Nz$“]
 In diesem Kapitel wird die Menge der natürlichen Zahlen, sofern die Null eingeschlossen ist, mit „$\Nz_0$“ bezeichnet, sofern die Null ausgeschlossen ist mit „$\Nz_{\geq 1}$“. In Situationen, in denen es keine Rolle spielt, ob die Null nun dabei ist oder nicht, schreiben wir einfach nur „$\Nz$“.
\end{bem}



\section{Zahlenfolgen}

\begin{de}[Folge] \label{folge}
Eine \textbf{Folge} ist eine Familie von Objekten $(a_n)_{n\in \Nz}$, deren Indexmenge die Menge $\Nz$ der natürlichen Zahlen ist. Ist $n\in \Zz$ eine beliebige ganze Zahl, so spricht man auch bei Familien, deren Indexmenge $\Zz_{\geq n}$ ist, von Folgen. In diesem Fall starten die Indizes nicht bei Eins oder Null, sondern bei $n$. Der Begriff der Folge ist also nicht klar umrissen, spricht man aber von „der beliebigen Folge $(a_n)$“ so ist damit gemeint, dass die Indexmenge gleich $\Nz$ (mit oder ohne Null) ist. \\
Sind $A$ eine Menge und $(a_n)_{n\in \Nz}$ eine Folge, deren Einträge allesamt in $A$ liegen, so spricht man von einer \emph{Folge von Elementen aus $A$} oder einer „$A$-wertigen Folge“. Nach \cref{familie} bezeichnet
\[ A^\Nz := \{ (a_n)_{n\in \Nz} \mid \forall n\in \Nz: a_n\in A \} \]
die Menge aller Folgen mit Einträgen aus $A$. Im Spezialfall $A=\Rz$ spricht man von \textbf{reellen Zahlenfolgen} oder auch \textbf{reellwertigen Folgen}. Analog spricht man von rationalen Zahlenfolgen, Folgen ganzer Zahlen oder Folgen komplexer Zahlen, falls es sich um Elemente von $\Qz^\Nz$, $\Zz^\Nz$ oder $\Cz^\Nz$ handelt. \\
Folgen lassen sich sowohl durch eine exakte Angabe definieren, etwa
\begin{align*}
  (n^2)_{n\in \Nz_0}
\end{align*}
als auch durch eine suggestive Aufzählung der ersten paar Folgenglieder, etwa
\[ 0,\quad 1,\quad 4,\quad 9,\quad 16,\quad 25,\quad\dots \]
Die „Definition durch Auflistung“ ist allerdings nicht mathematisch präzise und sollte nur dann benutzt werden, wenn wirklich unmissverständlich klar ist, welche Folge gemeint ist. Würdest du etwa erahnen, dass mit
\[ 0,\quad 2,\quad 12,\quad 36,\quad 80,\quad 150,\quad 252,\quad 392,\quad \dots\]
die Zahlenfolge $(n^2\cdot (n+1))_{n\in \Nz_0}$ gemeint sein soll?
\end{de}



\begin{bem}[Folge vs. Menge ihrer Einträge]
 Eine Folge $(a_n)_{n\in \Nz}$ ist nicht mit der Menge ihrer Einträge $\{a_n\mid n\in \Nz\}$ zu verwechseln.\footnote{vgl. \cref{mengeeinerfamilie}} Beispielsweise sind die beiden Zahlenfolgen
 \begin{align*}
  0,\quad 1,\quad 0,\quad 1,\quad 0,\quad \dots \\
  1,\quad 0,\quad 0,\quad 0,\quad 0,\quad \dots 
 \end{align*}
 voneinander verschieden, während die Mengen ihrer Einträge übereinstimmen und gleich der Menge $\{0,1\}$ sind.
\end{bem}



\begin{bsp}
 Beispiele für reelle Zahlenfolgen sind:
 \begin{enumerate}[a)]
  \item Die Folge der Primzahlen $2,3,5,7,11,\dots$. Also die Folge $(a_n)_{n\in \Nz_{\geq 1}}$ wobei für jedes $n\in \Nz_{\geq 1}$ mit $a_n$ die $n$-te Primzahl gemeint ist.
  \item Die Folge der natürlichen Zahlen $(n)_{n\in \Nz_0}$: $0,1,2,3,4,\dots$.
  \item Die Folge $0,1,-1,2,-2,3,-3,4,\dots$. Sie lässt sich definieren als diejenige Folge $(a_n)_{n\in \Nz_0}\in \Nz_0^{\Nz_0}$ mit den Einträgen
  \begin{align*}
  a_n := \begin{cases}
        -\frac{n}{2} & n\ \text{ist eine gerade Zahl} \\
        \frac{n+1}{2} & n\ \text{ist eine ungerade Zahl}
           \end{cases} && n \in \Nz_0
\end{align*}
  \item Die „alternierende Folge“ $((-1)^n)_{n\in \Nz_0}$. Also $1,-1,1,-1,1,-1,1,\dots$.
  \item Die Folge der Kehrwerte natürlicher Zahlen $(1/n)_{n\in \Nz_{\geq 1}}$. Das ist  $1,\frac{1}{2},\frac{1}{3},\frac{1}{4},\frac{1}{5},\dots$. Beachte, dass bei dieser Folge die Indizes erst bei Eins losgehen.
  \item Die Folge $\left(\frac{n}{n+1}\right)_{n\in \Nz_0}$. Also $0,\frac{1}{2},\frac{2}{3},\frac{3}{4},\frac{4}{5},\dots$.
  \item Die „konstante Folge“ $(3)_{n\in \Nz}$. Also: $3,3,3,3,3,\dots$.
  \item Für $q\in \Rz$ die Folge der $q$-Potenzen $(q^n)_{n\in \Nz_0}$ Im Fall $q=2$ erhielte man beispielsweise die Folge der Zweierpotenzen $1,2,4,8,16,\dots$.
 \end{enumerate}
In all diesen Beispielen gehorchen die Folgenglieder einer einfachen Regel. Dies muss aber nicht immer der Fall sein. Eine Folge darf auch völlig chaotisch sein und ihre Einträge brauchen keinem Muster zu gehorchen. Möglichst chaotische Folgen können mit einem Zufallsgenerator erzeugt werden. \\
Hier ist noch ein Beispiel für eine Folge, deren Einträge mal keine Zahlen sind:
\begin{itemize}
 \item Die Folge $(\{1,\dots , n\})_{n\in \Nz_0}$, deren Einträge die „Anfangsstücke“ von $\Nz_{\geq 1}$ sind. Also
 \[ \emptyset,\quad \{1\},\quad \{1,2\},\quad \{1,2,3\},\quad \{1,2,3,4\}\quad,\dots \]
 Dies ist keine Zahlenfolge, sondern eine Folge von Mengen. Sie besitzt die Eigenschaft, dass für jedes $n\in \Nz_0$ ihr $n$-ter Eintrag eine Menge ist, die genau $n$-viele Elemente enthält.
\end{itemize}
\end{bsp}



\begin{de}[Rechnen mit Folgen] \label{folgenrech}
Seien $(a_n)_{n\in \Nz}, (b_n)_{n\in \Nz}\in \Rz^\Nz$ zwei reelle Zahlenfolgen. Dann heißt die Folge
\[ (a_n)_{n\in \Nz}+(b_n)_{n\in \Nz} := (a_n + b_n)_{n\in \Nz} \]
die \textbf{Summe} von $(a_n)_{n\in \Nz}$ und $(b_n)_{n\in \Nz}$ und die Folge
\[ (a_n)_{n\in \Nz}\cdot (b_n)_{n\in \Nz}:= (a_n\cdot b_n)_{n\in \Nz} \]
heißt das \textbf{Produkt} der Folgen $(a_n)_{n\in \Nz}$ und $(b_n)_{n\in \Nz}$. \\
Ist ferner $\lambda\in \Rz$, so ist
\[ \lambda\cdot (a_n)_{n\in \Nz} := (\lambda\cdot a_n)_{n\in \Nz} \]
Da man Summe und Produkt zweier Zahlenfolgen dadurch erhält, dass man sie eintragsweise addiert bzw. multipliziert, spricht man auch von der \emph{komponentenweisen} Addition bzw. Multiplikation.
\end{de}


\begin{bsp}
Sind
\begin{align*}
 a_n & := (-1)^n \\
 b_n & := \frac{n}{n+1} && n \in \Nz_0
\end{align*}
so ist
\begin{align*}
 (a_n)_{n\in \Nz_0} + (b_n)_{n\in \Nz_0} & = \left((-1)^n+ \frac{n}{n+1}\right)_{n\in \Nz_0} &&= 1,\ \frac{-1}{2},\ \frac{5}{3},\ \frac{-1}{4},\ \frac{9}{5},\ \frac{-1}{6},\ \frac{13}{7},\dots \\
  (a_n)_{n\in \Nz_0} \cdot  (b_n)_{n\in \Nz_0} & = \left((-1)^n\cdot \frac{n}{n+1}\right)_{n\in \Nz_0} &&= 0,\ \frac{-1}{2},\ \frac{2}{3},\ \frac{-3}{4},\ \frac{4}{5},\ \frac{-5}{6},\ \frac{6}{7},\dots
\end{align*}
sowie
\begin{align*}
 3\cdot (b_n)_{n\in \Nz_0} & = \left( \frac{3n}{n+1}\right)_{n\in \Nz_0} && =0,\ \frac{3}{2},\ \frac{6}{3},\ \frac{9}{4},\ \frac{12}{5},\ \frac{15}{6},\dots \\
 0\cdot (a_n)_{n\in \Nz_0} & = \left(0\cdot (-1)^n\right)_{n\in \Nz_0} && = 0,0,0,0,0,\dots
\end{align*}

\end{bsp}



\begin{de}[Beschränktheit]
 Eine reelle Zahlenfolge $(a_n)_{n\in \Nz}\in \Rz^\Nz$ heißt \textbf{nach oben beschränkt} bzw. \textbf{nach unten beschränkt}, falls die Menge ihrer Einträge $\{a_n\mid n\in \Nz\}$ eine nach oben bzw. nach unten beschränkte Teilmenge von $\Rz$ im Sinne von \cref{schranken} ist. Konkret ist die Folge $(a_n)_{n\in \Nz}$ also genau dann
 \begin{itemize}
  \item \textbf{nach oben beschränkt}, falls es eine Zahl $M\in \Rz$ gibt, sodass $a_n\leq M$ für alle $n\in \Nz$ ist. In diesem Fall heißt ein solches $M$ eine \textbf{obere Schranke} für $(a_n)_{n\in \Nz}$.
  \item \textbf{nach unten beschränkt}, falls es eine Zahl $M\in \Rz$ gibt, sodass $a_n\geq M$ für alle $n\in \Nz$ ist. In diesem Fall heißt ein solches $M$ eine \textbf{untere Schranke} für $(a_n)_{n\in \Nz}$.
  \item \textbf{beschränkt}, falls sie sowohl nach oben als auch nach unten beschränkt ist.
  \item \textbf{unbeschränkt}, wenn sie nicht beschränkt ist.
 \end{itemize}
\end{de}


\begin{bsp} Es gilt:
\begin{itemize}
\item Die Folge $\left(\frac{n}{n+1}\right)_{n\in \Nz}$ ist nach unten durch $0$ und nach oben durch $1$ beschränkt.
\item Die Folge $2,3,5,7,11,\dots$ der Primzahlen ist nach oben unbeschränkt. Dies ist die Aussage des berühmten \emph{Satzes von Euklid}, siehe \cref{euklid}.
 \item Die alternierende Folge $((-1)^n)_{n\in \Nz}$ ist beschränkt, da sie nach oben durch $1$ und nach unten durch $-1$ beschränkt ist.
\end{itemize}
\end{bsp}



\begin{bem}
 Da die Beschränktheit einer Folge allein von der Menge ihrer Einträge abhängt, ist sie unempfindlich gegenüber einer Änderung der Reihenfolge der Folgeneinträge. Da beispielsweise die Folge
 \[ 0,\quad\frac{1}{2},\quad\frac{2}{3},\quad\frac{3}{4},\quad\frac{4}{5},\quad\frac{5}{6},\quad\frac{6}{7},\quad\dots \]
 nach unten durch $0$ und nach oben durch $1$ beschränkt ist, gilt dasselbe auch für die Folge
  \[ \frac{1}{2},\quad 0,\quad \frac{3}{4},\quad\frac{2}{3},\quad\frac{5}{6},\quad\frac{4}{5},\quad\frac{7}{8}\quad\dots \]
 die man durch eine Umordnung der Folgenglieder erhält.
\end{bem}


\begin{de}[Monotonie]
  Eine Folge reeller Zahlen $(a_n)_{n\in \Nz}\in \Rz^\Nz$ heißt
  \begin{itemize}
   \item \textbf{(monoton) wachsend}, falls für jedes $n\in \Nz$ gilt, dass $a_{n+1}\geq a_n$.
   \item \textbf{(monoton) fallend}, falls für jedes $n\in \Nz$ gilt, dass $a_{n+1}\leq a_n$.
   \item \textbf{monoton}, falls sie wachsend oder fallend ist.
  \end{itemize}
Gilt sogar $a_{n+1}>a_n$ für alle $n\in \Nz$, so sagt man auch, die Folge sei \emph{strikt wachsend}. Ähnlich definiert man auch „strikt fallend“.
\end{de}



\begin{bsp} Es gilt:
\begin{enumerate}[a)]
 \item Die alternierende Folge $((-1)^n)_{n\in \Nz}$ ist nicht monoton, da zum Beispiel $(-1)^3< (-1)^4$ aber auch $(-1)^4>(-1)^5$.
 \item Die Folge $\left(\frac{n}{n+1}\right)_{n\in \Nz}$ ist strikt wachsend.
 \begin{bew}
Für jedes $n\in \Nz$ ist
 \begin{align*}
  \frac{(n+1)}{(n+1)+1} - \frac{n}{n+1} & = \frac{(n+1)^2- n\cdot (n+2)}{(n+2)(n+1)} \\
  & = \frac{1}{(n+2)(n+1)} \\
  & >0
 \end{align*}
sodass $\frac{(n+1)}{(n+1)+1} > \frac{n}{n+1}$ ist. \qed
\end{bew}
 \item Für $q\in \Rz_{\geq 1}$ ist die Folge $(q^n)_{n\in \Nz}$ monoton wachsend.
 \begin{bew}
  Weil für jedes $n\in \Nz$ gilt:
  \begin{align*}
   q^{n+1} & = q \cdot q^n \\
   & \geq 1 \cdot q^n && (\text{wegen $q\geq 1$ und $q^n>0$})\\
   & = q^n \qed
  \end{align*}
 \end{bew}
 \item Eine reelle Folge ist genau dann sowohl monoton wachsend als auch monoton fallend, wenn sie konstant ist, d.h. wenn alle ihre Einträge identisch sind.
 \begin{bew}
  Sei $(a_n)_{n\in \Nz}\in \Rz^\Nz$ eine Folge, die gleichzeitig wachsend und fallend ist. Für jedes $n\in \Nz$ muss dann
  \[ a_{n+1}\geq a_n\qquad\text{und}\qquad a_{n+1}\leq a_n \]
  also insgesamt $a_{n+1}=a_n$ gelten. Aus der Tranitivität der Gleichheitsrelation\footnote{vgl. \cref{kettenfalten}} ergibt sich, das $a_n=a_0$ für jedes $n\in \Nz$ gilt. Also ist die Folge konstant. \qed
 \end{bew}
 \item Die Folge
 \[ 0,\quad -1,\quad -2,\quad -1,\quad 0,\quad 1,\quad 2,\quad 3,\quad 4, \quad 5,\dots \]
 ist nicht monoton. Sie könnte aber zu einer strikt wachsenden Folge gemacht werden, ließe man die ersten zwei Folgenglieder weg.
\end{enumerate}
\end{bsp}




\section{Abstand}

\begin{de}[Betrag und Abstand] \label{abstand}
Für eine reelle Zahl $a\in \Rz$ ist ihr \textbf{Betrag} definiert durch 
	\begin{align*}
		|a|=\begin{cases}
			a&a\geq 0\\
			-a&a<0
		\end{cases}
	\end{align*}
Beispielsweise ist
\[ \vert 3\vert = 3 \qquad\qquad \vert -7\vert =7 \qquad\qquad \vert 0\vert=0 \]
Sind $x,y\in \Rz$ zwei reelle Zahlen, so ist ihr \textbf{Abstand} definiert als
 \[ d(x,y) := \vert x -y\vert \]
 Beispielsweise ist
 \[ d(5,2)= 3 \qquad\qquad d(-2,5)=7 \qquad\qquad d(3,3)=0 \]
In der Literatur wird meistens der Buchstabe „$d$“ benutzt (für englisch ``distance'').
    \begin{figure}[H]
\begin{center}
\includegraphics[width=12cm]{./_img/Abstand.jpeg}
\end{center}
\centering \caption{Abstände auf der reellen Gerade}
\end{figure}
\end{de}


\begin{bem}
Für jede reelle Zahl $x\in \Rz$ gilt:
 \[ \vert x\vert = \vert x-0\vert = d(x,0) \]
 d.h. der Betrag einer Zahl ist genau ihr Abstand zur Null.
\end{bem}



\begin{bem}[Allgemeine Abstände]
 In der Ebene $\Rz^2$, im Raum $\Rz^3$ und allgemein im Hyperraum $\Rz^n$ (für ein $n\in \Nz$) ist der Abstand zweier Punkte definiert als die Länge ihrer Verbindungsstrecke. Dies wird in den Analysis-Vorlesungen rigoros definiert werden, wir werden es hier aber ab und zu schon in einem informellen Sinn ausnutzen, um nette Beispiele und Illustrationen beisteuern zu können. Alle präzisen Aussagen und Beweise in diesem Text handeln aber lediglich von reellen Zahlen.
\end{bem}





\begin{de}[Intervalle in $\Rz$]
 Seien $a,b\in \Rz$ mit $a\leq b$. Dann heißen
 \begin{itemize}
  \item $(a,b) := \{ x\in \Rz \mid a<x<b \}$ das \textbf{offene Intervall} mit den Randpunkten $a$ und $b$. 
  \item $[a,b] := \{ x\in \Rz \mid a\leq x\leq b \}$ das \textbf{abgeschlossene Intervall} mit den Randpunkten $a$ und $b$.
  \item Die Mengen $[a,b) := \{ x\in \Rz \mid a\leq x<b \}$ und $(a,b] := \{ x\in \Rz \mid a<x\leq b \}$ heißen \textbf{halboffene Intervalle} mit Randpunkten $a,$ und $b$. 
 \end{itemize}
\end{de}



\begin{bem}
 Das offene Intervall $(a,b)$ besteht also aus genau denjenigen Elementen, die zwischen $a$ und $b$ liegen. Es ist nicht zu verwechseln mit dem Paar $(a,b)\in \Rz^2$. Da sowohl für das Paar $(a,b)\in \Rz^2$ als auch für das offene Intervall $(a,b)\subseteq \Rz$ dieselbe Notation verwendet wird, musst du beim Lesen mathematischer Texte aus dem Kontext erraten, von welchem Objekt gerade die Rede ist. \\[0.5em]
 Das abgeschlossene Intervall $[a,b]$ unterscheidet sich vom offenen Intervall $(a,b)$ lediglich dadurch, dass in $[a,b]$ auch noch die beiden „Randpunkte“ $a$ und $b$ enthalten sind, während sie bei $(a,b)$ fehlen.
\end{bem}



\begin{de}[offene Bälle]
 Für $x\in \Rz$ und $r\in \Rz_{\geq 0}$ ist der \textbf{offene Ball} um $x$ mit Radius $r$ definiert durch
 \[ \mathbb{B}_r(x) := \{ y\in \Rz \mid d(x,y)< r\} \]
$\mathbb{B}_r(x)$ besteht also genau aus denjenigen Elementen, deren Abstand zu $x$ kleiner als $r$ ist.
\end{de}


\begin{bsp}
 Für $x=4$ und $r= \frac{3}{2}$ ist
 \[ \mathbb{B}_{3/2}(4) = \left\{ y\in \Rz\mid \vert 4-y\vert <\frac{3}{2} \right\} = \left\{ y\in \Rz\mid 5/2<y<11/2 \right\} = \left(5/2,11/2\right) \]
    \begin{figure}[H]
\begin{center}
\includegraphics[width=10cm]{./_img/1Dball.jpeg}
\end{center}
\centering \caption{Der offene Ball $\mathbb{B}_{3/2}(4)$}
\end{figure}
 Dies sind genau die Punkte in $\Rz$, deren Abstand zur $4$ kleiner als $3/2$ ist. Allgemein gilt für reelle Zahlen $x\in \Rz$ und $r\in \Rz_{\geq 0}$:
 \[ \mathbb{B}_r(x) = (x-r,x+r) \]
\end{bsp}


\begin{bem}[Entartete Fälle]
Es gilt:
\begin{itemize}
 \item Sind $a,b\in \Rz$ mit $a=b$, so ist $(a,b)=(a,a)=\emptyset$, da es kein Element gibt, das sowohl strikt kleiner als auch strikt größer als $a$ wäre. Ebenso wären $[a,a)=(a,a]=\emptyset$, wohingegen $[a,a]=\{a\}$.
 \item Für jedes $x\in \Rz$ ist $\mathbb{B}_0(x)=\emptyset$, da es kein Element gibt, dessen Abstand zu $x$ noch kleiner als Null wäre.
\end{itemize}
\end{bem}



\begin{bem}[zum Wort „Ball“]
 Zugegeben ergibt die Bezeichnung „offener Ball“ auf der (eindimensionalen) Zahlengerade nicht so viel Sinn, da es sich dabei um Intervalle handelt. Sobald man aber in der Ebene $\Rz^2$ oder im Raum $\Rz^3$ unterwegs ist, sehen die „Bälle“ auch wirklich aus wie Bälle: \\[0.5em]
 In der Ebene $\Rz^2$ ist zum Beispiel der offene Ball $\mathbb{B}_{1}\begin{pmatrix} 2 \\ 1\end{pmatrix}$ gegeben durch: \\
    \begin{figure}[H]
\begin{center}
\includegraphics[width=10cm]{./_img/2Dball.jpeg}
\end{center}
\centering \caption{Menge aller Punkte in der Ebene $\Rz^2$, deren Abstand zum Punkt mit den Koordinaten $(2,1)$ kleiner als $1$ ist.}
\end{figure}
Im Raum $\Rz^3$ ist beispielsweise der offene Ball $\mathbb{B}_{2}\begin{pmatrix} 1\\ 1 \\ 1\end{pmatrix}$ gegeben durch: \\
    \begin{figure}[H]
\begin{center}
\includegraphics[width=10cm]{./_img/3Dball.jpeg}
\end{center}
\centering \caption{Menge aller Punkte im Raum $\Rz^3$, deren Abstand zum Punkt mit den Koordinaten $(1,1,1)$ kleiner als $2$ ist.}
\end{figure}
Beachte aber, dass diese offenen Bälle keinen „Rand“ haben. Denn $\mathbb{B}_r(x)$ besteht ja nur aus den Elementen, deren Abstand zu $x$ strikt kleiner als $r$ ist. Diejenigen Elemente, deren Abstand zu $x$ genau gleich $r$ ist (im Zweidimensionalen also genau die Punkte auf dem Kreisrand, im Dreidimensionalen die Punkte auf der Kugeloberfläche), sind nicht in $\mathbb{B}_r(x)$ enthalten.
\end{bem}




\begin{comment}
\begin{sat}
	Für $x,y\in\Rz$ gilt
	\begin{resListeN}
		\item $x=0\lra |x|=0$
		\item $x\leq |x|$
		\item $|xy|=|x||y|$
		\item $|x+y|\leq |x|+|y|$
	\end{resListeN}
\end{sat}
\begin{bew}
	\begin{bewListeN}
		\item folgt direkt aus der Definition
		\item für $x\geq 0$ gilt $x=|x|$. Im anderen Fall folgt $x<0<-x=|x|$
		\item Für $x=0\vee y=0$ ist die Aussage trivial. Seien also $x,y\neq 0$.
			\begin{itemize}
				\item[$x,y>0$:] damit ist auch $xy>0$. Also: $|xy|=xy=|x||y|$
				\item[$x,y<0$:] auch hier ist $xy>0$. Das heißt $|xy|=xy=(-x)(-y)=|x||y|$
				\item[$x<0,y>0$:] hier ist $xy<0$ und damit $|xy|=-xy=(-x)y=|x||y|$.
				\item[$x>0,y<0$:] analog
			\end{itemize}
		\item Da beide Seiten positiv sind, ist das Quadrieren eine Äquivalenzumformung. Dann folgt mit ii)
			\begin{align*}
				|x+y|\leq |x|+|y|\lra x^2+2xy+y^2\leq x^2+2|xy|+y^2\lra 2xy\leq 2|xy|
			\end{align*}
	\end{bewListeN}
\bewEnd
\end{bew}
\end{comment}


% --------------------------------------------------------------------
% §3 Section <<Konvergenz>>
% --------------------------------------------------------------------
\section{Konvergenz}

\begin{bem}[Motivation]
 Betrachten wir einmal eine Folge von Punkten in der Ebene, sie sich spiralenförmig dem Ursprung annähert:
     \begin{figure}[H]
\begin{center}
\includegraphics[width=10cm]{./_img/Spirale.jpeg}
\end{center}
\centering \caption{Eine konvergente Folge von Punkten in der Ebene}
\end{figure}
Konkret könnte man
\begin{align*}
 a_n & := \frac{1}{n}\cdot \begin{pmatrix}
           \cos(n\cdot \pi/4) \\
           \sin(n\cdot \pi/4)
          \end{pmatrix} && n\in \Nz
\end{align*}
definieren, aber das soll gerade keine Rolle spielen. \\
Man sieht, dass die Folgenglieder dem Koordinatenursprung immer näher kommen, sich ihm geradezu „anschmiegen“.
\end{bem}



Dass eine Folge gegen einen Grenzwert konvergiert, heißt grob gesagt, dass sie diesem Grenzwert „irgendwann beliebig nahe kommt“. Die übliche mathematische Formalisierung dieser Vorstellung geht folgendermaßen:
\begin{de}[Konvergenz]
	Seien $(a_n)_{n\in \Nz}\in \Rz^\Nz$ eine Folge reeller Zahlen und $a\in \Rz$. Man sagt, die Folge $(a_n)_{n\in \Nz}$ \textbf{konvergiert} gegen $a$, falls es für jedes $\varepsilon \in \Rz_{>0}$ ein $N\in \Nz$ gibt, sodass ab dem $N$-ten Folgenglied alle Folgenglieder in $\mathbb{B}_\epsilon(a)$ liegen. Als Quantorenformel:
	\begin{align*}
	\forall\epsilon\in \Rz_{>0}\ \exists N\in\Nz\ \forall n\in \Nz_{\geq N}:\ a_n\in \mathbb{B}_\epsilon(a)
	\end{align*}
	In diesem Fall heißt $a$ ein \textbf{Grenzwert} der Folge $(a_n)_{n\in \Nz}$. Man schreibt
	\[ \limm{n\longra\infty}a_n=a \qquad\text{oder}\qquad a_n\xrightarrow[n\to \infty]{} a \]
	Eine Folge, die keinen Grenzwert besitzt, heißt \textbf{divergent}.
\end{de}



\begin{bem}[„Sei $\epsilon>0$“]
 Diese Grenzwertdefinition ist vergleichsweise jung: sie stammt aus dem 19. Jahrhundert und wurde durch Cauchy\footnote{\href{https://de.wikipedia.org/wiki/Augustin-Louis_Cauchy}{Augustin-Louis Cauchy (1789 - 1857)}} und Weierstraß\footnote{\href{https://de.wikipedia.org/wiki/Karl_Weierstra\%C3\%9F}{Karl Weierstraß (1815 - 1897)}} populär. Bis dahin herrschte in der Mathematik ein eher „intuitiver“ Umgang mit Grenzwerten vor. \\[0.5em]
 Seit dem Aufkommen der modernen Analysis hat es sich eingebürgert, in Definitionen und Beweisen jene Abstände, die „beliebig klein“ werden sollen, mit einem „$\epsilon$“ zu notieren. Aus diesem Grund spricht man bei Definitionen und Beweisen der Analysis, die viel Gebrauch von $\epsilon$'s machen, von „Epsilontik“.
\end{bem}





\begin{bsp}
 Die durch
 \begin{align*}
  a_n &:= \frac{n}{n+1} && n\in \Nz
 \end{align*}
 definierte reelle Folge $(a_n)_{n\in \Nz}\in \Rz^\Nz$ konvergiert gegen $1$.
\end{bsp}
     \begin{figure}[H]
\begin{center}
\includegraphics[width=14cm]{./_img/Konvergenzbsp.jpeg}
\end{center}
%\centering \caption{Eine konvergente Folge von Punkten in der Ebene}
\end{figure}
\begin{bem}
Bevor wir versuchen, für diese Aussage einen Bilderbuchbeweis hinzuschreiben, wollen wir „auf dem Skizzenblatt“ erstmal ein paar Überlegungen anstellen: \\
 Wir müssen für ein beliebiges $\varepsilon \in \Rz_{>0}$ ein $N\in \Nz$ finden, sodass für alle $m\in \Nz_{\geq N}$ gilt:
 \begin{align*}
   \epsilon &  > d(1,\frac{m}{m+1})
   \end{align*}
   Mit ein paar Umformungen ergibt sich:
   \begin{align*}
  d(1,\frac{m}{m+1}) & = \left| 1-\frac{m}{m+1}\right| \\
   & = \left| \frac{m+1}{m+1} - \frac{m}{m+1} \right| \\
   & = \left| \frac{1}{m+1}\right| \\
   & = \frac{1}{m+1} & (\text{da $m\in \Nz$})
 \end{align*}
Die Ungleichung $\epsilon >d(1,m/(m+1))$ lässt sich daher umformen zu
\[ m > \frac{1}{\epsilon} -1  \]
Wählen wir also für $N$ die nächstgrößere natürliche Zahl oberhalb von $1/\epsilon$, indem wir $1/\epsilon$ aufrunden (man könnte für $N$ auch jede weitere Zahl, die größer als $(1/\varepsilon -1)$ ist, verwenden). Dann sind sowohl $N$ als auch alle $m\in \Nz_{>N}$ größer als $1/\epsilon- 1$, sodass die Ungleichung aufgeht. \\[0.5em]
Damit haben wir die Aufgabe auf dem Schmierblatt gelöst. Im finalen Beweis lassen wir die Überlegungen, die uns zur Wahl von $N$ geführt haben, weg. Dies tritt häufig in Analysis-Beweisen auf: die Beweise unterdrücken den Gedankenprozess, der zu ihrem Auffinden geführt hat und verlaufen genau in die entgegengesetzte Richtung:
\end{bem}
\begin{bew}
Es sei $\varepsilon \in \Rz_{>0}$ beliebig. Sei $N\in \Nz$ die kleinste natürliche Zahl, die größer als $1/\epsilon$ ist. Für alle $m\in \Nz_{\geq N}$ gilt dann:
\begin{align*}
 d(1,a_m) & = \left| 1-\frac{m}{m+1}\right| \\
 & = \left| \frac{1}{m+1} \right| \\
 & = \frac{1}{m+1} \\
 & < \frac{1}{m} \\
 & \leq \frac{1}{N} & (\text{da $m\geq N$})\\
 & < \frac{1}{1/\epsilon} & (\text{da $N> 1/\epsilon$}) \\
 & = \epsilon
\end{align*}
Somit liegen ab dem $N$-ten Folgenglied alle Folgenglieder in $\mathbb{B}_\epsilon(1)$. Da $\epsilon \in \Rz_{>0}$ beliebig gewählt war, ist damit bewiesen, dass die $a_n$'s gegen $1$ konvergieren. \qed
\end{bew}



\begin{bem}
 Beachte, dass die Tatsache, ob eine Folge konvergiert oder divergiert, eine reine Eigenschaft des „Langzeitverhaltens“ dieser Folge ist. Die ersten paar Millionen Folgenglieder haben für sich allein keinen Einfluss auf das Konvergenzverhalten, weil die Folge ja ab dem dreimillionsten Eintrag plötzlich eine ganz andere Richtung einschlagen könnte. (Die Folgen, mit denen man es in der Praxis zu tun hat, unterliegen aber meist einem einfachen Muster, das spätestens nach ein paar Dutzend Folgengliedern ersichtlich sein sollte)
\end{bem}




\begin{bsp}
 Die Folge der Quadratzahlen $(n^2)_{n\in \Nz}$ besitzt keinen Grenzwert.
\end{bsp}
\begin{bew}
 Für einen Widerspruchsbeweis sei angenommen, es gebe einen Grenzwert $a\in \Rz$. Dann gäbe es ein $N\in \Nz$, sodass für alle $n\in \Nz_{\geq N}$ gälte, dass $n^2 \in \mathbb{B}_1(a)$. Wähle ein $k\in \Nz$, für das sowohl $k>N$ als auch $k^2>a+1$ gilt. Dann wäre $k^2\in \mathbb{B}_1(a)$ und wegen $\mathbb{B}_1(a)=(a-1,a+1)$ folgte der Widerspruch $a+1 < k^2 < a+1$. \qed
\end{bew}




\begin{bsp}[Konstante Folgen konvergieren]
 Sei $x \in \Rz$ beliebig. Dann konvergiert die konstante Folge $(x)_{n\in \Nz}$ gegen $x$.
 \[ x,x,x,x,\dots \qquad \xrightarrow[n\to \infty]{} x \]
\end{bsp}
\begin{bew}
Sei $\varepsilon \in \Rz_{>0}$ beliebig. Für jedes $n\in \Nz$ ist $\vert a_n-x\vert =x-x=0 < \varepsilon$. \qed
\end{bew}





\section{Einige Konvergenzgesetze}


\begin{sat}[Eigenschaften der Abstandsfunktion]
 Für alle reellen Zahlen $x,y,z \in \Rz$ gilt:
 \begin{align*}
  d(x,y)=0 \quad &\leftrightarrow\quad x=y && (\text{Definitheit}) \\
  d(x,y) & = d(y,x) && (\text{Symmetrie}) \\
  d(x,z) & \leq d(x,y)+d(y,z) && (\text{Dreiecksungleichung})
 \end{align*}
\end{sat}
\begin{bew}
 Wir werden diese Eigenschaften hier ohne Beweis verwenden. Einen Beweis kannst du z.B. in Satz I.11.4 im Analysis-Lehrbuch \citet{AE06} (das du über den Link im Literaturverzeichnis kostenlos aus dem Uni-Netz herunterladen kannst) nachschlagen. Du kannst aber auch abwarten, bis diese Eigenschaften in der Ana1-Vorlesung thematisiert werden.
\end{bew}



\begin{bem}[Metrische Räume]
 Die obigen drei Aussagen können auch an und für sich untersucht werden: Ist $X$ eine beliebige Menge und $d:X\times X\to \Rz_{\geq 0}$ eine Abbildung, die diese drei Aussagen erfüllt, so nennt man $d$ eine „Metrik“ und das Paar $(X,d)$ einen „\href{https://de.wikipedia.org/wiki/Metrischer_Raum}{metrischen Raum}“. Ein erheblicher Teil der Analysis reeller Zahlen verallgemeinert sich auf beliebige metrische Räume, was meist in der Zweitsemestervorlesung „Analysis 2“ thematisiert wird.
\end{bem}




\begin{bem}[zur Dreiecksungleichung]
 Der Name „Dreiecksungleichung“ ergibt, sofern man nur Abstände reeller Zahlen untersucht, nicht so viel Sinn. Vielmehr kommt er aus der ebenen bzw. räumlichen Geometrie. Sind $x,y,z\in \Rz^2$ drei Punkte in der Ebene, die ein Dreieck bilden, so besagt die Dreiecksungleichung, dass „der direkte Weg von $x$ nach $z$ nicht länger als der Umweg über $y$ sein kann“:
     \begin{figure}[H]
\begin{center}
\includegraphics[width=10cm]{./_img/Dreiecksungleichung.jpeg}
\label{zahlenkugel}
%\centering \caption{Die Summe zweier Dreiecksseitenlängen ist stets kleiner als die Länge der dritten Seite.}
\end{center}
\end{figure}
\end{bem}





\begin{sat}[Eindeutigkeit des Grenzwerts]
 Sofern eine Folge reeller Zahlen konvergiert, ist ihr Grenzwert eindeutig bestimmt.
\end{sat}
\begin{bew}
Seien $(a_n)_ {n\in \Nz}\in \Rz^\Nz$ eine reelle Zahlenfolge, $a,b\in \Rz$ und es gelte sowohl $a_n\xrightarrow[n\to \infty]{} a$ als auch $a_n\xrightarrow[n\to \infty]{} b$. Ferner sei $\varepsilon \in \Rz_{>0}$. \\
Weil die $a_n$'s gegen $a$ konvergieren, gibt es ein $N\in \Nz$, sodass für alle $m\in \Nz_{\geq N}$ gilt, dass $a_m\in \mathbb{B}_{\varepsilon/2}(a)$. Und weil die $a_n$'s auch gegen $b$ konvergieren, gibt es ebenso ein $M\in \Nz$, sodass für alle $m\in \Nz_{\geq M}$ gilt, dass $a_m\in \mathbb{B}_{\varepsilon/2}(b)$. \\
Sei nun $k\in \Nz$ eine Zahl, die sowohl größer als $N$ als auch größer als $M$ ist. Dann gilt
\[ a_k \in \mathbb{B}_{\varepsilon/2}(a) \qquad\text{und}\qquad  a_k \in \mathbb{B}_{\varepsilon/2}(b) \]
Mit der Dreiecksungleichung folgt
\[ d(a,b) \leq d(a,a_k) + d(a_k,b) < \frac{\epsilon}{2}+\frac{\epsilon}{2} = \epsilon \]
Da die Zahl $\varepsilon\in \Rz_{>0}$ beliebig gewählt war, ergibt sich, dass der Abstand $d(a,b)$ kleiner als jede beliebige positive reelle Zahl sein muss. Damit bleibt nur noch $d(a,b)=0$ übrig, woraus aufgrund der Definitheit der Abstandsfunktion folgt, dass $a=b$. \qed
\end{bew}



\begin{bem}[\textbf{Der} Grenzwert einer Folge]
 Dieser Satz berechtigt uns, bei einer konvergenten Folge $(a_n)_{n\in \Nz}\in \Rz^\Nz$ anstelle von „einem Grenzwert“ von \emph{dem} Grenzwert dieser Folge zu sprechen. Ist $a\in \Rz$ der Grenzwert der $a_n$'s, so suggeriert die Notation
 \[ \lim_{n\to \infty} a_n = a \]
 ja auch schon, dass „$\lim_{n\to \infty}a_n$“ ein wohlbestimmtes Objekt ist, das eben gleich $a$ oder ungleich $a$ sein kann.
\end{bem}



%------------------
%3.10 Satz
\begin{sat}[Rechenregeln für Folgengrenzwerte]
	Seien $(a_n)_{n\in \Nz},(b_n)_{n\in \Nz}\in \Rz^\Nz$ zwei konvergente Folgen mit $a=\limm{n\longra\infty}a_n$ und $b=\limm{n\longra\infty}b_n$. Dann gilt:
	\begin{enumerate}[a)]
		\item $\limm{n\longra\infty}(a_n+b_n)=a+b$
		\item $\limm{n\longra\infty}(\lambda \cdot a_n)=\lambda \cdot a$ für $\lambda\in\Rz$
		\item $\limm{n\longra\infty}(a_n\cdot b_n)=a\cdot b$.
	\end{enumerate}
\end{sat}

%------------------
% 3.11 Beweis
\begin{bew}
Wir werden hier nur die Gleichung für die Addition beweisen. Die Punkte b) und c) werden in der Ana1-Vorlesung bewiesen werden\footnote{und lassen sich etwa in Satz II.2.2 und Satz II.2.4 im hervorragenden Lehrbuch \citet{AE06} nachschlagen.}. \\[0.5em] 
Sei $\epsilon \in \Rz_{>0}$ eine beliebige positive reelle Zahl. Weil die $a_n$'s gegen $a$ konvergieren, existiert ein $N\in\Nz$ sodass $a_n\in\mathbb{B}_{\epsilon/2}(a)$ für alle $n\in \Nz_{\geq N}$ gilt. Und weil die $b_n$'s gegen $b$ konvergieren, existiert ein $M\in \Nz$, sodass $a_n\in\mathbb{B}_{\epsilon/2}(b)$ für alle $n\in \Nz_{\geq M}$ gilt. Setze $K:=\max\{N,M\}$ auf die größere der beiden Zahlen $M,N$. Für alle $n\in \Nz_{\geq K}$ ist dann $a_n \in \mathbb{B}_{\epsilon/2}(a)\cap\mathbb{B}_{\epsilon/2}(b)$, sodass
    \begin{align*}
    \betrag{(a_n+b_n)-(a+b)}& = \vert (a_n-a) + (b_n-b)\vert \\
    & \leq \betrag{a_n-a}+\betrag{b_n-b } &&(\text{ohne Beweis})\\
    & <\frac{\epsilon}{2}+\frac{\epsilon}{2} &&(\text{da $a_n\in  \mathbb{B}_{\epsilon/2}(a)\cap\mathbb{B}_{\epsilon/2}(b)$})\\
    & =\epsilon
    \end{align*}
Da $\varepsilon\in \Rz_{>0}$ beliebig gewählt war, folgt, dass die Folge $(a_n+b_n)_{n\in \Nz}$ gegen $a+b$ konvergiert. \qed
\end{bew}



\begin{bem}[Komplizierte Objekte in einfache Bausteine zerlegen]
 Sätze wie dieser sind von herausragender Bedeutung für die Analysis. Sofern man einen kleinen Vorrat an Folgen, für die man ihre Konvergenz bewiesen hat, aufgebaut hat, erlauben sie es, Grenzwerte für die kompliziertesten Folgen auszurechnen, ohne dass man für einen Beweis nochmal die $\epsilon$'s auskramen müsste. \\
Beispielsweise würde kein routinierter Mathematiker einen $\epsilon$-Beweis dafür führen, dass die Folge
 \[ \left( 1 + \frac{3n}{n+1} \right)_{n\in \Nz} \]
 gegen $4$ konvergiert. Sondern er würde schlicht bemerken, dass sich diese Folge als Linearkombination
 \[ (1)_{n\in \Nz} + 3\cdot \left(\frac{n}{n+1}\right)_{n\in \Nz} \]
 zusammensetzt und dann auf die Rechenregeln für $\lim_{n\to\infty}(-)$ verweisen. Denn es ist
 \[ 4 = 1+3\cdot 1 = \left( \lim_{n\to \infty} 1 \right)+3\cdot \left( \lim_{n\to \infty} \frac{n}{n+1} \right) = \lim_{n\to \infty} \left( 1+ \frac{3n}{n+1} \right) \]
 Diese Denkweise kennst du auch aus der Schule: Um beispielsweise die Ableitung der Funktion
 \begin{align*}
  f(x) &= (x^2+3x)\cdot e^x && x\in \Rz
 \end{align*}
 zu berechnen, würdest du ausnutzen, dass sich diese Funktion aus den Bestandteilen
 \begin{align*}
  g(x)& = x^2 \\
  h(x) & = 3x \\
  q(x) &= e^x && x\in \Rz
 \end{align*}
 zusammensetzt via
 \begin{align*}
  f(x) & = (g(x)+h(x))\cdot q(x) && x\in \Rz
 \end{align*}
Mittels Summen- und Produktregel würdest du schlussfolgern
\begin{align*}
 f'(x) & = (g'(x)+h'(x))\cdot q(x)  + (g(x)+h(x))\cdot q'(x) \\
 & = (2x+3)\cdot e^x + (x^2+3x)\cdot e^x \\
 & = (x^2+5x+3)\cdot e^x && x\in \Rz
\end{align*}
Die „analytische“ Methode, komplexe Funktionen in einfache Bestandteile zu zerlegen, ist auch an der Uni überlebensnotwendig. Kein erfahrener Mathematiker würde, um die Ableitung von $(x^2+3x)\cdot e^x$ zu berechnen, unmittelbar mit der Definition der Ableitung arbeiten und versuchen, den Differenzialquotienten
\begin{align*}
 f'(x)&:=\limm{h\longra 0} \frac{((x+h)^2+3(x+h))\cdot e^{x+h} - (x^2+3x)\cdot e^{x}}{h} && x\in \Rz
\end{align*}
direkt auszurechnen. \\[0.5em]
Ein Ziel der Analysis-Vorlesung ist es, dich mit Werkzeugen auszustatten, die das Berechnen von Grenzwerten bequemer machen und Epsilontik vermeiden. Versuche in Analysis-Beweisen, die Objekte immer soweit es geht in einfachste Bausteine zu zerlegen und einen $\epsilon$-Beweis erst wenn gar nichts anderes mehr geht als Ultima Ratio anzusetzen.
\end{bem}






% --------------------------------------------------------------------
% §5 Section <<Aufgabenvorschlaege>>
% --------------------------------------------------------------------
\newpage
\section{Aufgabenvorschläge}

%Das hier sind die Übungsaufgaben zum Folgen-Vortrag im Vorkurs WS 2014.



% --------------------------------------------------------------------
% §5.1 Subsection <<Aufgabe 1>>
% --------------------------------------------------------------------






\begin{aufg}[Konkrete Folgen]
Untersucht jede der folgenden Folgen darauf, ob sie (nach oben oder unten) beschränkt und/oder monoton ist.
\begin{enumerate}[a)]
  \item Die Folge der Kehrwerte natürlicher Zahlen $(1/n)_{n\in \Nz_{\geq 1}}$. Das ist  $1,\frac{1}{2},\frac{1}{3},\frac{1}{4},\frac{1}{5},\dots$.
  \item Die Folge $(n^2)_{n\in \Nz}$ der Quadratzahlen: $0,1,4,9,16,\dots$.
  \item Die Folge $0,1,-1,2,-2,3,-3,4,\dots$.
  \item[d)*] Die Folge $(\sin(n))_{n\in \Nz}$.
\end{enumerate}
\end{aufg}





\begin{aufg}[Teilmengen skizzieren]
Visualisiert die folgenden Teilmengen von $\Rz$ jeweils durch eine Zeichnung:
\begin{align*}
 &\text{a)} && \bigcup_{k\in \Zz} [3k+1,3k+2] \\
&\text{b)} && \bigcup_{n\in \Nz} \left[ -n, \frac{n}{n+1} \right] \\
 &\text{c)} && \bigcap_{n\in \Nz_{\geq 1}} \mathbb{B}_{1/n}(7) \\
 &\text{d)*} && \bigcup_{n\in \Nz_{\geq 1}} \left( \bigcup_{k\in \Zz} \left[\frac{3k+1}{3^n},\frac{3k+2}{3^n} \right] \right)
\end{align*}
\end{aufg}





\begin{aufg}[Intuition für $\Rz$]
Beurteile intuitiv, ob die folgenden Aussagen wahr oder falsch sind. Du brauchst in dieser Aufgabe keine wasserdichten Beweise formulieren und dir auch nicht hundertprozentig sicher sein; es geht hier um die Schärfung deiner Intuition für reelle Zahlen.
\begin{enumerate}[a)]
 \item Sind $x,y,z\in \Rz$ mit $d(x,y)<2$ und $d(y,z)<3$, so ist $d(x,z)<5$.
 \item Sind $x,y,z\in \Rz$ mit $d(x,y)>2$ und $d(y,z)>3$, so ist $d(x,z)>5$.
  \item Jede monoton wachsende Folge reeller Zahlen konvergiert.
 \item Jede beschränkte, monoton wachsende Folge reeller Zahlen konvergiert.
  \item Jede konvergente Folge reeller Zahlen ist beschränkt.
 \item Jede beschränkte Folge reeller Zahlen konvergiert.
 \item Ist $(a_n)_{n\in \Nz}\in \Rz^\Nz$ eine konvergente Zahlenfolge und ist $a_n\leq 4$ für alle $n\in \Nz$, so ist auch $\lim_{n\to\infty} a_n\leq 4$.
 \item Ist $(a_n)_{n\in \Nz}\in \Rz^\Nz$ eine konvergente Zahlenfolge und ist $a_n <4$ für alle $n\in \Nz$, so ist auch $\lim_{n\to\infty} a_n<4$.
\end{enumerate}
\end{aufg}





\begin{aufg}[Konvergenzbeweis]
Zeigt mithilfe eines $\epsilon$-Beweises, dass die Folge $(2^{-n})_{n\in \Nz}\in \Rz^\Nz$ gegen $0$ konvergiert.
\end{aufg}








\begin{comment}
\subsection{Aufgabe}
In dieser Aufgabe soll deutlich gemacht werden, dass das Konvergenzverhalten einer Folge unabhängig von den ersten paar Millionen Folgeneinträgen ist. \\
Dazu seien $(a_n)_{n\in \Nz},(b_n)_{n\in \Nz} \in \Rz^\Nz$ zwei reelle Folgen, für die es ein $N\in \Nz$ gibt, sodass $a_n=b_n$ für alle $n\geq N$ gilt. Mit anderen Worten: bis auf eine Handvoll Einträge am Anfang stimmen die Folgen ab einem gewissen Zeitpunkt überein. \\[0.5em]
Beweist, dass die Folge der $a_n$'s genau dann konvergiert, wenn die Folge der $b_n$'s konvergiert und dass in diesem Fall beide Grenzwerte übereinstimmen.






\subsection{Aufgabe}
Im Vortrag wurde bewiesen, dass eine reelle Folge, sofern sie konvergiert, auch nur genau einen Grenzwert hat. Daher stimmt etwas im folgenden „Beweis“ nicht:
\begin{sat}[Falsche Aussage]
 Die alternierende Folge $(a_n)_{n\in \Nz}:=((-1)^n)_{n\in \Nz}$ hat zwei verschiedene Grenzwerte, nämlich $1$ und $-1$.
\end{sat}
\begin{bew}
 Ich beweise zuerst, dass die $a_n$'s gegen $1$ konvergieren. Dazu sei $\epsilon \in \Rz_{>0}$. Für alle geraden Zahlen $n$ ist $a_n=(-1)^n=1$, sodass
 \[ d(a_n,1) = \vert a_n-1\vert = \vert 1-1\vert = 0<\epsilon \]
 Da $\epsilon\in \Rz_{>0}$ beliebig gewählt war, konvergieren die $a_n$'s also gegen $1$. \\[0.5em]
 Nun beweise ich, dass die $a_n$'s auch gegen $-1$ konvergieren. Dazu sei $\epsilon \in \Rz_{>0}$. Für alle ungeraden Zahlen $n$ ist $a_n=(-1)^n=-1$, sodass
  \[ d(a_n,-1) = \vert a_n-(-1)\vert = \vert (-1)-(-1)\vert = 0<\epsilon \]
 Da $\epsilon\in \Rz_{>0}$ beliebig gewählt war, konvergieren die $a_n$'s also gegen $-1$. 
\end{bew}

Findet den Fehler in der Argumentation.


\subsection{Aufgabe}

Zeigt oder widerlegt die Konvergenz der Folgen:
	\begin{enumerate}
		\item $a_n=\frac{n^2+4}{2n^2}$
		\item $b_n=\cos(\pi n)$
	\end{enumerate}

\end{comment}
% --------------------------------------------------------------------
% §5.4 Subsection <<Aufgabe 4>>
% --------------------------------------------------------------------

\begin{comment}
\subsection{Aufgabe}

Unter Annahme ihrer Konvergenz, berechnet die Grenzwerte der Folgen:
		\begin{align*}
			a_n&=\frac{4^{(2^n)}}{2^{(4^n)}}&
			b_n&=\frac{(-1)^n}{3^n+(-2)^n}&
			c_n&=\frac{(-1)^nn}{n+1}
		\end{align*}
\end{comment}




%%% Local Variables:
%%% mode: latex
%%% TeX-master: "Skript"
%%% End:


% ====================================================================
% 7. Vortrag <<Gruppen>>
% ====================================================================
\setchapterpreamble[c][.7\textwidth]{%
  \itshape\dgrau\small Der Begriff der Verknüpfung verallgemeinert die aus der Schule bekannten Rechenoperationen. In diesem Vortrag werden grundlegende Eigenschaften von Verknüpfungen untersucht und ein paar Konsequenzen daraus abgeleitet.
  \vspace{24pt}%
}
\chapter{Verknüpfungen}
% ====================================================================
%                                                                 
%
%       §07
%
%
% %%ts latex start%%[2019-03-07 Thu 14:45]%%ts latex end%%
% ====================================================================
% --------------------------------------------------------------------
% §1 Section <<Einführung>>
% --------------------------------------------------------------------
\section{Verknüpfungen}%\label{wrz}

In den Kapiteln 3 und 4 haben wir als zentrale mathematische Objekte Mengen und die Abbildungen zwischen ihnen eingeführt. Von sich aus tragen Mengen relativ wenig Information: nämlich nur, welche Elemente in ihnen enthalten sind und welche nicht. Als einen Weg, einer Menge mehr Struktur zu verleihen, wurden in Kapitel 5 Relationen thematisiert, die je zwei Elemente einer Menge in eine „Beziehung“ setzen können, wodurch die bereits aus der Schule bekannten Relationen $\leq$ und $=$ verallgemeinert wurden.

In diesem Kapitel möchten wir einen weiteren Weg, mehr Struktur in eine Menge zu bringen, vorstellen, der die bekannten Rechenoperationen wie \glqq$+$\grqq\ und \glqq$\cdot$\grqq\ verallgemeinert.

Was soll eine Verknüpfung tun? Als Input möchten wir ihr zwei Elemente $x,y$ unserer Menge geben, und als Output erwarten wir ein neues Element $z$, eben die Verknüpfung der Elemente $x$ und $y$; so wie auch die Addition „$+$“ aus zwei Zahlen $x,y$ die Zahl $z=x+y$ macht. Formal wird dies in einer Abbildung realisiert, die zwei Elemente aufnimmt und ein neues ausgibt:

\begin{de}
	Sei $X$ eine beliebige Menge. Eine (zweistellige) \textbf{Verknüpfung} auf $X$ ist eine Abbildung
		\[ X\times X \to X  \]
Eine zweistellige Verknüpung wird in der Regel mit einem „Verknüpfungszeichen“ notiert. Das heißt: ist der Name der Abbildung etwa „$*$“, so schreibt man
\[ x*y \qquad\text{anstelle von}\qquad *(x,y) \]
für den Funktionswert des Paares $(x,y)$ unter der Abbildung „$*$“.
\end{de}

\begin{bem}[*]
 Prinzipiell lassen sich auch dreistellige und höherstellige Verknüpfungen definieren. Allerdings sind so gut wie alle Verknüpfungen, die dir im Studium begegnen werden, zweistellig. Falls wir daher von „Verknüpfungen“ schreiben, sollen damit stets zweistellige Verknüpfungen gemeint sein.
\end{bem}



\begin{bsp}[Grundrechenarten]
Es gilt:
 \begin{enumerate}[a)]
 \item Auf $\Rz$ ist durch die Addition $(x,y)\mapsto x+y$ eine zweistellige Verknüpfung gegeben. Da die Summe zweier rationaler Zahlen ebenfalls rational ist, liefert „$+$“ auch eine Verknüpfung auf der Menge $\Qz$. Ebenso ist auch auf $\Nz$ und $\Zz$ durch die Addition „$+$“ jeweils eine Verknüpfung gegeben.
  \item Auf den Mengen $\Zz,\Qz,\Rz,\Cz$ ist durch die Subtraktion $(x,y)\mapsto x-y$ jeweils eine zweistellige Verknüpfung gegeben. Allerdings ergibt der Ausdruck
  \[ \text{„$\Nz \times \Nz \to \Nz \ ,\ (x,y) \mapsto x-y$“} \]
  \emph{keinen} Sinn, da etwa „$2-3$“ gar kein Element von $\Nz$ ist. Auf $\Nz$ ist die Subtraktion also \textbf{keine} zweistellige Verknüpfung. Zwar kann man für gewisse Zahlenpaare durchaus Differenzen in $\Nz$ bilden (z.B. „$3-2$“); dass „$-$“ eine Verknüpfung auf $\Nz$ wäre, scheitert aber daran, dass man eben nicht für \emph{jedes} Paar natürlicher Zahlen eine Differenz in $\Nz$ bilden kann.
  \item Auf den Mengen $\Nz,\Zz,\Qz,\Rz,\Cz$ ist durch die Multiplikation $(x,y)\mapsto x\cdot y$ jeweils eine zweistellige Verknüpfung gegeben.
  \item Auf den Mengen $\Qz\setminus \{0\},\Rz\setminus \{0\},\Cz\setminus \{0\}$ ist durch die Division $(x,y)\mapsto x:y$ jeweils eine zweistellige Verknüpfung gegeben. Beachte, dass die Null ausgelassen werden muss, weil bspw. nicht „$1:0$“ gebildet werden kann. Zwar könnte man $1:0=\infty$ setzen, aber „$\infty$“ wäre ja kein Element von $\Qz,\Rz$ bzw. $\Cz$, sodass man dadurch dennoch keine Verknüpfung auf $\Qz,\Rz$ bzw. $\Cz$ erhielte.
 \end{enumerate}
\end{bsp}





\begin{bsp}[Rechenoperationen auf Folgen]
 Auf der Menge $\Rz^\Nz$ der reellen Zahlenfolgen gibt es die folgenden Verknüpfungen, die bereits in \cref{folgenrech} eingeführt wurden:
 \begin{enumerate}[a)]
  \item Die Addition reeller Zahlenfolgen:
  \[ \Rz^\Nz\times \Rz^\Nz \to \Rz^\Nz \ ,\ ((a_n)_{n\in \Nz} , (b_n)_{n\in \Nz}) \mapsto (a_n+b_n)_{n\in \Nz} \]
  \item Die Multiplikation reeller Zahlenfolgen:
    \[ \Rz^\Nz\times \Rz^\Nz \to \Rz^\Nz \ ,\ ((a_n)_{n\in \Nz} , (b_n)_{n\in \Nz}) \mapsto (a_n\cdot b_n)_{n\in \Nz} \]
\item Ebenso kann man auch eine Subtraktion reeller Folgen definieren und, sofern man sich auf solche Folgen, in denen nirgends die Null als Eintrag auftritt, beschränkte, sogar eine Division.
 \end{enumerate}
Dies lässt sich weitreichend verallgemeinern. Wann immer auf einer Menge $X$ irgendeine Verknüpfung gegeben ist, erhält man durch „komponentenweises Verknüpfen“ eine Verknüpfung auf dem Folgenraum $X^\Nz$.
\end{bsp}





\begin{bsp}[Verkettung von Abbildungen]
 Sei $M$ eine beliebige Menge. Dann ist auf der Menge $\text{Abb}(M,M)$ der Selbstabbildungen von $M$ eine Verknüpfung gegeben durch die Verkettung\footnote{siehe \cref{verkettung}} von Abbildungen:
 \[ \text{Abb}(M,M)\times \text{Abb}(M,M)\to \text{Abb}(M,M) \ ,\ (f,g)\mapsto f\circ g \]
\end{bsp}


\begin{bem}
 Beachte, dass in diesem Beispiel von Selbstabbildungen (also Abbildungen, deren Definitions- und Wertebereich übereinstimmen) die Rede ist. Sind allgemein $A,B,C$ drei Mengen, so hat man zwar eine Abbildung
 \[ \text{Abb}(B,C) \times \text{Abb}(A,B) \to \text{Abb}(A,C) \ ,\ (f,g) \mapsto f\circ g \]
 sofern $A,B,C$ drei verschiedene Mengen sind, ist dies aber \textbf{keine} zweistellige Verknüpfung -- denn auf welcher Menge sollte diese Verknüpfung „leben“, wenn ja $\text{Abb}(B,C)$ und $\text{Abb}(A,B)$ zwei verschiedene Mengen sind? \\[0.5em]
 Nichtsdestotrotz besitzt auch das allgemeine Verketten von Abbildungen eine interessante, häufig auftretende Struktur, die man „\href{https://ncatlab.org/nlab/show/category}{Kategorie}“ nennt. Mehr darüber wirst du spätestens in fortgeschrittenen Algebra-Vorlesungen kennenlernen.
\end{bem}



\begin{bsp}[Operationen mit Mengen]
 Sei $M$ eine beliebige Menge. Dann gibt es unter Anderem die folgenden Verknüpfungen auf der Potenzmenge $\mathcal{P}(M)$:
 \begin{align*}
  \cap : \mathcal{P}(M)\times \mathcal{P}(M) \to \mathcal{P}(M) \ & ,\ (A,B)\mapsto A\cap B \\
    \cup : \mathcal{P}(M)\times \mathcal{P}(M) \to \mathcal{P}(M) \ &,\ (A,B)\mapsto A\cup B \\
\setminus : \mathcal{P}(M)\times \mathcal{P}(M) \to \mathcal{P}(M) \ &,\ (A,B)\mapsto A\setminus B
 \end{align*}
 Denn der Durchschnitt / die Vereinigung / die Differenzmenge zweier Teilmengen von $M$ ist ebenfalls eine Teilmenge von $M$.
\end{bsp}





\begin{bsp}[* Kleineres und Größeres zweier Elemente] \label{minmax}
 Sei $X$ eine beliebige totalgeordete Menge (z.B. $X=\Rz$ mit der gewöhnlichen Ordnung). Indem man von je zwei Elementen das kleinere bzw. größere auswählt, erhält man zwei Verknüpfungen auf $X$:
 \begin{align*}
  \min : X\times X\to X\ &,\ (x,y)\mapsto \min\{x,y\} \\
  \max : X\times X\to X\ &,\ (x,y)\mapsto \max\{x,y\}
 \end{align*}
Eine Verallgemeinerung sowohl dieses Beispiels als auch des Beispiels mit „$\cap$“ und „$\cup$“ stellen sogenannte „\href{https://de.wikipedia.org/wiki/Verband_(Mathematik)}{Verbände}“ dar.
\end{bsp}




\begin{bem}
Alle bisher beschriebenen Verknüpfungen besaßen ein eigenes Verknüpfungssymbol und waren relativ übersichtlich aufzuschreiben. Allgemeine Verknüpfungen auf einer Menge $X$ dürfen aber beliebig kompliziert sein. Es muss sich ja lediglich um \emph{irgendeine} Abbildung $X\times X\to X$ handeln, die beliebig chaotisch sein darf und keinem Muster gehorchen muss. Neben Addition und Multiplikation gibt es auf $\Nz$ unendlich viele weitere Verknüpfungen, von denen die meisten wohl niemals von mathematischem Interesse sein werden.
\end{bem}





\begin{bem}[Verknüpfungssymbole]
 In diesem Text werden wir, wenn wir über eine „allgemeine zweistellige Verknüpfung“ schreiben, die Verknüpfung mit einem „$*$“ notieren. Die vorigen Beispiele zeigen, dass konkrete Verknüpfungen auch mit ganz anderen Zeichen wie etwa $+,-,\cdot,:,\cap,\cup$ usw. notiert werden. Andere Bücher und Vorlesungen verwenden auch andere Symbole wie etwa „$\odot$“ oder „$\circ$“, um über „die allgemeine Verknüpfung“ zu reden. Oftmals schreiben Bücher auch gar kein Verknüpfungssymbol auf. Sind $X$ eine Menge mit einer zweistelligen Verknüpfung und $a,b\in X$ zwei Elemente, so schreiben sie einfach „$ab$“ für deren Verknüpfung, so wie du es aus der Schule auch schon von der Multiplikation zweier Zahlvariablen kennst. %\\[0.5em]
 %Fortgeschrittene Bücher notieren eine „allgemeine zweistellige Verknüpfung“ auch mit einem Multiplikationspunkt „$\cdot$“, was wir hier bewusst vermeiden, um die Verwechslungsgefahr mit der bekannten Multiplikation von Zahlen auszuschließen. \emph{Eine zweistellige Verknüpfung braucht nichts mit „Zahlen“ zu tun haben}! Oftmals schreiben solche Bücher auch gar kein Verknüpfungssymbol auf. 
\end{bem}





\section{Assoziativ- und Kommutativgesetz}


\begin{de}
 Seien $X$ eine Menge und $*$ eine Verknüpfung auf $X$. Die Verknüpfung $*$ heißt
 \begin{itemize}
  \item \textbf{assoziativ}, falls für alle $x,y,z\in X$ das sogenannte \emph{Assoziativgesetz} gilt:
  \[ x*(y*z) = (x*y)*z \]
    \item \textbf{kommutativ}, falls für alle $x,y\in X$ das sogenannte \emph{Kommutativgesetz} gilt:
  \[ x*y = y*x \]
 \end{itemize}
\end{de}





\begin{bsp}[Grundrechenarten]
 Es gilt:
 \begin{enumerate}[a)]
  \item Auf $\Nz,\Zz,\Qz,\Rz,\Cz$ sind die Addition $+$ und die Multiplikation $\cdot$ sowohl assoziativ als auch kommutativ. Denn für alle natürlichen/ganzen/rationalen/reellen/komplexen Zahlen $x,y,z$ gilt bekanntlich
  \begin{align*}
   x+(y+z)& = (x+y)+z  \\
   x+y & = y+x \\
  x\cdot (y\cdot z) & = (x\cdot y)\cdot z  \\
   x\cdot y & = y\cdot x
  \end{align*}
  \item Die Subtraktion auf $\Zz,\Qz,\Rz,\Cz$ ist weder assoziativ noch kommutativ. Beispielsweise ist
  \begin{align*}
   3-(2-1) &\neq  (3-2)-1  \\
  3-2 &\neq 2-3
  \end{align*}
\item Die Division auf $\Qz\setminus \{0\},\Rz\setminus \{0\},\Cz\setminus \{0\}$ ist weder assoziativ noch kommutativ. Beispielsweise ist
\begin{align*}
 2:(3:2) = 4:3 &\neq 1:3 = (2:3):2 \\
2 : 3 & \neq 3:2
\end{align*}
 \end{enumerate}
\end{bsp}


\begin{comment}
\begin{bsp}
 Die Grundrechenarten sind vergleichbar simple Verknüpfungen. Allgemeine Verknüpfungen dürfen aber beliebig kompliziert sein:
 \begin{itemize}
  \item Die Abbildung
  \[ \Rz_{>0}\times \Rz_{>0} \to \Rz_{>0} \ ,\ (x,y)\mapsto \frac{2xy}{x+y} \]
  ist eine zweistellige Verknüpfung auf $\Rz_{>0}$.
  \item Das \textbf{arithmetische Mittel} (umgangssprachlich „Durchschnitt“)
    \[ \Rz \times \Rz \to \Rz \ ,\ (x,y)\mapsto \frac{x+y}{2} \]
  ist eine zweistellige Verknüpfung auf $\Rz$.
 \end{itemize}
\end{bsp}
\end{comment}


\begin{bsp}[Verketten von Abbildungen] \label{verkett}
Sei $M$ eine beliebige Menge. Dann ist die Verkettung von Abbildungen eine assoziative Verknüpfung auf $\text{Abb}(M,M)$, die aber, sofern $M$ mindestens zwei Elemente enthält, nicht kommutativ ist. 
\end{bsp}
\begin{bew}[(*)]
 (Assoziativität) Dies wurde in \cref{abbass} bewiesen. \\[0.5em]
 (Nicht-Kommutativität) Die Menge $M$ enthalte mindestens zwei verschiedene Elemente $a,b\in M$. Betrachte die beiden konstanten Abbildungen
 \begin{align*}
  f_a : M\to M \ &,\ x\mapsto a \\
  f_b : M\to M \ &,\ x\mapsto b  
 \end{align*}
d.h. $f_a$ bildet jedes Element von $M$ auf $a$ ab und $f_b$ bildet alles auf $b$ ab. Dann gilt:
\begin{align*}
 (f_a\circ f_b)(a) = f_a(f_b(a)) = f_a(b)=a \neq b = f_b(a)=f_b(f_a(a))=(f_b\circ f_a)(a)
\end{align*}
Also ist $f_a\circ f_b\neq f_b\circ f_a$, sodass die Verknüpfung „$\circ$“ nicht kommutativ ist. \qed
\end{bew}






\begin{bsp}[Das Verketten von Abbildungen ist nicht kommutativ!]
 Hier ist ein weiteres, weniger abstraktes Beispiel dafür, dass das Verketten von Abbildungen im Allgemeinen nicht kommutativ ist: \\
 Betrachte die beiden Abbildungen
 \begin{align*}
  f: \Rz \to \Rz \ &,\ x\mapsto 2\cdot x \\
  g  : \Rz \to \Rz \ &,\ x\mapsto x^2
 \end{align*}
Dann gilt für jedes $x\in \Rz$:
\begin{align*}
(g\circ f)(x) & = (2\cdot x)^2 \\
(f\circ g)(x) & = 2\cdot x^2
\end{align*}
und es ist beispielsweise
\[ (g\circ f)(3)  = 36 \neq 18 = (f\circ g)(3) \]
sodass $g\circ f\neq f\circ g$ ist. Es macht einen Unterschied, ob ich erst verdoppele und dann quadriere oder erst quadriere und dann verdopple.
\end{bsp}



\begin{bsp}[Mengenverknüpfungen]
 Ist $M$ eine beliebige Menge, so sind die beiden Verknüpfungen $\cap$ und $\cup$ auf $\mathcal{P}(M)$ sowohl assoziativ als auch kommutativ. Die Operation $\setminus$ ist aber, sofern $M$ nichtleer ist, weder assoziativ noch kommutativ.
\end{bsp}
\begin{bew}
Dies ist der Inhalt von \cref{capcupgesetze}.
\end{bew}




\begin{bem}[Die tiefere Bedeutung des Assoziativgesetzes]
\label{klammerfrei}
Sei $X$ eine Menge mit einer assoziativen Verknüpfung $*$. Der tiefere Grund dafür, dass das Assoziativgesetz so eine wichtige Rolle in der Uni-Mathematik spielt, ist, dass man bei einer assoziativen Verknüpfung keine Klammern setzen braucht. Das Assoziativgesetz selbst
 \[ (x*y)*z = x*(y*z) \qquad x,y,z\in X\]
 besagt schonmal, dass bei solchen Termen, die nur drei Elemente involvieren, jede Art von Klammernplatzierung auf dasselbe Ergebnis hinausläuft. Daher kann man auch einfach
 \[ x*y*z \]
 schreiben. Mit fortgeschrittenen Techniken lässt sich beweisen, dass bei assoziativen Verknüpfungen sogar in Termen mit beliebig vielen Elementen jede Art von Klammerung auf dasselbe Ergebnis hinausläuft. Beispielsweise gilt für $a,b,c,d,e\in X$:
 \begin{align*}
  (a*(b*c))*(d*e) & =((a*b)*c)*(d*e) && (\text{Assoziativgesetz für $a$, $b$ und $c$})\\
& = (((a*b)*c)*d)*e && (\text{Assoziativgesetz für $(a*b)*c$, $d$ und $e$})  \\
  & = ((a*b)*(c*d))*e && (\text{Assoziativgesetz für $a*b$, $c$ und $d$})\\
  &=\ \text{usw.}
 \end{align*}
 Daher kann man bei assoziativen Verknüpfungen ganz allgemein überall die Klammern weglassen und in diesem Fall einfach
 \[ a*b*c*d*e \]
 schreiben. \\
 Aus der Schule bist du es ja auch gewohnt, einfach
 \[ 1+3+2+4 \qquad\text{anstelle von}\qquad (1+3)+(2+4)\ \text{oder}\ (1+(3+2))+4 \]
 zu schreiben und daran ändert sich auch an der Uni nichts. Sind beispielsweise $A,B,C,D$ vier Mengen, so schreibt man schlicht
 \[ A\cup B\cup C \cup D \]
 für deren Vereinigung, was unproblematisch ist, da $\cup$ eine assoziative Verknüpfung ist. Sind $f:A\to B$, $g:B\to C$, $h:C\to D$ drei Abbildungen, so schreibt man schlicht
 \[ h\circ g\circ f \]
 für deren Verkettung, wobei man auch hier wegen der Assoziativität keine Klammern setzen muss. \\[0.5em]
 Aber Achtung: Bei nicht-assoziativen Verknüpfungen kannst du in der Regel nicht auf Klammerung verzichten. Beispielsweise ergäbe der Ausdruck
 \[ \Rz\setminus \Qz\setminus \Rz_{>0}\]
keinen Sinn, solange du nicht klarstellst, ob
 \[ (\Rz\setminus \Qz)\setminus \Rz_{>0} \qquad\text{oder}\qquad \Rz\setminus (\Qz\setminus \Rz_{>0}) \]
 gemeint ist.
\end{bem}




\begin{comment}
\begin{bsp}[* Minimum und Maximum]
 Ist $X$ eine totalgeordnete Menge, so sind die beiden Verknüpfungen $\min$ und $\max$ aus \cref{minmax} sowohl assoziativ als auch kommutativ.
\end{bsp}
\end{comment}






\begin{comment}
\begin{bsp}[*]
 Auf jeder beliebigen Menge $X$ ist durch
 \begin{align*}
 x*y &:= y && x,y\in X  
 \end{align*}
eine Verknüpfung gegeben. Diese Verknüpfung ist assoziativ, aber, sofern $M$ mindestens zwei verschiedene Elemente enthält, nicht kommutativ.
\end{bsp}
\begin{bew}
 (Assoziativität) Seien $x,y,z\in M$ drei beliebige Elemente. Dann ist
 \[ x*(y*z) = x*z = z = (x*y)*z \]
 Also ist $*$ eine assoziative Verknüpfung. \\[0.5em]
 (Nicht-Kommutativität) Die Menge $M$ enthalte mindestens zwei verschiedene Elemente $a,b\in M$. Dann ist
 \[ a*b = b\neq a = b*a \]
 sodass die Verknüpfung nicht kommutativ ist. \qed
\end{bew}
\end{comment}




\begin{bsp}[Rechnen mit Rundungsfehlern] \label{fehlerrech}
So gut wie alle Verknüpfungen in den ersten Semestern Mathematikstudium sind assoziativ. Ein für die Informatik wichtiges Beispiel für eine nicht-assoziative Verknüpfung ist die „fehlerbehaftete Multiplikation“. Da ein Computer eine Zahl nicht mit beliebig vielen Nachkommastellen speichern kann, muss er nach solchen Rechenschritten, die die Anzahl der Nachkommastellen übers Maximum erhöhen würde, die letzte verfügbare Nachkommastelle runden. Für ein vereinfachtes Beispiel betrachte die Menge
 \[ \frac{1}{10}\Zz := \left\{ x\in \Qz \mid \exists z\in \Zz:\ x=\frac{z}{10} \right\} \]
all derjenigen rationalen Zahlen, die höchstens eine Nachkommastelle besitzen (im Dezimalsystem -- Computer würden im Binärsystem rechnen und erheblich mehr Nachkommastellen einbeziehen). Auf dieser Menge ist folgendermaßen eine zweistellige Verknüpfung „$*$“ gegeben:
\begin{quote}
 Für $a,b\in \frac{1}{10}\Zz$ bilde zuerst das gewöhnliche Produkt rationaler Zahlen $a\cdot b \in \Qz$. Runde dieses Produkt nun auf die erste Nachkommastelle. Diese gerundete Zahl sei $a*b$.
\end{quote}
Für diese Verknüpfung gilt beispielsweise:
\begin{align*}
1,5* 0,5 = 0,8 \qquad 0,1* 0,1 = 0 \qquad 2*3 = 6 \qquad 1,5 * (-0,3)= -0,5 
\end{align*}
Diese „ungenaue Multiplikation“ ist zwar kommutativ, aber nicht assoziativ, da beispielsweise:
\[ (0,1 * 0,1) * 10 = 0*10 = 0 \neq 0,1 = 0,1*1 = 0,1*(0,1*10) \qedhere \]
 \end{bsp}


 
 
 \begin{de}[Potenzen]
  Seien $X$ eine Menge mit einer assoziativen Verknüpfung $*$ und $a\in X$ irgendein Element. Für ein $n\in \Nz_{\geq 1}$ nennt man das Element, das durch $n$-faches Verknüpfen von $a$ mit sich selbst entsteht
  \[ a^n := \underbrace{a * \ldots * a}_{n\text{-mal}} \]
  (beachte, dass hier aufgrund der Assoziativität keine Klammern gesetzt werden müssen) die \textbf{$n$-te Potenz von $a$}. Beispielsweise gilt:
  \begin{align*}
   a^1 & = a \\
   a^2 & = a* a \\
   a^3 & = a*a*a \\
   a^{n+1} & = a^n*a && n\in \Nz_{\geq 1}
  \end{align*}
 \end{de}
 
 
 
 \begin{bem}\label{potlaw}
    Seien $X$ eine Menge mit einer assoziativen Verknüpfung $*$ und $a,b\in X$ mit $a*b=b*a$. Für $n,m\in \Nz_{\geq 1}$ gelten die folgenden \emph{Potenzgesetze}:
    \begin{align*}
    a^1 & = a \\
     a^{n+m} & = a^n* a^m \\
     (a^m)^n & = a^{n\cdot m} \\
    (a*b)^n &= a^n*b^n
    \end{align*}
Beachte, dass für die letzte Gleichung wichtig ist, dass $a*b=b*a$ gilt. Nur so kann man beispielsweise ohne Weiteres die Umformungen
\[ (a*b)^2 = (a*b)*(a*b) = a*(b*a)*b \overset{!}{=} a*(a*b)*b = (a*a)*(b*b) = a^2*b^2 \] 
durchführen.
 \end{bem}
\begin{bew}
 Diese Gleichungen beweist man komfortabel mit einem sogenannten \href{https://de.wikipedia.org/wiki/Vollst%C3%A4ndige_Induktion}{Induktionsbeweis}. Da diese Beweistechnik im Vorkurs nicht behandelt wurde, könntest du noch abwarten, bis sie in den ersten beiden Semesterwochen durchgenommen wurde und daraufhin nochmal hierher zurückkehren.
\end{bew}



 
 \section{Neutrales Element}


\begin{de}[Neutrales Element] \label{neutrales}
Seien $X$ eine Menge und $*$ eine zweistellige Verknüpfung auf $X$. Ein Element $e\in X$ heißt \textbf{neutrales Element}, falls für jedes $x\in X$ die folgenden beiden Gleichungen gelten:
\begin{align*}
 e*x & = x && (\text{Linksneutralität}) \\
 x*e & = x && (\text{Rechtsneutralität})
\end{align*}
\end{de}




\begin{bsp} \label{neutralbsp}
Es sei $M$ eine beliebige Menge. Dann gilt:
 \begin{enumerate}[a)]
  \item Die Addition auf $\Nz_0,\Zz,\Qz,\Rz$ bzw. $\Cz$ hat die Zahl Null als neutrales Element. Denn für jede natürliche/ganze/rationale/reelle/komplexe Zahl $x$ gilt ja
  \[ x+0=0+x=x \]
  \item Die Multiplikation auf $\Nz,\Zz,\Qz,\Rz$ bzw. $\Cz$ hat die Zahl Eins als neutrales Element. Denn für jede natürliche/ganze/rationale/reelle/komplexe Zahl $x$ ist
  \[ x\cdot 1=1\cdot x = x \]
  \item Das Verketten von Abbildungen aus $\text{Abb}(M,M)$ hat die Identität $\iden_M$ als neutrales Element. Denn für jede beliebige Abbildung $f:M\to M$ gilt
  \[ \iden_M\circ f = f\circ \iden_M = f \]
  was in \cref{idneutral} bewiesen wurde.
  \item Bezüglich der Verknüpfung $\cap$ hat $\mathcal{P}(M)$ das neutrale Element $M$. Denn es gilt:
  \begin{align*}
   A\cap M & = M\cap A = A && \text{für jede Teilmenge}\ A\subseteq M
  \end{align*}
  \item Bezüglich der Verknüpfung $\cup$ hat $\mathcal{P}(M)$ das neutrale Element $\emptyset$. Denn für jede Teilmenge $A\subseteq M$ gilt
  \[ A\cup \emptyset = \emptyset \cup A = A \]
  \item Die fehlerbehaftete Multiplikation aus \cref{fehlerrech} hat die $1$ als neutrales Element. Denn wenn ich eine rationale Zahl, die höchstens eine Nachkommastelle besitzt, mit $1$ multipliziere, ändert sich nichts, sodass auch das nachfolgende Runden nichts am Zahlenwert ändert.
  \item Hinsichtlich der Subtraktion auf $\Zz,\Qz,\Rz$ bzw. $\Cz$ gibt es \emph{kein} neutrales Element. Denn die einzige Zahl $e\in \Cz$, die
  \begin{align*}
    x-e&=x && \text{für alle}\ x\in \Zz 
  \end{align*}
  erfüllt, ist die $0$. Also käme höchstens die $0$ als neutrales Element in Betracht. Aber z.B. wegen
  \[ 0-5 \neq 5 \]
  ist die $0$ kein neutrales Element.
  \item Hinsichtlich der Division auf $\Qz\setminus \{0\},\Rz\setminus \{0\},\Cz\setminus \{0\}$ gibt es ebenfalls kein neutrales Element.
  \item Sofern $M$ eine nichtleere Menge ist, gibt es bezüglich der Verknüpfung „$\setminus$“ auf $\mathcal{P}(M)$ auch kein neutrales Element.
 %\item Hinsichtlich der Verknüpfung
 %\[ M\times M \to M \ ,\ (x,y)\mapsto y \]
 %ist jedes Element von $M$ linksneutral.
 \end{enumerate}
\end{bsp}



\begin{bem}[Beweisarbeit einsparen im kommutativen Fall]
Seien $X$ eine Menge mit einer Verknüpfung $*$ und $e\in X$ ein Element, von dem du vermutest, dass es ein neutrales Element ist. Sofern $*$ eine \emph{kommutative Verknüpfung} ist, brauchst du, um zu beweisen, dass $e$ ein neutrales Element ist, von den beiden Gleichungen
\[ e*x=x \qquad\text{und}\qquad x*e=x \qquad\qquad x\in X \]
nur eine zu beweisen. Die andere folgt dann direkt aus dem Kommutativgesetz.
\end{bem}



\begin{comment}
\begin{bsp}[* Neutrales Element bei $\min$ und $\max$]
 Sei $X$ eine totalgeordnete Menge. Dass ein Element $m\in X$ neutral bezüglich der $\max$-Verknüpfung aus \cref{minmax} ist, hieße, dass für jedes $x\in X$ gelten muss
 \[ \max \{m,x\} = x \]
 was wiederum äquivalent zu $m\leq x$ ist. Also ist ein Element genau dann neutral bezüglich $\max$, wenn es das kleinste Element von $X$ ist (sofern eines existiert). Beispielsweise wäre $0$ neutral bezüglich der $\max$-Verknüpfung auf $\Nz_0$ -- dagegen gäbe es bezüglich der $\max$-Verknüpfung auf $\Zz$ kein neutrales Element. \\
 Analog ist ein Element genau dann neutral bezüglich der $\max$-Verknüpfung, wenn es das größte Element von $X$ ist.
\end{bsp}
\end{comment}




\begin{sat}[Eindeutigkeitssatz für neutrale Elemente] \label{neutreind}
Seien $X$ eine Menge und $*$ eine Verknüpfung auf $X$, die ein neutrales Element $e\in X$ besitzt. Dann ist $e$ auch das einzige neutrale Element.
\end{sat}
\begin{bew}
Es sei $d\in X$ ein beliebiges neutrales Element bezüglich der Verknüpfung $*$. Dann gilt:
\begin{align*}
 d & = d*e && (\text{weil $e$ neutrales Element}) \\
 & = e && (\text{weil $d$ neutrales Element})
\end{align*}
Also gibt es neben $e$ keine weiteren neutralen Elemente. \qed
\end{bew}




\begin{bem}[\textbf{Das} neutrale Element]
Der Eindeutigkeitssatz berechtigt uns, beim Vorhandensein eines neutralen Elements statt von „einem neutralen Element“ von \emph{dem} neutralen Element zu reden. \\
An diesem Satz wird vielleicht deutlich, wie vorteilhaft die axiomatische Arbeit mit abstrakten Verknüpfungen sein kann. Denn er garantiert uns auf einen Schlag, dass die neutralen Elemente aus allen Beispielen in \cref{neutralbsp} auch jeweils die einzigen neutralen Elemente sind, ohne dass wir dies in jedem Fall einzeln beweisen müssten.
\end{bem}







\section{Monoide}



\begin{de}[Monoid]
 Ein \textbf{Monoid} ist ein Paar $(M,*)$ bestehend aus einer Menge $M$ und einer Verknüpfung $*$ auf $M$, für das gilt:
 \begin{enumerate}[(M1)]
  \item $*$ ist eine assoziative Verknüpfung.
  \item $M$ enthält ein neutrales Element (bezüglich der Verknüpfung $*$).
 \end{enumerate}
In diesem Fall ist das neutrale Element nach \cref{neutreind} automatisch eindeutig bestimmt. \\
 Ist überdies die Verknüpfung $*$ auch noch kommutativ, so nennt man $(M,*)$ ein \textbf{kommutatives Monoid}.
\end{de}





\begin{bsp} Es gilt:
\begin{enumerate}[a)]
 \item(Addition von Zahlen) $(\Nz_0,+),(\Zz,+),(\Qz,+),(\Rz,+),(\Cz,+)$ sind jeweils kommutative Monoide. Denn die Addition ist assoziativ und kommutativ und die Zahl $0$ ist ihr neutrales Element.
 \item(Multiplikation von Zahlen) $(\Nz,\cdot),(\Zz,\cdot),(\Qz,\cdot),(\Rz,\cdot),(\Cz,\cdot)$ sind jeweils kommutative Monoide. Denn die Multiplikation ist assoziativ und kommutativ und die Zahl $1$ ist ihr neutrales Element.
 \item Die Subtraktion und die Division liefern keine Monoide, weil sie nicht assoziativ sind. Und ein neutrales Element besitzen sie ja auch nicht.
 \item Ist $X$ eine beliebige Menge, so ist $(\text{Abb}(X,X),\circ)$ ein Monoid mit neutralem Element $\iden_X$. Sofern $X$ mindestens zwei verschiedene Elemente enthält, ist es aber nicht kommutativ, siehe \cref{verkett}
 \item Ist $X$ eine beliebige Menge, so sind $(\mathcal{P}(X),\cup)$ und $(\mathcal{P}(X),\cap)$ zwei kommutative Monoide mit den jeweiligen neutralen Elementen $\emptyset$ und $X$. Sofern $X$ nichtleer ist, ist $(\mathcal{P}(X),\setminus)$ aber kein Monoid, da es weder assoziativ ist noch ein neutrales Element enthält.
 %\item Ist $X$ eine totalgeordnete Menge, so sind $(X,\min)$ und $(X,\max)$ zwei kommutative Monoide.
 \item Die „ungenaue Multiplikation“ aus \cref{fehlerrech} besitzt zwar ein neutrales Element -- sie liefert aber kein Monoid, weil sie nicht assoziativ ist.
 \item Die Addition „$+$“ ist zwar eine assoziative Verknüpfung auf der Menge $\Nz_{\geq 1}$, aber $(\Nz_{\geq 1},+)$ ist kein Monoid, da es kein neutrales Element besitzt.
\end{enumerate}
\end{bsp}





\begin{bem}[Trägermenge]
 Beachte, dass ein Monoid immer ein Paar $(M,*)$ ist, in das sowohl die „Trägermenge“ $M$ als auch die Verknüpfung $*$ kodiert ist. Ein und dieselbe Menge kann durchaus als Trägermenge für verschiedene Monoide herhalten. Beispielsweise sind $(\Nz_0,+)$ und $(\Nz_0,\cdot)$ zwei verschiedene Monoide, die dennoch dieselbe Trägermenge $\Nz_0$ besitzen. \\
 Andererseits gibt es Mengen mit „kanonischen“ Verknüpfungen, wie z.B. $\text{Abb}(X,X)$ (wobei $X$ irgendeine Menge ist). Sprechen Mathematiker von „dem Monoid $\text{Abb}(X,X)$“, so meinen sie damit grundsätzlich das Monoid $(\text{Abb}(X,X),\circ)$, also $\text{Abb}(X,X)$ mit der Verkettung von Abbildungen als Verknüpfung. Solche Konventionen wirst du mit der Zeit durch Erfahrung und Gewohnheit verinnerlichen.
\end{bem}





\begin{bem}[Die tiefere Bedeutung des Kommutativgesetzes]
Es sei $(M,*)$ ein kommutatives Monoid. Weil dann $*$ assoziativ ist, brauchen wir nach \cref{klammerfrei} keine Klammern setzen. Sind beispielsweise $a,b,c,d\in M$, so können wir einfach
\[ a*b*c*d \]
schreiben. Das Kommutativgesetz
\begin{align*}
  x*y&=y*x && x,y\in M
\end{align*}
sagt nun aus, dass es bei der Verknüpfung \emph{zweier} Elemente nicht auf die Reihenfolge ankommt. Es lässt sich zeigen, dass es damit sogar bei der Verknüpfung beliebig vieler Elemente nicht auf die Reihenfolge ankommt. Beispielsweise gilt
\begin{align*}
 a*b*c*d & = a*c*b*d && (\text{Kommutativgesetz für $b$ und $c$}) \\
 & = c*a*b*d && (\text{Kommutativgesetz für $a$ und $c$}) \\
 & = b*d*c*a && (\text{Kommutativgesetz für $c*a$ und $b*d$}) \\
 & \text{usw.}
\end{align*}
Bei kommutativen Monoiden brauchst du also weder aufs Klammernsetzen, noch auf die Reihenfolge, in der du die Elemente verknüpfst, achten.
\end{bem}



\begin{de}[Nullte Potenz]
 Seien $(M,*)$ ein Monoid mit neutralem Element $e\in M$ und $a\in M$ irgendein Element. In \cref{potlaw} wurden bereits die $n$-ten Potenzen von $a$ für $n\in \Nz_{\geq 1}$ definiert. In Monoiden lässt sich eine „nullte Potenz“ definieren durch
 \[ a^0 := e \]
Mit dieser Definition sind die Potenzgesetze aus \cref{potlaw} allgemeiner für alle $n,m\in \Nz_0$ gültig.
\end{de}



\begin{bem}[„multiplikativ geschriebene“ Monoide]
 In fortgeschrittenen Büchern und Vorlesungen wird als Verknüpfungszeichen für ein „allgemeines Monoid $M$“ oftmals der Malpunkt „$\cdot$“ verwendet. Wir haben dies hier vermieden, damit du nicht auf die Idee kämest, ein Monoid müsse unbedingt etwas mit Zahlen und deren Multiplikation zu tun haben. \\
 Das neutrale Element notieren solche Texte dann oft mit dem Zeichen „$1$“ und nennen es das „Einselement“ des Monoids $M$. In dieser Notation gilt also
 \begin{align*}
  1\cdot x & = x \\
  x \cdot 1 & = x && x\in M
 \end{align*}
 Beachte, dass es sich hier trotz Verwendung des Zeichens „$1$“ \textbf{nicht} um die natürliche Zahl Eins handelt! Um hervorzuheben, dass es sich nicht um die Zahl Eins, sondern um das neutrale Element des Monoids $M$ handelt, kann man anstelle von „$1$“ auch „$1_M$“ schreiben.
\end{bem}





\section{Inverse Elemente}



\begin{de}[Inverse Elemente] \label{inverse}
Sei $X$ eine Menge mit einer zweistelligen Verknüpfung $*$, die ein (nach \cref{neutreind} automatisch eindeutig bestimmtes) neutrales Element $e$ besitzt. Ein Element $b\in M$ heißt \textbf{invers} zu $a$, falls es die folgenden beiden „Inversengleichungen“ erfüllt:
\begin{align*}
 b*a & = e && (\text{„$b$ ist linksinvers zu $a$“})\\
 a*b & = e && (\text{„$b$ ist rechtsinvers zu $a$“})
\end{align*}
Das Element $a$ heißt \textbf{invertierbar} (oder auch: \emph{Einheit}\footnote{Abgesehen von der Benennung hat das aber erstmal nichts mit den „Einheiten“ aus der Physik, wie z.B. „Meter“ und „Kilogramm“, zu tun.}), falls es mindestens ein zu $a$ inverses Element in $X$ gibt.
\end{de}



\begin{bem}[Keine Inversen ohne Neutrales]
Beachte, dass es nur bei Vorhandensein eines neutralen Elements überhaupt Sinn ergibt, von inversen Elementen zu sprechen. 
\end{bem}




\begin{bsp}[Addition und Multiplikation]
 Es gilt:
 \begin{enumerate}[a)]
  \item In den Monoiden $(\Zz,+),(\Qz,+),(\Rz,+),(\Cz,+)$ ist jedes Element invertierbar. Denn für jede ganze/rationale/reelle/komplexe Zahl $x$ gilt
  \begin{align*}
   x + (-x) & = 0 \\
   (-x) + x & = 0
  \end{align*}
  sodass $-x$ invers zu $x$ ist.
  \item Das einzige invertierbare Element im Monoid $(\Nz_0,+)$ ist die $0$. Denn als neutrales Element ist die $0$ invertierbar und für jede Zahl $n\in \Nz_{\geq 1}$ kann es kein $m\in \Nz_0$ mit $n+m=0$ geben.
  \item \label{inversebsp} In den Monoiden $(\Qz,\cdot),(\Rz,\cdot),(\Cz,\cdot)$ ist jedes Element $\neq 0$ invertierbar. Denn für jede rationale/reelle/komplexe Zahl $x\neq 0$ gilt
  \begin{align*}
   x \cdot \frac{1}{x} & = 1 \\
   \frac{1}{x}\cdot x & = 1
  \end{align*}
  sodass $\frac{1}{x}$ invers zu $x$ ist. Die Null ist dagegen nicht invertierbar. Denn für jede beliebige Zahl $x$ ist
  \[ 0\cdot x = 0 \neq 1 \]
 \end{enumerate}
\end{bsp}




\begin{bem}[Beweisarbeit einsparen im kommutativen Fall]
 Sei $X$ eine Menge mit einer Verknüpfung $*$, die ein neutrales Element $e\in X$ besitzt. Außerdem seien $a,b\in X$ und du vermutest, dass $b$ invers zu $a$ ist. Sofern $*$ eine \emph{kommutative} Verknüpfung ist, brauchst du, um zu beweisen, dass $b$ invers zu $a$ ist, von den beiden Gleichungen
 \[ b*a=e\qquad\text{und}\qquad a*b=e \]
 nur eine zu beweisen. Die andere folgt dann aus dem Kommutativgesetz.
\end{bem}





\begin{bem}[Das neutrale Element ist selbstinvers] \label{selbstinvers}
 Sei $X$ eine Menge mit einer zweistelligen Verknüpfung $*$, die ein neutrales Element $e\in X$ besitzt. Dann ist $e$ invertierbar und ein inverses Element zu sich selbst.
\end{bem}
\begin{bew}
 Da $e$ ein neutrales Element ist, gilt
 \[ e*e = e \]
 Und diese Gleichung entspricht beiden Inversengleichungen aus \cref{inverse} zugleich. \qed
\end{bew}




\begin{bsp}[Inverse Abbildungen]
Es gilt:
\begin{enumerate}[a)]
 \item Betrachte das Monoid $\text{Abb}(\Rz_{\geq 0},\Rz_{\geq 0})$ und darin die beiden Abbildungen
 \begin{align*}
  f : \Rz_{\geq 0}\to \Rz_{\geq 0} \ &,\ x\mapsto x^2 \\
  g: \Rz_{\geq 0}\to \Rz_{\geq 0} \ &,\ x\mapsto \sqrt{x}
 \end{align*}
Dann ist $g$ invers zu $f$.
\begin{bew}
 Für jedes $x\in \Rz_{\geq 0}$ ist
 \begin{align*}
  (f\circ g)(x) & = \sqrt{x}^2 =x = \iden_{\Rz_{\geq 0}}(x) \\
  (g\circ f)(x) & = \sqrt{x^2} = x = \iden_{\Rz_{\geq 0}}(x)
 \end{align*}
sodass bereits die Gleichheit von Abbildungen $f\circ g=\iden_{\Rz_{\geq 0}}$ und $g\circ f=\iden_{\Rz_{\geq 0}}$ gilt. Also ist $g$ invers zu $f$. \qed
\end{bew}
\item Betrachte das Monoid $\text{Abb}(\Rz,\Rz)$ und darin die Abbildung
\[ f : \Rz\to \Rz \ ,\ x\mapsto x^2 \]
Dann besitzt $f$ keine Inverse. Denn nach \cref{invkriterium} müsste dann $f$ eine bijektive Abbildung sein; aber wegen $f(1)=f(-1)$ ist $f$ nicht einmal injektiv.
\item Betrachte das Monoid $\text{Abb}(\Nz_0,\Nz_0)$ und darin die beiden Abbildungen
 \begin{align*}
  f : \Nz_0 \to \Nz_0 \ &,\ n \mapsto n+1 \\
  g: \Nz_0 \to \Nz_0 \ &,\ n\mapsto \begin{cases}
                                     0 & n=0 \\
                                     n-1 & n\geq 1
                                    \end{cases}
 \end{align*}
Dann gilt zwar $g\circ f=\iden_{\Nz_0}$. Allerdings ist $f\circ g\neq \iden_{\Nz_0}$, sodass $g$ nicht invers zu $f$ ist. Tatsächlich kann $f$ gar keine Inverse besitzen, da $f$ wegen $0\notin \im(f)$ nicht surjektiv, also erst recht nicht bijektiv, ist.
\end{enumerate}
\end{bsp}




\begin{sat}[Eindeutigkeitssatz für inverse Elemente] \label{inveind}
Seien $(M,*)$ ein Monoid und $a\in M$ ein invertierbares Element. Dann gibt es auch nur genau ein inverses Element zu $a$.\footnote{vgl. \cref{umkehreind}}
\end{sat}
\begin{bew}
 Es seien $e\in M$ das neutrale Element von $M$ und $b,b'\in X$ zwei beliebige Inverse zu $a$. Dann gilt:
 \begin{align*}
  b & = b*e && (\text{da $e$ neutral ist}) \\
  & = b* a*b' && (\text{da $b'$ invers zu $a$ ist}) \\
  %& = a'*a*a'' && (\text{Assoziativgesetz}) \\
  & = e*b' && (\text{da $b$ invers zu $a$ ist}) \\
  & = b' && (\text{da $e$ neutral ist}) \qed
 \end{align*}
\end{bew}




\begin{bem}[\textbf{Das} inverse Element]
  Seien $(M,*)$ ein Monoid und $a\in M$ ein invertierbares Element. Da dann $a$ auch nur genau ein Inverses besitzt, ergibt es Sinn, anstelle von „einem Inversen zu $a$“ von \emph{dem} Inversen von $a$ zu sprechen.
\end{bem}



\begin{bem}
 Beachte, dass wir im Beweis von \cref{inveind} implizit Gebrauch vom Assoziativgesetz gemacht haben (erkennst du, wo?). Bei einer nicht-assoziativen Verknüpfung wie z.B. der fehlerbehafteten Multiplikation aus \cref{fehlerrech}, die sowohl kommutativ ist als auch ein neutrales Element besitzt, brauchen Inverse nicht eindeutig sein. Beispielsweise gilt dort
 \begin{align*}
  0,4*2,5 & = 2,5*0,4 = 1 \\
  0,4 * 2,6 & = 2,6*0,4 = 1
 \end{align*}
sodass das Element $0,4$ mindestens zwei verschiedene Inverse besitzt, nämlich $2,5$ und $2,6$.
\end{bem}




\begin{de}[negative Potenzen]
 Seien $(M,*)$ ein Monoid mit neutralem Element $e\in M$ und $a\in M$ ein invertierbares Element mit Inversem $h\in M$. Für $n\in \Zz$ setzt man in Verallgemeinerung der bisher definierten Potenzen
 \begin{align*}
  a^n & := \begin{cases}
            \underbrace{a* \ldots * a}_{n\text{-mal}} & n \geq 1 \\
            e & n= 0 \\
            \underbrace{h * \ldots * h}_{-n\text{-mal}} & n \leq -1
           \end{cases}
 \end{align*}
Mit dieser Definition gelten in Verallgemeinerung von \cref{potlaw} für alle invertierbaren Elemente $a,b\in M$ mit $a*b=b*a$ folgende Potenzgesetze:
\begin{align*}
 a^1 & = a \\
 a^0\ & \text{ist das neutrale Element}  \\
 a^{-1}\ & \text{ist die Inverse zu $a$} \\
 a^{n+m} & = a^n*a^m \\
 (a^m)^n & = a^{n\cdot m} \\
 (a*b)^n & = a^n*b^n && n,m\in \Zz
\end{align*}
Beachte, dass die Voraussetzung $a*b=b*a$ essenziell für die letzte Gleichung ist.
\end{de}




\begin{bem}
 Ab sofort werden wir die Inverse eines invertierbaren Monoidelements $a$ meist mit „$a^{-1}$“ notieren. Viele Lehrbücher tun das auch schon von Anfang an, ohne es mit Potenzen zu motivieren.
\end{bem}







\begin{bem}[„additiv geschriebene“ Monoide]
Als Verknüpfungszeichen \emph{kommutativer} Monoide verwenden Bücher gerne ein „$+$“-Zeichen. Bis jetzt haben wir das vermieden, damit du nicht auf die Idee kommst, es müsse etwas mit der Addition von Zahlen zu tun haben. \\[0.5em]
Sofern die Verknüpfung in einem kommutativen Monoid $M$ mit „$+$“ notiert wird, ändern sich auch einige weitere Schreibweisen. Beispielsweise wird das neutrale Element dann gerne mit dem Zeichen „$0$“ notiert (es braucht aber nichts mit der Zahl Null zu tun haben!) und das „Nullelement von $M$“ genannt. Zur Abgrenzung von der natürlichen Zahl Null schreibt man manchmal auch „$0_M$“. Es gilt dann also
\begin{align*}
 x+0_M & = x && \\
 0_M + x & = x && x\in M
\end{align*}
Überdies wird für ein invertierbares Element $a\in M$ dessen Inverse nicht mit „$a^{-1}$“, sondern mit „$-a$“ notiert. Für $b\in M$ schreibt man dann
\[ b-a \qquad\text{anstelle von}\qquad b + (-a) \]
Die Inversengleichungen aus \cref{inverse} nehmen für $-a$ dann die Gestalt
\begin{align*}
 a-a & = 0 \\
 (-a) + a & = 0
\end{align*}
an, wobei die zweite Gleichung aufgrund der Kommutativität redundant ist. \\
Außerdem wird für ein $n\in \Nz_0$ (bzw. $n\in \Zz$, falls $a$ invertierbar ist) die $n$-te Potenz von $a$ nicht mit „$a^n$“, sondern mit „$n\cdot a$“ notiert. Im Fall $n\geq 1$ gilt also:
\[ n\cdot a := \underbrace{a+\ldots + a}_{n\text{-mal}} \]
Hier ist eine Gegenüberstellung der Potenzgesetze in multiplikativer und in additiver Schreibweise:
\[ \begin{tabular}{c|c r}
Multiplikative Notation & Additive Notation & $a,b\in M$,\ $n,m\in \Nz_0$\\
    \hline 
$a^1 = a$ & $1\cdot a=a$ &\\
$a^0 = 1_M$ & $0\cdot a=0_M$ &\\
$a^{-1} =$ Inverses zu $a$ & $-a =$ Inverses zu $a$& (sofern $a$ invertierbar ist) \\
$a^{n+m} = a^n\cdot a^m$ & $(n+m)\cdot a = n\cdot a + m\cdot a$ & \\
$(a^m)^n = a^{n\cdot m}$ & $n\cdot (m\cdot a) = (n\cdot m)\cdot a$ & \\
$(a\cdot b)^n =a^n\cdot b^n$ & $n\cdot (a+b) = n\cdot a+n\cdot b$ & (sofern $a\cdot b=b\cdot a$ ist)
   \end{tabular} \]
   \begin{comment}
\begin{align*}
 1 \cdot a & = a \\
 0_{\Nz}\cdot a & = 0_M \\
 (-1)\cdot a & = -a && (\text{sofern $a$ invertierbar ist})\\
 (n+m) \cdot a & = n\cdot a + m\cdot a \\
 (n\cdot m) \cdot a & = n\cdot (m\cdot a) \\
 n\cdot (a+b) & = n\cdot a + n\cdot b && a,b\in M,\ n,m\in \Nz_0
\end{align*}
\end{comment}
Beachte, dass es sich hierbei um ein- und dieselbe Sache, nur in zwei verschiedenen Schreibweisen, handelt.
\end{bem}








\begin{sat}[Regel von Hemd und Jacke]
Seien $(M,*)$ ein Monoid und $a,b\in M$ zwei invertierbare Elemente. Dann ist auch $a*b$ invertierbar und es gilt
\[ (a*b)^{-1} = b^{-1} * a^{-1} \]
Beim Invertieren dreht sich also die Reihenfolge um.
\end{sat}
\begin{bew}
 Es gilt
 \begin{align*}
  b^{-1}*a^{-1} * a*b  & =  b^{-1} * e* b && \\
  & = b^{-1} * b \\
  & = e
 \end{align*}
sowie
  \begin{align*}
 a*b* b^{-1}*a^{-1}  & =  a*e*a^{-1} && \\
  & = a*a^{-1} \\
  & = e
 \end{align*}
Insgesamt erfüllt $b^{-1}*a^{-1}$ somit beide Inversengleichungen. \qed
\end{bew}





\begin{bem}
 Dieser Satz wird „Regel von Hemd und Jacke“ genannt aufgrund folgender Analogie: Habe ich mir erst ein Hemd und daraufhin eine Jacke angezogen und möchte mich nun wieder entkleiden -- so muss ich zuerst die Jacke und dann das Hemd ausziehen. \\[0.5em]
Im kommutativen Fall kommt es nicht auf die Reihenfolge an, sodass man dann auch $(a*b)^{-1} = a^{-1}*b^{-1}$ schreiben kann. Insofern ist dir die Regel von Hemd und Jacke gar nicht fremd: schließlich hast du auch in der Schule regelmäßig benutzt, dass:
\begin{align*}
 -(x+y) & = -x -y && x,y\in \Rz \\
 \frac{1}{x\cdot y} &= \frac{1}{x}\cdot \frac{1}{y} && x,y\in \Rz\setminus \{0\}
\end{align*}
\end{bem}




\begin{sat}[Inverses vom Inversen] \label{doppelinvers}
Seien $(M,*)$ ein Monoid und $a\in M$ ein invertierbares Element. Dann ist auch $a^{-1}$ invertierbar und es gilt
\[ (a^{-1})^{-1} = a \]
\end{sat}
\begin{bew}
Sei $e\in M$ das neutrale Element von $M$. Da $a^{-1}$ invers zu $a$ ist, gelten die beiden Gleichungen
\[ a*a^{-1} = e \qquad\text{und}\qquad a^{-1}*a=e \]
Die erste dieser Gleichungen besagt, dass $a$ linksinvers zu $a^{-1}$ ist und die zweite Gleichung besagt, dass $a$ rechtsinvers zu $a^{-1}$ ist. Insgesamt ist also $a$ die Inverse von $a^{-1}$. \qed
\end{bew}


\begin{bem}
 Auch diese Regel ist dir in Spezialfällen aus der Schule bekannt. Schließlich gilt
 \begin{align*}
  -(-x) & = x && x\in \Rz \\
  \frac{1}{\frac{1}{x}} & = x && x\in \Rz\setminus \{0\}
 \end{align*}
\end{bem}





\section{Gruppen}

\begin{de}[Gruppe]
Eine \textbf{Gruppe} ist ein Monoid, in dem jedes Element invertierbar ist. Konkret handelt es sich bei einer Gruppe also um ein Paar $(G,*)$ bestehend aus einer Menge $G$ und einer Verknüpfung $*$ auf $G$, für das die sogenannten \emph{Gruppenaxiome} gelten:
\begin{enumerate}[(G1)]
 \item Die Verknüpfung $*$ ist assoziativ.
 \item $G$ enthält ein neutrales Element (bezüglich der Verknüpfung $*$).
 \item Jedes Element von $G$ ist invertierbar.
\end{enumerate}
Ist überdies die Verknüpfung auch noch kommutativ, so spricht man von einer \textbf{abelschen Gruppe}\footnote{\href{https://de.wikipedia.org/wiki/Niels_Henrik_Abel}{Niels Henrik Abel (1802-1829)}}.
\end{de}





\begin{bsp}
 Es gilt:
 \begin{enumerate}[a)]
  \item $(\Zz,+),(\Qz,+),(\Rz,+),(\Cz,+)$ sind jeweils abelsche Gruppen. Denn die Addition ist assoziativ, kommutativ, besitzt die $0$ als neutrales Element und für jede ganze/rationale/reelle/komplexe Zahl $x$ ist $-x$ ebenfalls eine ganze/rationale/reelle/komplexe Zahl und invers zu $x$.
  \item Das Monoid $(\Nz_0,+)$ ist keine Gruppe, da beispielsweise das Element $5\in \Nz_0$ nicht invertierbar ist.
%  \item $(\Qz\setminus \{0\},\cdot),(\Rz\setminus \{0\},\cdot),(\Cz\setminus \{0\},\cdot)$ sind jeweils abelsche Gruppen.
 % \item Sofern $X$ eine nichtleere Menge ist, sind die Monoide $(\mathcal{P}(X),\cap)$ und $(\mathcal{P}(X),\cup)$ keine Gruppen.
  \item Sofern $X$ eine mindestens zweielementige Menge ist, ist das Monoid $\text{Abb}(X,X)$ keine Gruppe.
\begin{bew}[(*)]
 Seien $a\in X$ irgendein Element und $f:X\to X$ die konstante Abbildung, die alles auf $a$ abbildet. Weil $X$ mindestens zwei Elemente enthält, ist $f$ injektiv und nach \cref{invkriterium} somit auch nicht invertierbar in $\text{Abb}(X,X)$. Also ist $\text{Abb}(X,X)$ keine Gruppe. \qed
\end{bew}
 \end{enumerate}
\end{bsp}



\begin{comment}
\begin{bsp}[*]
	Als ein Beispiel neben den Zahlenräumen betrachten wir einmal die Symmetrien eines Vierecks:
	\begin{figure}[h]
		\begin{tikzpicture}
			
		\foreach \xtext / \xangle / \xshift /\yshift in
			{
			$\iden$ (ändert nichts)/0/-5cm/3cm,
			$r_1$ (Rotation um $90^\circ$)/90/0cm/3cm,
			$r_2$ (Rotation um $180^\circ$)/180/5cm/3cm,
			$r_3$ (Rotation um $270^\circ$)/270/-5cm/-1cm
			}
			{
			\begin{scope}[xshift=\xshift, yshift=\yshift]
				\begin{scope}[xshift=-.5cm,yshift=.5cm,gray,thin]
				\draw (0,0) rectangle (2,2);
				\filldraw (-.1,1.9) rectangle (.1,2.1) (2,2) circle (.1cm);
				\filldraw[fill=white, draw=gray] (2,0) circle (.1cm);
			\end{scope}
			\begin{scope}[thick, rotate around={\xangle:(1,1)}]
				\draw (0,0) rectangle (2,2);
				\filldraw (-.1,1.9) rectangle (.1,2.1) (2,2) circle (.1cm) (2,0);
				\filldraw[fill=white, draw=black] (2,0) circle (.1cm);
			\end{scope}
			\path (1,-.2) node[below] {\xtext};
			\end{scope}
			};
			
		\foreach \xtext / \xscale / \yscale / \xangle / \xshift / \yshift / \xxshift / \yyshift in
			{
				$f_1$ (vertikale Spiegelung)/-1/1/0/0cm/-1cm/-2cm/0cm,
				$f_2$ (horizontale Spiegelung)/1/-1/0/5cm/-1cm/0cm/-2cm,
				$f_3$ (diagonale Spiegelung)/-1/1/270/-2.5cm/-5cm/-2cm/0cm,
				$f_4$ (diagonale Spiegelung)/-1/1/90/2.5cm/-5cm/-2cm/0cm
			}
			{
				\begin{scope}[xshift=\xshift, yshift=\yshift]
					\begin{scope}[xshift=-.5cm,yshift=.5cm,gray,thin]
						\draw (0,0) rectangle (2,2);
						\filldraw (-.1,1.9) rectangle (.1,2.1) (2,2) circle (.1cm);
						\filldraw[fill=white, draw=gray] (2,0) circle (.1cm);
					\end{scope}
					\begin{scope}[thick, xscale=\xscale,yscale=\yscale, xshift=\xxshift, yshift=\yyshift, rotate around={\xangle:(1,1)} ]
						\draw (0,0) rectangle (2,2);
						\filldraw (-.1,1.9) rectangle (.1,2.1) (2,2) circle (.1cm) (2,0);
						\filldraw[fill=white, draw=black] (2,0) circle (.1cm);
					\end{scope}
					\path (1,-.2) node[below] {\xtext};
				\end{scope}
			};
			
		\end{tikzpicture}
	\centering \caption{Symmetrien eines Quadrates}
	\end{figure}
	
	Auf der Menge $S=\{\iden,r_1,r_2,r_3,f_1,f_2,f_3,f_4\}$ betrachten wir die Hintereinanderausführung von Abbildungen $\circ$ als Verknüpfung; etwa gilt $f_1\circ f_2=r_2$, wie man sich mittels eines Bildchens schnell verdeutlicht. Die Verkettung zweier Symmetrien ergibt wieder eine Symmetrie, also ist $\circ$ tatsächlich eine Verknüpfung auf $S$. [Tabelle?]
	
	Wir möchten nun zeigen, dass $(S,\circ,\iden)$ eine Gruppe bildet, und prüfen dazu die geforderten Eigenschaften.
	Dass $\circ$ assoziativ ist, gilt ganz allgemein für Abbildungen [Referenz?].
	Die Identität $\iden$ ist neutrales Element bezüglich $\circ$, siehe [Referenz].
	Jetzt müssen wir noch zu jedem Element Inverse finden. Die Identität hat als Inverses sich selbst, denn es gilt
		\[ \iden\circ\iden=\iden \,.\]
	Die Rotation um $x$ Grad hat als Inverses die Rotation um $360-x$ Grad (denn Hintereinanderausführung ergibt dann die Rotation um 360 Grad, die der Identität entspricht). Konkret:
		\begin{align*}
			r_1\circ r_3 &= \iden = r_3\circ r_1 \,, \\
			r_2\circ r_2 &= \iden \,.
		\end{align*}
	Führt man dieselbe Achsenspiegelung zwei mal hintereinander aus, landen man wieder, wo man gestartet ist, also sind die vier Achsenspiegelungen alle selbstinvers:
		\[ \iden=f_1\circ f_1=f_2\circ f_2=f_3\circ f_3 = f_4\circ f_4\,. \]
		
	Damit ist bewiesen, dass $(S,\circ,\iden)$ eine Gruppe ist.
\end{bsp}
\end{comment}



Viele interessante Verknüpfungen liefern lediglich Monoide aber keine Gruppen. Allerdings kann aus jedem Monoid eine (mehr oder weniger große) Gruppe extrahiert werden, indem man sich einfach auf die invertierbaren Elemente einschränkt:


\begin{de}[Einheitengruppe eines Monoids]
Sei $(M,*)$ ein Monoid. Die Teilmenge
\[ M^\times := \{x\in M\mid x\ \text{ist invertierbar} \} \]
heißt die \textbf{Einheitengruppe} von $M$.
\end{de}


\begin{sat} \label{einheitengruppe}
 Sei $(M,*)$ ein Monoid. Dann kann die Verknüpfung von $M$ auf $M^\times$ eingeschränkt werden. Auf diese Weise wird $M^\times$ zu einer Gruppe. Ist $M$ ein kommutatives Monoid, so ist $M^\times$ eine abelsche Gruppe.
\end{sat}
\begin{bew}
(Einschränkbarkeit) Nach der Regel von Hemd und Jacke ist für alle $a,b\in M^\times$ auch $a*b\in M^\times$. Somit liefert $*$ durch Einschränkung eine zweistellige Verknüpfung auf der Menge $M^\times$. \\[0.5em]
(Assoziativität): Da $*$ eine assoziative Verknüpfung ist, gilt
\begin{align*}
 x*(y*z) & = (x*y)*z && x,y,z\in M
\end{align*}
Also gilt diese Gleichung auch erst recht für alle Elemente von $M^\times$. Somit ist $*$ auch auf $M^\times$ eine assoziative Verknüpfung. \\[0.5em]
(Neutrales Element): Sei $e\in M$ das neutrale Element von $M$. Nach \cref{selbstinvers} ist $e\in M^\times$. Weil für alle $x\in M$ gilt
\[ e*x=x*e=x \]
gilt dies erst recht auch für alle $x\in M^\times$. Somit ist $e$ ein neutrales Element in $M^\times$. \\[0.5em]
(Inverse): Sei $a\in M^\times$. Dann ist $a$ ein invertierbares Element von $M$. Nach \cref{doppelinvers} ist auch $a^{-1}$ invertierbar, also $a^{-1}\in M^\times$. Wegen
\begin{align*}
 a*a^{-1}=a^{-1}*a=e
\end{align*}
und weil $e$ das neutrale Element in $M^\times$ ist, ist dann $a^{-1}$ auch in $M^\times$ invers zu $a$. \\[0.5em]
(Kommutativität) Sei $M$ ein kommutatives Monoid. Dann gilt für alle $x,y\in M$:
\[ x*y=y*x \]
Also gilt diese Gleichung erst recht auch für alle Elemente von $M^\times$, sodass $M^\times$ in diesem Fall eine abelsche Gruppe ist. \qed
\end{bew}







\begin{bsp}
Es gilt:
\begin{itemize}
 \item Die Einheitengruppe des Monoids $(\Nz_0,+)$ ist $\{0\}$. Insbesondere ist $(\{0\},+)$ eine Gruppe, die nur ein einziges Element enthält. Solche Gruppen nennt man auch \emph{triviale Gruppen}.\footnote{vgl. \cref{verkbsp}d)}
 \item Die Einheitengruppe des Monoids $(\Zz,\cdot)$ ist $\{1,-1\}$. Insbesondere ist $(\{1,-1\},\cdot)$ eine Gruppe, die aus genau zwei Elementen besteht.
 \item Nach \cref{inversebsp} ist die Einheitengruppe des Monoids $(\Rz,\cdot)$ genau $\Rz\setminus \{0\}$. Somit ist $(\Rz\setminus \{0\},\cdot)$ eine Gruppe.
\end{itemize}
\end{bsp}




\begin{de}[Permutationsgruppe]
 Sei $M$ eine beliebige Menge. Die Menge der bijektiven Selbstabbildungen von $M$
 \[ S(M) := \{f\in \text{Abb}(M,M)\mid f\ \text{ist bijektiv} \} \]
 heißt die \textbf{symmetrische Gruppe} von $M$. Ihre Elemente heißen \textbf{Permutationen} von $M$.
\end{de}


\begin{sat}
 Sei $M$ eine Menge. Dann ist $S(M)$ mit der Verkettung von Abbildungen als Verknüpfung eine Gruppe. Es handelt sich genau um die Einheitengruppe von $\text{Abb}(M,M)$.
\end{sat}
\begin{bew}
 Aus \cref{invkriterium} folgt, dass eine Abbildung $f:M\to M$ genau dann bijektiv ist, wenn sie invertierbar im Monoid $\text{Abb}(M,M)$ ist. Also ist $S(M)$ tatsächlich die Einheitengruppe von $\text{Abb}(M,M)$ und nach \cref{einheitengruppe} somit eine Gruppe. \qed
\end{bew}




\begin{bem}[Endliche Permutationsgruppen]
 Für eine Zahl $n\in \Nz_0$ schreibt man
 \[ S_n := S(\{1,\dots , n\}) \]
 für die Permutationsgruppe der Menge $\{1,\dots , n\}$. Diese Gruppen sind von großer Bedeutung in der Gruppentheorie, da sie nach dem sogenannten „\href{https://de.wikipedia.org/wiki/Satz_von_Cayley}{Satz von Cayley}“ in einem gewissen Sinne „universell“ sind unter allen Gruppen, die nur endlich viele Elemente besitzen. Die $S_n$-Gruppen werden dir bereits in der LA1-Vorlesung wieder begegnen, dort spätestens im Kontext von \emph{Matrixdeterminanten}.
\end{bem}




\begin{bem}[* Grothendieck\footnote{\href{https://de.wikipedia.org/wiki/Alexander_Grothendieck}{Alexander Grothendieck (1928-2014)}}-Gruppe]
 Neben dem Konzept „Einheitengruppe“ gibt es ein weiteres Rezept, um aus Monoiden Gruppen zu machen: die sogenannte „\href{https://de.wikipedia.org/wiki/Grothendieck-Gruppe}{Grothendieck-Gruppe}“. Während man, um von einem Monoid zu seiner Einheitengruppe zu gelangen, die Trägermenge soweit verkleinert, bis nur noch die invertierbaren Elemente übrigbleiben, fügt man bei der Grothendieck-Gruppe „künstliche Inverse“ hinzu. Beispielsweise kann die Gruppe $(\Zz,+)$ dadurch konstruiert werden, dass man dem Monoid $(\Nz_0,+)$ für jedes $n\in \Nz_0$ eine „künstliche Inverse $-n$“ beilegt. Auch die Zahlbereichserweiterung $\Zz \mapsto \Qz$ geschieht durch die Hinzufügung künstlicher Inverser, diesmal bezüglich der Multiplikation. Solche Techniken, bei denen man für eine Struktur gewisse wünschenswerte Eigenschaften künstlich erzwingt, sind typisch für die abstrakte Algebra und tauchen dort beispielsweise bei den Konzepten „Quotientenkörper“, „Lokalisierung eines Rings“, „Tensoralgebra“ oder „Zerfällungskörper“ auf.
\end{bem}






\newpage
\section{Aufgabenvorschläge}




\begin{aufg} \label{verkbsp}
Entscheidet für jede der folgenden Verknüpfungen, ob sie assoziativ ist, kommutativ ist, ob sie ein neutrales Element besitzt und ob sie ein Monoid oder gar eine Gruppe liefert. Sofern ein Monoid vorliegt, bestimmt dessen Einheitengruppe.
\begin{enumerate}[a)]
 \item Die auf der Menge $\Nz_0$ durch
 \[ \Nz_0 \times \Nz_0 \to \Nz_0 \ ,\ (n,m) \mapsto n^m \]
 gegebene Verknüpfung (wobei $n^0=1$ für alle $n\in \Nz_0$ sei).
 \item Auf der Menge $\Rz_{>0}$ der positiven reellen Zahlen die Multiplikation $(x,y)\mapsto x\cdot y$.
 \item Die auf der Menge $\Rz$ durch
 \[ \Rz\times \Rz \to \Rz \ ,\ (x,y) \mapsto 96 \]
 gegebene Verknüpfung (d.h. je zwei beliebige Elemente verknüpfen sich immer zu $96$).
\item Auf einer beliebigen einelementigen Menge eine beliebige zweistellige Verknüpfung.
 \item Es sei $V$ die Gesamtheit aller Mengen. Betrachte darauf die Verknüpfung
 \[ V\times V\to V \ ,\ (A,B) \mapsto \{A,B\} \]
\end{enumerate}
\end{aufg}




\begin{aufg}
Seien $(M,*)$ ein Monoid und $a\in M$ ein invertierbares Element. Beweist, dass „Multiplikation mit $a$“ eine Äquivalenzumformung ist, d.h. dass für alle $x,y\in M$ gilt:
\begin{align*}
x&=y \quad \leftrightarrow\quad a*x=a*y \\
x&=y \quad \leftrightarrow\quad x*a=y*a
\end{align*}
Gilt dies auch, wenn $a$ nicht invertierbar ist? Könnt ihr ein Gegenbeispiel für diesen Fall finden?
\end{aufg}





\begin{aufg}
Sei $X$ eine beliebige Menge. Beweist die Äquivalenz der folgenden Aussagen:
\begin{enumerate}[(i)]
 \item Die symmetrische Gruppe $S(X)$ ist eine abelsche Gruppe.
 \item $X$ enthält höchstens zwei verschiedene Elemente.
\end{enumerate}
\end{aufg}



\begin{aufg}
Sei $X$ irgendeine Menge.
\begin{enumerate}[a)]
 \item Beweist, dass die Einheitengruppe des Monoids $(\mathcal{P}(X),\cup)$ aus nur einem einzigen Element besteht. Welchem?
 \item Beweist, dass das Monoid $(\mathcal{P}(X),\cup)$ dann und nur dann eine Gruppe ist, wenn $X=\emptyset$.
\end{enumerate}
\end{aufg}
\begin{comment}
\subsection{Aufgabe}
Sei $X$ eine beliebige Menge. Es bezeichne
\[ \text{Rel}(X) := \{{\sim}\mid {\sim}\ \text{ist eine zweistellige Relation auf $X$} \} \]
die Menge aller Relationen auf $X$. Für zwei Relationen ${\sim_1},{\sim_2}\in \text{Rel}(X)$ sei die Relation „$\sim_1\land \sim_2$“ definiert durch
\begin{align*}
 x\ (\sim_1 \land \sim_2)\ y  \qquad:\Leftrightarrow\qquad x\sim_1 y\ \text{und}\ x\sim_2 y && x,y\in X
\end{align*}
\begin{enumerate}[a)]
 \item Sei $X=\Rz$. Bestimmt die Relation $\sim_1\land \sim_2$ in den folgenden drei Fällen:
 \begin{align*}
  {\sim_1} = \text{„$\leq$“} \quad \text{und}\quad {\sim_2} = \text{„$\geq$“} \\
  {\sim_1} = \text{„$\leq$“} \quad \text{und}\quad {\sim_2} = \text{„$\neq$“} \\
  {\sim_1} = \text{„$<$“} \quad \text{und}\quad {\sim_2} = \text{„$>$“} 
 \end{align*}
\item Zeigt, dass $(\text{Rel}(X),\land)$ ein kommutatives Monoid ist. Was ist das neutrale Element?
\end{enumerate}





\begin{aufg}[* Große Abschlussaufgabe (nur für Profis)] \label{abschlussaufgabe}
Diese Aufgabe beinhaltet Konzepte aus nahezu allen Vorkursvorträgen. Komm eventuell nochmal in ein paar Wochen hierher zurück, wenn du dich besser an diese Denkweisen gewohnt hast.\\[0.5em]
Es sei
\[ \mathcal{A} := \{ A\mid A\ \text{ist eine Aussage} \} \]
die Menge aller Aussagen. Betrachtet darauf die folgende Relation „$\sim$“:
\begin{align*}
A\sim B\qquad &:\Leftrightarrow\qquad \text{Es lässt sich beweisen, dass $A\leftrightarrow B$ gilt}  && A,B\in \mathcal{A} 
\end{align*}
Nach \cref{aussagenequiv} ist „$\sim$“ eine Äquivalenzrelation auf $\mathcal{A}$. Die zugehörige Faktormenge bezeichnen wir mit „$L$“\footnote{Man nennt $L$ die \emph{Lindenbaum-Algebra}}:
\[ L := \{ k\subseteq\mathcal{A} \mid k\ \text{ist eine Äquivalenzklasse bezüglich}\ {\sim} \} \]
\begin{enumerate}[a)]
 \item Beweist, dass es auf der Menge $L$ genau eine Relation „$\leq$“ gibt, bei der für alle $A,B\in \mathcal{A}$ gilt:
 \[ [A]\leq [B]\qquad\Leftrightarrow\qquad \text{Es lässt sich beweisen, dass $A\to B$ gilt} \]
Zeigt außerdem, dass „$\leq$“ eine Ordnungsrelation, also reflexiv, transitiv und antisymmetrisch, ist.
\item Beweist, dass es eindeutig bestimmte zweistellige Verknüpfungen „$\land$“ und „$\lor$“ auf $L$ gibt, die für alle $A,B\in \mathcal{A}$ die folgenden Gleichungen erfüllen:
\begin{align*}
[A]\land [B] & = [A\land B] \\
[A]\lor [B] & = [A\lor B]
\end{align*}
\item Beweist, dass in der geordneten Menge $(L,\leq)$ jede zweielementige Menge ein Infimum und ein Supremum besitzt. Um genau zu sein, dass für alle $x,y\in L$ gilt:
\[ \inf \{x,y\} = x\land y \qquad\text{und}\qquad \sup \{x,y\} = x\lor y \]
\item Sei $E$ eine beliebige Eigenschaft. Zeigt, dass die Menge $\{ [E(a)] \mid a\ \text{ist irgendein Objekt} \}$ jeweils ein Infimum und ein Supremum in $L$ besitzt. Um genau zu sein:
\begin{align*}
\inf \{ [E(a)] \mid a\ \text{ist ein Objekt} \}&= [ \forall x: E(x)]  \\
\sup \{ [E(a)] \mid a\ \text{ist ein Objekt} \} & =  [ \exists x: E(x)] 
\end{align*}
\item Zeigt, dass $(L,\land)$ und $(L,\lor)$ jeweils kommutative Monoide sind.
\end{enumerate}
\end{aufg}
\end{comment}


\begin{comment}
Auf der Menge $\Rz^\Nz$ der reellen Zahlenfolgen wurde im letzten Vortrag\footnote{siehe \cref{folgenrech}} eine Addition eingeführt:
\[ + : \Rz^\Nz\times \Rz^\Nz\to \Rz^\Nz \ ,\ ((a_n)_{n\in \Nz} , (b_n)_{n\in \Nz} )\mapsto (a_n+b_n)_{n\in \Nz} \]
\begin{enumerate}[a)]
 \item Beweist, dass $(\Rz^\Nz,+)$ eine Gruppe ist, indem ihr nacheinander die Gruppenaxiome durchgeht.
 \item Sei nun $(G,*)$ eine beliebige Gruppe. Auf der Menge $G^\Nz$ der $G$-wertigen Folgen lässt sich die Verknüpfung
 \[ * : G^\Nz\times G^\Nz\to G^\Nz \ ,\ ((g_n)_{n\in \Nz} , (h_n)_{n\in \Nz} )\mapsto (g_n* h_n)_{n\in \Nz} \]
 definieren. Beweist, dass $(G^\Nz,*)$ eine Gruppe ist.
 \item Macht euch klar, dass a) ein Spezialfall von b) ist. Lässt sich die Aussage aus b) noch weiter verallgemeinern?
\end{enumerate}
\end{comment}








\begin{comment}
\newpage
\section{Elementare Eigenschaften}

%------------------
% 1.1 Def. Gruppe
\begin{de}[(abelsche) Gruppe]
Sei $G$ eine  nicht-leere Menge, $e\in G$ ein (ausgezeichnetes) Element und 
\[
 *\colon G\times G \to G
\]
eine Abbildung.
Wir nennen das Tripel $(G,*,e)$ eine \swdde{Gruppe}, falls für beliebige $a,b,c\in G$ gilt
\begin{enumerate}[\colde\upshape(i)]
\item $(a*b)*c=a*(b*c)$ \hfill (Assoziativität)
\item $a*e=a$ \hfill (Ex. rechtsneutrales El.)
\item Es gibt $d\in G$ mit $a*d=e$ \hfill (Ex. rechtsinverses El.)
\end{enumerate}
wir nennen die Gruppe zusätzlich \swdde{abelsch}, falls gilt
\begin{enumerate}[\colde\upshape(iv)]
\item $a*b=b*a$ \hfill (Kommutativität)
\end{enumerate}
%\deEnd
\end{de}

%------------------
% 1.2 Bsp. Gruppe
\begin{bsp}
\begin{bspListeN}[]%\leavevmode%
\item $(\Zz,+,0)$ ist eine abelsche Gruppe
\item $(\Qz\backslash\{0\},\cdot,1)$ ist eine abelsche Gruppe
\item $(\{-1,1\},\cdot,1)$ ist eine abelsche Gruppe
\item $(\Nz, \cdot, 1)$ ist keine Gruppe
\end{bspListeN}
\end{bsp}

%------------------
% 1.3 Bsp. Gruppe
\begin{bsp}
Wir betrachten die Menge der bijektiven Abbildungen $M=\{1,2,3\} \ra M$
\begin{align*}
S_3=\{e=\id[3],\;
d_1=(1 \,2\, 3),\;
d_2=(1\, 3\, 2),\;
\tau_1=(2\,3),\;
\tau_2=(1\,3),\;
\tau_3=(1\,2)\}
\end{align*}\
Als erstes fällt uns auf, dass für $f,g\in S_3\colon f \circ g \in S_3$.
Z.\,B.: $d_1 \circ \tau_1=(1\,2)=\tau_3\in S_3$ oder $\tau_1 \circ d_1=(1\,3)\in S_3$.
Insbesondere gilt hier also nicht die Kommutativität.
\\
Außerdem gilt für $f,g,h\in S_3\colon(f\circ g)\circ h=f \circ(g\circ f)$, denn es gilt für $x\in M$ beliebig:
\begin{align*}
	((f\circ g)\circ h)(x) &\overset{\text{Def}}{=} (f\circ h)(h(x)) \overset{\text{Def}}{=} f(g(h(x))) \\
	(f\circ(g\circ h))(x) &= f((g\circ h)(x)) = f(g(h(x))) 
\end{align*}
Also gilt die behauptete Gleichheit. \\
Da $e\in S_3$ jedes Element wieder auf sich selber abbildet, gilt für $f\in S_3\colon f\circ e=f$.
\\
Außerdem können jede dieser Abbildungen wieder umkehren:
\[
	e\circ e=e,\; d_1\circ d_2=e,\; d_2\circ d_1=e\; \tau_1\circ \tau_1=e,\;\tau_2 \circ \tau_2=e\;\tau_3\circ \tau_3=e
\]

Insgesamt handelt es sich bei $S_3$ also um eine Gruppe.
\\
Wenn wir nun die bijektiven Abbildungen $S_n$ von $M_n=\{1, \dots, n\}$ für $n\geq 3$ betrachten, fällt uns auf, dass diese auch nicht-abelsche Gruppen sind.
Deshalb betrachten wir jetzt Gruppen, als Verallgemeinerung dieses Konzeptes, um Aussagen über all diese Mengen zu treffen.
\end{bsp}


% --------------------------------------------------------------------
% §2 Section <<Elementare Eigenschaften>>
% --------------------------------------------------------------------







%\label{pot}

%------------------
% 2.1 Lemma
\begin{lem}
Sei $G$ eine Gruppe, $a,b\in G$, s.\,d., $a*b=e$. Dann gilt auch $b*a=e$, $b$ ist also auch ein linksinverses El. von $a$.
\end{lem}

%------------------
% 2.2 Beweis
\begin{bew}
Sei $c \in G$ rechtsinvers von $b$, also $b*c=e$.
Dann gilt:
\[
	b*a \overset{\text{1ii}}{=} (b*a)*e \overset{b*c=e}{=} (b*a)*(b*c) \overset{\text{1i}}{=} b*(a*b)*c \overset{a*b=e}{=} b*e*c \overset{\text{1ii}}{=} b*c \overset{n.V}{=} e.
\]
\bewEnd
\end{bew}

%------------------
% 2.3 Lemma
\begin{lem}
Sei $G$ eine Gruppe, dann ist $e\in G$ auch linksneutral, also für $a\in G\colon e*a=a$.
\end{lem}

%------------------
% 2.4 Beweis
\begin{bew}
Sei $a \in G$ bel. und $b\in G$, s.\,d., $a*b=e$. Dann gilt:
\[
	e*a \overset{\text{n.\,V.}}{=} (a*b)*a \overset{\text{1i}}{=} a*(b*a) \overset{\text{4}}{=} a*e \overset{\text{1ii}}{=} a
\]
\bewEnd
\end{bew}

%------------------
% 2.5 Lemma
\begin{lem}
Sei $G$ eine Gruppe. Dann ist $e$ das einzige neutrale Element, d.\,h., für $\tilde{e}\in G$ ein neutrales Element gilt bereits $e=\tilde{e}$.
\end{lem}

%------------------
% 2.6 Beweis
\begin{bew}
Es gilt $e=e*\tilde{e}=\tilde{e}$
\bewEnd
\end{bew}

%------------------
% 2.7 Lemma
\begin{lem}
Sei $G$ eine Gruppe, $a\in G$. Dann gilt es nur ein zu $a$ inverses Element.
D.\,h., für $b,c\in G$ mit $a*b=e=a*c$ gilt bereits $b=c$ 
\end{lem}

%------------------
% 2.8 Beweis
\begin{bew}
Es ergibt sich
\[
	b\overset{\text{5}}{=}e*b\overset{\text{n.\,V.}}{=}(c*a)*b\overset{\text{1i}}{=}c*(a*b)\overset{\text{n.\,V.}}{=}c*e\overset{\text{1ii}}{=}c
\]
\bewEnd
\end{bew}

%------------------
% 2.9 Bem.
\begin{bem}
Wir haben bis jetzt gesehen, dass ein rechtsneutrales Element auch ein linksneutrales Element ist und es nur ein Element mir dieser Eigenschaft gibt.
Dieses Element nennen wir das \swdde{neutrale Element}.

Außerdem ist ein rechtsinverses Element auch ein linksinverses Element und zu jedem $a\in G$ gibt es genau ein $b\in G$ mit dieser Eigenschaft.
Wir nennen dieses Element das \swdde{inverse Element von $a$} und wir schreiben $a^{-1}:=b$.
Insbesondere gilt $(a^{-1})^{-1}=a$.
\bemEnd
\end{bem}

%------------------
% 2.10 Lemma Kuerzungsregel
\begin{lem}[Kürzungsregel]
Sei $G$ eine Gruppe. $a, b, c \in G$ mit $a*b = a * c$. Dann gilt $b = c$
\end{lem}

%------------------
% 2.11 Beweis
\begin{bew}
Seien $a, b, c \in G$ wie oben, dann gilt:
\[
	b\overset{\text{5}}{=}e*b\overset{\text{1iii}}{=}(a^{-1}*a)*b\overset{\text{1ii}}{=}a^{-1}*(a*b)\overset{\text{n.\,V.}}{=}a^{-1}*(a*c)\overset{\text{1i}}{=}(a^{-1}*a)*c \overset{\text{1i/iii}}{=} c
\]
\bewEnd
\end{bew}






% --------------------------------------------------------------------
% §3 Section <<Untergruppen und Nebengruppen>>
% --------------------------------------------------------------------
\section{Untergruppen und Nebengruppen}%\label{pot}

%------------------
% 3.1 Def. Untergruppe
\begin{de}%\label{defi::12}
Sei $G$ eine Gruppe und $H\subseteq G$ nicht-leer. Wir nennen $H$ eine \swdde{Unterguppe} von $G$, falls für alle $a,b\in H$ gilt
\begin{enumerate}[\colde\upshape(i)]
\item $a*b \in H$
\item $a^{-1}\in H$
\end{enumerate}
\deEnd
\end{de}

%------------------
% 3.2 Beispiel Untergruppen
\begin{bsp}
\begin{bspListeN}[]
\item $2\Zz = \{2a \mid a \in \Zz\}\subseteq\Zz$ ist eine Untergruppe
\item $\{\pm 1\} \subseteq \Qz$ ist eine Untergruppe
\item Für $G$ eine Gruppe, ist $\{e\}\subseteq G$ eine Untergruppe
\item $A_3=\{e,d_1,d_2\}\subseteq S_3$ ist eine Untergruppe (sogar abelsch)
\end{bspListeN}
\end{bsp}

%------------------
% 3.3 Lemma
\begin{lem}
Seien $G$ Gruppe, $H\subseteq G$ Untergruppe.
Dann gilt $e\in H$ und $(H,*_{|H},e)$ mit der auf $H$ eingeschränkten Verknüpfung selbst eine Gruppe.
Gilt zusätzlich $G$ abelsch, dann ist $H$ abelsch.
\end{lem}

%------------------
% 3.4 Beweis
\begin{bew}
$H$ ist nicht-leer, also gibt es $a\in  H$. Damit ist $a^{-1}\in H$ nach 10ii.
Dann gilt $e=a*a^{-1}\in H$ nach 10i.
\begin{enumerate}
\item[(0)] Wohldefiniertheit: es muss gelten $\forall a,b \in H : a*b\in H$, dies gilt nach 10i
\item Assoziativ: Seien $a,b,c\in H$, dann gilt $a,b,c\in G$, also gilt
\[
	(a*b)*c=a*(b*c)
\]
\item Rechtsneutrale Element: Es ist $e\in H$ und für $a\in H\colon a*e=a$
\item Inverses Element: folgt aus 10ii
\item Kommutativität: falls G abelsch ist, gilt für $a,b\in H\colon a*b=b*a$ in $G$, also auch in $H$ 
\end{enumerate}
\bewEnd
\end{bew}

%------------------
% 3.5 Lemma
\begin{lem}
Sei $G$ Gruppe, $H\subseteq G$, $e\in H$, $(H,*,e)$ Gruppe. Dann gilt $H\subseteq G$ Untergruppe.
\end{lem}

%------------------
% 3.6 Beweis
\begin{bew}
\begin{bewListeN}
\item gilt, da $H$ wohldefiniert
\item folgt aus 1(iii)
\end{bewListeN}
\bewEnd
\end{bew}

%------------------
% 3.7 Bem.
\begin{bem}
$G$ eine Gruppe, $a\in G$. Für $n\in\Zz$ definieren wir:
\begin{align*}
a^n:=
\begin{cases}
\underbrace{a*a*\cdots *a}^{n\text{-mal}} & n>0\\
e & n=0\\
\underbrace{a^{-1}*a^{-1}*\cdots *a^{-1}}_{(-n)\text{-mal}} & n<0,
\end{cases}
\end{align*}
es gilt
\[
\langle a \rangle := \{a^n| n\in \Zz \} \subseteq G
\]
ist eine Untergruppe. Wir nennen diese, die \swdde{von $a$ erzeugte Untergruppe}.
\end{bem}

%------------------
% 3.8 Beweis
\begin{bew}
Übungsaufgabe
\bewEnd
\end{bew}

%------------------
% 3.9 Def. 
\begin{de}%\label{defi::17}
$G$ Gruppe, $H\subseteq G$ Untergruppe. Wir definieren eine Relation auf $G$ via
\[
	a\sim_H b :\lrP a^{-1}b \in H
\]
\deEnd
\end{de}

%------------------
% 3.10 Beispiel
\begin{bsp}
Wir betrachten $G = \Zz$ und $H = 2\Zz$.
Für $a, b \in \Zz$ erhalten wir also
\[
	a \sim b \lrP b - a \in 2\Zz
\]
Also muss die Differenz von $a$ und $b$ gerade sein.
Das ist genau dann der Fall, wenn $a$ und $b$ beide gerade sind oder beide ungerade sind.
\bspEnd
\end{bsp}

%------------------
% 3.11 Lemma
\begin{lem}%\label{lemm::18}
Die Relation aus \cref{defi::17} ist eine Äquivalenzrelation.
\end{lem}

%------------------
% 3.12 Beweis
\begin{bew}
Seien $a,b,c\in G$ beliebig
\begin{enumerate}
\item Reflexiv: $a^{-1}a=e\in H$, also $a\sim_H a$
\item Symmetrie: Gelte $a\sim_H b$, also $a^{-1}b\in H$\\
Es gilt $(b^{-1}a) = (a^{-1}b)^{-1}$, denn:
\[
	(a^{-1}b)(b^{-1}a)= a^{-1} *(bb^{-1})*a=a^{-1}a=e
\]
Also $ b^{-1}a =(a^{-1}b)^{-1} \in H$, folgt aus \cref{defi::12}ii, also $b\sim_H a$
\item Transitiv:
Gelte $a\sim_H b$ und $b\sim_H c$. Dann gilt $$ a^{-1}c = a^{-1} (bb^{-1}) c=
\underbrace{(a^{-1} b)}_{\in H} \underbrace{(b^{-1}c)}_{\in H} \in H$$
also $a\sim_H c$
\end{enumerate}
\end{bew}

%------------------
% 3.13 Lemma
\begin{lem}%\label{lemm::19}
$G$ Gruppe $H\subseteq G$ UG, $a\in G$.
Dann ist die Äquivalenzklasse von $a$ bzgl $\sim_H$ gegeben durch
\[
	[a]=aH:=\{ah\mid h\in H\}
\]
\end{lem}

%------------------
% 3.14 Beweis
\begin{bew}
\begin{bewListeN}[]
\item[\glqq$\subseteq$\grqq] Sei $b \in [a]$, also $a \sim b$, also $a^{-1}*b = h \in H$. Damit erhalten wir
\[
	b = (aa^{-1})b = a(\underbrace{a^{-1}b}_{= h}) = ah
\]
Also $b \in aH$.
\item[\glqq$\supseteq$\grqq] Sei $b \in aH$, also $b = ah$ für $h \in H$. Wir erhalten
\[
	a^{-1}b = a^{-1}ah = h \in H
\]
Also $a \sim b$ und $b \in [a]$.
\end{bewListeN}
\end{bew}

%------------------
% 3.15 Notation
\begin{sch}
Wir schreiben
\[
	G/H := G/\sim_H = \{[a] \mid a \in G\}
\]
\end{sch}

%------------------
% 3.16 Beispiel
\begin{bsp}
$\Zz/2\Zz = \{\{\ldots,-4,-2,0,2,4,\ldots\}, \{\ldots,-5,-3,-1,1,3,5,\ldots\}\}$
\end{bsp}


% --------------------------------------------------------------------
% §4 Section <<Der Satz von Lagrange>>
% --------------------------------------------------------------------
\section{Der Satz von Lagrange}

%------------------
% 4.1 Def. 
\begin{de}
Sei $G$ eine Gruppe. Wir nennen $G$ \swdde{endlich}, falls die Menge $G$ nur endlich viele Elemente besitzt.
Sei $H \subseteq G$ eine Untergruppe. Ist $G/H$ eine endliche Menge, so nennen wir
\[
	(G:H) := \#(G/H)
\]
den \swdde{Index von $H$ in $G$}.

Sei $a \in G$. Wir definieren die \swdde{Ordnung} von $a$ durch
\[
	\ord(a) := \min\{n\in \NN \mid a^n = e\} \qquad \text{wobei } \min\emptyset := \infty
\]
\deEnd
\end{de}

%------------------
% 4.2 Bem.
\begin{bem}
Sei $G$ eine Gruppe, $a \in G$. Dann gilt
\[
	\ord(a) = \#\langle a \rangle
\]
\end{bem}

%------------------
% 4.3 Beweis
\begin{bew}
Übungsaufgabe
\bewEnd
\end{bew}

%------------------
% 4.4 Lemma
\begin{lem}
Sei $G$ eine Gruppe, $H \subseteq G$ eine Untergruppe. Dann gilt
\begin{enumerate}
\item Je zwei Äquivalenzklassen sind gleichmächtig
\item Je zwei Äquivalenzklassen sind disjunkt oder gleich
\item $G$ ist die disjunkte Vereinigung der Äquivalenzklassen
\end{enumerate}
\end{lem}

%------------------
% 4.5 Beweis
\begin{bew}
\begin{enumerate}
\item Es genügt für $a \in G$ bel zu zeigen, dass $\#[a] = \#[e]$.
Dafür betrachten wir $f\colon H \ra aH, h \mapsto ah$. $f$ ist injektiv, denn seien $h_1, h_2 \in H$ mit $f(h_1) = f(h_2)$ dann gilt
\[
	ah_1 = f(h_1) = f(h_2) = ah_2 \overset{\text{9}}{\rP} h_1 = h_2
\]
Da $f$ injektiv ist, gilt nun $\#[a] \geq \#[e]$.

Außerdem ist $f$ surjektiv, denn sei $b = ah \in aH$. Dann gilt
\[
	b = f(h)
\]
Damit ist $\#[a] \leq \#[e]$.

Also muss bereits $\#[a] = \#[e]$ gelten. Damit folgt nun für $a, b \in G$ beliebig
\[
	\#[a] = \#[e] = \#[b]
\]
\item folgt da $\sim$-Äquivalenzrelation
\item folgt da $\sim$-Äquivalenzrelation
\end{enumerate}
\end{bew}

%------------------
% 4.6 Satz von Lagrange
\begin{sat}[Lagrange]
Sei $G$ eine endliche Gruppe, $H \subseteq G$ eine Untergruppe. Dann gilt:
\[
	\#G = \#H \cdot (G:H)
\]
\end{sat}

%------------------
% 4.7 Beweis
\begin{bew}
Wir schreiben zunächst
\[
	G/H = \{[a_1], \ldots, [a_n]\}
\]
mit disjunkten $[a_1], \ldots, [a_n]$ (Lemma 23ii) und $n = \# G/H = (G:H)$ (Definition 21). Wir haben
\[
	G = \mathop{\overset\bullet\bigcup} \limits _ {i=1,\ldots,n} [a_i] \qquad \text{Lemma 23iii}
\]
und damit folgt
\[
	\# G = \# \left(\mathop{\overset\bullet\bigcup} \limits _ {i=1,\ldots,n} [a_i]\right) = \sum \limits _{i=1}^n \#[a_i] \overset{\text{23}}{=} \sum \limits _{i=1}^n \#[e] = n \cdot \# [e] = (G:H) \cdot \# H
\]
\bewEnd
\end{bew}

%------------------
% 4.8 Korollar
\begin{kor}
Sei $G$ eine endliche Gruppe, $a \in G$. Dann gilt $\ord(a) | ord(G)$ bzw.
\[
	a^{\#G} = e
\]
\end{kor}
\end{comment}

% --------------------------------------------------------------------
% §5 Section <<Übungsaufgaben>>
% --------------------------------------------------------------------






%%% Local Variables:
%%% mode: latex
%%% TeX-master: "Skript"
%%% End:



\backmatter
\cleardoublepage
\appendix
\chapter{Anhang}


\section*{Einige wichtige Mengen}
\begin{itemize}
 \item $\Nz= \{ (0,) 1,2,3,4,\dots \}$. Menge der natürlichen Zahlen. Ob die Null dazugehört oder nicht, hängt vom Autor ab.
 \item $\Zz= \{0,1,2,3,\dots , -1,-2,-3,\dots \}$. Menge der ganzen Zahlen.
\item $\Qz = \left\{\frac{p}{q} \mid p, q\in \Zz,\ q\neq 0\right\}$. Menge der rationalen Zahlen.
\item $\Rz$. Menge der reellen Zahlen, d.h. aller Zahlen, die sich als Kommazahl (mit möglicherweise unendlich vielen Nachkommastellen) schreiben lassen.
\item $\Cz= \{a+bi \mid a,b\in \Rz\}$. Menge der komplexen Zahlen.
\item $\emptyset$. Leere Menge. Manche Autoren schreiben auch $\{ \}$.
\end{itemize}





\section*{Aussagenlogische Tautologien}
 Seien $A,B,C$ drei beliebige Aussagen. Hier ist eine Liste aussagenlogischer Tautologien. Versuche besser nicht, für jede einzelne Formel eine Wahrheitstafel aufzustellen. Wenn du Lust hast, kannst du
 \begin{itemize}
  \item versuchen, dir für die ein oder andere Formel intuitiv klarzumachen, dass es sich um eine Tautologie handeln muss. So kannst du ein besseres Verständnis für die Junktoren erwerben.
  \item mithilfe der Beweistechniken aus dem zweiten Vortrag versuchen, die eine oder andere dieser Formeln zu beweisen.
 \end{itemize}
 \begingroup
\allowdisplaybreaks
 \begin{align*}
% \begin{split}
%  A & \to (B \to A) \\
%  (A\to (B\to C)) & \to ((A\to B) \to (A\to C))
%  \end{split} && (\text{$\to$-Axiome}) \\[1em]
 %(A\to B) & \to ((B\to C) \to (A\to C)) && (\text{Syllogismusaxiom}) \\[1em]
 %   \begin{split}(A\land B) & \to A \\
 % (A \land B) & \to B\end{split} \\[1em] %&& (\text{$\land$-Elimination}) \\[1em]
 %     \begin{split}A & \to (A \lor B) \\
 % B & \to (A \lor B)\end{split} \\[1em] %&& (\text{$\lor$-Einführung}) \\[1em]
%   \begin{split}
%(A \to C) & \to ((B \to C) \to ((A \lor B) \to C)) \\
%     (C \to A) & \to ((C \to B) \to (C \to (A \land B)))
%  \end{split} \\[1em] %&& (\text{Fallunterscheidungsaxiome}) \\[1em]
%  (A\to B) & \to (\neg B\to \neg A) && (\text{Kontrapositionsformel})
%A& \to \top  \\
%    \bot & \to A  \\
%     & \top \\[1em]
%      (A \to B) & \to (\neg B \to \neg A)  && (\text{Kontrapositionsformel}) \\[1em]
 %      A & \leftrightarrow ( \top \to A)  \\
%\neg A  & \leftrightarrow (A\to \bot) &&\\[1em]
%    \neg (A & \land \neg A) && (\text{Satz vom Widerspruch}) \\
%  A & \lor \neg A && (\text{Satz vom ausgeschlossenen Dritten}) \\[1em]
         (A \leftrightarrow B)     & \leftrightarrow ((A \to B) \land (B \to A) ) \\[1em] %\\[1em]
     A & \to (B \to (A \land B)) \\[1em] % && (\text{$\land$-Einführungsformel}) \\[1em]
(A \to (B \to C)) & \leftrightarrow ((A \land B) \to C) && (\text{Import-/Exportformel}) \\[1em]
    \begin{split} ((A\to C) \land (B\to C)) & \leftrightarrow ((A\lor B) \to C) \\
    ((C\to A) \land (C\to B)) & \leftrightarrow (C\to (A\land B))
    \end{split} && (\text{Fallunterscheidungsformeln}) \\[1em]
    ((A\to B) \land A) & \to B && (\text{Modus-ponens-Formel}) \\[1em]
   (A \to (B \to C)) & \leftrightarrow (B \to (A \to C))  && ( \text{Umordnung der Prämissen}) \\[1em]
    ((A \to B) \land (B \to C)) & \to (A \to C) && \\ % (\text{Syllogismusformel}) \\
 ((A\leftrightarrow B)\land (B\leftrightarrow C)) & \to (A\leftrightarrow C) \\[1em]
 %   A& \to A \\
 %            A & \leftrightarrow A \\[1em]
    (A\leftrightarrow (A\to B)) & \to B && (\text{Curry-Paradoxon}) \\[1em]
  \begin{split} ((A \land B) \land C) & \leftrightarrow (A \land (B \land C)) \\
  ((A \lor B) \lor C) & \leftrightarrow (A \lor (B \lor C)) \end{split} && (\text{Assoziativgesetze}) \\[1em]
  \begin{split} (A \land B) & \leftrightarrow (B \land A) \\
  (A \lor B) & \leftrightarrow (B \lor A) \\
  (A\leftrightarrow B) & \leftrightarrow (B \leftrightarrow A) \end{split} && ( \text{Kommuativgesetze}) \\[1em]
    \begin{split} (A \lor (A \land B)) & \leftrightarrow A \\
  (A \land (A \lor B)) & \leftrightarrow A \end{split} && ( \text{Absorptionsgesetze}) \\[1em]
  \begin{split}
   (A \land A) & \leftrightarrow A \\
   (A \lor A) & \leftrightarrow A
  \end{split} && (\text{Idempotenzgesetze}) \\[1em]
  \begin{split} (A \land (B \lor C)) & \leftrightarrow ((A \land B) \lor (A \land C)) \\
   (A \lor (B \land C)) & \leftrightarrow ((A \lor B) \land (A \lor C)) \end{split} && ( \text{Distributivgesetze}) \\[1em]
  \begin{split} (A \to B) & \leftrightarrow ((A \land B) \leftrightarrow A) \\
(A \to B) & \leftrightarrow ((A \lor B) \leftrightarrow B)
    \end{split} &&  \\[1em]
  \begin{split}         (A \land B) & \leftrightarrow (A \leftrightarrow (A\to B)) \\
    (A \lor B) & \leftrightarrow (B \leftrightarrow (A\to B))
    \end{split} &&  \\[1em]
 %   (A \leftrightarrow B) & \to (\neg A \leftrightarrow \neg B) \\
%         (A\leftrightarrow B) & \leftrightarrow ((A\lor B) \to (A\land B)) \\[1em]
 %   (A\land \neg A)&\to \neg \top \\[1em]
 %  (A\to \neg B) & \leftrightarrow (B\to \neg A)\\% && (\text{Adjunktionsformel}) \\
 %          A & \to \neg \neg A  \\
 %   (A\to(B\land \neg B)) &\leftrightarrow \neg A \\[1em]  %&&(\text{Reductio ad absurdum})  \\[1em]
                (A\land \neg A) & \to B \\ %&& (\text{Ex contradictione quodlibet}) \\
    ((A\lor B) \land \neg A) & \to B && (\text{Modus-tollendo-ponens-Formel})\\[1em]
 %    (A\to B) & \leftrightarrow (\neg B\to \neg A)\\
 %   A & \leftrightarrow \neg\neg A && (\text{Regel der doppelten Verneinung}) \\[1em]
  (A\to B) & \leftrightarrow (\neg A \lor B) \\
  \neg (A\to B) & \leftrightarrow (A \land \neg B) \\[1em]
       \begin{split} ( \neg A \land \neg B) & \leftrightarrow \neg (A \lor B) \\
  (\neg A \lor \neg B) & \leftrightarrow \neg(A \land B) 
  \end{split} && (\text{Regeln von De Morgan}) \\[1em]
      (A\to B) & \lor (B\to C) \\
      ((A\to B)\to A) & \to A && (\text{Peirce’sche Regel}) \\
    (A \to(A\land B)) & \lor (B\to (A\land B)) &&  (\text{Aussagenlogisches }``\text{Drinker paradox}")
 \end{align*}
 \endgroup

 
 

 \section*{Prädikatenlogische Tautologien}
  Seien $A$ eine Aussage, $\varphi,\psi$ zwei einstellige Prädikate und $\chi$ ein zweistelliges Prädikat. Bei den folgenden Aussagen handelt es sich um Tautologien. Wenn du Lust hast, kannst du
 \begin{itemize}
  \item versuchen, dir für die ein oder andere Formel intuitiv klarzumachen, dass es sich um eine Tautologie handeln muss. So kannst du deine Intuition für die Quantoren trainieren.
  \item mithilfe der Beweistechniken aus dem zweiten Vortrag versuchen, die eine oder andere dieser Formeln zu beweisen.
 \end{itemize}
 \begingroup
 \allowdisplaybreaks
   \begin{align*}
  \begin{split} (\forall x : \varphi(x)) & \leftrightarrow (\forall y : \varphi(y)) \\
  (\exists x : \varphi(x)) & \leftrightarrow (\exists y : \varphi(y)) \end{split} &  \begin{split}
  (\text{Austauschbarkeit der} \\
  \text{gebundenen Variable})
  \end{split}  \\[1em]
  \begin{split} (\forall x,y : \chi(x,y)) & \to (\forall x : \chi(x,x)) \\
    (\exists x : \chi(x,x)) & \to (\exists x, y: \chi(x,y)) \end{split} \\[1em]
      \begin{split} ((\forall x : \varphi(x)) \land A) & \leftrightarrow (\forall x :(\varphi(x) \land A)) \\
           ((\forall x : \varphi(x)) \land (\forall y : \psi(y))) & \leftrightarrow (\forall x,y : (\varphi(x) \land \psi(y)))\\
     ((\forall x : \varphi(x)) \land (\forall y : \psi(y))) & \leftrightarrow (\forall x : (\varphi(x) \land \psi(x))) 
     \end{split} && (\text{$\land$ und $\forall$}) \\[1em]
              \begin{split}
((\exists x : \varphi(x)) \lor A) & \leftrightarrow (\exists x : (\varphi(x) \lor A)) \\
      ((\exists x : \varphi(x)) \lor (\exists y : \psi(y))) & \leftrightarrow (\exists x,y : (\varphi(x) \lor \psi(y))) \\
      ((\exists x : \varphi(x)) \lor (\exists y : \psi(y))) & \leftrightarrow (\exists x : (\varphi(x) \lor \psi(x)))
      \end{split} && (\text{$\lor$ und $\exists$}) \\[1em]
          \begin{split}
((\exists x : \varphi(x)) \land A) & \leftrightarrow (\exists x : (\varphi(x) \land A)) \\
   ((\exists x : \varphi(x)) \land (\exists y : \psi(y))) & \leftrightarrow (\exists x,y:(\varphi(x)\land \psi(y))) \\
      ((\exists x : \varphi(x)) \land (\exists y : \psi(y))) & \leftarrow (\exists x : (\varphi(x) \land \psi(x)))
      \end{split} && (\text{$\land$ und $\exists$}) \\[1em]
  \begin{split} ((\forall x : \varphi(x)) \lor A) & \leftrightarrow (\forall x :(\varphi(x) \lor A)) \\ %nur in klassischer Logik
      ((\forall x : \varphi(x)) \lor (\forall y : \psi(y))) & \leftrightarrow (\forall x,y : (\varphi(x)\lor \psi(y))) \\%nur in klassischer Logik
     ((\forall x : \varphi(x)) \lor (\forall y : \psi(y))) & \to (\forall x : (\varphi(x) \lor \psi(x)))
     \end{split} && (\text{$\lor$ und $\forall$}) \\[1em]
     \begin{split}
 (\forall x : (\varphi(x) \to A)) & \leftrightarrow ((\exists x : \varphi(x)) \to A) \\
  (\forall x : (A \to \varphi(x))) & \leftrightarrow (A \to (\forall x : \varphi(x))) \\
  (\forall x : (\varphi(x) \to \psi(x))) & \leftarrow ((\exists x : \varphi(x)) \to (\forall x : \psi(x)))
\end{split} && (\text{$\to$ und $\forall$}) \\[1em]
\begin{split}
( \exists x : (\varphi(x) \to A)) & \leftrightarrow ((\forall x : \varphi(x)) \to A) \\ %nur in klassischer Logik
( \exists x : ( A\to \varphi(x))) & \leftrightarrow (A \to( \exists x : \varphi(x))) \\ %nur in klassischer Logik
(\exists x : (\varphi(x)\to \psi(x))) & \leftrightarrow ((\forall x : \varphi(x)) \to (\exists x : \psi(x)))
\end{split} && (\text{$\to$ und $\exists$}) \\[1em] %nur in klassischer Logik
\forall x\ \exists y: (\varphi(x) & \to \varphi(y)) \\
\exists x\ \forall y : (\varphi(x) & \to \varphi(y)) && (``\text{\href{https://en.wikipedia.org/wiki/Drinker_paradox}{Drinker paradox}}")%\\nur in klassischer Logik
%\nexists x \ \forall y : (\chi(x,y) & \leftrightarrow \neg \chi(y,y)) && (\text{Russellsche Antinomie})\\[1em]
%\forall x : (x & = x) && (\text{Identitätsaxiom}) \\
%\forall x,y : (x=y & \to (\varphi(x) \leftrightarrow \varphi(y))) && (\text{Ununterscheidbarkeitsaxiom}) \\
%\forall x,y : (x=y & \leftrightarrow y=x) && (\text{Symmetrie von $=$}) \\
%\forall x,y,z : ((x=y \land y=z) & \to x=z) && (\text{Transitivität von $=$})
 \end{align*}
 \endgroup
%Ist das Diskursuniversum die Menge der Leute in einer Bar und ist $\varphi(x):\Leftrightarrow$ „$x$ trinkt“, so besagt das \href{https://en.wikipedia.org/wiki/Drinker_paradox}{Drinker paradox}: „Es gibt eine Person $x$, für die gilt: Falls $x$ trinkt, dann trinkt jede Person $y$ in der Bar.“ Es lässt sich nicht in konstruktiver Logik beweisen.


%\section*{A.1 Normalverteilung}\label{Ztable}
%\begin{center}
%\includegraphics[scale=1]{./_img/Ztable.pdf}
%\end{center}

\cleardoublepage
\bibliography{_src/Skript-ST1}
  \addcontentsline{toc}{chapter}{Literaturverzeichnis}
\cleardoublepage
\printindex
\end{document}

%%% Local Variables:
%%% mode: latex
%%% TeX-master: t
%%% End:
