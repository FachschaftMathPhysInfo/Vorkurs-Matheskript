
\newacronym{sym:N}{$\N$}{Menge der natürlichen Zahlen.}
\newacronym{sym:Z}{$\Z$}{Menge der ganzen Zahlen}
\newacronym{sym:Q}{$\Q$}{Menge der rationalen Zahlen}
\newacronym{sym:R}{$\R$}{Menge der reellen Zahlen}
\newacronym{sym:C}{$\C$}{Menge der komplexen Zahlen}
\newacronym{sym:nZ}{$n\Z$}{Menge aller ganzzahligen Vielfachen von $n$}


\newacronym{sym:perdef}{\quad\\[0.5em] $:=$}{Per Definition gleich} % Das \quad ist zum Einfügen eines Phantom-Zeilenabstands (sorry, kriegs nicht besser hin)
\newacronym{sym:perdefiff}{$:\Leftrightarrow$}{Per Definition äquivalent zu}
\newacronym{sym:land}{$\land$}{Und}
\newacronym{sym:lor}{$\lor$}{Oder}
\newacronym{sym:neg}{$\neg$}{Aussagenlogische Negation}
\newacronym{sym:implikation}{$\rightarrow$\quad bzw.\quad $\Rightarrow$}{Aussagenlogische Implikation}
\newacronym{sym:iff}{$\leftrightarrow$\quad bzw.\quad $\Leftrightarrow$}{Aussagenlogische Äquivalenz}
\newacronym{sym:forall}{$\forall$}{Allquantor}
\newacronym{sym:exists}{$\exists$}{Existenzquantor}


\newacronym{sym:in}{\quad\\[0.5em] $\in$}{Element}
\newacronym{sym:subseteq}{$\subseteq$}{Teilmenge}
\newacronym{sym:subsetneq}{$\subsetneq$}{Echte Teilmenge}
\newacronym{sym:emptyset}{$\emptyset$}{Leere Menge}
\newacronym{sym:potenzmenge}{$\calP(X)$}{Potenzmenge von $X$}
\newacronym{sym:familie}{$(a_i)_{i\in I}$}{Durch die Menge $I$ indizierte Familie von Objekten $a_i$}
\newacronym{sym:mengenpotenz}{$M^I$}{Menge der durch die Menge $I$ indizierten Familien mit Einträgen aus der Menge $M$}
\newacronym{sym:matrizenmenge}{$M^{n\times m}$}{Menge der $(n\times m)$-Matrizen mit Einträgen aus $M$}
\newacronym{sym:paar}{$(x,y)$}{Geordnetes Paar bestehend aus zwei Objekten $x,y$}
\newacronym{sym:cap}{$\cap$}{Durchschnitt von Mengen}
\newacronym{sym:cup}{$\cup$}{Vereinigung von Mengen}
\newacronym{sym:setminus}{$\setminus$}{Differenz zweier Mengen}
\newacronym{sym:komplement}{$M^c$}{Komplement von $M$ (in einer fixierten Obermenge)}
\newacronym{sym:times}{$\times$}{Kartesisches Produkt}
\newacronym{sym:sqcup}{$\sqcup$}{Disjunkte Vereinigung}


\newacronym{sym:funktion}{\quad\\[0.5em] $f:X\to Y$\quad bzw.\quad $X\xrightarrow{f} Y$}{$f$ ist eine Abbildung von $X$ nach $Y$}
\newacronym{sym:zuordnung}{$x\mapsto t(x)$}{Dem Objekt $x$ wird das Objekt $t(x)$ zugeordnet}
\newacronym{sym:funktionswert}{$f(x)$}{Funktionswert von $f$ an der Stelle $x$}
\newacronym{sym:abb}{$\Abb(X,Y)$}{Menge der Abbildungen von $X$ nach $Y$}
\newacronym{sym:circ}{$\circ$}{Verkettung von Abbildungen}
\newacronym{sym:id}{$\id_X$}{Identität auf $X$}
\newacronym{sym:inklusion}{$\iota: U\hookrightarrow X$}{Inklusion der Teilmenge $U\subseteq X$}
\newacronym{sym:faser}{$f^{-1}(x)$}{Faser von $x$ oder aber Funktionswert unter der inversen Abbildung (sofern diese existiert)}
\newacronym{sym:im}{$\im(f)$}{Bild von $f$}
\newacronym{sym:urbild}{$f^{-1}(B)$}{Urbild von $B$ unter $f$}
\newacronym{sym:bild}{$f(A)$}{Bild von $A$ unter $f$}
\newacronym{sym:domres}{$f\vert_A$}{Einschränkung von $f$ im Definitionsbereich}
\newacronym{sym:codomres}{$f\vert^A$}{Einschränkung von $f$ im Wertebereich}


\newacronym{sym:le}{\quad\\[0.5em] $\le$}{Generisches Zeichen für eine Ordnungsrelation}
\newacronym{sym:min}{$\min$}{Minimum, kleinstes Element einer geordneten Menge}
\newacronym{sym:max}{$\max$}{Maximum, größtes Element einer geordneten Menge}
\newacronym{sym:inf}{$\inf$}{Infimum, größte untere Schranke}
\newacronym{sym:sup}{$\sup$}{Supremum, kleinste obere Schranke}
\newacronym{sym:intervall}{$(a,b),\ [a,b],\ (a,b],\ [a,b)$}{Offenes, abgeschlossenes und halboffene Intervalle mit Endpunkten $a$ und $b$}
\newacronym{sym:teilt}{$a\mid b$}{$a$ teilt $b$}
\newacronym{sym:aequiklasse}{$[x]$\quad bzw.\quad $\bar x$}{Äquivalenzklasse von $x$ bzgl. einer Äquivalenzrelation}
\newacronym{sym:faktormenge}{$X/{\sim}$}{Faktormenge modulo $\sim$}
\newacronym{sym:projektion}{$\pi:X\twoheadrightarrow X/{\sim}$}{Projektion auf die Faktormenge $X/{\sim}$}


\newacronym{sym:addinv}{\quad\\[0.5em] $-x$}{Additiv Inverses von $x$}
\newacronym{sym:multinv}{$x^{-1}$}{Multiplikativ Inverses von $x$}
\newacronym{sym:sum}{$\sum\limits_{k=m}^n a_k$}{Summe der $a_m,\dots , a_n$}
\newacronym{sym:prod}{$\prod\limits_{k=m}^n a_k$}{Produkt der $a_m,\dots , a_n$}
\newacronym{sym:einheitengruppe}{$M^\times$}{Einheitengruppe von $M$}
\newacronym{sym:symgruppe}{$S_n$}{$n$-te symmetrische Gruppe}


\newacronym{sym:Rdach}{\quad\\[0.5em] $\bar \R$}{Um $\pm\infty$ erweiterte reelle Zahlen}
\newacronym{sym:betrag}{$\vert x\vert$}{Betrag von $x$}
\newacronym{sym:sgn}{$\sgn(x)$}{Vorzeichen von $x$}
\newacronym{sym:abstand}{$d(x,y)$}{Abstand von $x$ zu $y$}
\newacronym{sym:ball}{$\bbB_r(a)$}{Offener Ball mit Radius $r$ um den Punkt $a$}
\newacronym{sym:lim}{$\lim\limits_{n\to\infty} a_n$}{Limes der Folge der $a_n$}
\newacronym{sym:konvergenz}{$a_n \xrightarrow[n\to \infty]{} a$}{Die $a_n$'s konvergieren gegen den Punkt $a$}
