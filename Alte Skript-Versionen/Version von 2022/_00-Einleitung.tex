

\chapter{Vorwort: Wie dieses Buch zu lesen ist}

Gleich zu Beginn sei gesagt: die beiden Vorkurswochen sind vor allem dazu da, dass du Menschen kennenlernst, die Uni und die Stadt erkundest. Verbringe also besser nicht so viel Zeit mit diesem Skript!

Entgegen der Bezeichnung „Skript“ handelt es sich bei diesem Text eigentlich schon um ein regelrechtes Lehrbuch. Dementsprechend ist es bei Weitem ausführlicher als die Vorträge, die du während des Vorkurses hören wirst. Es soll folgende Funktionen erfüllen:
\begin{enumerate}[1.]
    \item Dir während der Zeit zwischen Schule und Studium zur selbständigen Vorbereitung auf die Uni-Mathematik dienen.
    \item Dir während des Vorkurses ermöglichen, die Vorträge zu vertiefen.
    \item Dir während des ersten Semesters als Nachschlagewerk mathematischer Grundlagen dienen.
\end{enumerate}
Abschnitte, die ich für die Vorträge für ungeeignet halte, sind mit einem * markiert. Sie sollten von den Dozenten am ehesten im Vortrag ausgelassen werden.

Weitere Tipps für die Lektüre:
\begin{itemize}
    \item Anfänger neigen dazu, einen mathematischen Text ähnlich einem Roman von vorne bis hinten linear durchzulesen und sich an einer Stelle, die sie nicht verstehen, solange festzubeißen, bis die Aussage nachvollziehbar wird. Damit kannst du allerdings viel Zeit vergeuden. Wenn du mit einer Definition nichts anfangen kannst, versuche erstmal nicht, alle Details der Definition nachzuvollziehen, sondern lies dir die folgenden Beispiele durch, um ein Gespür dafür zu bekommen, worum es überhaupt gehen soll. Wenn dir der Sinn eines Satzes beim ersten Lesen nicht klar wird, überspringe seinen Beweis erstmal und schau dir an, welche Schlussfolgerungen aus dem Satz gezogen werden und wie seine Aussage durch die Beispiele illustriert wird. Wenn ich ein Kapitel in einem Mathebuch lese, sind die längeren Beweise meist, was ich zuletzt lese.
    \item Manche verlinkten Artikel sowie die als „Vorschau“ gekennzeichneten Abschnitte enthalten Inhalt, der für dich zu fortgeschritten ist, bei dem du Vieles nicht verstehen wirst und der erst in späteren Semestern für dich Sinn ergeben wird. Ich habe diese Abschnitte und Links hinzugefügt, um dir eine Referenz an die Hand und eine Vorschau auf die Weitläufigkeit der Mathematik zu geben. Setz dich bloß nicht unter Druck, die verlinkten Texte nachvollziehen zu müssen!
    \item Dieses Skript wurde in {\LaTeX} geschrieben. Sofern es dein PDF-Viewer unterstützt, kannst du allerlei mögliche Referenzen und Hyperlinks (z.B. die Einträge des Inhaltsverzeichnisses) einfach anklicken, um zum verwiesenen Ort zu gelangen.
\end{itemize}
Nach sechs Jahren geht mein eigenes Studium nun zu Ende. Ich wünsche dir alles Gute für Deines!

\quad

Luka \hfill September 2022





\clearpage
\section*{Vorwort von 2021}

Das Skript zum mathematischen Vorkurs wurde zum Wintersemester 2021/2022 von Luka Thomé, Matthis Scholz, Nikolaus Betker, Maximilian Bur und Luna Cielibak grundlegend neu verfasst. Autoren der einzelnen Kapitel sind:
\begin{enumerate}[label={Kapitel \arabic*}, labelindent=1.5em, leftmargin=*]
    \item Logik -- Luka Thomé
    \item Beweise -- Luka Thomé
    \item Mengen und Familien -- Matthis Scholz
    \item Abbildungen -- Nikolaus Betker, Luka Thomé
    \item Relationen -- Matthis Scholz
    \item Folgen, Abstand und Grenzwerte -- Maximilian Bur, Luka Thomé
    \item Verknüpfungen -- Luka Thomé
\end{enumerate}
Wir wünschen euch zwei schöne Vorkurswochen, einen guten Start ins Studentenleben und hoffen, euch mit diesem Skript einen ersten Einblick in die Uni-Mathematik bieten zu können, der neugierig auf mehr macht.

\quad

Nikolaus, Max, Luna, Matthis, Luka \hfill September 2021
